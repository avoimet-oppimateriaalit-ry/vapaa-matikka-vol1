\begin{tehtavasivu}

\subsubsection*{Opi perusteet}

\begin{tehtava}
    Mitkä seuraavista ovat polynomeja?
    \alakohdatm{
        § $\frac{1}{x}$
        § $5x-125$
        § $\sqrt{x}+1$
        § $3x^4+6x^2+9$
        § $\sqrt{2}x-x$
        § $4^x+5x+6$
        § $y^4+2\pi$
        § $\frac{4y^2-y}{2y}$
        § $\frac{6x+40}{x^3}$
        }
    \begin{vastaus}
        \alakohdatm{
            § Ei ole.
            § On.
            § Ei ole.
            § On.
            § On.
            § Ei ole.
            § On.
            § On. (Sieventämällä saadaan $2y-\frac{1}{2}$, vaikkakin $y\neq 0$.)
            § Ei ole.
        }
    \end{vastaus}
\end{tehtava}

\begin{tehtava}
	Mikä on/mitkä ovat polynomin $P(x) = x^5-3x^3+2x-1$
	\alakohdat{
		§ aste
		§ termit
		§ kolmannen asteen termi
		§ kolmannen termin aste
		§ vakiotermi?
	}
	\begin{vastaus}
		\alakohdatm{
			§ $5$
			§ $x^5$, $-3x^3$, $2x$, $-1$
			§ $-3x^3$
			§ $1$
			§ $-1$
		}
	\end{vastaus}
\end{tehtava}

\begin{tehtava}
    Täydennä taulukko. Polynomeissa on vain yksi muuttuja, $x$.
        
    \begin{tabular}{|c|c|c|c|c|}
                                                                         \hline
polynomi     & \begin{sideways}termien lukumäärä\end{sideways}%
& \begin{sideways}korkeimman asteen termin kerroin\end{sideways}%
& \begin{sideways}polynomin asteluku\end{sideways}%
& \begin{sideways}vakiotermi\end{sideways} \\ \hline
$-2x^2+6x$   &        $2$  &         $-2$      &       $2$   &    $0$       \\ \hline 
$7x^3-x-15$  &           &                   &           &            \\ \hline 
             &        $2$  &          $-9$     &       $2$   &    $5$       \\ \hline 
%             &        3  &          $-1$     &       5   &    $-17$   \\ \hline 
%             &        4  &                   &       3   &            \\ \hline 
             &        $1$  &         $ -5$       &       $99$  &            \\ \hline                           
    \end{tabular}

%      \begin{tabular}{|l|c|c|c|c|c|c|}
%                                                                                            \hline
% polynomi     & \begin{sideways}$-2x^2+6x$\end{sideways} & \begin{sideways}$7x^3-x-15$\end{sideways}    &     &          &     &     \\ \hline
% termien      &            &                &     &          &     &     \\ \hline 
% lukumäärä    &        2   &                & 2   &    3     &  4  &  1  \\ \hline 
% korkeimman & & & & & & \\  
% asteen & & & & & & \\  
% termin & & & & & & \\  
% kerroin      &    $-2$    &                &$-9$ &   $-1$   &     &$-5$ \\ \hline 
% polynomin & & & & & & \\  
% asteluku     &        2   &                & 2   &    5     &  3  & 99  \\ \hline 
% vakiotermi   &        0   &                & 5   &    $-17$ &     &     \\ \hline 
%     \end{tabular}
%      \begin{tabular}{|c|c|c|c|c|}
%                                                                                           \hline
%              & termien   & korkeimman asteen & polynomin &            \\
% polynomi     & lukumäärä & termin kerroin    & asteluku  & vakiotermi \\ \hline
% $-2x^2+6x$   &        2  &         $-2$      &       2   &    0       \\ \hline 
% $7x^3-x-15$  &           &                   &           &            \\ \hline 
%              &        2  &          $-9$     &       2   &    5       \\ \hline 
%              &        3  &          $-1$     &       5   &    $-17$   \\ \hline 
%              &        4  &                   &       3   &            \\ \hline 
%              &        1  &          -5       &       99  &            \\ \hline                           
%     \end{tabular}

    \begin{vastaus}
    \begin{footnotesize}
	    \begin{tabular}{|c|c|c|c|c|}
	     \hline
polynomi     & \begin{sideways}termien lukumäärä\end{sideways}%
& \begin{sideways}korkeimman asteen termin kerroin\end{sideways}%
& \begin{sideways}polynomin asteluku\end{sideways}%
& \begin{sideways}vakiotermi\end{sideways} \\ \hline
$-2x^2+6x$   &        $2$          &         $-2$      &       $2$             &    $0$       \\ \hline 
$7x^3-x-15$  &        $3$          &           $7$       &       $3$             &    $-15$   \\ \hline 
$-9x^2+5$    &        $2$          &          $-9$     &       $2$             &    $5$       \\ \hline 
%$-x^5\textcolor{blue}{+4x}-17$%
%             &        3          &          $-1$     &       5             &    $-17$   \\ \hline 
%$\textcolor{blue}{8}x^3\textcolor{blue}{-x^2+4x}-17$%
%             &        4          &\textcolor{blue}{8}  &       3             &\textcolor{blue}{17}\\ \hline 
$-5x^{99}$   &        $1$          &          $-5$     &       $99$            &         $0$      \\ \hline                           
   	  \end{tabular}
      \end{footnotesize}
     \end{vastaus}
\end{tehtava}

\begin{tehtava}
    Olkoot $P(x)=x^2+5$ ja $Q(x)=x^3-1$. Laske
    \alakohdat{
        § polynomin $P(x)$ arvo, kun $x=2$
        § polynomin $Q(x)$ arvo, kun $x=1$
        § $P(-7)$
        § $Q(-1)$.
    }
    \begin{vastaus}
        \alakohdatm{
            § $9$ % 2^2 + 5 = 4 + 5 
            § $0$ % 1^3 - 1
            § $54$ % (-7)^2 + 5 = 49 + 5 
            § $-2$ % (-1)^3 - 1 = -1 -1
        }
    \end{vastaus}
\end{tehtava}

\subsubsection*{Hallitse kokonaisuus}

\begin{tehtava}
	Mitkä ovat seuraavien polynomien asteet?
	\alakohdat{
		§ $x^2 + 3x - 5$
		§ $100 + x$
		§ $3x^3 + 90x^8 + 2x$
		§ $12x^1 + 34x^2 + 56x^3 + 78x^5 + 90x^5$
	}
	\begin{vastaus}
		\alakohdatm{
			§ $2$
			§ $1$
			§ $8$
			§ $5$
		}
	\end{vastaus}
\end{tehtava}

\begin{tehtava}
    Olkoot $P(x)=x^2+3x+4$ ja \\ $Q(x)=x^3-10x+1$. Laske
    \alakohdatm{
        § $P(-1)$
        § $Q(-2)$
        § $P(3)$
        § $Q(0)$
        § $P(2x)$.
    }
    \begin{vastaus}
        \alakohdatm{
            § $2$
            § $13$
            § $22$
            § $1$
            § $4x^2+6x+4$
        }
    \end{vastaus}
\end{tehtava}

\subsubsection*{Lisää tehtäviä}

\begin{tehtava}
Täydennä taulukko. Polynomeissa on muuttujat $x$ ja $y$.
    \begin{tabular}{|c|c|c|c|c|}
 \hline
polynomi     & \begin{sideways}termien lukumäärä\end{sideways}%
& \begin{sideways}korkeimman asteen termin kerroin\end{sideways}%
& \begin{sideways}polynomin asteluku\end{sideways}%
& \begin{sideways}2. asteen termin kerroin\end{sideways} \\ \hline
$ x+xy^2+6y^2-1$ &        4  &         1      &       3   &    1       \\ \hline 
$ ax^6+y^4-4by^2+\sqrt{2}$  &           &                   &           &            \\ \hline 
$ 100x+ x^3y^4-99y$           &         &               &          &           \\ \hline 
$ 8y-\frac{\sqrt{15}}{2}x^3$    &          &            &         &       \\ \hline 
$ \sqrt{3}xy^6 + y^8 - 4x^2$ &          &                   &          &            \\ \hline 
    \end{tabular}

%      \begin{tabular}{|l|c|c|c|c|c|c|}
%                                                                                            \hline
% polynomi     & \begin{sideways}$-2x^2+6x$\end{sideways} & \begin{sideways}$7x^3-x-15$\end{sideways}    &     &          &     &     \\ \hline
% termien      &            &                &     &          &     &     \\ \hline 
% lukumäärä    &        2   &                & 2   &    3     &  4  &  1  \\ \hline 
% korkeimman & & & & & & \\  
% asteen & & & & & & \\  
% termin & & & & & & \\  
% kerroin      &    $-2$    &                &$-9$ &   $-1$   &     &$-5$ \\ \hline 
% polynomin & & & & & & \\  
% asteluku     &        2   &                & 2   &    5     &  3  & 99  \\ \hline 
% vakiotermi   &        0   &                & 5   &    $-17$ &     &     \\ \hline 
%     \end{tabular}
%      \begin{tabular}{|c|c|c|c|c|}
%                                                                                           \hline
%              & termien   & korkeimman asteen & polynomin &            \\
% polynomi     & lukumäärä & termin kerroin    & asteluku  & vakiotermi \\ \hline
% $-2x^2+6x$   &        2  &         $-2$      &       2   &    0       \\ \hline 
% $7x^3-x-15$  &           &                   &           &            \\ \hline 
%              &        2  &          $-9$     &       2   &    5       \\ \hline 
%              &        3  &          $-1$     &       5   &    $-17$   \\ \hline 
%              &        4  &                   &       3   &            \\ \hline 
%              &        1  &          -5       &       99  &            \\ \hline                           
%     \end{tabular}

    
    \begin{vastaus}
    \begin{footnotesize}
    \begin{tabular}{|c|c|c|c|c|}
                                                                         \hline
polynomi
& \begin{sideways}termien lukumäärä\end{sideways}%
& \begin{sideways}korkeimman asteen termin kerroin\end{sideways}%
& \begin{sideways}polynomin asteluku\end{sideways}%
& \begin{sideways}2. asteen termin kerroin\end{sideways} \\ \hline
$ x+xy^2+6y^2-1$ 		&      4  &         1      &       3   &    1       \\ \hline 
$ ax^6+y^4-4by^2+\sqrt{2}$  	&     4    &    $a$       &    6   &     4b       \\ \hline 
$ 100x+ ix^3y^4-99y$          	&    3     &      $i$    &     7     &    0       \\ \hline 
$ 8y-\frac{\sqrt{15}}{2}x^3$	 &     2     &    $-\frac{\sqrt{15}}{2}$    &     3    &    0   \\ \hline 
$ \sqrt{3}xy^6 + y^8 - 4x^2$ 	&   3       &         1        &    8      &     -4       \\ \hline 
    \end{tabular}
    \end{footnotesize}
     \end{vastaus}
\end{tehtava}

\begin{tehtava}
	$\star$ Määritellään kahden reaalimuuttujan polynomifunktio kaavalla $f(x,y)=xy^2+x^2y$.
		\alakohdat{
		§ Laske funktion arvo $f(-1,2)$.
		§ Onko $f(x,y)=f(y,x)$ kaikilla $x$ ja $y$?
		}
	\begin{vastaus}
		\alakohdat{
			§ $f(-1,2)=(-1)\cdot 2^2+(-1)^2 \cdot 2=-4+2=-2$
			§ Kyllä; termi muuttuvat toisikseen, jos muuttujat vaihtavat paikkaa, ja lauseke pysyy samana.
		}
	\end{vastaus}
\end{tehtava}

\begin{tehtava}
	$\star$ Kahden muuttujan ($x$ ja $y$) binomista tiedetään, että sen asteluku on kaksi, vakiotermejä ei ole, ja kaikkien termien kertoimet ovat ykkösiä. Luettele kaikki mahdolliset polynomit, jotka toteuttavat nämä ehdot.
	\begin{vastaus}
		$x^2+y$, $x^2+xy$, $y^2+x$, $y^2+xy$, $x^2+y^2$
	\end{vastaus}
\end{tehtava}

%\begin{tehtava} rationaalilausekkeista polynomeja :)
%    Mitkä seuraavista ovat polynomeja?
%    \alakohdat{
%        § $\frac{1}{x}$
%       %§ $x^3+4x$
%        § $5x-125$
%       %§ $2^x$
%        § $\sqrt{x}+1$
%        § $3x^4+6x^2+9$
%        § $\sqrt{2}x-x$
%        § $4^x+5x+6$
%    }
%    \begin{vastaus}
%        \alakohdat{
%            § Ei ole.
%           %§ On.
%            § On.
%           %§ Ei ole.
%            § Ei ole.
%            § On.
%            § On.
%            § Ei ole.
%        }
%    \end{vastaus}
%\end{tehtava}

\begin{tehtava}
	$\star$ Määritä kolmen muuttujan ($x$, $y$ ja $z$) polynomifunktion $R(x,y,z)=ax^6+2y^8-\sqrt{56}xy^2+y^2z-4z$
	arvo, kun $x=0$, $y=\sqrt{2}$ ja $z=\frac{\pi}{2}$.
	\begin{vastaus}
	$R(0,\sqrt{2},\frac{\pi}{2})$ \\ $=0+2\cdot2^4-0+\pi-2\pi=32-\pi$
	\end{vastaus}
\end{tehtava}

\end{tehtavasivu}