\begin{tehtavasivu}

\subsubsection*{Opi perusteet}

\begin{tehtava}
    Kerro sulut auki muistikaavan avulla.
    \alakohdat{
        § $(x+y)^2$
        § $(x-y)^2$
        § $(x+y)(x-y)$
    }
    \begin{vastaus}
        \alakohdat{
        § $x^2 +2xy+y^2$
        § $x^2 -2xy +y^2$
        § $x^2-y^2$
        }
    \end{vastaus}
\end{tehtava}

\begin{tehtava}
    Kerro sulut auki muistikaavan avulla.
    \alakohdat{
        § $(x+3)^2$
        § $(y-5)^2$
        § $(x-4)(x+4)$
        § $(3x+2)^2$
    }
    \begin{vastaus}
        \alakohdat{
        § $x^2 +6x+9$
        § $y^2 - 10y+25$
        § $x^2 -16$
        § $9x^2 +12x +4$
        }
    \end{vastaus}
\end{tehtava}

\begin{tehtava}
Osoita, että seuraavat laskukaavat eivät päde etsimällä esimerkki reaaliluvuista $a$ ja $b$, joilla yhtälö ei päde.
        \alakohdat{
        § $(a+b)^2=a^2+b^2$
        § $\sqrt{a+b}=\sqrt{a}+\sqrt{b}$
        § $\dfrac{3a}{3b+c}=\dfrac{a}{b+c}$
        }
        \begin{vastaus}
        \alakohdat{
            § Kaava ei päde, sillä esimerkiksi $(1+2)^2=3^2=9$, mutta $1^2+2^2=1+4=5$.
            § Kaava ei päde, sillä esimerkiksi $\sqrt{1+1}=\sqrt{2}$, mutta $\sqrt{1}+\sqrt{1}=1+1=2$.
            § Kaava ei päde, sillä esimerkiksi $\frac{3 \cdot 2}{3 \cdot 1+1}=\frac{6}{5}$, mutta $\frac{2}{1+1}=1$.
        }
        \end{vastaus}
\end{tehtava}

\begin{tehtava}
    Esitä tulona eli käytä muistikaavaa toiseen suuntaan.
    \alakohdat{
        § $a^2+2ab+b^2$
        § $x^2+2x+1$
        § $y^2-4y+4$
        § $a^2-25$
    }
    \begin{vastaus}
        \alakohdat{
            § $(a+b)^2$
            § $(x+1)^2$
            § $(y-2)^2$
            § $(a+5)(a-5)$
        }
    \end{vastaus}
\end{tehtava}

\subsubsection*{Hallitse kokonaisuus}

\begin{tehtava}
    Sievennä.
    \alakohdat{
            § $(a+b)^3$
            § $(a+b)^4$
            § $(a-b)^3$
        }
    \begin{vastaus}
        \alakohdat{
            § $(a+b)^3 = a^3 + 3a^2b + 3ab^2 + b^3$
            § $(a+b)^4 = a^4 + 4a^3b + 6a^2b^2 + 4ab^3 + b^4$
            § $(a-b)^3 = a^3 - 3a^2b + 3ab^2 - b^3$
        }
    \end{vastaus}
\end{tehtava}

\begin{tehtava}
    Osoita, että
    \alakohdat{
        § $(t+v)^2+(t-v)^2=2t^2+2v^2$
        § $(t+v)^2-(t-v)^2 = 4tv$.
    }
    \begin{vastaus}
    	Sievennetään yhtälöiden vasempia puolia:
        \alakohdat{
            § $(t+v)^2+(t-v)^2 = t^2+2tv+v^2+t^2-2tv+v^2 = 2t^2+2v^2$.
            § $(t+v)^2-(t-v)^2 = t^2+2tv+v^2-t^2+2tv-v^2 = 4tv$.
        }
    \end{vastaus}
\end{tehtava}

\begin{tehtava}
	Tiedetään, että $(a+b)^2=12$ ja $(a-b)^2=4$. Ratkaise tulon $ab$ suuruus.
    \begin{vastaus}
	$ab = 2$
    \end{vastaus}
\end{tehtava}

\begin{tehtava}
Tutkitaan funtiota $f$, jonka arvot lasketaan kaavalla $f(n)=n^2+2n+2$, missä $n$ on mielivaltainen kokonaisluku.
	\alakohdat{
		§ Osoita, että $f$ kuvaa parillisen kokonaisluvun aina uudeksi parilliseksi luvuksi.
		§ Tutki, mitä vastaavasti parittomille kokonaisluvuille käy.
	}
	\begin{vastaus}
		\alakohdat{
		§ Parilliset luvut voidaan aina kirjoittaa muodossa $2k$, missä $k$ on jokin kokonaisluku. Sijoitetaan lauseke muuttujan paikalle, niin funktion arvoksi saadaan $f(2k)=(2k)^2+2\cdot 2k+2$ \\ $=4k^2+4k+2=2(2k^2+2k+1)$. Koska $k$:n neliö on kokonaisluku, $k$ kerrottuna toisella kokonaisluvulla on kokonaisluku, ja kokonaislukujen summat ovat kokonaislukuja, on $2k^2+2k+1$ myös kokonaisluku. Funktion arvo saatiin siis kirjoitettua muodossa ''kaksi kertaa kokonaisluku'', joten mielivaltaisesta parillisesta luvusta tuli funktion arvoksi myös parillinen luku. 
		§ Mielivaltaisen parittoman luvun voi kirjoittaa muodossa $2k+1$, missä $k$ on kokonaisluku. Nyt funktion arvoksi saadaan $f(2k+1)=(2k+1)^2+2(2k+1)+2$ \\ $=4k^2+4k+1+4k+2+2=4k^2+8k+5$ \\ $=2(2k^2+4k+2)+1$. Sulkeissa oleva lauseke on a-kohdan päättelyllä aina kokonaisluku, joten funktion arvo on muotoa ''kaksi kertaa kokonaisluku plus yksi'', eli aina pariton luku. Parittomasta luvusta tulee siis funktion arvona aina pariton luku.
	}
	\end{vastaus}
\end{tehtava}

\begin{tehtava}
(YO 1888/1) Mikä on $a/b$:n arvo, jos $(\sqrt{a}+\sqrt{b}):(\sqrt{a}-\sqrt{b})=\sqrt{2}$?
	\begin{vastaus}
$(3-2\sqrt{2}):(3+2\sqrt{2})$
	\end{vastaus}
\end{tehtava}

\subsubsection*{Lisää tehtäviä}

\begin{tehtava}
    Sievennä muistikaavojen avulla.
    \alakohdat{
        § $(x+6)^2$
        § $(3y-1)^2$
        § $(-3-x)(x-3) $
    }
    \begin{vastaus}
        \alakohdat{
            § $x^2 + 12x + 36$
            § $9y^2 - 6y + 1$
            § $9-x^2$
        }
    \end{vastaus}
\end{tehtava}

\begin{tehtava}
    Sievennä muistikaavojen avulla.
    \alakohdat{
        § $(5-x)^2$
        § $(7x + 4)^2$
        § $(9 - 7x)(9 + 7x)$
        § $(8x - 8)(8x + 8)$
    }
    \begin{vastaus}
        \alakohdat{
            § $x^2 - 10x + 25$
            § $49x^2 + 56x + 16$
            § $-49x^2 + 81$
            § $64x^2 - 64$
        }
    \end{vastaus}
\end{tehtava}

\begin{tehtava}
   	Muistikaavat on opittava ulkoa ja niiden käytön tulee automatisoitua. Laske siis nämä käyttäen muistikaavoja. Tavoite on kirjoittaa vastaus suoraan ilman välivaiheita. Jos se ei vielä onnistu, yritä selvitä yhdellä välivaiheella.
    \alakohdat{
        § $(x+2)^2$
        § $(x-5)^2$
        § $(b+4)(b-4)$
        § $(x-1)^2$
        § $(2x+1)^2$
        § $(1-a)(1+a)$
        § $(x-7)^2$
        § $(y+1)^2$
        § $(x-3y)^2$
        § $(3x-1)^2$                                                                                                                             
    }
    \begin{vastaus}
        \alakohdat{
            § $x^2+4x+4$
            § $x^2-10x+25$
            § $b^2-16$
            § $x^2-2x+1$
            § $4x^2+4x+1$
            § $1-a^2$
            § $x^2-14x+49$
            § $y^2+2y+1$
            § $x^2+6xy+9y^2$
            § $9x^2-6x+1$                                                                                                                                    
        }
    \end{vastaus}
\end{tehtava}

\begin{tehtava}
    Muistikaavan mukaisen lausekkeen tunnistaminen on tärkeää. Tunnista edellisessä tehtävässä laskemasi muistikaavat ja esitä lausekkeet tulomuodossa.
    \alakohdat{
            § $1-a^2$ 
            § $x^2+6xy+9y^2$
            § $4x^2+4x+1$
            § $x^2-14x+49$                                                                                                                                                
            § $y^2+2y+1$
            § $x^2+4x+4$
            § $b^2-16$
            § $x^2-2x+1$         
            § $x^2-10x+25$
            § $9x^2-6x+1$
    }
    \begin{vastaus}
        \alakohdat{
            § $(1-a)(1+a)$
            § $(x-3y)^2$
            § $(2x+1)^2$
            § $(x-7)^2$
            § $(y+1)^2$
            § $(x+2)^2$
            § $(b+4)(b-4)$
            § $(x-1)^2$            
            § $(x-5)^2$  
            § $(3x-1)^2$
         }
    \end{vastaus}
\end{tehtava}

\begin{tehtava}
    Muistikaavat vaativat paljon harjoitusta, joten laske vielä nämä.
    \alakohdat{
        § $(2x+1)(2x-1)$
        § $(t-2)^2$
        § $(2x+3)^2$
        § $(2-a)^2$
        § $(5x+2)(5x-2)$
        § $(3-c)(3+c)$
        § $(10x-1)^2$
        § $(2a+b)^2$            
        § $(x+6)^2$
        § $(4x+3)(4x-3)$ 
    }
    \begin{vastaus}
    \alakohdat{
        § $4x^2-1$
        § $t^2-4t+4$
        § $4x^2+12x+9$
        § $a^2-4a+4$
        § $25x^2-4$
        § $9-c^2$                                                                                                                                    
        § $100x^2-20x+1$            
        § $4a^2+4ab+b^2$                                                                                                                                       
        § $x^2+12x+36$
        § $16x^2-9$
    }
    \end{vastaus}
\end{tehtava}


\begin{tehtava}
   	Tunnista myös edellisessä tehtävässä laskemasi muistikaavat ja esitä nämäkin lausekkeet tulomuodossa.
    \alakohdat{
	        § $a^2-4a+4$
            § $100x^2-20x+1$
            § $4a^2+4ab+b^2$                                                                                                                                                      
            § $4x^2-1$
            § $t^2-4t+4$
            § $25x^2-4$
            § $16x^2-9$
            § $9-c^2$                                                                                                                                
            § $x^2+12x+36$                                                                                                                                                        
            § $4x^2+12x+9$
    }
    \begin{vastaus}
        \alakohdat{
            § $(2-a)^2$
            § $(10x-1)^2$
            § $(2a+b)^2$   
            § $(2x+1)(2x-1)$
            § $(t-2)^2$
            § $(5x+2)(5x-2)$
            § $(4x+3)(4x-3)$
            § $(3-c)(3+c)$
            § $(x+6)^2$                                                                                                                                                         
            § $(2x+3)^2$
         }
    \end{vastaus}
\end{tehtava}

\begin{tehtava}
    Esitä tulona, eli käytä muistikaavaa toiseen suuntaan.
    \alakohdat{
        § $4b^2-4b+1$
        § $16t^2-9$
        § $9x^2+12xy+4y^2$
        § $x^2-x+\frac{1}{4}$
        § $x^2-2$
    }
    \begin{vastaus}
        \alakohdat{
            § $(2b-1)^2$
            § $(4t+3)(4t-3)$
            § $(3x+2y)^2$
            § $(x-\frac{1}{2})^2$
            § $(x-\sqrt{2})(x+\sqrt{2})$
        }
    \end{vastaus}
\end{tehtava}

\begin{tehtava}
    Laske ovelasti muistikaavojen avulla:
    \alakohdat{
        § $101^2$
        § $99^2$
        § $49\cdot 51$
        § $35^2-25^2$
        § $170^2-30^2$
    }
    \begin{vastaus}
        \alakohdat{
            § $(100+1)^2=100^2+2\cdot 100 \cdot 1 + 1^2=10\,201$
            § $99^2=(100-1)^2=100^2-2\cdot 100 \cdot 1 + 1^2=9\,801$
            § $49\cdot 51=(50-1)(50+1)=50^2-1^2=2\,500-1=2\,499$
            § $35^2-25^2 = (35+25)(35-25)=60 \cdot 10 = 600$
            § $170^2-30^2 = (170+30) \cdot (170-30)^2 = 200 \cdot 140 = 28\,000$
        }
    \end{vastaus}
\end{tehtava}

\begin{tehtava}
    Sievennä. (Ohje: käytä summakaavoja.)
    \alakohdat{
        § $63^2+37^2$
        § $101^2+99^2$
    }
    \begin{vastaus}
        \alakohdat{
            § $63^2+37^2 = (50+13)^2+(50-13)^2 = 2\cdot 50^2 + 2\cdot 13^2 = 2\cdot 2\,500 +2\cdot 169 = 5\,000 + 338 = 5\,338$
            § $101^2+99^2 = (100+1)^2+(100-1)^2 = 2\cdot 100^2 + 2\cdot 1^2 = 2\cdot 10\,000 + 2\cdot 1 = 20\,000 + 2 = 20\,002$
        }
    \end{vastaus}
\end{tehtava}

\begin{tehtava}
Sievennä lauseke $(x^2+2y^2+2xy)(x^2+2y^2-2xy)$ muistikaavojen avulla.
	\begin{vastaus}
	\begin{align*}
		&(x^2+2y^2+2xy)(x^2+2y^2-2xy) \\
		=&(x^2+2y^2)^2-(2xy)^2 \\
		=&x^4+4x^2y^2+4y^2-4x^2y^2 \\
		=&x^4+4y^4
	\end{align*}		
	(Yhtälöä $(x^2+2y^2+2xy)(x^2+2y^2-2xy)=x^4+4y^2$ kutsutaan \textit{Sophie Germainen} identiteetiksi.)
	\end{vastaus}
\end{tehtava}

\begin{tehtava}
Luvun summa käänteislukunsa kanssa on kaksi. Määritä kyseinen luku.
	\begin{vastaus}
	Jos tuntematonta lukua merkataan $a$:lla, saadaan yhtälöksi $a+\frac{1}{a}=2$. Tästä voidaan päätellä, että $a$ ei voilla ainakaan nolla (määrittelyehto). Kertomalla yhtälö puolittain $a$:lla, saadaan muoto $a^2+1=2a$ ja termejä järjestelemällä $a^2-2a+1=0$. Huomataan vasemman puolen lausekkeen vastaavan binomin neliöä $(a-1)^2$. Ratkaistaan siis potenssiyhtälö $(a-1)^2=0$:
	\begin{align*}
(a-1)^2&=0 &&\sqrt{ } \\
a-1&=0 &&+1 \\
a&=1 &&
	\end{align*}
	Ainut ehdon toteuttava luku on siis $1$.
	\end{vastaus}
\end{tehtava}

\begin{tehtava} 
	Yllättäviä yhteyksiä:
    \alakohdat{
            § Perustele, että $\left(2+\sqrt{3}\right)^{-1}= 2-\sqrt{3}$. 
            § Perustele, että $\left(4+\sqrt{3}\right)^{-1} \neq 4-\sqrt{3}$. 
	        § $\star$ Etsi lisää a-kohdan kaltaisia lukuja. Mikä on niiden
	        yleinen muoto?
    }

    \begin{vastaus}
    \alakohdat{
            § Tutki lukujen $2+\sqrt{3}$ ja $2-\sqrt{3}$ tuloa. 
            § Lukujen $4+\sqrt{3}$ ja $4-\sqrt{3}$ tulo on $13$, joten ne
            eivät ole käänteislukuja.
	        § Yleisesti $\left(a+\sqrt{a^2-1}\right)^{-1}= a-\sqrt{a^2-1}$
    }
    \end{vastaus}
\end{tehtava}

\begin{tehtava}
    $\star$ Etsi kaikki positiiviset kokonaisluvut $x$ ja $y$, joille pätee $9x^2-y^2=17$.
    \begin{vastaus}
    Ainoa ratkaisu on $x = 3$, $y=8$. Opastus: jaa yhtälön vasen puoli tekijöihin muistikaavalla. 
    \end{vastaus}
\end{tehtava}

\begin{tehtava}
    $\star$ Kahden luvun keskiarvo on $7$. Kuinka suuri niiden tulo voi korkeintaan olla? Perustele. (Keksit vastauksen todennäköisesti helpommin kuin sen perustelun.)
    \begin{vastaus}
        $49$. Perustelu muistikaavoilla: Jos lukujen keskiarvo on $7$, ne ovat muotoa $7+a$ ja $7-a$. Lasketaan tulo: $(7+a)(7-a)=7^2-a^2 = 49-a^2 \geq 49$. Siis $49$ on suurin mahdollinen.
    \end{vastaus}
\end{tehtava}

\end{tehtavasivu}