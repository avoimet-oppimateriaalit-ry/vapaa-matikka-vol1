\begin{tehtavasivu}

\subsubsection*{Opi perusteet}

\begin{tehtava}
  Aukeavatko seuraavat paraabelit ylös- vai alaspäin? %hnnnggh!
  \alakohdat{
    § $4x^2 + 100x - 3$
    § $-x^2 + 1\,337$
    § $5x^2 - 7x + 5$
    § $-6(-3x^2 + 5)$
    § $-13x(9 - 17x)$
    § $100(1-x^2)$
  }

  \begin{vastaus}
    \alakohdat{
      § Ylös
      § Alas
      § Ylös
      § Ylös
      § Ylös
      § Alas
    }
  \end{vastaus}
\end{tehtava}

\begin{tehtava}
Funktiot $P(x)$ ja $Q(x)$ ovat toisen asteen polynomeja.\\
\begin{kuvaajapohja}{1}{-2}{3}{-1}{3}
\kuvaaja{x*(2-x)}{$P(x)$}{black}
\kuvaaja{x**2+1}{$Q(x)$}{black}
\end{kuvaajapohja} \\
Päättele kuvaajan perusteella
\alakohdat{
§ mihin suuntaan paraabelit aukeavat
§ funktion $P$ nollakohdat
§ yhtälön $Q(x)=2$ ratkaisu
§ polynomin $Q(x)$ vakiotermi
}

\begin{vastaus}
\alakohdat{
§ $P$ alaspäin, $Q$ ylöspäin.
§ nollakohdat: $x=0$ ja $x=2$
§ $x=-1$ tai $x=1$
§ $1$, sillä kun $x=0$, $Q(x)=1$.
}
\end{vastaus}
\end{tehtava}

\subsubsection*{Hallitse kokonaisuus}

\begin{tehtava}
Kuvassa on funktion $P(x)=x^2$ kuvaaja.\\
\begin{kuvaajapohja}{1.5}{-1.5}{1.5}{-1}{3}
\kuvaaja{x**2}{$P(x)=x^2$}{black}
\end{kuvaajapohja} \\
Hahmottele kuvaajan avulla funktioiden
\alakohdat{
§ $x^2-1$
§ $2-x^2$
§ $\frac{1}{2}x^2$
§ $(x-2)^2$
}
kuvaajat.
\begin{vastaus}
\alakohdat{
§
\begin{kuvaajapohja}{1}{-1.5}{1.5}{-2}{2}
\kuvaaja{x**2-1}{$P(x)=x^2-1$}{black}
\end{kuvaajapohja}
§
\begin{kuvaajapohja}{1}{-1.5}{1.5}{-1}{3}
\kuvaaja{2-x**2}{$P(x)=2-x^2$}{black}
\end{kuvaajapohja}
§
\begin{kuvaajapohja}{1}{-1.5}{1.5}{-1}{3}
\kuvaaja{0.5*x**2}{$P(x)=\frac12x^2$}{black}
\end{kuvaajapohja}
§
\begin{kuvaajapohja}{1}{-0.5}{3.5}{-1}{3}
\kuvaaja{(x-2)**2}{$P(x)=(x-2)^2$}{black}
\end{kuvaajapohja}
}
\end{vastaus}
\end{tehtava}

\end{tehtavasivu}