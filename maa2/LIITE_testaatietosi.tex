\subsection*{Osaatko selittää?}
\begin{multicols}{2}

\begin{tehtava}
Miksi polynomin vakiotermin (tai vakiopolynomin) asteluku on nolla?
	\begin{vastaus}
	Koska mielivaltaisen vakion $a$ voidaan aina tulkita olevan kerroin muuttujan potenssille $x^0$.
	\end{vastaus}
\end{tehtava}

\begin{tehtava}
Miksi puhutaan vain termien summasta, vaikka polynomeissa voi esiintyä miinusmerkkejä?
	\begin{vastaus}
	Vähennyslasku määritellään $a-b=a+(-b)$, eli vastaluvun avulla kaikki vähenyslaskut voidaan tulkita yhteenlaskuina.
	\end{vastaus}
\end{tehtava}

\begin{tehtava}
Miten usean muuttujan polynomin asteluku lasketaan?
	\begin{vastaus}
	Niin kuin yhdenkin muuttujan polynomin tapauksessa, korkein termin aste eli muuttujan potenssin eksponentti määrää koko polynomin asteen. Jos termissä on kerrottuna eri muuttujia keskenään, termin asteluku on näiden muuttujien potenssien eksponenttien summa.
	\end{vastaus}
\end{tehtava}

\begin{tehtava}
Mitä tarkoitetaan tulon nollasäännöllä?
	\begin{vastaus}
\textit{Kahta} asiaa: toisaalta mikä tahansa tulo, jossa tekijänä on nolla, on arvoltaan nolla. Toisaalta, jos tulo on nolla, niin tiedetään, että varmasti ainakin yksi tulon tekijöistä on nolla.
	\end{vastaus}
\end{tehtava}

\begin{tehtava}
Kun polynomi jaetaan tekijöihin (jotka eivät ole vakioita), mitä voit varmasti sanoa tekijäpolynomien asteista verrattuna alkuperäiseen polynomiin?
	\begin{vastaus}
Tekijät ovat alhaisempaa astetta kuin alkuperäinen polynomi, kunhan tekijänä ei ole vakiopolynomi.
	\end{vastaus}
\end{tehtava}

\begin{tehtava}
Miten selvität polynomin $(x^3-2x)^{15}$ asteen purkamatta sulkuja?
	\begin{vastaus}
Potenssiinkorotettava lausekkeen suurin aste määrää koko polynomin termin. Koska sulkeita avattaessa $x^3$ lopulta korotettaisiin potenssin $15$, sinne tulisi korkeimmaksi termiksi $x^{3\cdot 15}=x^{45}$. Asteen saa siis selville kertomalla korkeimman asteen termin eksponentti potenssilla, johon sulkeisiin laitettu polynomi korotetaan.
	\end{vastaus}
\end{tehtava}

\begin{tehtava}
Miksi epäyhtälön ''suunta muuttuu'', kun se kerrotaan puolittain negatiivisella luvulla?
	\begin{vastaus}
Epäyhtälö (muu kuin $\neq$) kuvaa järjestystä, ja kahden toisistaan poikkeavan reaaliluvun keskinäinen järjestys muuttuu aina, jos molemmat muutetaan vastaluvuikseen (eli on kerrottu negatiivisella luvulla) -- aiemmin pienemmästä luvusta on tullutkin nyt suurempi. Totuusarvon säilyttämiseksi tulee sen vuoksi kääntää epäyhtälömerkin suunta.
	\end{vastaus}
\end{tehtava}

\end{multicols}

\subsection*{Monivalinta}

\begin{multicols}{2}
Valitse kussakin monivalintatehtävässä yksi paras vastaus.

\begin{tehtava}
Mitkä ovat polynomin $x^4-x^2+\sqrt{2}x-1\,337$ termien kertoimet neljännestä asteesta pienempään luetellen?
	\alakohdat{
	§ $1, -1, \sqrt{2}$ ja $-1\,337$
	§ $1, 0, -1, \sqrt{2}$ ja $-1\,337$
	§ $1, 0, -1, \sqrt{2}$ ja $1\,337$
	§ $1, 1, \sqrt{2}$ ja $1\,337$
	}
	\begin{vastaus}
	b) $1, 0, -1, \sqrt{2}$ ja $-1\,337$
	\end{vastaus}
\end{tehtava}

\begin{tehtava}
Tulosta $ab$ tiedetään, että $a = 0$. Mikä seuraavista väitteistä pitää välttämättä paikkansa?
		\alakohdatm{
		§ $b \geq a$
		§ $b = 0$
		§ $|b| \geq a$
		§ $ab = 0$
		}
		\begin{vastaus}
d) $ab = 0$
	\end{vastaus}
\end{tehtava}

\begin{tehtava}
Mikä seuraavista epäyhtälöistä pitää paikkansa?
		\alakohdat{
		§ $5,2\cdot 10 < 52$
		§ $52 > 520:100$
		§ $52>5\,200$
		§ $520<52$
		§ $52\cdot 10<52$
		}
		\begin{vastaus}
b) $52 > 520:100$
	\end{vastaus}
\end{tehtava}

\begin{tehtava}
Mitkä ovat funktion $f(x)=(x+3)(x-50)$ nollakohdat?
		\alakohdatm{
		§ $3$ ja $-50$
		§ $-3$ ja $-50$
		§ $-3$ ja $50$
		§ $3$ ja $50$
		}
		\begin{vastaus}
c) $-3$ ja $50$ %FIXME: opeta, sanotaanko "nollakohta on -3" vai "nollakohta on x=-3"
	\end{vastaus}
\end{tehtava}

\begin{tehtava}
Monettako astetta on kahden muuttujan polynomi $x^5y-x^4y^3-715\,517y$?
		\alakohdatm{
		§ $1.$
		§ $3.$
		§ $4.$
		§ $5.$
		§ $7.$
		}
		\begin{vastaus}
e) $7.$
	\end{vastaus}
\end{tehtava}

\begin{tehtava}
Mikä on polynomin $P(x)=-x^3-x^2-x-1$ arvo kohdassa $x=-1$?
		\alakohdatm{
		§ $-4$
		§ $-2$
		§ $0$
		§ $1$
		§ $2$
		}
		\begin{vastaus}
c) $0$
	\end{vastaus}
\end{tehtava}

\begin{tehtava}
Mikä polynomi saadaan, kun avataan sulut lausekkeesta $(2x-1)^2$ ja sievennetään?
		\alakohdatm{
		§ $2x^2+1$
		§ $4x^2+1$
		§ $2x^2-1$
		§ $4x^2-1$
		§ $4x^2-4x+1$
		§ $4x^2+4x+1$
		§ $4x^2-4x-1$
		}
		\begin{vastaus}
e) $4x^2-4x+1$
	\end{vastaus}
\end{tehtava}

\begin{tehtava}
Määritellään kahden funktion arvot kaavoilla $f(x) = x^2 + 1$ ja $g(x) = x^2 - x$. Mikä väitteistä ei pidä paikkaansa?
	\alakohdat{
	§ $f(x) + g(x) = 2x^2 - x + 1$
	§ $f(x) \cdot g(x) = x^4-x^3+x^2-x$
	§ $f(0) = g(0)$
	§ $f(-1) = g(-1)$
	}
	\begin{vastaus}
	c) $f(0) = g(0)$
	\end{vastaus}
\end{tehtava}
%monivalinta,jossa "mikä väitteistä on epätosi"?
\begin{tehtava}
Eräs reaalimuuttujan polynomi saa kahdella erillisellä, suljetulla reaalilukuvälillä negatiivia arvoja ja kaikkialla muualla epänegatiivisia arvoja. Kuinka monta nollakohtaa polynomilla on?
	\alakohdatm{
	§ nolla
	§ yksi
	§ kaksi
	§ kolme
	§ neljä
	}
	\begin{vastaus}
	e) neljä
	\end{vastaus}
\end{tehtava}

% Pysyykö epäyhtälö yhtäpitävänä, jos
%a) lisäämällä kummallekkin puolelle sama lauseke
%b) kertomalla kummatkin puolet samalla luvulla
%c) korotetaan kummatkin puolet potenssiin n
%d) kerrotaan negatiivisellä luvulla ja käännetään epäyhtälön suunta.

\end{multicols}
\subsection*{Tosi vai epätosi?}

Pitävätkö väitteet paikkansa?

\begin{multicols}{2}

\begin{tehtava}
	\alakohdat{
	§ Potenssifunktiot ovat polynomifunktioita.
%	§ Jokainen polynomi on jaollinen nollakohdillaan.
	§ Jos epäyhtälön molemmat puolet korotetaan kolmanteen potenssiin, epäyhtälö säilyy yhtäpitävänä.
	§ $5\leq x$ esittää, että $x$ kuuluu puoliavoimelle välille.
	}
	\begin{vastaus}
	\alakohdatm{
	§ Tosi
%	§ Epätosi
	§ Tosi
	§ Tosi
	}
	\end{vastaus}
\end{tehtava}

\begin{tehtava}
	\alakohdat{
	§ Jos trinomin kertoo trinomilla, sulkeet avaamalla saadaan kuusi termiä (ennen mahdollista sieventämistä eli samanasteisten termien yhdistämistä).
	§ Jos kolmannen asteen polynomi korotetaan kokonaisuudessaan kolmanteen potenssiin, saadaan tulomuotoinen yhdeksännen asteen polynomi.
	§ Kaikki lukua $100$ suuremmat reaaliluvut muodostavat avoimen välin.
	}
	\begin{vastaus}
	\alakohdatm{
	§ Epätosi
	§ Tosi
	§ Tosi
	}
	\end{vastaus}
\end{tehtava}

\end{multicols}
\newpage