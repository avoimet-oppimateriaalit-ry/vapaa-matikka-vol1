Epäyhtälöllä tarkoitetaan väitettä, jossa esitetään kahden lausekkeen arvon välinen suuruusjärjesty lukuunottamatta tilannetta, että ne ovat yhtä suuret. Suuruusjärjestyksien esittämiseen käytetään muun muassa seuraavia merkintöjä:

\begin{center}
\begin{tabular}{l|l}
\emph{Merkintä} & \emph{Merkitys} \\
\hline
$a<b$ &  $a$ on pienempi kuin $b$ \\
$a>b$ & $a$ on suurempi kuin $b$ \\
$a \leq b$ & $a$ on pienempi tai yhtäsuuri kuin $b$ \\
$a \geq b$ & $a$ on suurempi tai yhtäsuuri kuin $b$ \\
$a \neq b$ & $a$ ei ole yhtä suuri kuin $b$ \\
\end{tabular}
\end{center}

Kahden luvun välinen epäyhtälöriippuvuus eli -relaatio voidaan esittää myös kielteisesti: $a \nless b $. Epäyhtälöiden relaatio-ominaisuudet poikkeavat merkittävästi yhtäsuuruudesta. Yhtäsuuruus on sekä refleksiivinen, symmetrinen että transitiivinen (ks. MAA1), mutta $<$, $>$ ovat vain transitiivisia, ja $\leq$ ja $\geq$ ovat transitiivisia ja symmetrisiä.

\begin{esimerkki}
Jos $a>b$ ja $b>c$, niin aivan varmasti $a>c$. (transitiivisuus)
Koska $a \leq a$ kaikilla luvuilla, niin $\leq$ on symmetrinen relaatio.
\end{esimerkki}

Sama epäyhtälö voidaan ilmaista usealla tavalla.

\begin{esimerkki}
\alakohdat{
§ $a < b$ tarkoittaa samaa kuin $b > a$
§ $a \leq b$ tarkoittaa samaa kuin $b \geq a$.
§ Epäyhtälö $a < b$ pätee täsmälleen silloin, kun epäyhtälö $a \geq b$ ei päde.
}
\end{esimerkki}

Epäyhtälön totuusarvo voi riippua epäyhtälön puolilla esiintyvien muuttujien arvoista. Tämän perusteella epäyhtälöt voidaan jakaa kolmeen tyyppiin:

\luettelolaatikko{Epäyhtälöiden tyypit}{
§ \emph{Aina tosi} -- pätee kaikilla muuttujien arvoilla. Esimerkiksi epäyhtälöt $5 < 6$ tai $x + 1 \leq x + 3$ pätevät riippumatta muuttujan $x$ arvosta.
§ \emph{Ehdollisesti tosi} -- pätee vain joillain muuttujien arvoilla. Esimerkiksi epäyhtälö $x < 5$ pätee, kun $x = 4$, mutta ei päde, kun $x = 7$.
§ \emph{Aina epätosi} -- ei päde millään muuttujien arvoilla. Esimerkiksi epäyhtälöt $3 < 1$ ja $x < x$ eivät päde koskaan.
}

\begin{esimerkki}
Kaverisi katsoo teini-ikäisiä mutanttininjakilpikonnia (ja tietenkin fanittaa Michelangeloa), ja sinua kiinnostaa, kuinka monta jaksoa sarjaa on jo julkaistu. Kaverisi vastaa tiedusteluusi epämääräisesti, että viimeisin julkaistu jakso oli toisen tuotantokauden $19$. jakso. Mitä tästä voidaan päätellä, eli kuinka monta jaksoa ($n$) sarjaa on julkaistu enintään ja vähintään? Oletetaan, että jokaisessa tuotantokaudessa on yhtä monta jaksoa. Esitä vastaus epäyhtälönä.
	\begin{esimratk}
	Jos toisesta tuotantokaudesta on julkaistu jo $19$ jaksoa ja oletamme, että tuotantokausissa on yhtä monta jaksoa, niin silloin jaksoja on kokonaisuudessaan julkaistu \textit{vähintään} $2\cdot 19 = 38$. Emme tiedä, kuinka paljon ensimmäisessä tuotantokaudessa on ollut jaksoja, joten toisen tuotantokauden jakso $19$ saattaa olla niin alku- kuin loppupäästäkin. Ylärajaa näillä tiedoilla ei siis voi antaa, eli voimme vain sanoa $n\geq38$. (Muita lisätietoja hankkimalla tai esimerkiksi arvioimalla, että jaksoja ei tuotantokaudessa voi olla enempää kuin vuodessa viikkoja, jonkinlainen yläraja toki saadaan.)
	\end{esimratk}
\end{esimerkki}

\subsection{Epäyhtälöiden muokkaaminen}

Kuten yhtälöiden ratkaisemisessa, epäyhtälön ratkaisemisessa selvitetään ne muuttujien arvot, joilla epäyhtälö on tosi. Kuten yhtälöitä, myös epäyhtälöitä voidaan ratkaista muokkaamalla niitä sellaisilla operaatioilla, joilla muokattu epäyhtälö on yhtäpitävä alkuperäisen kanssa.

Kahden luvun kasvattaminen saman verran siirtää lukuja lukusuoralla, mutta säilyttää niiden keskinäisen järjestyksen:

\begin{kuva}
lukusuora.pohja(-1, 10, 11.5, n = 2)
lukusuora.kohta(0, "$0$", 0)

with vari("red"):
	lukusuora.nuoli(2, 2+4, 1, 2)
	lukusuora.nuoli(3, 3+4, 1, 2)

lukusuora.piste(2, "$2$", 1)
lukusuora.piste(3, "$3$", 1)

lukusuora.piste(2+4, "$2\!+\!4$", 2)
lukusuora.piste(3+4, "$3\!+\!4$", 2)
\end{kuva}

Tämän perusteella epäyhtälö, joka saadaan lisäämällä epäyhtälön molemmille puolille sama lauseke, on yhtäpitävä alkuperäisen epäyhtälön kanssa.

\begin{esimerkki}
Luvun lisääminen epäyhtälöön.
  \begin{align*}
     3 - x &< 5 - x && \ppalkki +x\\
     3 -x +x &< 5 -x +x \\
     3 &< 5 && \textrm{tosi}
  \end{align*}
\end{esimerkki}

Myös lukujen kertominen samalla positiivisella kertoimella säilyttää niiden keskinäisen järjestyksen. Jos kerroin on pienempi kuin yksi, luvut lähenevät toisiaan:

\begin{kuva}
lukusuora.pohja(-1, 10, 11.5, n = 2)
lukusuora.kohta(0, "$0$", 0)

with vari("red"):
	lukusuora.nuoli(2, 1, 1, 2)
	lukusuora.nuoli(4, 2, 1, 2)

lukusuora.piste(2, "$2$", 1)
lukusuora.piste(4, "$4$", 1)

lukusuora.piste(1, r"$2 \cdot \frac{1}{2}$", 2)
lukusuora.piste(2, r"$4 \cdot \frac{1}{2}$", 2)
\end{kuva}

Jos kerroin on suurempi kuin yksi, luvut etääntyvät toisistaan:

\begin{kuva}
lukusuora.pohja(-1, 10, 11.5, n = 2)
lukusuora.kohta(0, "$0$", 0)

with vari("red"):
	lukusuora.nuoli(2, 4, 1, 2)
	lukusuora.nuoli(2*2, 4*2, 1, 2)

lukusuora.piste(2, "$2$", 1)
lukusuora.piste(4, "$4$", 1)

lukusuora.piste(2*2, r"$2 \cdot 2$", 2)
lukusuora.piste(4*2, r"$4 \cdot 2$", 2)
\end{kuva}

Luvulla jakaminen on sama asia kuin jakajan käänteisluvulla kertominen, joten positiivisella luvulla jakaminen säilyttää järjestyksen kertolaskun tavoin. Näin ollen alkuperäisen epäyhtälön kanssa yhtäpitävä epäyhtälö saadaan kertomalla tai jakamalla molemmat puolet positiivisella luvulla.

\begin{esimerkki}
Epäyhtälön kertominen lukua yksi pienemmällä positiivisella luvulla.
\begin{align*}
     2 &< 4 && \ppalkki \cdot \frac{1}{2} \\
   2\cdot\frac{1}{2} &< 4\cdot\frac{1}{2}  \\
     1 &< 2 && \textrm{tosi}
\end{align*}
\end{esimerkki}

\begin{esimerkki}
Epäyhtälön kertominen lukua yksi suuremmalla luvulla.
\begin{align*}
     2 &< 4 && \ppalkki \cdot 2 \\
   2\cdot 2 &< 4\cdot 2  \\
     4 &< 8 && \textrm{tosi}
\end{align*}
\end{esimerkki}

Sen sijaan negatiivisella luvulla kertominen ei säilytä suuruusjärjestystä.

\begin{esimerkki}
Kun luvut $2$ ja $5$ kerrotaan luvulla $-1$, niiden suuruusjärjestys kääntyy:

\begin{kuva}
lukusuora.pohja(-6, 6, 11.5, n = 2)
lukusuora.kohta(0, "$0$", 0)

with vari("red"):
	lukusuora.nuoli(2, -2, 1, 2)
	lukusuora.nuoli(5, -5, 1, 2)

lukusuora.piste(2, "$2$", 1)
lukusuora.piste(5, "$5$", 1)

lukusuora.piste(-2, r"$2 \cdot (-1)$", 2)
lukusuora.piste(-5, r"$5 \cdot (-1)$", 2)
\end{kuva}
\end{esimerkki}

Jos epäyhtälöä kerrotaan tai jaetaan negatiivisella luvulla, epäyhtälömerkin suunta täytyy kääntää, jotta saataisiin yhtäpitävä epäyhtälö.
%: esimerkissä $2 < 5$ muuttuu muotoon $-2 > -5$.

\begin{esimerkki}
Epäyhtälön kertominen negatiivisella luvulla.
\begin{align*}
     2 &< 5 && \ppalkki \cdot (-1) \\
   2\cdot (-1) &> 4\cdot (-1)  \\
     -2 &> -5 && \textrm{tosi}
\end{align*}
\end{esimerkki}

\luettelolaatikko{Epäyhtälön muokkaaminen}{
§ Yhtäpitävä epäyhtälö saadaan
§§ lisäämällä molemmille puolille sama lauseke,
§§ kertomalla tai jakamalla molemmat puolet samalla positiivisella luvulla tai
§§ kertomalla tai jakamalla molemmat puolet samalla negatiivisella luvulla ja kääntämällä epäyhtälömerkin suunta.
}
%tarkennus/laajennus! ^

Kuten yhtälöiden tapauksessa, epäyhtälön kertominen puolittain nollalla ei tuota yhtäpitävää epäyhtälöä.

\begin{esimerkki}
$a \leq b$ tulee $0 \leq 0$, joka on aina tosi, ja epäyhtälöstä $a < b$ tulee $0 < 0$, joka on aina epätosi.
\end{esimerkki}

\begin{esimerkki}
Muokataan epäyhtälöä $-2x+4<6$ käyttämällä esitettyjä operaatioita.
\begin{align*}
-2x+4&<6 && \ppalkki -4 \\
-2x&<2 && \ppalkki :(-2) \\
x&>-1
\end{align*}
Tehdyt operaatiot tuottavat yhtäpitäviä epäyhtälöitä, joten epäyhtälö $-2x+4<6$ on yhtäpitävä epäyhtälön $x>-1$ kanssa. Voidaan päätellä, että lukua $-1$ suuremmat luvut ovat täsmälleen epäyhtälön ratkaisut.
\end{esimerkki}

\subsection{Reaalilukuvälit}

Ratkaistaessa yhtälöitä ratkaisuksi saadaan yleensä pieni joukko lukuja. Epäyhtälöiden tapauksessa on tyypillistä, että ratkaisu on \termi{väli}{väli}, eli kaikki kahden luvun väliset luvut. Reaalilukuvälejä merkitään usein laittamalla välin ala- ja ylärajat hakasulkujen sisään: merkintä $[a, b]$ tarkoittaa niiden kaikkien lukujen (esim. $x$) joukkoa, jotka toteuttavat kaksoisepäyhtälön $a \leq x \leq b$. Mikäli ala- tai yläraja ei kuulu väliin, vastaava hakasulku käännetään.

\begin{esimerkki}
$[2,5[$ tarkoittaa lukuja $x$, joille pätee $2 \leq x < 5$. %lisää!
\end{esimerkki}

Huomaa, että pelkkä merkintä $[a,b]$ ja sen variaatiot tarkoittavat vain joukkoa reaalilukuja. Jos tarkoitus on esittää väite, että jokin luku (esim. $y$) kuuluu tuolle välille, se kirjoitetaan $y \in [a,b]$, missä $\in$ luetaan ''kuuluu joukkoon'' tai ''kuuluu välille''.

Väliä kutsutaan \termi{suljettu väli}{suljetuksi väliksi}, mikäli ala- ja yläraja kuuluvat väliin, ja \termi{avoin väli}{avoimeksi väliksi}, mikäli ala- ja yläraja eivät kuulu väliin. Jos vain toinen rajoista kuuluu väliin, väli on \termi{puoliavoin väli}{puoliavoin}.

Väli voidaan piirtää lukusuoralle kahden luvun välisenä janana. Päätepisteet merkitään täytetyllä ympyrällä, mikäli luku kuuluu väliin, ja muuten tyhjällä ympyrällä.

\begin{esimerkki}
Väli $[a, b[$ piirretään seuraavasti:

\begin{kuva}
lukusuora.pohja(0, 10, 8)
lukusuora.vali(2, 8, True, False, "$a$", "$b$")
\end{kuva}
\end{esimerkki}

Jos halutaan, että väli ei ole alhaalta tai ylhäältä rajoitettu, merkitään välimerkintään rajaksi $-\infty$ tai $\infty$. Koska ääretön ei ole reaaliluku eikä näin ollen kuulu väliin, on sitä vastaava hakasulku kirjoitettava ulospäin käännettynä.

\begin{esimerkki}
Väli $]{-\infty}, a[$ on avoin. 
\end{esimerkki}

Seuraavaan taulukkoon on koottu reaalilukuvälien olennainen käsitteistö ja merkinnät.

\begin{luoKuva}{vali1}
lukusuora.pohja(-5, 7, 3, varaa_tila = False)
lukusuora.kohta(0, "$0$")
lukusuora.vali(-3, 5, False, False, "$-3$", "$5$")
\end{luoKuva}
\begin{luoKuva}{vali2}
lukusuora.pohja(-5, 7, 3, varaa_tila = False)
lukusuora.kohta(0, "$0$")
lukusuora.vali(-3, 5, False, True, "$-3$", "$5$")
\end{luoKuva}
\begin{luoKuva}{vali3}
lukusuora.pohja(-5, 7, 3, varaa_tila = False)
lukusuora.kohta(0, "$0$")
lukusuora.vali(-3, 5, True, False, "$-3$", "$5$")
\end{luoKuva}
\begin{luoKuva}{vali4}
lukusuora.pohja(-5, 7, 3, varaa_tila = False)
lukusuora.kohta(0, "$0$")
lukusuora.vali(-3, 5, True, True, "$-3$", "$5$")
\end{luoKuva}
\begin{luoKuva}{vali5}
lukusuora.pohja(-5, 7, 3, varaa_tila = False)
lukusuora.kohta(0, "$0$")
lukusuora.vali(-3, None, True, False, "$-3$", "$5$")
\end{luoKuva}
\begin{luoKuva}{vali6}
lukusuora.pohja(-5, 7, 3, varaa_tila = False)
lukusuora.kohta(0, "$0$")
lukusuora.vali(-3, None, False, False, "$-3$", "$5$")
\end{luoKuva}
\begin{luoKuva}{vali7}
lukusuora.pohja(-5, 7, 3, varaa_tila = False)
lukusuora.kohta(0, "$0$")
lukusuora.vali(None, 5, False, True, "$-3$", "$5$")
\end{luoKuva}
\begin{luoKuva}{vali8}
lukusuora.pohja(-5, 7, 3, varaa_tila = False)
lukusuora.kohta(0, "$0$")
lukusuora.vali(None, 5, False, False, "$-3$", "$5$")
\end{luoKuva}

\begin{tabular}{|p{2.0cm}|p{2.0cm}|c|c|}
\hline
Epäyhtälö\-merkintä & Joukko-opillinen merkintä & Esitys lukusuoralla & Välin nimitys \\
\hline
 $-3<x<5$ & $x \in {]-3, 5[}$ & \naytaKuva{vali1} & Avoin väli  \\
\hline
 $-3<x \leq 5$ & $x \in {]-3, 5]}$ & \naytaKuva{vali2} & Puoliavoin väli  \\
\hline
 $-3\leq x < 5$ & $x \in {[-3, 5[}$ & \naytaKuva{vali3} & Puoliavoin väli  \\
\hline
$-3\leq x \leq 5$ & $x \in {[-3, 5]}$ & \naytaKuva{vali4} & Suljettu väli \\
\hline
$-3\leq x$ & $x \in {[-3, \infty[}$ & \naytaKuva{vali5} & Puoliavoin väli  \\
\hline
 $-3<x$ & ${x \in {]-3, \infty[}}$ & \naytaKuva{vali6} & Avoin väli \\
\hline
$x \leq 5$ & $x \in {]{-\infty}, 5]}$ & \naytaKuva{vali7} & Puoliavoin väli  \\
\hline
$x < 5$ & $x \in {]{-\infty}, 5[}$ & \naytaKuva{vali8} & Avoin väli  \\
\hline
\end{tabular}

\begin{esimerkki}
	\alakohdat{
	§ Epäyhtälö $2<x<10$ vaatii, että $x$ saa arvoja kahden ja kymmenen väliltä, mutta se ei koskaan saa täsmälleen näitä reuna-arvoja. Kyseessä on avoin väli kahdesta kymmeneen, $]2,10[$. Annetulle epäyhtälölle yhtäpitävä ilmaisu on $x \in ]2,10[$.
	§ Epäyhtälö $0\leq y \leq 2$ rajaa muuttujan $y$ välille suljetulle välille $[0,2]$. Väli on suljettu, koska $y$ voi myös saada täsmälleen arvot $0$ ja $2$.
	§ Joskus kirjallisuudessa näkee äärettömyyssymbolin käyttöä myös kaksoisepäyhtälöissä, esimerkiksi $3<x<\infty $, mutta ilmaistaan yleisemmin muodossa $x \in ]3,\infty[$. Kyseessä on avoin väli.
	§ Epäyhtälöt $-100<k\leq 0$ ja $u\leq 90$ ovat puoliavoimia välejä, koska ne rajaavat muuttujan yhtäsuuruuden avulla vain toiselta puolelta.
	§ Kaksoisepäyhtälö $\frac{1}{5}\geq x>-\sqrt{3}$ tarkoittaa samaa kuin kaksoisepäyhtälö $-\sqrt{3}<x\leq \frac{1}{5}$. Kaksoisepäyhtälö vaatii, että muuttujalle $x$ pätee erikseen sekä epäyhtälö $-\sqrt{3}<x$ että $x\leq \frac{1}{5}$.
	}
\end{esimerkki}

\begin{esimerkki}
Merkitse väli
	\alakohdat{
	§ $[-2,4]$
	§ $[4,5[$
	§ $]6,\infty[$
	}
epäyhtälömerkintänä. Onko väli suljettu, avoin vai puoliavoin?
	\begin{esimratk}
\alakohdat{
§ Kun $x\in [-2,4]$, pätee $-2\leq x \leq 4$. Väli on suljettu, koska päätepisteet kuluvat väliin.
§ Kun $x\in [4,5[$, pätee $4 \leq x < 5$. Väli on puoliavoin.
§ Kun $x\in ]6,\infty[$, pätee $6<x$. Väli on avoin, koska kumpikaan päätepiste ei kuulu väliin.
}

Huomaa, että ei ole väliä, millä symbolilla mielivaltaista välille kuuluuvaa lukua merkkaa -- tässä niin tehtiin $x$:llä.
	\end{esimratk}
\end{esimerkki}

Joskus -- erityisesti englanninkielisissä matematiikan teksteissä -- käytetään avoimen välin merkintänä käännetyn hakasulkeen sijasta tavallista kaarisuljetta. Merkintää ei suositella käytettävän, koska samaa merkintää käytetään kahdessa ulottuvuudessa siajitsevan pisteen koordinaattien ilmaisuun.

\begin{esimerkki}
	\alakohdat{
	§ Avoin väli $]1,9[$ kirjoitetaan joskus $(1,9)$.
	§ Puoliavoin väli $]3,4]$ kirjoitetaan joskus $(3,4]$.
	}
\end{esimerkki}

Myös kaksoisepäyhtälöitä voi muokata laskutoimituksilla.

\begin{esimerkki}
\begin{align*}
& a+1<b+1<c && ||-1 \\
&a<b<c-1 && ||\cdot(-1) \\
&-a>-b>1-c &&
\end{align*}
\end{esimerkki}