\begin{tehtavasivu}

\subsubsection*{Opi perusteet}

\begin{tehtava}
	Laske diskriminanttien arvot. Kuinka monta ratkaisua yhtälöillä on?
	\alakohdat{
		§ $3x^2+4x+1=0$
		§ $x^2+2x+5=0$
		§ $3x^2-6x+3=0$
		§ $x^2-7x-40=0$
	}
	\begin{vastaus}
		\alakohdat{
			§ $D=4$, eli kaksi ratkaisua.
			§ $D=-1$, eli ei ratkaisuja.
			§ $D=0$, eli $1$ ratkaisu
			§ $D=209$, kaksi ratkaisua
		}
	\end{vastaus}
\end{tehtava}


\begin{tehtava}
	Kuinka monta ratkaisua yhtälöillä on?
	\alakohdat{
		§ $12x^2+12x-4=0$
		§ $12x^2+12x+4=0$
	}
	\begin{vastaus}
		\alakohdat{
			§ Kaksi. $D=12^2-4 \cdot 12 \cdot (-4) = 336 >0$
			§ Ei yhtään. $D=12^2-4 \cdot 12 \cdot 4 = -48 <0$
		}
	\end{vastaus}
\end{tehtava}

\begin{tehtava}
	Tulkitse polynomifunktion lauseketta: Onko kyseessä ylös- vai alaspäin aukeava paraabeli? Kuinka monta nollakohtaa funktiolla on?
	\alakohdat{
		§ $P(x)=-3x^2+9x-5$
		§ $Q(y)=5y^2-2y+1$
		§ $R(z)=z^2-7z-40$
		§ $S(w)=3w^2-6w+3$
	}
	\begin{vastaus}
	Nollakohtien määrä voidaan päätellä diskriminantin arvosta.
		\alakohdat{
			§ alaspäin, 2 nollakohtaa
			§ ylöspäin, ei yhtään nollakohtaa
			§ ylöspäin,2 nollakohtaa
			§ ylöspäin, 1 nollakohta
		}
	\end{vastaus}
\end{tehtava}

\begin{tehtava}
	Millä luvuilla $c$ yhtälöllä $x^2+5x+c = 0$ ei ole ratkaisua?
	\begin{vastaus}
		 $c> \frac{25}{4} =6,25$. (Halutaan $D < 0$, eli $5^2-4\cdot 1 \cdot c <0$.)
	\end{vastaus}
\end{tehtava}

\subsubsection*{Hallitse kokonaisuus}

\begin{tehtava}
	Kuinka monta ratkaisua yhtälöillä on?
	\alakohdat{
		§ $10x^2-8x-35=14x-10$
		§ $-6x^2+15x-59=5x+17$
%		§ $7x^2-6x+2=10$
		§ $3x^2+7x=2x^2+x-9$
	}
	\begin{vastaus}
		\alakohdat{
			§ Kaksi. $D=(-22)^2-4 \cdot 10 \cdot (-25) = 1484 >0$
			§ Nolla. $D=10^2-4\cdot (-6) \cdot (-76) = -1724 <0$
%			§ Kaksi. $D=(-6)^2-4\cdot 7\cdot (-8) 260 > 0$
			§ Yksi. $D=6^2-4\cdot 1 \cdot 9 = 0$
		}
	\end{vastaus}
\end{tehtava}

\begin{tehtava}
	Millä vakion $a$ arvoilla yhtälöllä $(2a-1)x^2+(a+1)x+3=0$ on täsmälleen yksi juuri?
	\begin{vastaus}
		Sopivat $a$:n arvot ovat $\frac{1}{2}$, $11+6\sqrt{3}$ ja $11-6\sqrt{3}$.
	\end{vastaus}
\end{tehtava}

\begin{tehtava}
Osoita, että polynomi $-2x^2+6x-6$ saa vain negatiivisia arvoja.
	\begin{vastaus}
	Tutkitaan nollakohtia diskriminantin avulla: $D=6^2-4\cdot (-2)\cdot(-6)=36-48=-12<0$, joten polynomilla ei ole nollakohtia. Se ei siis voi vaihtaa merkkiä missään. Korkeimman asteen termin kerroin on negatiivinen, joten polynomin kuvaaja on alaspäin avautuva paraabeli. Koska nollakohta ei diskriminantin perusteella ole, on paraabelin huippu $x$-akselin alapuolella, ja siten kaikki polynomin arvot negatiivisia.
	\end{vastaus}
\end{tehtava}

\begin{tehtava}
	Mitä voit sanoa ratkaisujen lukumäärästä vaillinaisten yhtälöiden $ax^2+c=0$ ja $ax^2+bx=0$ tapauksessa? Oletetaan, että $a$,$b$,$c \neq 0$.
	\begin{vastaus}
		\begin{description}
			\item[$ax^2+c=0$] Joko kaksi ratkaisua tai ei yhtään ratkaisua. ($D \neq 0$)
			\item[$ax^2+bx=0$] Aina kaksi ratkaisua. ($D > 0$)
		\end{description}
	\end{vastaus}
\end{tehtava}

\subsubsection*{Lisää tehtäviä}

\begin{tehtava}
	Kuinka monta ratkaisua yhtälöillä on?
	\alakohdat{
		§ $9x^2+12x-4=0$
		§ $5x^2+4x-10=0$
		§ $3x^2-12x+12=0$
		§ $5x^2+10x-30=0$
	}
	\begin{vastaus}
		\alakohdat{
			§ Kaksi. $D=12^2-4 \cdot 9 \cdot (-4) = 288 >0$
			§ Kaksi. $D=4^2-4\cdot 5 \cdot (-10) = 216 >0$
			§ Yksi. $D=(-12)^2-4\cdot 3\cdot 12 =0$
			§ Kaksi. $D=10^2-4\cdot 5 \cdot (-30) = 700 >0$
		}
	\end{vastaus}
\end{tehtava}

\begin{tehtava}
	Millä vakion $k$ arvoilla yhtälöllä $-x^2-x-k = 0$ on ratkaisuja?
	\begin{vastaus}
		Pitää olla $D=(-1)^2-4 \cdot (-1) \cdot (-k) \geq 0$. Siis $k \leq \frac{1}{4}$.
	\end{vastaus}
\end{tehtava}

\begin{tehtava}
	Millä vakion $a$ arvoilla yhtälöllä $ax^2+x=ax-5$ on täsmälleen yksi ratkaisu?
	\begin{vastaus}
		$a =11 \pm 2\sqrt{30}$.
	\end{vastaus}
\end{tehtava}

\begin{tehtava}
	Kuinka monta ratkaisua yhtälöillä on vakion $a$ eri arvoilla?
	\alakohdat{
		§ $x^2+6x+a+1=0$
		§ $ax^2+4x-1=0$
	}
	\begin{vastaus}
		\alakohdat{
			§ Kaksi, kun $a<8$, yksi, kun $a=8$, ja nolla, kun $a>8$.
			§ Kaksi, kun $-4<a$ ja $a\neq 0$; yksi, kun $a=-4$ tai $a=0$, ja nolla, kun $a<-4$.
		}
	\end{vastaus}
\end{tehtava}

\begin{tehtava}
	$\star$ Osoita, että diskriminantti on $0$, jos ja vain jos yhtälö voidaan esittää muodossa \\ $(c_1 x+ c_2)^2=0$, missä $c_1$ ja $c_2$ ovat reaalilukuja.
	\begin{vastaus}
		Suunta "$\Rightarrow$": $(c_1 x+ c_2)^2=0 \Leftrightarrow c_1^2 x^2 + 2c_1 c_2 x+ c_2^2 =0 \Rightarrow
		D=(2 c_1 c_2)^2-4 c_1^2 c_2^2 =4 c_1^2 c_2^2 -4 c_1^2 c_2^2 =0$ \\
		Suunta "$\Leftarrow$": $D=0 \Leftrightarrow b^2-4ac=0 \Leftrightarrow b^2=4ac \Leftrightarrow c=\frac{b^2}{4a} \Rightarrow ax^2+bx+\frac{b^2}{4a}=0 \Leftrightarrow 4a^2x^2+4abx+b^2=0 \Leftrightarrow (2ax+b)^2=0$
	\end{vastaus}
\end{tehtava}

\end{tehtavasivu}