\begin{tehtavasivu}

\subsubsection*{Opi perusteet}

\begin{tehtava}
	Ratkaise yhtälöt, ja jaa vastaavat polynomifunktiot (joiden nollakohdat olet juuri laskenut) tekijöihinsä.
	\alakohdat{
		§ $x^2+6x-6=0$
		§ $-8x^2+10x-2=0$
		§ $x^2-4x+4=0$
	}
	\begin{vastaus}
		\alakohdat{
			§ nollakohdat $x=1$ ja $x=-6$, $P(x)=(x-1)(x+6)$
			§ nollakohdat $x= \frac {1}{4}$ ja $x=1$, $P(x)=-2(4x-1)(x-1)$
				§ nollakohta $x=2$, $P(x)=(x-2)^2$
		}
	\end{vastaus}
\end{tehtava}

\begin{tehtava}
Jaa tekijöihin
\alakohdat{
§ $x^2-2x-15$
§ $\frac{1}{3}x^2-2x+3$
§ $4x^2+2x+2$
}
\begin{vastaus}
\alakohdat{
§ $(x-5)(x+3)$
§ $\frac{1}{3}(x-3)^2$
§ $2(2x^2+x+1)$, ei jakaudu 1. asteen tekijöihin.
}
\end{vastaus}
\end{tehtava}

\begin{tehtava}
    Millä seuraavista polynomeista on yhteisiä tekijöitä?

    \begin{kuvaajapohja}{1.5}{-1.5}{2.5}{-3.5}{1.5}
	\kuvaaja{(x-1)*(x+1)}{$P(x)$}{red}
	\kuvaaja{2*(x-2)*(x+0.5)}{$Q(x)$}{blue}
	\kuvaaja{0.25*(x-2)*(x-1)*(x+1)}{$R(x)$}{black}
	\kuvaaja{-(x-0.25)*(x-1.5)*(x+0.75)}{$S(x)$}{black}
    \end{kuvaajapohja}
    \begin{vastaus}
	$P(x)$:llä ja $R(x)$:llä on kaksi yhteistä tekijää, koska on kaksi kohtaa, jossa molemmat saavat arvon nolla. Vastaavasti $Q(x)$:llä ja $R(x)$:llä on yksi yhteinen tekijä. $S(x)$:llä ei ole yhteisiä tekijöitä minkään muun polynomin kanssa.
    \end{vastaus}
\end{tehtava}

\subsubsection*{Hallitse kokonaisuus}

\begin{tehtava}
    Jaa tekijöihin helpoimmalla tavalla
    \alakohdat{
        § $x^2-9$
        § $4x^2-4x+1$
        § $4x^2-4x-8$
    }
    \begin{vastaus}
    	\alakohdat{
        § $(x+3)(x-3)$ (muistikaava)
        § $(2x-1)^2$ (muistikaava)
        § $4(x-2)(x+1)$
        }
    \end{vastaus}
\end{tehtava}

\begin{tehtava}
    Toisen asteen polynomille $P$ pätee $P(-3)=P(4)=0$ ja $P(1)=12$. Ratkaise $P(x)$.
    \begin{vastaus}
        $P(x)=-(x+3)(x-4)=-x^2+x+12$
    \end{vastaus}
\end{tehtava}

\begin{tehtava}
    Osoita, että jos toisen asteen polynomin toisen asteen termin kerroin on $1$, niin sen vakiotermi on yhtä suuri kuin sen nollakohtien tulo.
    \begin{vastaus}
        Kirjoitetaan polynomi tekijämuodossa ja kerrotaan auki: $(x-a)(x-b)=x^2-2(a+b)x+ab$. Nyt syntyneen polynomin vakiotermi on $ab$.
    \end{vastaus}
\end{tehtava}

\begin{tehtava}
    Toisen asteen polynomille $P$ pätee $P(0)=P(5)=3$ ja $P(3)=-27$. Ratkaise $P(x)$.
    \begin{vastaus}
        $P(x)=5x(x-5)+3=5x^2-25x+3$
    \end{vastaus}
\end{tehtava}

\begin{tehtava}
    Paraabelin kuvaajia katsomalla voidaan huomata, että paraabelin huippu löytyy aina nollakohtien puolivälistä. (Tarkempi perustelu saadaan esimerkiksi kurssilla MAA4.) Symmetrian nojalla huipun $x$-koordinaatti on siis nollakohtien $x$-koordinaattien keskiarvo.
    \alakohdat{
        § Etsi paraabelin $-10x^2+5x+5$ huipun $x$- ja $y$-koordinaatit.
        § Johda lauseke yleisen paraabelin $ax^2+bx+c$ huipun $x$-koordinaatille.
        % myös y-koordinaattia voisi kysyä
    }
    \begin{vastaus}
        \alakohdat{
            § $x=\frac14$ ja $y=5\frac58$.
            § $x=-\frac{b}{2a}$ % ja $y=c-\frac{b^2}{4a$
        }
    \end{vastaus}
\end{tehtava}

\subsubsection*{Lisää tehtäviä}

\begin{tehtava}
    Jaa tekijöihin.
    \alakohdat{
        § $4x^2 +4x +1$
        § $4x^2 +4x +4$
        § $9-x^2$
    }
    \begin{vastaus}
        \alakohdat{
        § $(2x+1)^2$
        § $4(x^2 +x +1)$
        § $(3+x)(3-x)$
        }
    \end{vastaus}
\end{tehtava}

\begin{tehtava}
	Jaa tekijöihin.
	\alakohdat{
		§ $x^2-x-6$ % \{[ \star ]\
		§ $(x-4)^2-9$ %  \{[ \star ]\
	}
	\begin{vastaus}
		\alakohdat{
			§ $(x-3)(x+2)$
			§ $(x-1)(x-7)$
		}
	\end{vastaus}
\end{tehtava}

\begin{tehtava}
Jaa tekijöihin.
\alakohdat{
§ $5x^2+10x-15$
§ $12x^2-10x+2$
§ $-8x^2+8x+6$
}
\begin{vastaus}
\alakohdat{
§ $5(x+3)(x-1)$
§ $2(2x-1)(3x-1)$
§ $-2(2x-3)(2x+1)$
}
\end{vastaus}
\end{tehtava}

\begin{tehtava}
    Jaa tekijöihin.
    \alakohdat{
        § $-10x^2+5x+5$
        § $8x^3-12x^2+4x$
    }
    \begin{vastaus}
    	\alakohdat{
        § $-5(2x+1)(x-1)$
        § $(4x)(2x-1)(x-1)$
        }
    \end{vastaus}
\end{tehtava}

\begin{tehtava}
    Kolmannen asteen polynomille $P$ pätee $P(-1)=P(2)=P(3)=0$ ja $P(1)=-8$. Ratkaise $P(x)$.
    \begin{vastaus}
        $P(x)=-2(x+1)(x-2)(x-3)=-2x^3+8x^2-2x-2$
    \end{vastaus}
\end{tehtava}

\begin{tehtava}
   Määritä toisen asteen yhtälö jonka juuret ovat yhtälön $ x^2+7x+49 =0$ 
 \alakohdat{
    	§ juurien käänteisluvut
        § juurien neliöiden käänteisluvut.
    }
    \begin{vastaus}
        \alakohdat{
            § $k(49x^2+7x+1)=0$
            § $k(7x^2 \pm x\sqrt{7} +1)=0$
        }
    \end{vastaus}
\end{tehtava}

\begin{tehtava}
    $\star $ Tutki, onko seuraava väite totta vai ei: Jos polynomilla on nollakohta $x=1$, sen kerrointen summa on $0$. 
    \begin{vastaus}
        Totta, sillä polynomin $P(x)$ kerrointen summa on $P(1)$. (Koska $1^n=1$.)
    \end{vastaus}
\end{tehtava}

\begin{tehtava}
    $\star$ $P$ on toisen asteen polynomi, jonka vakiotermi on $1$. Polynomi $Q$ määritellään lausekkeella $Q(x)=P(x+1)-P(x)$ ja siitä tiedetään, että $Q(0)=7$ ja $Q(1)=13$. Määritä polynomin $P$ lauseke.
    \begin{vastaus}
        $P(x) = 3x^2+4x+1$
    \end{vastaus}
\end{tehtava}

\end{tehtavasivu}