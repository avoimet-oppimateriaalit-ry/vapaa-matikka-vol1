Seuraavissa harjoituskokeissa on kussakin kahdeksan tehtävää, jotka on suunniteltu niin, että jokaisesta (alakohtineen) saa korkeintaan kuusi pistettä. %tulevaisuudessa täydet ratkaisut pisteytysohjeineen %tarkista pisteytykset % puuttuu kuvia vastauksista

\subsection*{Harjoituskoe 1} %prosenttilaskuja pitäisi ympätä...
\begin{multicols}{2}

\begin{tehtava}
	\alakohdat{
	§ Perustele, onko $t^{\pi}+1$ polynomi. ($1$\,p.)
	§ Trinomi kerrotaan toisella trinomilla. Kuinka monta termiä syntyneeseen uuteen polynomiin tulee (ennen mahdollista sieventämistä)? ($0,5$\,p.)
	§ Mikä on tulomuodossa esitetyn polynomin $(x^2+1)^{10}(1-x)^2$ aste? ($0,5$\,p.)
	§ Mainitse jokin luku, joka kuuluu reaalilukuvälille $]1,966;1,967[$. ($0,5$\,p.)
	§ Mikä on korkeimman asteen termin kerroin polynomiyhtälössä $9\,001k^2-\frac{1}{4}k-\sqrt{2}k^3=-k^2+k^3$? ($0,5$\,p.)
	§ Polynomi voidaan esittää tulomuodossa $(x-2)x(x+3)$. Mikä on kyseisen polynomin suurin nollakohta? ($1$\,p.)
	§ Minkälainen (muoto \& mahdollinen suunta) kuvaaja on polynomifunktiolla $P$, jonka arvot lasketaan yhtälöllä $P(t)=-100-\frac{1}{2}t$? ($1$\,p.)
	§ Jaa tekijöihin polynomi $x^2+2x+1$. ($1$\,p.)
	}
 			\begin{vastaus}
 				\alakohdat{
	§ Ei ole polynomi, sillä muuttujan $t$ eksponentti ei ole luonnollinen luku.
	§ Yhdeksän
	§ $22$
	§ Esimerkiksi $1,9665$ (reuna-arvot eivät kuulu välille!)
	§ $1+\sqrt{2}$
	§ $x=2$
	§ laskeva suora
	§ Suoraan muistikaavalla saadaan $x^2+2x+1=(x+1)^2$.
	}
 			\end{vastaus}
 \end{tehtava}
 
 \begin{tehtava}
	\alakohdat{
§ Näytä, että polynomiyhtälöt $(t+1)^2=t+1$ ja $t^2=-t$ ovat samat.
§ Osoita, että $\sqrt{11+4\sqrt{6}}=\sqrt{3}+2\sqrt{2}$. %esimerkkitehtävä+harjoituksia!
	}
	\begin{vastaus}
		\alakohdat{
§ Avaa ensimmäisestä yhtälöstä sulut, kumoa ykköset ja yhdistä ensimmäisen asteen termit oikealle -- totea yhtäläisyys
	§ Korottamalla $\sqrt{3}+2\sqrt{2}$ neliöön ja käyttämällä muistikaavoja saadaan $11+4\sqrt{6}$.
	}
	\end{vastaus}
\end{tehtava}

\begin{tehtava}
Ratkaise epäyhtälöistä $x$.
	\alakohdat{
	§ $\frac{2x}{3}-1\geq \frac{3}{2}x$
	§ $(\frac{1}{2}x-1)^2<1-(x-1)^2$
	}
	\begin{vastaus}
		\alakohdatm{
		§ $x\leq -\frac{6}{5}$
	§ $\frac{2}{5}<x<2$
	}
	\end{vastaus}
\end{tehtava}

\begin{tehtava}
Ratkaise yhtälöistä tuntematon reaaliluku $z$.
	\alakohdat{
	§ $(2+\frac{\sqrt{2}}{2})z^2-z=\sqrt{2}z^3$
	§ $z^4-2z^2+1=0$
	}
		\begin{vastaus}
	\alakohdat{
	§ $z=0$ tai $z=\frac{1}{2}$ tai $z=\sqrt{2}$
	§ $z=-1$ tai $z=1$
	}
		\end{vastaus}
\end{tehtava}

\begin{tehtava}
Millä parametrin $k$ arvoilla yhtälöllä $kx^2-(k+1)x+1=0$ on kaksi erisuurta reaalijuurta?
	\begin{vastaus}
Kaikilla paitsi $k=0$
	\end{vastaus}
\end{tehtava}

\begin{tehtava} %tästä pari perustehtävää ja kertaustehtäävää, myös suhteellisena, eikä tunnetuilla absoluuttisilla + kuinka paljon rahaa on tilillä kolmen vuoden kuluttua ilman lisätlalletusta?
Sofie tallettaa $1\,000$ euroa uudelle pankkitilille vuoden alussa ja uudelleen saman verran vuoden kuluttua. Pankki maksaa talletukselle vuosikorkoa siten, että kahden vuoden kuluttua ensimmäisestä talletuksen tilillä on rahaa $2\,100$ euroa. Kuinka suuri on pankin tarjoama korkokanta prosentin kymmenesosan tarkkuudella? Inflaatiota ja korkotuotoista maksettavaa lähdeveroa ei oteta huomioon. Muita tilitapahtumia ei ole.
	\begin{vastaus}
	Jos pankin korkokerroin on $x=(1+\frac{p}{100})$, niin vuoden kuluttua tilillä on rahaa $1\,000x$. Kahden vuoden kuluttua tilillä on rahaa $((1\,000x)+1\,000)x)$, mikä sisältää molemmat talletukset ja kaksi vuosikorkoa. Saadaan yhtälö $((1\,000x)+1\,000)x)=2\,100$ eli sievennettynä ja uudelleen järjesteltynä $1\,000x^2+1\,000x-2\,100=0$. Ratkaisukaavalla tai laskimella saadaan kaksi ratkaisua, joista hyväksytään vain positiivinen: $x=\frac{1}{10}(\sqrt{235}-5)\approx1,033$. Pankin tarjoama vuosikorkokanta on siten $3,3$\,\%.
	\end{vastaus}
\end{tehtava}

\begin{tehtava}
Mikä on funktion $f:\mathbb{R}\rightarrow \mathbb{R}$, pienin arvo, kun funktion arvot määritellään kaavalla $f(x)=3x^2-12x+6$?
	\begin{vastaus}
	$-6$ (vetoaminen paraabelin symmetrisyyteen, että pohja on nollakohtien puolivälissä, tai vastaus nähdään neliöön täydennetystä lausekkeesta perustellen, että reaaliluvun neliö voi olla vähintään nolla)
	\end{vastaus}
\end{tehtava}

\begin{tehtava}
	\alakohdat{
	§ Osoita, että kolmannen asteen polynomi $x^3 \pm y^3$ voidaan esittää tulomuodossa $(x \pm y)(x^2 \mp xy+y^2)$. ($2$\,p.)
	§ Ratkaise epäyhtälö $z^3-8\geq z-2$. ($4$\,p.)
	}
	\begin{vastaus}
	\alakohdat{
	§ Purkamalla sulkeet lausekkeesta $(x \pm y)(x^2 \mp xy+y^2)$ saadaan vaadittu $x^3 \pm y^3$. (Huomaa, toisiaan vastaavat etumerkit!)
	§ Kirjoitetaan epäyhtälö $z^3-8\geq z-2$ a-kohdan kaavan avulla muodossa $(z-2)(z^2+2z+4) \geq z-2$. Epäyhtälöstä voidaan jakaa tekijä $z-2$ pois, mutta sitä varten tutkitaan erikseen tilanteet $z=2$, $z<2$,  ja $z>2$.
	
	Jos $z=2$, niin alkuperäinen epäyhtälö yksinkertaistuu muotoon $0\geq0$, mikä pitää paikkansa.
	
	Jos $z>2$, niin $z-2>0$, eli lauseke $z-2$ on positiivinen. Jakamalla epäyhtälö puolittain kyseisellä lausekkeella säilyttää suuruusjärjestyksen, eli epäyhtälömerkkiä ei tarvitse kääntää. Siis epäyhtälöstä $(z-2)(z^2+2z+4) \geq z-2$ on yhtäpitävä epäyhtälön $z^2+2z+4 \geq 1$ kanssa. Tästä saadaan edelleen toisen asteen polynomiepäyhtälö $z^2+2z+3 \geq 0$. Vasemman puolen polynomin diskrimanttia tutkimalla selviää, että kyseisellä polynomilla ei ole nollakohtia. Koska lisäksi kyseisen polynomin kuvaaja on ylöspäin aukeava paraabeli, voidaan päätellä, että polynomi $z^2+2z+3$ saa vain positiivisia arvoja. Epäyhtälö $z^2+2z+3 \geq 0$ pitää siis paikkansa kaikilla kahta suuremmilla muuttujan $z$ arvoilla.

Jos $z<2$, epäyhtälö $(z-2)(z^2+2z+4) \geq z-2$ muuttuu muotoon $z^2+2z+4 \leq 1$ eli $z^2+2z+3 \leq 0$. Aiemmin jo todettiin, että vasemman puolen polynomi on aina positiivinen, joten epäyhtälö ei pidä paikkaansa millään kahta pienemmillä $z$:n arvoilla.

Yhdistämällä osatulokset päädytään päätelmään, että alkuperäisen epäyhtälön $z^3-8\geq z-2$ ratkaisu on $z\geq 2$.
	}
	\end{vastaus}
\end{tehtava}

\end{multicols}

\newpage
\subsection*{Harjoituskoe 2}

\begin{tehtava}
	\alakohdat{
	§ Perustele, onko $\frac{1}{t^2}+\pi$ polynomi. ($1$\,p.)
	§ Binomi kerrotaan trinomilla. Kuinka monta termiä syntyneeseen uuteen polynomiin tulee (ennen mahdollista sieventämistä)? ($0,5$\,p.)
	§ Mikä on tulomuodossa esitetyn polynomin $(x^3+1)^5(1-x)^3$ aste? ($0,5$\,p.)
	§ Esitä ehto $x\in ]-\pi,42]$ kaksoisepäyhtälönä. ($0,5$\,p.)
	§ Jaa polynomi $y^2-2y+1$ tekijöihin. ($1$\,p.)
	§ Sievennä $\frac{x^2-10^{102}}{x-10^{51}}$ ($1$\,p.)
	§ Minkälainen (muoto \& mahdollinen suunta) kuvaaja on polynomifunktiolla $P$, jonka arvot lasketaan yhtälöllä $P(t)=12,34^{56}-789t^2$? ($0,5$\,p.)
	§ Polynomi voidaan esittää tulomuodossa $(x-32)x(x+23)$. Mikä on kyseisen polynomin pienin nollakohta? ($0,5$\,p.)
	§ Kuinka monta nollakohtaa voi korkeintaan olla kuudennen asteen polynomilla? ($0,5$\,p.)

	}
 			\begin{vastaus}
 				\alakohdat{
	§ Ei ole, sillä muuttujan $t$ eksponentti ($-2$) ei ole luonnollinen luku.
	§ kuusi
	§ $18$
	§ $\pi < x \leq 42$
	§ Suoraan muistikaavalla saadaan $y^2-2y+1=(y-1)^2$.
	§ $x+10^{51}$
	§ alaspäin avautuva paraabeli
	§ $x=-23$
	§ kuusi
	}
 			\end{vastaus}
 \end{tehtava}
 
 \begin{tehtava}
 Ratkaise yhtälöistä reaalinen tuntematon $t$.
 	\alakohdat{
 	§ $(9-t^2)(t-1)t=0$
 	 § $(2t^2-1)(t-1)=1$
 	}
 	\begin{vastaus}
 	\alakohdat{
 	 		§ $t=\pm 3$, $t=0$ tai $t=1$
 		§ $t=0$ tai $t=\frac{1}{2}(1-\sqrt{3})$ tai $xt\frac{1}{2}(1+\sqrt{3})$
 	}
 	\end{vastaus}
 \end{tehtava}

\begin{tehtava}
Millä parametrin $a$ arvoilla yhtälöllä $x^2-4ax-a=0$ on täsmälleen yksi reaaliratkaisu?
	\begin{vastaus}
	Diskriminantin lauseke on $16a^2+4a$, ja yhtälöllä on yksi reaalijuuri (kaksoisjuuri) jos ja vain jos diskriminantin arvo on nolla. Ratkaistaan siis yhtälö $16a^2+4a=0$.
	\begin{align*}
	16a^2+4a&=0 \\
	4a(4a+1)&=0 \\
	\end{align*}
	Eli $4a=0$ tai $4a+1=0$. Ensimmäisestä yhtälöstä saadaan ratkaistua $a=0$, ja toisesta $a=-\frac{1}{4}$.
	\end{vastaus}
\end{tehtava}

\begin{tehtava}
Reaaliluvuista $a$ ja $b$ tiedetään, että $a<b$. Osoita kyseistä epäyhtälöä muokkaamalla, että lukujen $a$ ja $b$ aritmeettinen keskiarvo on todellakin lukujen $a$ ja $b$ välissä.
	\begin{vastaus}
	Osoitetaan ensin, että keskiarvo $\frac{a+b}{2}$ on suurempi kuin $a$:
	\begin{align*}
	a&<b && ||+a \\
	2a&<a+b && ||:2 \\
	a&<\frac{a+b}{2} && 
	\end{align*}
	
	Osoitetaan sitten, että keskiarvo on pienempi kuin $b$:
	\begin{align*}
	a&<b && ||+b \\
	a+b&<2b && ||:2 \\
	\frac{a+b}{2}&<b &&
	\end{align*}
	
	Koska keskiarvo on suurempi kuin $a$ mutta pienempi kuin $b$, on se $a$:n ja $b$:n välissä. (Todistuksen kannalta ei ole väliä, minkä merkkisiä $a$ ja $b$ ovat.) 
	\end{vastaus}
\end{tehtava}

\begin{tehtava}
Määritä pienin kokonaisluku, joka toteuttaa epäyhtälön $\frac{1}{4}n^4<240+n^2$.
	\begin{vastaus}
	$n=-5$
	\end{vastaus}
\end{tehtava}

\begin{tehtava} %PALJON LISÄÄ NÄITÄ TEHTÄVIÄ KIRJAAN JA ESIMERKKEJÄ ('' on myytävä näin monta,jotta pääsee omilleen, millä hinalla möi...'')
Karkkikauppias ostaa joka kuukausi irtokarkkeja tukusta hintaan $10$\,€/kg ja myy niitä tavallisesti asiakkailleen hintaan $12,9$\,€/kg. Makeisten ostamisen lisäksi kauppiaalla menee joka kuukausi $500$ euroa niin sanottuihin kiinteisiin kustannuksiin. Kiinteät kustannukset ovat aina samat riippumatta myynnin määrästä. Kysynnän hän osaa arvioida niin hyvin, että mitään makeisia ei jää myymättä.
\alakohdat{
	§ Merkitään myynnin kuukausittaista määrää kilogrammoina $q$:lla. Muodosta lausekelle niin sanotulle voittofunktiolle $P$ eli kuvaukselle myynnin määrästä tuottoon (euroina), josta on vähennetty kaikki myyntitoiminnan kustannukset. Funktion arvot määräävään lausekkeeseen ei tarvitse erikseen merkitä yksiköitä.
	§ Kauppias on pitkän myyntikokemuksen perusteella huomannut, että kilogrammahinnan kasvattaminen eurolla vähentää aina kuukausittaista myyntiä $50$ kilogrammaa ja toisaalta kilogrammahinnan vähentäminen eurolla lisää aina kuukausittaista myyntiä samat $50$ kilogrammaa. Mikä tulisi asettaa kilohinnaksi, jotta $180$ kilogramman kuukausimyynnillä kauppias ei jäisi liiketoiminnassaan tappiolle?
}
	\begin{vastaus}
		\alakohdat{
	§ $P(q)=12,9q-10q-500=2,9q-500$
	§ Merkataan kilogrammahinnan euromääräistä lisäystä $x$:llä. Nyt voittofunktion määrittelevä yhtälö on $P(q,x)=(12,9+x)(q-50x)-10(q-50x)-500$, eli uusi kilogrammahinta on $12,9+x$ ja myynnin määrä $q-50x$. Sijoitetaan tehtävänannossa annettu $q=180$, niin saadaan $P(180,x)=(12,9+x)(180-50x)-10(180-50x)-500$. Ottamalla tästä tai jo aiemmasta muodosta yhteinen tekijä saadaan lyhyempi muoto $P(180,x)=(12,9+x-10)(180-50x)-500$ eli $P(180,x)=(2,9+x)(180-50x)-500$ (vrt. a-kohdan sievennetty lauseke).
	
Liiketoiminta on tappiotonta, kun $P(x)\geq0$, missä $P(x)=0$ tarkoittaisi tilannetta, että kauppiaan tulot kattavat täsmälleen kaikki kustannukset. Ratkaisemme siis epäyhtälön $(2,9+x)(180-50x)-500\geq0$, joka sievennettynä on $-50x^2+35x+22\geq0$. Polynomin $-50x^2+35x+22$ nollakohdat ovat ratkaisukaavan tai laskimen ratkaisuohjelman perusteella $x=-0,4$ ja $x=1,1$. Koska polynomin korkeimman asteen termin kerroin on negatiivinen, olisi polynomin kuvaaja alaspäin aukeava paraabeli, ja siten kuvaajan perusteella voidaan päätellä, että funktio $P$ saa positiiviset arvonsa nollakohtiensa välissä. Nollakohtia $x=-0,4$ ja $x=1,1$ vastaavat kilohinnat ovat $12,9-(-0,4)=13,3$ ja $12,9-1,1=11,8$, molemmat yksikössä €/kg.

Jotta myyntitappiolta vältyttäisiin, kauppiaan tulee asettaa kilohinnaksi vähintään $11,8$\,€/kg ja korkeitaan $13,3$\,€/kg. (Näiden rajojen välillä tulee voittoa., rajojen ulkopuolella tappiota.)
		}
	\end{vastaus}
\end{tehtava}

\begin{tehtava}
	\alakohdat{
§ Avaa sulut lausekkeesta $(a-b)^3$ ja sievennä tulos. ($2$\,p.)
§ Ratkaise reaaliluku $x$ yhtälöstä $x^3-3x^2+3x=9$. ($4$\,p.)
	}
	\begin{vastaus}
		\alakohdat{
§ $a^3-3a^2b+3ab^2-b^3$
§ Binomin kuutioksi täydentämällä saadaan yhtälö potenssiyhtälö $(x-1)^3=8$, jonka ratkaisuna $x=3$
	}
	\end{vastaus}
\end{tehtava}

\begin{tehtava}
Muodosta muuttujan $a$ funktio $f$, joka antaa toisen asteen polynomin $ax^2+2x+2a$ nollakohtien etäisyyden. Määritä funktion $f$ laajin reaalinen määrittelyjoukko ja arvojoukko.
	\begin{vastaus}
	Nollakohtien erotusfunktion määrittelee yhtälö $f(a)=\frac{\sqrt{4-8a^2}}{a}$. Määrittelyjoukkoa rajoittavia tekijöitä ovat neliöjuuri ja jakolasku. Nimittäjän perusteella vaaditaan $a\neq 0$, neliöjuuren perusteella $4-8a^2\geq 0$. Epäyhtälöstä saadaan ratkaisuksi $-\frac{1}{\sqrt{2}}\leq a \leq \frac{1}{\sqrt{2}}$. Yhdistämällä tiedot saadaan määrittelyjoukoksi kaksiosainen reaalilukuväli, jonka osat erottaa luku nolla: $[-\frac{1}{\sqrt{2}},\frac{1}{\sqrt{2}}] \setminus \lbrace 0 \rbrace$. % $[-\frac{1}{\sqrt{2}},0[\cup ]0,\frac{1}{\sqrt{2}}]$.
	
Arvojoukon voi päätellä asettamalla funktion arvoksi tuntematon $t$ ja ratkaisemalla yhtälöstä muuttuja $x$:
	$$\frac{\sqrt{4-8x^2}}{x}=t$$
	$$\sqrt{4-8x^2}=tx$$
	$$4-8x^2=(tx)^2=t^2x^2$$
	$$t^2x^2+8x^2=4$$
	$$(t^2+8)x^2=4$$
	$$x^2=\frac{4}{t^2+8}$$
	$$x=\pm \frac{2}{\sqrt{t^2+8}}$$
	
	Huomataan, että saadun lausekkeen perusteella ei ole mitään syytä rajoittaa $t$:n arvoja. Funktion arvojoukko on siis koko reaalilukujen joukko $\mathbb{R}$.
	\end{vastaus}
\end{tehtava}
\newpage
\subsection*{Harjoituskoe 3}

\begin{tehtava}
	\alakohdat{
	§ Perustele, onko $\sqrt{k}+\sqrt{2}$ polynomi. ($1$\,p.)
	§ Binomi kerrotaan trinomilla. Kuinka monta termiä syntyneeseen uuteen polynomiin tulee (ennen mahdollista sieventämistä)? ($1$\,p.)
	§ Mikä on tulomuodossa esitetyn polynomin $(x^3+1)^5(1-x)^3$ aste? ($1$\,p.)
	§ Selitä, miksi kerrottaessa epäyhtälöä negatiivisella luvulla epäyhtälön merkki kääntyy. ($1$\,p.)
		§ Esitä muuttujaa $x$ koskeva kaksoisepäyhtälönä esitetty ehto $-\frac{2}{3}<x\leq100$ joukko-opillisesti. ($1$\,p.)
		§ Muuttujan $x$ arvot ovat suoraan verrannollisia $y$:n arvoihin. Perustele, voiko $x$:n ja $y$:n välinen yhteys olla polynomiaalinen. ($1$\,p.)
	}
 			\begin{vastaus}
 				\alakohdat{
	§ Ei ole, sillä muuttujan $k$ eksponentti $\frac{1}{2}$ ei ole luonnollinen luku.
	§ kuusi
	§ $18$
	§ $\pi < x \leq 42$
	§ $x\in ]\frac{2}{3},100]$
	§ Jos $x\propto y$, niin voidaan kirjoittaa $x=ky$, missä $k$ on vakio. Yksiterminen lauseke $ky$ on polynomi.
	}
 			\end{vastaus}
 \end{tehtava}
 
 \begin{tehtava}
Olkoot $a$, $b$, $c$, $d$ ja $n$ lukuja. Osoita, että $$(a^2+nb^2)(c^2+nd^2)=(ac+nbd)^2+n(ad-bc)^2.$$
	\begin{vastaus}
	Puretaan sulkeet auki ja täydennetään binomien neliöiksi.
		\begin{align*}
		&(a^2+nb^2)(c^2+nd^2) \\
		&=a^2c^2+na^2d^2+nb^2c^2+n^2b^2d^2 \\
		&= a^2c^2+n^2b^2d^2+na^2d^2+nb^2c^2 \\
		&= a^2c^2+n^2b^2d^2+na^2d^2+nb^2c^2 \\
		&+2nabcd-2nabcd \\
		&= a^2c^2+2nabcd+n^2b^2d^2+na^2d^2\\
		&-2nabcd+nb^2c^2 \\
		&= (ac+nbd)^2+na^2d^2-2nabcd+nb^2c^2 \\
		&= (ac+nbd)^2+n(a^2d^2-2abcd+b^2c^2) \\
		&= (ac+nbd)^2+n(ad-bc)^2 \\
\end{align*}
Huom.! Voi myös ''lähteä lopusta'', eli purkaa lauseke $(ac+nbd)^2+n(ad-bc)^2$ ja sitten ryhmitellä vastaus tulomuotoon $(a^2+nb^2)(c^2+nd^2)$. Kumpikin käy.
	\end{vastaus}
\end{tehtava}

 \begin{tehtava}
Tykin ammuksen lentokorkeus $h$ ajanhetkellä $t$ (sekunteina) laukaisun jälkeen noudattaa likimain yhtälöä on $h(t)=h_0+v_0t-\frac{1}{2}gt^2$, missä $h_0$ on laukaisukorkeus metreinä, $v_0$ ylöspäin suuntautuva pystysuora lähtönopeus yksikössä m/s, ja $g$ on putoamiskiihtyvyys $9,81$\,m\,s$^{-2}$. Ammus poistuu laukaisussa tykin piipusta metrin korkeudelta ylöspäin suuntautuvalla nopeudella $15,5$\,m/s. Kuinka kauan kestää ennen kuin tykinkuula osuu maahan? Piirrä tilanteesta kuva.
	\begin{vastaus}
	Ammus osuu maahan noin $3,22$ sekunnin päästä. %FIXME: kuva!
	\end{vastaus}
\end{tehtava}

\begin{tehtava}
Millä vakion $t$ arvoilla yhtälöllä $tx^2+tx-6=0$ ei ole lainkaan reaalisia ratkaisuja?
	\begin{vastaus}
Ei millään $t$:n reaalilukuarvoilla.
	\end{vastaus}
\end{tehtava}

\begin{tehtava}
Tutkitaan polynomifunktiota $P$, jonka arvot määritellään kaavalla $P(x)=x^3-2x^2-x$. Millä $x$:n arvoilla polynomi saa positiivisia arvoja?
	\begin{vastaus}
	Kun $1-\sqrt{2}<x<0$ tai $x>1+\sqrt{2}$
	\end{vastaus}
\end{tehtava}

\begin{tehtava}
Ratkaise yhtälö $t^5-t^4=t^2-t^3$.
	\begin{vastaus}
$t^5-t^4=t^2-t^3$ vastaa yhtälöä $t^5-t^4+t^3-t^2=0$, jonka taasen saa yhteisen tekijän (useaan kertaan) ottamalla muotoon $t^2(t-1)(t^2+1)=0$, eli reaaliset nollakohdat ovat $t=0$ ja $t=1$.
	\end{vastaus}
\end{tehtava}

\begin{tehtava}
Ratkaise yhtälöstä $y^{2n}=y^n$ molemmat reaalimuuttujat $y$ ja $n$.
	\begin{vastaus}
Alkuperäinen yhtälö on yhtäpitävä yhtälön $y^{2n}-y^n=0$ kanssa. Jaetaan tekijöihin ja sovelletaan tulon nollasääntöä:
	\begin{align*}
	y^{2n}-y^n&=0 \\
	(y^n)^2-y^n&=0 \\
	y^n \cdot y^n-y^n&=0 \\
	y^n(y^n-1)&=0\\
	\end{align*}
Eli joko $y^n=0$ tai $y^n-1=0$.	 Ensimmäinen yhtälö voi päteä vain, jos $y=0$. (Jos käsitellään $n$:n eri arvoja, niin eksponenttifunktio ei voi saada milloinkaan arvoa $0$.) Tällöin vaaditaan yksikäsitteisyyden vuoksi $n\neq0$, koska $0^0$ ei ole määritelty.

Toisesta yhtälöstä saadaan $y^n=1$, joka voi olla tosi vain, jos joko $n=0$ ja $y \neq 0$ (minkä tahansa nollasta poikeavan luvun nollas potenssi on arvoltaan yksi) tai sitten $y=1$ ja $n$ on mielivaltainen.

Yhtälön ratkaisut ovat siis: \\
$y=0$ ja $n \neq 0$ \\
$n=0$ ja $y \neq 0$ \\
$y=1$ ja mikä tahansa $n$
	\end{vastaus}
\end{tehtava}

\begin{tehtava}
Polynomin $Q$ muuttuja on $x$, ja polynomi voidaan esittää tulomuodossa $(ax-b)(\frac{2ab}{a^2+b^2}x-1)(bx-a)$, missä $a$ ja $b$ ovat reaalisia vakioita, joille pätee $0<a<b$. Perustele, mikä on funktion nollakohtien suuruusjärjestys.
		\begin{vastaus}
Lausekkeen perusteella nollakohdat ovat $\frac{b}{a}$,$\frac{a^2+b^2}{2ab}$ ja $\frac{a}{b}$. Koska $0<a<b$, niin $a$ on positiivinen, ja kaksoisepäyhtälö voidaan jakaa sillä niin, että järjestys säilyy (''merkkien suuntaa ei tarvitse kääntää''): $0<1<\frac{b}{a}$. Samoin voidaan tehdä $b$:llä: $0<\frac{a}{b}<1$. Vertailemalla nähdään, että selvästi $\frac{a}{b}<\frac{b}{a}$. Kolmas nollakohta on kahden muun aritmeettinen keskiarvo, joten se on varmasti niiden välissä.
		\end{vastaus}
\end{tehtava}
\newpage

\subsection*{Harjoituskoe 4}

\begin{tehtava}
	\alakohdat{
	§ Perustele, onko $\frac{2t^0+5t}{3}$ polynomi. ($1$\,p.)
	§ Trinomi kerrotaan toisella trinomilla. Kuinka monta termiä syntyneeseen uuteen polynomiin tulee (ennen mahdollista sieventämistä)? ($0,5$\,p.)
	§ Mikä on tulomuodossa esitetyn polynomin $2(1-2y)^2(y^2+2)^2$ aste? ($0,5$\,p.)
	§ Esitä muuttujaa $t$ koskeva ehto $t\leq \pi$ joukko-opillisesti. ($0,5$\,p.)
	§ Polynomi voidaan esittää tulomuodossa $(x-2)x(x+3)$. Mikä on kyseisen polynomin suurin nollakohta? ($0,5$\,p.)
		§ Sievennä lauseke $\frac{x^3-4x^2-4x}{x^2-2x}$. ($3$\,p.)
	}
 			\begin{vastaus}
 				\alakohdat{
	§ On, sillä sievennetyssä muodossa $\frac{2}{3}+\frac{5}{3}t$ on selvästi ensimmäisen asteen polynomi.
	§ Yhdeksän
	§ $6$
	§ $t \in ]\infty,\pi]$
	§ $x=2$
	§ $x-2$
	}
 			\end{vastaus}
 \end{tehtava}

\begin{tehtava}
Polynomifunktion $P$ arvot määritellään kaavalla $P(x)=(k^2+2k)x+2k$. Määritä vakio $k$ siten, että ehto $P(2)=0$ pätee. Piirrä ehdon täyttävän polynomifunktion kuvaaja.
	\begin{vastaus}
$P(2)=0 \Rightarrow$ $(k^2+2k)\cdot 2+2k=0$
\begin{align*}
k^2\cdot 2+2k\cdot 2+2k&=0 \\
2k^2+4k+2k&=0 \\
k^2+2k+k&&=0 \\
(k+1)^2&=0 \\
k+1=&0 \\
k&=-1  \\
\end{align*}

Tällöin siis $P(x)=(1-2)x-2=-x-2$, jonka kuvaaja on laskeva suora. %FIXME: lisääkuva
	\end{vastaus}
\end{tehtava}

\begin{tehtava}
Millä vakion $a$ arvoilla yhtälöllä $x^2+ax-a=0$ on kaksoisjuuri? Mikä tämä kaksoisjuuri tällöin on?
	\begin{vastaus}
Diskriminantista $a^2+4a=0$, josta kaksi juurta: $a=0$ tai $a=-4$.

Tapaus $a=0$:
\begin{align*}
x^2+0x-0&=0 \\
x^2&=0 \\
x&=0 \\
\end{align*}
Tapaus $a=-4$:
\begin{align*}
x^2-4x+4&=0 \\
(x-2)^2&=0 \\
x-2&=0 \\
x&=2
\end{align*}
	\end{vastaus}
\end{tehtava}

\begin{tehtava} %tästä esimerkkejä...
Erääseen ulkomaalaiseen matkapuhelinliittymään on saatavilla kaksi maksusuunnitelmaa mobiilidatan käyttöön. Tarjouksen A mukaan maksat kuukaudessa vakiohinnan $40$ euroa riippumatta siitä, kuinka paljon dataa kulkee. Tarjouksen B mukaan kuukauden ensimmäiset $100$ megatavua maksaa $25$ euroa, minkä jälkeen ylimenevästä osasta maksetaan $0,12$ euroa megatavua kohden. Millä kuukausittaisilla datamäärillä tarjous A on tarjousta B edullisempi? Hinnoittelussa ei tehdä eroa vastaanotetun ja lähetetyn datan välillä.
	\begin{vastaus}
	Merkataan $x$:llä kulkeneen datan määrää megatavuissa. Tällöin tarjouksen A hinta on $A(x)=40$ (hinta aina $40$ euroa, vakiofunktio) ja tarjouksen B hinta määritellään paloittain:
	$$B(x)=\begin{cases}
	25, & \mbox{kun } x\leq 100 \\
	0,12(x-100)+25, & \mbox{kun } x>100
	\end{cases}$$
Yhtäsuuruuden voi merkitä vaihtoehtoisesti myös jälkimmäisen lausekkeen ehtoon: $x\geq 100$. Jälkimmäisessä ehdossa on pelkän $x$:n sijaan $x-100$, jotta datamäärän lasku alkaisi nollasta, eikä heti sadan megatavun rajan ylitettyä tulisi $0,12\cdot 100=12$ euron lisälaskua.

Alle sadan megatavun käytössä tarjous A ei voi olla halvempi, koska $40>25$. Sitä suuremmalla käytöllä sen sijaan tämä on mahdollista. Ratkaistaan siis epäyhtälö $40<0,12(x-100)+25$, josta selviää, millä $x$:n arvoilla tarjous A on halvempi.
\begin{align*}
40&<0,12(x-100)+25 \\
15 &<0,12(x-100) \\
125&<x-100 \\
225&<x \\
\end{align*}
	Tarjous A on siis halvempi, kun dataa siirretään kuukaudessa yli 225 megatavua.
	\end{vastaus}
\end{tehtava}

\begin{tehtava}
Ratkaise $y$ yhtälöstä $\frac{1}{2}n^n y^2-n^{2n}y-n^{3n+1}=0$, kun $n=50$. Esitä vastaus kymmenpotenssimuodossa kolmen merkitsevän numeron tarkkuudella.
	\begin{vastaus}
	Sieventämällä saadaan, että ratkaisut ovat muotoa $y=n^n (1\pm \sqrt{1+2n})$. Sijoittamalla $n=50$ saadaan laskimesta kymmenpotenssimuodoiksi $-8,04\cdot 10^{85}$ ja $9,81\cdot 10^{85}$.
	\end{vastaus}
\end{tehtava}

\begin{tehtava} %tästä harjoituksia!
Ratkaise yhtälö $x^4-\frac{1}{2}x^3-4x^2+2x=0$.
	\begin{vastaus}	
$x^4-\frac{1}{2}x^3-4x^2+2x=x(x-\frac{1}{2})(x^2-4)$, eli $x=0$, $x=\frac{1}{2}$ tai $x=\pm 2$.
	\end{vastaus}
\end{tehtava}

\begin{tehtava}
Funktion $y$ arvot määritellään kaavalla $y(t)=\dfrac{t^2+\left(\frac{1}{3}+\pi\right)t-\frac{\pi}{3}}{t^2+\left(\frac{1}{3}-\pi\right)t-\frac{\pi}{3}}$.
	\alakohdat{
	§ Mikä on funktion määrittelyjoukko?
	§ Määritä ne $ty$-koordinaatiston pisteet, joissa funktion kuvaaja leikkaa jomman kumman koordinaattiakselin.
	}
		\begin{vastaus}
	\alakohdat{
	§ Reaalilukujen joukko poislukien $t=-\frac{1}{3}$ ja $t=\pi$
	§ $y$-akseli leikkautuu, kun $t=0$, eli pisteessä $(0,\frac{\pi}{3})$. $t$-akseli leikkautuu vain yhdessä pisteessä: $(-\pi, 0)$, koska määrittejoukko ei sisällä funktion lausekkeen osoittajan toista nollakohtaa $-\frac{1}{3}$.
	}
		\end{vastaus}
\end{tehtava}

\begin{tehtava}
Johda toisen asteen yhtälön ratkaisukaava lähtien yhtälön normaalimuodosta $ax^2+bx+c=0$, missä $a, b$ ja $c$ ovat reaalisia vakioita, ja $a \neq 0$.
	\begin{vastaus}
	Katso luku Toisen asteen yhtälön ratkaisukaava. (Johdosta on olemassa erilaisia variantteja.)
	\end{vastaus}
\end{tehtava}

\newpage

%KERTAUSTEHTÄVIKSI
%\alakohdat{
%§ Ratkaise yhtälöt.\\ a) $-x-5+2x=0$\\ b) $8x^2=-2x+4$\\ c) $(-3x)^2-36=0$
%§ Ratkaise epäyhtälöt.\\ a) $-2x^2-2>2$\\ b) $x^2-1\geq0$\\ c) $ax^2<bx$
%§ Kolme lohikäärmettä ja yksi kissa painaa saman verran kuin 10 kissaa ja 18 Jarkkoa. Muodosta yhtälö tilanteesta yhtälö ja ratkaise yhden lohikäärmeen paino. Tiedot eivät ole faktuaalisia.
%§ Montako juurta on yhtälöllä\\ a) $9x^2+3x+1=0$\\ b) $6x^2-3x=-2$\\ c) $3x^2-32$\\ d) $3x-30=0$
%§ Millä parametrin r arvoilla yhtälölle $rx^2-rx+1=0$ ei ole ratkaisuja.
%§ 
%§ 
%§ Ratkaise yhtälö $\frac{21x^2}{700}-\frac{7x}{1400}-\frac{14x}{2800}=0$ ilman laskinta.
%}

%\alakohdat{
%§ Ratkaise yhtälöt.\\ a) $4x-1=0$\\ b) $x=-3x+2$\\ c) $2g-3g=g-8$
%§ Ratkaise yhtälöt.\\ a) $4x^2-1=0$\\ b) $x^3=-3x$\\ c) $2y^2=y-8$
%§ Ratkaise epäyhtälöt.\\ a) $1-\dfrac{1-x}{6}<x$\\ b) $(x+1)(x^2-2x-1)\geq0$\\ c) $\frac{x}{2}>\frac{x}{5}$
%§ Täydennä neliöksi. \\ a) $x^2+2x+1$\\ b) $9x^2-6x+1$\\ c) $3x+4x^2+\frac{9}{16}$
%§ Ratkaise $x$. \\ a) $\frac{x}{2}(x-1)=0$\\ b) $(3x^2-3)(3x^2+1)=0$\\ c) $(x-a)(x-b)=0$
%§ Ratkaise yhtälö. $(x^2-4x+4)^2=0$
%§ Millä $x$:n arvoilla polynomi $x^2-2x-3$ saa positiivisia arvoja?
%§ Ratkaise yhtälöt.\\ a) $x-5x=0$\\ b) $x^4-1=0$\\ c) $(x-1)(x+4) = x(x-5)$
%§ Ratkaise epäyhtälöt.\\ a) $x^2-8\geq0$\\ b) $x^2-8\geq(x-3)^2$\\ c) $x^2-6x+9\leq0$
%§ Kolmannen asteen polynomifunktiolle pätee $P(-1)=0$, $P(0)=0$ ja $P(1)=0$. Lisäksi $P(3)=3$. Määritä polynomi $P$.
%\item
%§ Laske. \\ a) $(3x+1)^2+5=6x$\\ b) $5x>15x$\\ c)$3x^2=(3x)^2+6$ 
%§ Lukujen $a$ ja $b$ erotus $a-b=2$ ja tulo $ab=4$, laske käänteislukujen erotus $\frac{1}{a}-\frac{1}{b}$.
%§ Millä vakion $c$ arvoilla polynomi $x^2+cx+c$ saa sekä positiivia että negatiivisia arvoja?
%§ Jaa polynomi tekijöihin.\\ a) $x^2+x-30$\\ b)  $x^2-x-30$\\ c)  $-2x^2+5x-3$ 
%§ Ratkaise epäyhtälöt. \\
%a) $2x^3 \geq x$ \\
%b) $y^2 \leq 3y -9 $

%§ Ratkaise yhtälöt.\\ a) $3x-5=6$\\ b) $20x-20=20$\\ c) $2x-5=\frac{5x+2}{2}$
%§ Ratkaise yhtälöt.\\ a) $-5x^2-5=0$\\ b) $8x^2=-2x$\\ c) $x^2+10=0$
%§ Ratkaise epäyhtälöt.\\ a) $-5x^2-5>0$\\ b) $x^2+10\geq0$\\ c) $x^2<x$
%§ Millä $h$:n arvoilla yhtälöllä $h^2x^2+hx+\frac{1}{4}=0$ on tasan yksi ratkaisu?
%§ Ratkaise yhtälö $x^2-6x=0$\\ a) tulon nollasäännöllä\\ b) toisen asteen yhtälön ratkaisukaavalla
%§ Ratkaise epäyhtälö $(\frac{x+1}{-4})x-3>1$
%§ Jaa polynomi tekijöihin.\\ a) $x^2-x-2$\\ b) $4x^2-2x-2$
%}