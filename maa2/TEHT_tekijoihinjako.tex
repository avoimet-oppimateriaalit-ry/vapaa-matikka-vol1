\begin{tehtavasivu}

\subsubsection*{Opi perusteet}

\begin{tehtava}
    Esitä tulona ottamalla yhteinen tekijä.
    \alakohdatm{
        § $2x+6$
        § $x^2 -4x$
        § $3x^2 - 6x$
    }
    \begin{vastaus}
        \alakohdatm{
        § $2(x+3)$
        § $x(x-4)$
        § $3x(x-2)$
        }
    \end{vastaus}
\end{tehtava}

\begin{tehtava}
    Jaa tekijöihin.
    \alakohdatm{
        § $10a+5ab$
        § $x^4 -x^3$
        § $xy+x^2y$
    }
    \begin{vastaus}
        \alakohdatm{
        § $5a(2+b)$
        § $x^3(x-1)$
        § $xy(1+x)$
        }
    \end{vastaus}
\end{tehtava}

\begin{tehtava}
    Jaa tekijöihin muistikaavojen avulla.
    \alakohdatm{
        § $x^2+6x+9$
        § $y^2 - 2y+1$
        § $x^2 -25$
    }
    \begin{vastaus}
        \alakohdatm{
        § $(x+3)^2$
        § $(y-1)^2$
        § $(x-5)(x+5)$
        }
    \end{vastaus}
\end{tehtava}

\begin{tehtava}
    Sievennä.
    \alakohdat{
        § $\dfrac{3x-9}{3}$
        § $\dfrac{x^2-4x}{5x}$
        § $\dfrac{ab+a}{b^2+b}$
    }
    \begin{vastaus}
        \alakohdatm{
        § $x-3$
        § $\frac{x-4}{3}$
        § $\frac{a}{b}$
        }
    \end{vastaus}
\end{tehtava}

\begin{tehtava}
    Jaa tekijöihin ryhmittelemällä sopivasti.
    \alakohdat{
        § $x^3 +x^2 +x +1$
        § $a^3 +a^2b +2a +2b$
        § $8m^6-2m^4+4m^2-1$
    }
    \begin{vastaus}
        \alakohdat{
        § $(x^2+1)(x+1)$
        § $(a^2+2)(a+b)$
        § $(2m^4 +1)(4m^2 -1)=(2m^4 +1)(2m+1)(2m-1)$
        }
    \end{vastaus}
\end{tehtava}

\begin{tehtava}
	Ratkaise yhtälö jakamalla tekijöihin.
	\alakohdat{
		§ $x^2-16 = 0$
		§ $x^2+7x = 0$
		§ $x^2-6x+9 = 0$
	}
	\begin{vastaus}
		\alakohdatm{
			§ $x=4$ tai $x=-4$
			§ $x=0$ tai $x=-7$
			§ $x=3$
		}
	\end{vastaus}
\end{tehtava}


\subsubsection*{Hallitse kokonaisuus}

\begin{tehtava}
    Sievennä.
    \alakohdat{
        § $\dfrac{x^3-2x^2}{2-x}$
        § $\dfrac{x^2+6x+9}{x^2+3x}$
        § $\dfrac{4-x^2}{x^2-2x}$
    }
    \begin{vastaus}
        \alakohdatm{
        § $-x^2$
        § $\frac{x+3}{x}$
        § $-\frac{x+2}{x}$
        }
    \end{vastaus}
\end{tehtava}

\begin{tehtava}
	Jaa tekijöihin.
	\alakohdat{
		§ $x^3-x$
		§ $5ab+ b+10a+2$
		§ $16x^2y^2+8xy+1$
	}
	\begin{vastaus}
		\alakohdat{
			§ $x(x+1)(x-1)$
			§ $(5a+1)(b+2)$
			§ $(4xy+1)^2$
		}
	\end{vastaus}
\end{tehtava}

\begin{tehtava} 
Jaa tekijöihin $(3x^2-7y^2+5)^2-(x^2-9y^2-5)^2$.
    \begin{vastaus}
		$8(x-2y)(x+2y)(x^2+y^2+7)$. (Opastus: Älä kerro aluksi sulkuja auki vaan käytä heti muistikaavaa.)
    \end{vastaus}
\end{tehtava}

\subsubsection*{Lisää tehtäviä}

\begin{tehtava}
    Jaa tekijöihin.
    \alakohdatm{
    	§ $x^2 -4$
    	§ $x^2 -3$
    	§ $5x^2 -3$
		§ $16-x^4$
    }
    \begin{vastaus}
        \alakohdat{
            § $(x+2)(x-2)$
            § $(x+\sqrt{3})(x-\sqrt{3})$
            § $(\sqrt{5}x+\sqrt{3})(\sqrt{5}x-\sqrt{3})$
            § $(4+x^2)(4-x^2)=(4+x^2)(2-x)(2+x)$
        }
    \end{vastaus}
\end{tehtava}

\begin{tehtava}
    Jaa tekijöihin.
    \alakohdat{
        § $-15x^5 +10y$
        § $x^3y^2 +x^2y^3$
        § $-4a^3 -2a^2 +2ab$
    }
    \begin{vastaus}
        \alakohdat{
        § joko $5(-3x^5 +2y)$ tai $-5(3x^5 -2y)$
        § $x^2y^2(x+y)$
        § joko $2a(-2a^2 -a +b)$ tai $-2a(2a^2 +a -b)$
        }
    \end{vastaus}
\end{tehtava}

\begin{tehtava}
	Ratkaise yhtälöt.
	\alakohdat{
		§ $-x^4+4x^2=0$
		§ $x^5-16x^3=0$
	}
	\begin{vastaus}
		\alakohdat{
			§ $x=-2$, $x=0$ tai $x=2$ (Tekijöihin jakamalla yhtälö sievenee muotoon $x^2(2+x)(2-x)=0$.)
			§ $x=-4$, $x=0$ tai $x=4$ (Tekijöihin jakamalla yhtälö sievenee muotoon $x^3(x+4)(x-4)=0$.)
		}
	\end{vastaus}
\end{tehtava}

\end{tehtavasivu}