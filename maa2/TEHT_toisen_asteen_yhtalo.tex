\begin{tehtavasivu}

\subsubsection*{Opi perusteet}

\begin{tehtava}
    Ratkaise yhtälöt.
    \alakohdatm{
        § $x^2 = 16$
        § $x^2 = - 16$
        § $x^2 - 13 = 0$
        § $3x^2 - 12 = 0$

    }
    \begin{vastaus}
        \alakohdatm{
            § $x=\pm 4$
            § Ei reaaliratkaisuja
            § $x = \pm \sqrt{13}$
            § $x=\pm 2$
        }
    \end{vastaus}
\end{tehtava}

\begin{tehtava}
    Ratkaise yhtälöt.
    \alakohdat{
        § $x(x-3)= 0$
        § $x^2 + 4x = 0$
        § $7x^2-3x = 0$
    }
    \begin{vastaus}
        \alakohdatm{
            § $x=0$ tai $x=3$
            § $x =0$ tai $x=-4$
            § $x=0$ tai $x=\frac{3}{7}$
        }
    \end{vastaus}
\end{tehtava}

\begin{tehtava}
    Kirjoita neliöksi tunnistamalla muistikaava
    \alakohdatm{
        § $x^2 +2x +1$
        § $x^2 +6x +9$
        § $x^2 -4x +4$
    }
    \begin{vastaus}
        \alakohdatm{
            § $(x+1)^2$
            § $(x+3)^2$
            § $(x-2)^2$
        }
    \end{vastaus}
\end{tehtava}

\begin{tehtava}
    Ratkaise yhtälö täydentämällä neliöksi
    \alakohdat{
        § $x^2 -2x +1 = 4$
        § $x^2 +4x = 5$
        § $x^2 -3x + 10 = 0$
    }
    \begin{vastaus}
        \alakohdat{
            § $x = 3$ tai $x= -1$. Neliöksi täydennettynä $(x-1)^2=4$
            § $x = -5$ tai $x = 1$. Neliöksi täydennettynä $(x+2)^2=9$
            § Ei ratkaisua. Neliöksi täydennettynä $(x-3)^2=-1$
        }
    \end{vastaus}
\end{tehtava}

\subsubsection*{Hallitse kokonaisuus}

\begin{tehtava}
    Ratkaise seuraavat yhtälöt.
    \alakohdatm{
        § $x^2 - 100 = 0$
        § $x^2 + 100 = 0$
       § $x^2 - 10 = 0$
%        § $x^2 + 10 = 0$
        § $-x^2 - 25 = 0$
%        § $-x^2 + 25 = 0$
        § $2x^2 - 98 = 0$
%        § $2x^2 + 98 = 0$
    }
    \begin{vastaus}
        \alakohdatm{
            § $x=\pm10$
            § Ei ratkaisuja.
            § $x=\pm\sqrt{10}$
%            § Ei ratkaisuja.
            § Ei ratkaisuja.
%            § $x=\pm5$
            § $x=\pm7$
%            § Ei ratkaisuja.
        }
    \end{vastaus}
\end{tehtava}

\begin{tehtava}
    Ratkaise seuraavat yhtälöt.
    \alakohdatm{
        § $x^2 - 72x = 0$
%        § $x^2 + 72x = 0$
%        § $x^2 - 56x = 0$
        § $x^2 + 56x = 0$
        § $-x^2 - 13x = 0$
%        § $-x^2 + 13x = 0$
%        § $2x^2 - 43x = 0$
        § $2x^2 + 43x = 0$
    }
    \begin{vastaus}
        \alakohdatm{
            § $x=0$ tai $x=72$
%            § $x=-72$ tai $x=0$
%            § $x=0$ tai $x=56$
            § $x=-56$ tai $x=0$
            § $x=-13$ tai $x=0$
%            § $x=0$ tai $x=13$
%            § $x=0$ tai $x=21,5$
            § $x=-21,5$ tai $x=0$
        }
    \end{vastaus}
\end{tehtava}

\begin{tehtava}
    Ratkaise seuraavat yhtälöt.
    \alakohdatm{
        § $x^2 - 36 = 0$
        § $x^2 - 85x = 0$
        § $x^2 + 11x = -6x$
        § $x^2 + 10x = -4x^2$
    }
    \begin{vastaus}
        \alakohdatm{
            § $x=\pm6$
            § $x=0$ tai $x=85$
            § $x=0$ tai $x=-17$
            § $x=0$ tai $x=-2$
        }
    \end{vastaus}
\end{tehtava}

\begin{tehtava}
    Ratkaise yhtälö täydentämällä neliöksi.
    \alakohdat{
        § $x^2 -6x +9 = 1$
        § $x^2 +2x +4 = 0$
        § $x^2 -4x - 7 = 0$
        § $x^2 +x = \frac{3}{4}$
        § $2x^2 +3x -2 = 0$
    }
    \begin{vastaus}
        \alakohdat{
            § $x = 4$ tai $x= 2$. Neliöksi täydennettynä $(x-3)^2=1$.
            § Ei ratkaisua. Neliöksi täydennettynä $(x+1)^2=-3$
            § $x = 2 \pm \sqrt{11}$ Neliöksi täydennettynä $(x-2)^2=11$.
            § $x = \frac{1}{2}$ tai $x=-1\frac{1}{2}$ Neliöksi täydennettynä $(x+\frac{1}{2})^2=1$.
            § $x = -2$ tai $x = \frac{1}{2}$.  Neliöksi täydennettynä $(x+\frac{3}{4})^2=\frac{25}{16}$
        }
    \end{vastaus}
\end{tehtava}

\begin{tehtava}
Ollessaan leirikoulussa Lapissa lukiolaisryhmä saapuu järvelle ja havaitsee, että järven halkaisijan suuntainen maiseman poikkileikkaus on likimain paraabelin $\frac{1}{2\,500}x^2-\frac{1}{5}x$ muotoinen, jos $x$-akseli on on vedenpinnan taso ja yksikkönä on metri. Kuinka pitkä matka vastarannalle on?
\begin{vastaus}
$500$ metriä
\end{vastaus}
\end{tehtava}

\subsubsection*{Lisää tehtäviä}

\begin{tehtava}
    Ratkaise yhtälöt käyttämällä tulon nollasääntöä.
    \alakohdat{
        § $(x^2-1)(x-7)=0$
        § $(x^2-9)(x^2-16)=0$
        § $(x-4)=x(x-4)$
    }
    \begin{vastaus}
        \alakohdat{
            § $x=-1$, $x=1$ tai $x=7$
            § $x=-4$, $x=-3$, $x=3$ tai $x=4$
            § $x=1$ tai $x=4$
        }
    \end{vastaus}
\end{tehtava}

\begin{tehtava}
    Ratkaise seuraavat yhtälöt.
    \alakohdat{
        § $x^2 - 9 = 0$
        § $2x^2 + 8 = 0$
        § $-x^2 + 11 = -5$
        § $3 - x^2 = -1 + 3x^2$
    }
    \begin{vastaus}
        \alakohdatm{
            § $x=\pm3$
            § Ei ratkaisuja
            § $x=\pm4$
            § $x=\pm1$
        }
    \end{vastaus}
\end{tehtava}

\begin{tehtava}
    Ratkaise yhtälöt.
    \alakohdat{
        § $x^2(4x^2-1)^2 = 0 $
        § $-x^4(3x-1)^2 = 0$
    }
    \begin{vastaus}
        \alakohdat{
            § $x=0$, $x= \frac{1}{2}$ tai $x= -\frac{1}{2}$
            § $x=0$ tai $x= \frac{1}{3}$
        }
    \end{vastaus}
\end{tehtava}

\begin{tehtava}
    Ratkaise seuraavat yhtälöt.
    \alakohdat{
        § $x^2 - 3x = 0$
        § $10x + 2x^2 = 0$
        § $2x^2 - x^3 = 0$
        § $-3x^2 + 8x = -2x$
    }
    \begin{vastaus}
        \alakohdatm{
            § $x=0$ tai $x=3$
            § $x=0$ tai $x=-5$
            § $x=0$ tai $x=2$
            § $x=0$ tai $x=\frac{10}{3}$
        }
    \end{vastaus}
\end{tehtava}

\begin{tehtava}
    Elokuvassa \emph{Dredd} pudotetaan ihmisiä kuolemaan noin $1$ kilometrin korkeudesta. Ennen pudotusta heille annetaan huumausainetta, joka hidastaa aikakäsityksen $1$ prosenttiin normaalista. Vapaassa pudotuksessa pudottu matka ajanhetkellä $t$ on $\frac{1}{2} gt^2$, jossa $g$ on putoamiskiihtyvyytenä tunnettu vakio, jolle voimme tässä hyvin käyttää arviota $g \approx 10\frac{\text{m}}{\text{s}^2}$.
    \alakohdat{
    § Olettaen, että huumausaineen vaikutus kestää koko putoamisen ajan, kuinka pitkältä aika pudotuksesta kuolemaan \textbf{uhrista} tuntuu? (Oleta annetut arvot tarkoiksi ja muodosta relevantti toisen asteen yhtälö.)
    § $\star$ Mikä menee fataalisti pieleen, jos a-kohdan laskee suoraan kuvatulla tavalla?
    }
    \begin{vastaus}
        \alakohdat{
            § Vastaukseksi saadaan $1\,414\,\text{s} = 23\,\text{min}\,34\,\text{s}$. Käytännössä hyvä vastaustarkkuus voisi olla esimerkiksi $25\,\text{min}$.
            § Tehtävä ei huomioi ilmanvastusta. Ihminen saavuttaa korkeimmillaan rajanopeuden $v_\textrm{raja} \approx 55\frac{\text{m}}{\text{s}}$. Tehtävän mallissa putoavan ihmisen nopeus nousee $v_\textrm{max} \approx 141\frac{\text{m}}{\text{s}}$ asti. Todellisuudessa putoaminen siis kestää vieläkin kauemmin.
        }
    \end{vastaus}
\end{tehtava}

\begin{tehtava}
$\star$ Toisen asteen yhtälön vakiotermi on $4$ ja sen ratkaisut ovat $2$ ja $3$. Mikä yhtälö on kyseessä?
    \begin{vastaus}
		Olkoon yhtälö muotoa $ax^2+bx+4=0$. \\      
      Muodostetaan yhtälöpari:
      \[
        \left\{
          \begin{aligned}
            a\cdot 2^2 + b\cdot 2 + 4 &= 0 \\
            a\cdot 3^2 + b\cdot 3 + 4 &= 0
          \end{aligned}
        \right.
      \]
      
      Yhtälöparin ratkaisuna saadaan $a=\frac23$ ja $b=-3\frac13$. Yhtälö on siis $\frac{2}{3}x^2-3\frac{1}{3}x+4=0$. Vastauksen voi saada myös ilmaisemalla polynomiyhtälön tekijämuodossa $a(x-2)(x-3)=0$. Tässä tarvitaan kuitenkin juurten ja tekijöiden välistä yhteyttä, joka opetetaan vasta myöhemmin tässä kirjassa.
    \end{vastaus}
\end{tehtava}

\begin{tehtava}
%täydentyy kahdeksi neliöksi, joiden summa on 0
    $\star$ Ratkaise $x$ ja $y$ yhtälöstä $y^2+2xy+x^4-3x^2+4=0$.
    \begin{vastaus}
        $x=\sqrt{2}, y=-\sqrt{2}$ tai $x=-\sqrt{2}, y=\sqrt{2}$
    \end{vastaus}
\end{tehtava}

\begin{tehtava} % Kaunis
    $\star$ Ratkaise $x$ ja $y$ yhtälöstä $2x^4+2y^4=4xy-1$. %lisää tai vähennä kiva termi puolittain
    \begin{vastaus}
        $x=\frac{\sqrt{2}}{2}, y=\frac{\sqrt{2}}{2}$ tai $x=-\frac{\sqrt{2}}{2}, y=-\frac{\sqrt{2}}{2}$
    \end{vastaus}
\end{tehtava}

\end{tehtavasivu}
