\qrlinkki{http://opetus.tv/maa/maa2/toisen-asteen-yhtalon-ratkaisukaava/}{Opetus.tv: \emph{toisen asteen ratkaisukaava} ($9.03$, $11.06$ ja $10.09$)}

Edellisessä kappaleessa opittiin, että toisen asteen yhtälö voidaan aina ratkaista täydentämällä se neliöksi.
Neliöksi täydentämistä käytetään kuitenkin harvoin, sillä saman ajatuksen voi ilmaista valmiina kaavana.
Johdetaan seuraavassa toisen asteen yhtälön ratkaisukaava.

Lähdetään liikkeelle täydellisestä toisen asteen yhtälöstä $ax^2+bx+c=0$. Kuten edellisen kappaleen esimerkeissä, haluamme muokata sen muotoon, josta se on helppo täydentää binomin neliöksi. Huomataan binomin neliön muistikaavan sovelluksena, että $(\alpha x + \beta)^2 = \alpha^2 x^2 + 2\alpha\beta x +\beta^2$; verrataan tätä toisen asteen yhtälöön:
\begin{align*}
ax^2+bx+c &= 0 \\
\alpha^2 x^2 + 2\alpha\beta x +\beta^2 &= \text{?}
\end{align*}
(Tässä käytetään kreikkalaisia kirjaimia sekaannuksen välttämiseksi, sillä $a$ muistikaavassa $(a + b)^2$ ei välttämättä ole sama kuin toisen asteen yhtälön kerroin $a$.)

Nähdään että yhtälöt muistuttavat hieman toisiaan. Haluamme kuitenkin toisen asteen yhtälön vasemman puolen samanlaiseen muotoon kuin alemman:
\begin{align*}
ax^2+bx+c&=0 &&\textnormal{\footnotesize{vähennetään puolittain}} \ c \\
ax^2+bx &= -c &&\textnormal{\footnotesize{kerrotaan puolittain termillä}} \ 4a\\
4a^2 x^2 + 4abx &= -4ac \\
(2a)^2 x^2 +4abx &= -4ac 
\end{align*}
%\begin{align*}
%ax^2+bx+c&=0 &&\textnormal{\footnotesize{kerrotaan molemmat puolet termillä}} \ 4a \\
%4a \cdot ax^2+4a \cdot bx + 4a \cdot c&=0 \\
%4a^2x^2+4abx+4ac&=0 &&\textnormal{\footnotesize{vähennetään puolittain termi}} \ 4ac  \\
%4a^2x^2+4abx&=-4ac
%\end{align*}
Nyt voidaan täydentää vasen puoli neliöksi, mistä saadaan esimerkkien tapaan ratkaisu $x$:n suhteen.
\begin{align*}
(2a)^2 x^2 +4abx &=-4ac &&\textnormal{\footnotesize{lisätään puolittain termi}} \ b^2 \\
(2a)^2 x^2 +2\cdot 2a\cdot bx +b^2&=b^2-4ac &&\textnormal{\footnotesize{neliö:}} \ (2a)^2 x^2 +2\cdot 2a\cdot bx+b^2=(2ax+b)^2 \\
(2ax+b)^2&=b^2-4ac &&\textnormal{\footnotesize{otetaan puolittain neliöjuuri, jos}} \ b^2 -4ac \geq 0 \\
2ax+b&= \pm \sqrt[]{b^2-4ac} &&\textnormal{\footnotesize{vähennetään puolittain termi}} \ b \\
2ax&=-b \pm \sqrt[]{b^2-4ac} &&\textnormal{\footnotesize{jaetaan puolittain termillä}} \ 2a \neq 0 \\
x&= \frac{-b \pm \sqrt[]{b^2-4ac}}{2a}
\end{align*}
Toisen asteen yhtälön ratkaisukaava on siis \[x= \frac{-b \pm \sqrt[]{b^2-4ac}}{2a}\] oletuksella, että $b^2-4ac \geq 0$. Oletus tarvitaan, koska negatiivisille luvuille ei ole reaaliluvuilla määriteltyä neliöjuurta.\\
\laatikko[Toisen asteen yhtälön ratkaisukaava]{
Yhtälön
$ax^2+bx+c=0$, missä $a \neq 0$ ja $b^2 - 4ac \geq 0$, ratkaisut ovat
muotoa \\
\[ x=\frac{-b \pm \sqrt{b^2-4ac}}{2a}.\]
}
\begin{esimerkki}
Ratkaise yhtälö $x^2-8x+16=0$.
	\begin{esimratk}
Yhtälö on muodossa $ax^2+bx+c=0$:
\begin{align*}
\underbrace{1}_{=a}x^2 +\underbrace{(-8)}_{=b}x+\underbrace{16}_{=c}=0
\end{align*}
Sijoitetaan vakioiden $a=1$, $b=-8$ ja $c=16$ arvot toisen asteen yhtälön
ratkaisukaavaan.
\begin{align*}
x&=\frac{-(-8)\pm \sqrt[]{(-8)^2-4\cdot 1 \cdot 16}}{2 \cdot 1} \\
x&=\frac{8 \pm \sqrt{64- 64}}{2} \\
x&=\frac{8 \pm 0}{2} \\
x&=4
\end{align*}
	\end{esimratk}
\begin{esimvast}
$x=4$
\end{esimvast}
\end{esimerkki}

\begin{esimerkki}
Ratkaise yhtälö $15x^2+24x=-10$.
\begin{esimratk}
Siirretään kaikki termit samalle puolelle yhtälöä:
\begin{align*}
\underbrace{15}_{=a}x^2+\underbrace{24}_{=b}x+\underbrace{10}_{=c}=0
\end{align*}
Sijoitetaan  $a=15$, $b=24$ ja $c=10$ toisen asteen yhtälön ratkaisukaavaan.
\begin{align*}
x&=\frac{-24 \pm \sqrt[]{24^2-4 \cdot 15 \cdot 10}}{2 \cdot 15} \\
x&=\frac{-24 \pm \sqrt[]{576-600}}{30} \\
x&=\frac{-24 \pm \sqrt[]{-24}}{30}
\end{align*}
Koska juurrettava on negatiivinen ($-24<0$), niin yhtälöllä ei ole reaalilukuratkaisuja.
\end{esimratk}
\begin{esimvast}
Ei ratkaisua.
\end{esimvast}
\end{esimerkki}

\begin{esimerkki}
Ratkaise yhtälö $x^2+2x-3=0$.
\begin{esimratk}
Sijoitetaan  $a=1$, $b=2$ ja $c=-3$ ratkaisukaavaan.
\begin{align*}
x&=\frac{-2 \pm \sqrt[]{2^2-4 \cdot 1 \cdot (-3)}}{2 \cdot 1} \\
x&=\frac{-2 \pm \sqrt[]{4+12}}{2} \\
x&=\frac{-2 \pm \sqrt[]{16}}{2} \\
x&=\frac{-2 \pm 4}{2} \\
x&=-1 \pm 2 \\
x&= -1+2 \text{ tai } x = -1-2 \\
x&=1 \text{ tai } x=-3 
\end{align*}
\end{esimratk}
\begin{esimvast}
$x=1$ tai $x=-3$.
\end{esimvast}
\end{esimerkki}

\begin{esimerkki}
Kahden luvun erotus on $7$ ja tulo $330$. Mitkä luvut ovat kyseessä?
\begin{esimratk}
Olkoon pienempi luvuista $x$, jolloin suurempi on $x+7$. Saadaan yhtälö
\begin{align*}
x(x+7) &= 330 && \ppalkki -330 \\
x^2+7x-330 &= 0 && \ppalkki \text{ 2. asteen yhtälön ratkaisukaava}\\
x&=\frac{-7 \pm \sqrt{7^2-4\cdot 1 \cdot 330}}{2\cdot 1} \\
x&=\frac{-7 \pm \sqrt{1\,369}}{2}  &&	\ppalkki \sqrt{1\,369} = 37\\
x&=\frac{-7 \pm 37}{2}  \\
x&=\frac{-7 + 37}{2}\  \text{ tai } \  x = \frac{-7 - 37}{2}  \\
x&=15\  \text{ tai } \ x = -21
\end{align*}
Koska $x$ oli pienempi alkuperäisistä luvuista, suurempi on $15+7=22$ tai
$-21+7=-14$.
\end{esimratk}
\begin{esimvast}
Luvut ovat $15$ ja $22$ tai $-21$ ja $-14$.
\end{esimvast}
\end{esimerkki}

%\begin{tehtava}
%Ratkaise $y$ yhtälöstä $y^{2n}-y^n=2$.
%	\begin{vastaus}
%$y=(-1)^{\frac{1}{n}}$ (Tämä ei ole välttämättä reaalinen -- riippuu $n$:n arvosta.) tai $y=2^{\frac{1}{n}}$
%	\end{vastaus}
%\end{tehtava}

%\begin{esimerkki}
%Ratkaistaan yhtälö $-\sqrt{2}x+\frac{1}{2}=x^2$.
%\end{esimerkki}

%Yleinen toisen asteen yhtälö on muotoa $ax^2+bx+c=0$.
%Kerrotaan yhtälön molemmat puolet vakiolla $4a$: $4a^2x^2+4abx+4ac=0$.
%Siirretään termi $4ac$ toiselle puolelle: $4a^2x^2+4abx=-4ac$.
%Pyritään täydentämään vasen puoli neliöksi.
%Lisätään puolittain termi $b^2$: $4a^2x^2+4abx+b^2=b^2-4ac$.
%Havaitaan vasemmalla puolella neliö: $(2ax+b)^2=b^2-4ac$.
%Otetaan puolittain neliöjuuri: $2ax+b=\pm\sqrt{b^2-4ac}$.
%Vähennetään puolittain termi $b$: $2ax=-b\pm\sqrt{b^2-4ac}$.
%Jaetaan puolittain vakiolla $2a$: $x=\frac{-b\pm\sqrt{b^2-4ac}}{2a}$.