\begin{tehtavasivu}

\subsubsection*{Opi perusteet}

\begin{tehtava}
    Ratkaise yhtälöistä reaaliluku $x$.
    \alakohdat{
        § $x^2 + 6x + 5 = 0$
        § $x^2 - 2x - 3 = 0$
        § $2x^2+3x+5= 0$
		§ $9x^2 - 15x + 6 = 0$
	    § $9x^2 - 12x + 4 = 0$
    }
    \begin{vastaus}
        \alakohdatm{
            § $x = -5$ tai $x = -1$
            § $x = 3$ tai $x = -1$
            § Ei reaaliratkaisuja
			§ $x = 1$ tai $x = \frac{2}{3}$
            § $x = \dfrac{2}{3}$
        }
    \end{vastaus}
\end{tehtava}

\begin{tehtava} %järjestele paremmin (ensin kokonaisluvut, sitten rationaalilukuja, sitten ei-normaalimuodossa olevia, sitten irrationaalilukuja ja potensseja
    Ratkaise
    \alakohdat{
        § $2x+x^2 = -4$
        § $x^2+3x=5$
        § $3x^2 - 13x + 50 = -2x^2 + 17x + 5$
        § $6x^2 + 1 = -18x$.
    }
    \begin{vastaus}
        \alakohdatm{
            § Ei reaalijuuria
            § $x = \frac{-3 \pm \sqrt{29}}{2}$
            § $x = 3$
            § $x = \frac{-9 \pm 5\sqrt{3}}{6}$
        }
    \end{vastaus}
\end{tehtava}

\begin{tehtava}
    Ratkaise.
    \alakohdat{
		§ $-\frac{5}{7} x^2 + \frac{4}{11} x - \frac{1}{2} = 0$
		§ $\frac{2}{3} x^2 - \frac{18}{5} x + \frac{3}{10} = 0$
	}
    \begin{vastaus}
        \alakohdat{
			§ Ei ratkaisuja.
			§ $\frac{27 \pm \sqrt{684}}{10} = \frac{27 \pm 6 \sqrt{19}}{10}$
        }
    \end{vastaus}
\end{tehtava}

\begin{tehtava}
Muokkaa yhtälöt normaalimuotoisiksi toisen asteen yhtälöiksi ja merkitse tarvittaessa määrittelyehto.
    \alakohdat{
        § $x=\frac{1}{x}$
        § $\frac{1}{x+1}=x$
        § $\frac{y+1}{y^2}=2$
        § $ \frac{1}{t}=1-\frac{1}{t+1}$.
    }
    \begin{vastaus}
        \alakohdatm{
            § $x^2-1=0, x\neq 0$
            § $x^2+x-1=0, x \neq -1$
            § $2y^2-y-1=0, y \neq 0 $
            § $t^2-t-1=0, t\neq0,-1 $
        }
    \end{vastaus}
\end{tehtava}

\begin{tehtava}
Ratkaise $c$ yhtälöstä $c^2/(0,01-c)=1,8\cdot 10^{-5}$ sadastuhannesosien tarkkuudella, kun tiedetään, että $c$ on positiivinen.
	\begin{vastaus}
	$c\approx 0,00042$ ($c \approx -0,00043$ ei käy)
	\end{vastaus}
\end{tehtava}

%\begin{tehtava}
%    Suorakulmaisen muotoisen alueen piiri on $34$\,m ja pinta-ala $60$\,m$^2$. Selvitä alueen mitat.
%    \begin{vastaus}
%		Alueen toinen sivu on $5$\,m ja toinen $12$\,m.
%    \end{vastaus}
%\end{tehtava}

\subsubsection*{Hallitse kokonaisuus}

\begin{tehtava}
    Kahden luvun summa on $8$ ja tulo $15$. Määritä luvut.
    \begin{vastaus}
		Luvut ovat $3$ ja $5$.
    \end{vastaus}
\end{tehtava}

\begin{tehtava}
    Suorakulmion muotoisen talon mitat ovat $6,0\,\text{m} \times 10,0$\,m. Talon kivijalan ympärille halutaan levittää tasalevyinen sorakerros. Kuinka leveä kerros saadaan, kun soraa riittää $20$\,m$^2$ alalle?
    \begin{vastaus}
		$58$\,cm:n levyinen
    \end{vastaus}
\end{tehtava}

\begin{tehtava}
    Määritellään reaalifunktio $f$ kaavalla $f(t)=\frac{1}{600}t^2-\frac{1}{100}t+\frac{3}{200}$. Määritä funktion $f$ ainut nollakohta. Kuinka monen prosentin suhteellinen virhe vastaukseen tulee, jos kuitataan toisen asteen termin kerroin $\frac{1}{600}$ merkityksettömän pienenä ja unohdetaan toisen asteen termi kokonaan?
    \begin{vastaus}
Funktiolla $f$ on nollakohta täsmälleen kohdassa $t=3$. Jos toisen asteen termi kuitataan merkityksettömänä, voidaan käsitellään uutta funktiota $g$, jonka arvot määritellään kaavalla $g(t)=-\frac{1}{100}t+\frac{3}{200}$. Tämän funktion ainut nollakohta on $t=\frac{3}{2}$. Absoluuttinen virhe on tällöin $3-\frac{3}{2}=\frac{3}{2}$ ja suhteellinen $\frac{\frac{3}{2}}{3}=\frac{1}{2}=50\,\%$.
    \end{vastaus}
\end{tehtava}

\begin{tehtava}
	Jalkapallo potkaistiin ilmaan tasaiselta kentältä. Sivusta katsottuna pallon lentorata oli paraabeli, jonka yhtälö oli muotoa
	$$y=-x^2+15x-36,$$
	missä $x$ on etäisyys metreinä kenttää pitkin mitattuna ja $y$ korkeus kentän pinnasta. Kuinka pitkan matkan pallon lensi?
	\begin{vastaus}
		Paraabelin nollakohdat ovat $x=3$ ja $x=12$, joten pallo lensi $12-3 = 9$ metriä. 
	\end{vastaus}
\end{tehtava}

\begin{tehtava}
Ratkaise yhtälö $(4t+1)x^2-8tx+(4t-1)=0$ vakion $t$ kaikilla reaaliarvoilla.
	\begin{vastaus}
		Kun $t=-\frac{1}{4}$, toisen asteen termin kerroin on $0$, ja ainoa ratkaisu on $x = 1$  \\
		Kun $t \neq \frac{1}{4}$ kyseessä on toisen asteen yhtälö ja ratkaisu on $x= 1$ tai $x=\frac{4t-1}{4t+1}$.
    \end{vastaus}
\end{tehtava}

\begin{tehtava}
	Viljami sijoittaa $1\,000$\,€ korkorahastoon, jossa korko lisätään pääomaan vuosittain. Rahasto perii aina koronmaksun yhteydessä $20$ euron vuosittaisen hoitomaksun, joka vähennetään summasta koronlisäyksen jälkeen. Viljami laskee, että hän saisi yhteensä $2,4\,\%$ lisäyksen pääomaansa kahden vuoden aikana.
        \alakohdat{
            § Mikä on rahaston korkoprosentti? Ilmoita tarkka arvo ja likiarvo mielekkäällä tarkkuudella.
            § Paljonko rahaa Viljamin pitäisi sijoittaa, että hänen sijoituksensa kasvaisi yhteensä $5,0\,\%$ kahden vuoden aikana?
        }
	\begin{vastaus}
	    \alakohdat{
		§
		Merkitään korkokerrointa $x$:llä.
		$$(1\,000x -20)x-20=1,024\cdot 1\,000$$
		$$x = \dfrac{1+\sqrt{10\,441}}{100}$$
		$$\approx 1,03181211580212$$
		Vastaus: $3,2\,\%$
		§
		Merkitään Viljamin sijoittamaa summaa $a$:lla.
		\begin{flalign*}
		(ax -20)x-20 &= 1,050a\\
		a(x^2 -1,050) &= 20x+20\\
		a &= \dfrac{20x-20}{x^2-1,050}\\
        &\approx 2\,776,41224
		\end{flalign*}
		Vastaus: $2\,800$\,€
	    }
	\end{vastaus}
\end{tehtava}

\begin{tehtava}
    Kun kappaleen kiihtyvyys $a$ on vakio, pätee $x = v_0t + \dfrac{1}{2}at^2$ ja $v = v_0 + at$, missä $x$ on kuljettu matka, $v_0$ alkunopeus, $v$ loppunopeus ja $t$ aika.
		\alakohdat{
            § Kivi heitetään suoraan alas Olympiastadionin tornista (korkeus $72$ metriä) nopeudella $3,0$\,m/s. Kuinka monen sekunnin kuluttua kivi osuu maahan? Putoamiskiihtyvyys on noin $10$\,m/$\text{s}^2$
            § Bussin nopeus on $20$\,m/s. Bussi pysähtyy jarruttamalla tasaisesti. Se pysähtyy $10$ sekunnissa. Laske jarrutusmatka.
        }
    \begin{vastaus}
        \alakohdat{
            § Jarrutusmatka on $100$ metriä.
            § Noin $3,5$ sekunnin kuluttua. (Todellisuudessa putoamisessa kestää kauemmin, sillä lasku ei huomioi ilmanvastusta.)
        }
    \end{vastaus}
\end{tehtava}

\begin{tehtava}
	Johda neliöksi täydentämällä ratkaisukaava yhtälölle
	\[ x^2 +px+q=0. \]
	Tarkista sijoittamalla tavanomaiseen ratkaisukaavaan.
	\begin{vastaus}
		$x=\frac{-p \pm \sqrt{p^2-4q}}{2}$.
	\end{vastaus}
\end{tehtava}

\begin{tehtava} % toisen asteen yhtälö
On olemassa viisi peräkkäistä positiivista kokonaislukua, joista kolmen ensimmäisen neliöiden summa on yhtä suuri kuin kahden jälkimmäisen neliöiden summa. Mitkä luvut ovat kyseessä?
    \begin{vastaus}
		$10^2+11^2+12^2 = 13^2 + 14^2$.
    	(Jos negatiivisetkin luvut sallittaisiin, $(-2)^2+(-1)^2+0^2 = 1^2 + 2^2$ kävisi myös.) Löytyykö vastaava $4 + 3$ luvun sarja? Entä pidempi?
    \end{vastaus}
\end{tehtava}

\begin{tehtava}
    Kultaisessa leikkauksessa jana on jaettu siten, että pidemmän osan suhde lyhyempään on sama kuin koko janan suhde pidempään osaan. Tämä suhde ei riipu koko janan pituudesta ja sitä merkitään yleensä kreikkalaisella aakkosella fii eli $\varphi$. Kultaista leikkausta on taiteessa kautta aikojen pidetty ''jumalallisena suhteena''. %varupphi?
		\alakohdat{
            § Laske kultaiseen leikkauksen suhteen $\varphi$ tarkka arvo ja likiarvo.
            § Napa jakaa ihmisvartalon pituussuunnassa suunnilleen kultaisen leikkauksen suhteessa. Millä korkeudella napa on $170$\,cm pitkällä ihmisellä?
        }
    \begin{vastaus}
        \alakohdat{
            § $ \varphi = \dfrac{\sqrt{5}+1}{2} \approx 1,618$
            § Noin $105$\,cm:n korkeudella.
        }
    \end{vastaus}
\end{tehtava}

\subsubsection*{Lisää tehtäviä}

\begin{tehtava}
    Ratkaise oheiset \textit{vaillinaiset} toiseen asteen yhtälöt ratkaisukaavan avulla.
    \alakohdat{
        § $x^2 + 23x = 0$
        § $4x^2 - 64 = 0$
    }
    \begin{vastaus}
        \alakohdat{       
            § $x = 0$ tai $x = -23$
            § $x = 4$ tai $x = -4$
        }
    \end{vastaus}
\end{tehtava}

\begin{tehtava}
    Ratkaise
    \alakohdat{
		§ $-x^2 + 4x + 7 = 0$
		§ $x^2 - 13x + 1 = 0$
		§ $4x^2 - 3x - 5 = 0$
		§ $\frac{5}{6} x^2 + \frac{4}{7} x - 1 = 0$.
    }
    \begin{vastaus}
        \alakohdatm{
			§ $x = 2\pm \sqrt{11}$
			§ $x = \frac{13\pm \sqrt{165}}{2}$
			§ $x = \frac{3\pm \sqrt{89}}{8}$
			§ $x = \frac{-12\pm \sqrt{1614}}{35}$
        }
    \end{vastaus}
\end{tehtava}

\begin{tehtava} %esimerkkitehtävä, malli sijoituksesta
	$\star$ Ratkaise yhtälö $(x^3-2)^2+x^3-2=2$.
	\begin{vastaus}
		Kirjoitetaan yhtälö muotoon $(x^3-2)^2+(x^3-2)-2=0$ ja sovelletaan ratkaisukaavaa niin, että ratkaistaan pelkän $x$ sijaan yhtälöstä lauseke $x^3-2$. Näin saadaan $x^3-2=1$ tai $x^3-2=-2$, joista voidaan edelleen ratkaista $x=\sqrt[3]{3}$ tai $x=0$.
	\end{vastaus}
\end{tehtava}

\begin{tehtava}
	$\star$ Ratkaise yhtälö $(x^2-2)^6=(x^2+4x+4)^3$.
	\begin{vastaus}
		$x=-1$, $x=0$ tai $x=\frac{1 \pm \sqrt{17}}{2}$
	\end{vastaus}
\end{tehtava}

\end{tehtavasivu}