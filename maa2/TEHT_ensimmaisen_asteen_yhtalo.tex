\begin{tehtavasivu}

\subsubsection*{Opi perusteet}

\begin{tehtava}
    Ratkaise yhtälöt.
    \alakohdat{
        § $x + 5 = 47$
        § $2x = 64$
        § $3x - 5 = 16$
    }
    \begin{vastaus}
        \alakohdat{
            § $x = 42$
            § $x = 32$
            § $x = 7$
        }
    \end{vastaus}
\end{tehtava}

\begin{tehtava}
    Ratkaise yhtälöt.
    \alakohdat{
        § $x + 8 = 2x - 1$
        § $2x + 4 = 60$
        § $3x - 5 = -x + 11$
    }
    \begin{vastaus}
        \alakohdat{
            § $x = 9$
            § $x = 28$
            § $x = 4$
        }
    \end{vastaus}
\end{tehtava}

\begin{tehtava}
    Antero pitää hauskaa keksimällä luvun mielikuvituksessaan. Hän kirjoittaa luvun mielikuvituspaperille vaaleanpunaisella mielikuvituskynällä. Tämän jälkeen hän kertoo luvun silmiensä lukumäärällä ja vähentää siitä varpaidensa lukumäärän. Antero on innoissaan, sillä hän saa tulokseksi luvun $8$, joka on hänen ikänsä vuosina. Selvitä yhtälönratkaisulla Anteron keksimä luku, kun oletetaan, että hänellä on yhtä monta silmää ja varvasta kuin hänen ikätovereillaan yleensä on.
    \begin{vastaus}
        $2x-10=8 \Leftrightarrow 2x=18$ \\ $\Leftrightarrow x=9$. Vastaus: Antero keksi luvun $9$.
    \end{vastaus}
\end{tehtava}

\begin{tehtava}
    Kun eräs luku kerrotaan kolmella, ja siihen sen jälkeen lisätään viisi, saadaan tulokseksi puolet alkuperäisestä luvusta. Kirjoita yhtälö ja selvitä kyseinen luku.
    \begin{vastaus}
        $3x+5=\frac12x$, $x=-2$
    \end{vastaus}
\end{tehtava}

%ei ole yhtälötehtävä, mutta mallinnusharjoituksena ok? 
\begin{tehtava}
    Muodosta tilannetta kuvaavat lausekkeet.
    \alakohdat{
        § Kuinka paljon maksaa hilavitkuttimen vuokraus $x$ tunniksi, kun vuokra on $42$\,€/tunti. Vuokraajan tulee myös ottaa pakollinen $25$ euron laitteistovakuutus.
        § Kuinka monta euroa saa $x$ dollarilla, kun $1$\,EUR vastaa $1,23$\,USD:a, ja halutaan vaihtaa dollareita euroiksi. Valuutanvaihtaja veloittaa lisäksi palvelumaksun $0,50$ euroa.
    }
    \begin{vastaus}
        \alakohdat{
            § $42x + 25$
            § $\frac{1}{1,23}x + 0,5$
        }
    \end{vastaus}
\end{tehtava}

\subsubsection*{Hallitse kokonaisuus}

\begin{tehtava}
    Ratkaise yhtälöt.
    \alakohdat{
        § $3(x+7)=7x$
        § $2(3x-1)=-7x $
        § $3-2x-(4-x)=2 $
    }
    \begin{vastaus}
        \alakohdat{
            § $x = \frac{7}{6} =1\frac{1}{6} $
            § $x = \frac{2}{13}$
            § $x = -3$
        }
    \end{vastaus}
\end{tehtava}

\begin{tehtava}
    Ratkaise yhtälöt.
    \alakohdat{
        § $-2\cdot\frac{x-5}{3}-\frac{5}{7}(1-x)=5x+3$
        § $\frac{4x-5}{3}-\frac{3}{2}(x-8)=-\frac{x+5}{6}$
        § $3(x-3)+x=4x-9$
    }
    \begin{vastaus}
        \alakohdat{
            § $x = -\frac{1}{13}$
            § ei ratkaisuja
            § yhtälö on toteutuu kaikilla reaaliluvuilla
        }
    \end{vastaus}
\end{tehtava}

\subsubsection*{Lisää tehtäviä}

\begin{tehtava}
   $\star$ Suorakulmaisen kolmion sivujen pituuden kateettien pituudet ovat $x+1$ ja $4$. Hypotenuusan pituus $x+3$. Mikä $x$ on?
    \begin{vastaus}
		$x=2$
    \end{vastaus}
\end{tehtava}

\begin{tehtava}
    $\star$ Määritä sekunnin tarkkuudella se ajanhetki, kun kellotaulun minuutti- ja tuntiviisarit ovat päällekkäin ensimmäisen kerran klo $12.00$:n jälkeen.
    \begin{vastaus}
		kello $13.05.27$
    \end{vastaus}
\end{tehtava}

\end{tehtavasivu}