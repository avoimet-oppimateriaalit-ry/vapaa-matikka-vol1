\subsubsection*{Polynomi}

\begin{tehtava} 
Laske.
		\alakohdat{
		§ $x+x+x$ 
		§ $x\cdot x \cdot x$
		§ $-5x^2-5x^2$
		§ $2x\cdot 7x$
		§ $\frac{1}{2}x^2\cdot(-6x^3)$
		}
    \begin{vastaus}
		\alakohdat{
		§ $3x$
		§ $x^3$
		§ $-10x^2$
		§ $14x^2$
		§ $-3x^5$						
		}
    \end{vastaus}
\end{tehtava}

\begin{tehtava} 
Laske.
		\alakohdat{
		§ $-2x^3+4x-x^2-x+5x^3$
		§ $x-3-(5-x)$
		§ $(x^2+5x+2)-(-x^2+3x-5)$
		§ $3x(x-7)$
		§ $-2x^2(3x^5-12x^2+1)$					
		}
    \begin{vastaus}
		\alakohdat{
		§ $3x^3-x^2+3x$
		§ $2x-8$
		§ $2x^2+2x+7$
		§ $3x^2-21x$
		§ $-6x^7+24x^4-2x^2$						
		}
    \end{vastaus}
\end{tehtava}

\begin{tehtava} 
Kerro sulut auki.
		\alakohdat{
		§ $(x-2)(x+4)$
		§ $(3a+b)(a-b)$
		§ $(a+3)^2$
		§ $(x-1)^2$
		§ $(4x+1)(4x-1)$						
		}
    \begin{vastaus}
		\alakohdat{
		§ $x^2+2x-8$
		§ $3a^2-2ab-b^2$
		§ $a^2+6a+9$
		§ $x^2-2x+1$
		§ $16x^2-1$						
		}
    \end{vastaus}
\end{tehtava}

\begin{tehtava} 
Jaa tekijöihin.
		\alakohdat{
		§ $4x^2+x$
		§ $5x^3y+10xy^2$
		§ $y^2-9$
		§ $x^2-4x+4$
		§ $x^4-5x^3+10x-50$
		}
    \begin{vastaus}
		\alakohdat{
		§ $x(4x+1)$
		§ $5xy(x^2+2y)$
		§ $(y+3)(y-3)$
		§ $(x-2)^2$
		§ $(x^3+10)(x-5)$						
		}
    \end{vastaus}
\end{tehtava}

\begin{tehtava}
    Jaa tekijöihin.
    \alakohdat{
    	§ $x^3 - x$
        § $x^2 - x + \frac{1}{4}$
        § $9-x^4$
    }
    \begin{vastaus}
        \alakohdat{
            § $x(x-1)^2$
            § $(x-\frac{1}{2})^2$
            § $(3+x^2)(3-x^2)$
        }
    \end{vastaus}
\end{tehtava}

\begin{tehtava}
Sievennä muistikaavan avulla
    \alakohdat{
            § $(x^2-1)^2$ 
	        § $(a^6+3b^3)^2$
            § $(-12-3x)(12-3x)$
            § $(x+\frac{1}{x})^2$
    }
    \begin{vastaus}
        \alakohdat{
            § $x^4-2x^2+1$ 
            § $a^{12}+6a^6b^3+9b^6$
            § $-(144-9x^2)=9x^2-144$
            § $x^2+2+\frac{1}{x^2}$
         }
    \end{vastaus}
\end{tehtava}

\begin{tehtava}
    Jaa tekijöihin muistikaavojen avulla.
    \alakohdat{
        § $x^2-4x+4$
        § $9y^2 + 6y+1$
        § $49-4x^2$
    }
    \begin{vastaus}
        \alakohdat{
        § $(x-2)^2$
        § $(3y+1)^2$
        § $(7-2x)(7+2x)$
        }
    \end{vastaus}
\end{tehtava}

\begin{tehtava}
	Sievennä. 
	\alakohdat{
		§ $(x-3)(2x^3-3x+4)$
		§ $(x^2+1)(x^3-2x-4)$
		§ $(x-1)(x^4+x^3+x^2+x+1)$
		§ $(\frac x5-\frac23)(x^2+x+1)$
	}
	\begin{vastaus}
		\alakohdat{
			§ $2x^4-6x^3-3x^2+13x-12$
			§ $x^5-x^3-4x^2-2x-4$
			§ $x^5-1$
			§ $\frac15x^3-\frac{7}{15}x^2-\frac{7}{15}x-\frac23$
		}
	\end{vastaus}
\end{tehtava}

\begin{tehtava}
	Ratkaise yhtälö jakamalla tekijöihin.
	\alakohdat{
		§ $x^3-x^2 = 0$
		§ $x^3+3x^2-4x-12 = 0$
		§ $x^2-4x+4 = 4$
	}
	\begin{vastaus}
		\alakohdat{
			§ $x=0$ tai $x=1$
			§ $x=-3$, $x=2$ tai $x=-2$
			§ $x=0$ tai $x=4$
		}
	\end{vastaus}
\end{tehtava}

\begin{tehtava} 
Ratkaise yhtälö $(x-3)(x+2)(x-1)=0$.
    \begin{vastaus}
		$x=3$, $x=-2$ tai $x=1$
    \end{vastaus}
\end{tehtava}

\begin{tehtava} 
Miksi polynomi $x^6+3x^2+5$ ei voi saada negatiivisia arvoja?
    \begin{vastaus}
		Koska $x^6\geq 0$ ja $x^2 \geq 0$. (Parilliset potenssit.)
    \end{vastaus}
\end{tehtava}

\begin{tehtava} 
		\alakohdat{
		§ Osoita oikeaksi kaavat $a^3-b^3=(a-b)(a^2+ab+b^2)$ ja $a^3+b^3=(a+b)(a^2-ab+b^2)$.
		§ Jaa $x^6-y^6$ neljään tekijään.						
		}
    \begin{vastaus}
		\alakohdat{
		§ Opastus: Kerro sulut auki.
		§ $(x-y)(x^2+xy+y^2)(x+y)(x^2-xy+y^2)$						
		}
    \end{vastaus}
\end{tehtava}

\begin{tehtava} 
Kuvassa on polynomin $P(x)$ kuvaaja.

\begin{kuvaajapohja}{0.8}{-3}{3}{-2}{4}
				\kuvaaja{x*(x-1)*(x+2)}{}{black}
\end{kuvaajapohja}

		\alakohdat{
		§ Mitkä ovat polynomin nollakohdat?
		§ Millä muuttujan $x$ arvoilla $P(x)>0$?
		§ Kuinka monta ratkaisua yhtälöllä $P(x)=1$ on?
		}
    \begin{vastaus}
		\alakohdat{
		§ $x=-2$, $x=0$ ja $x=1$
		§ $-2<x<0$ tai $1 < x$
		§ Vähintään $3$. (Kuvaajassa näkyvän alueen ulkopuolella voisi olla lisää.)
		}
    \end{vastaus}
\end{tehtava}

\begin{tehtava} 
Jaa tekijöihin $(a+b)^2-16$.
    \begin{vastaus}
		$(a+b+4)(a+b-4)$. (Opastus: Älä kerro aluksi sulkuja auki. $16=4^2$)
    \end{vastaus}
\end{tehtava}

\begin{tehtava} 
Sievennä:
		\alakohdat{
			§ $(a^2-1)^2+(a^2+1)^2-2(a^4+1)$
			§ $(x+y)^2-4xy$
		}
	\begin{vastaus}
		\alakohdat{
			§ $0$
			§ $(x-y)^2$
		}
    \end{vastaus}
\end{tehtava}

\subsubsection*{Ensimmäinen aste}

\begin{tehtava} 
Ratkaise yhtälöt.
		\alakohdat{
		§ $3x-5(x-2)=3-(-x)$
		§ $x(x-6) = x^2+5$
		§ $ \frac{2x}{3}-\frac{x-3}{4}=7$
		}
    \begin{vastaus}
		\alakohdat{
		§ $x = \frac{7}{3}$
		§ $x=-\frac{5}{6}$
		§ $15$			
		}
    \end{vastaus}
\end{tehtava}

\begin{tehtava} 
Kirjoita joukko-opilliset ehdot (kaksois)epäyhtälöinä.
		\alakohdat{
		§ $x \in [2,7]$
		§ $y \in ]-3,0]$
		§ $z \in ]-\infty, 5[$?
		}
    \begin{vastaus}
		\alakohdat{
		§ $2 \leq x \leq 7$
		§ $-3 < y \leq 0$
		§ $z < 5$			
		}
    \end{vastaus}
\end{tehtava}

\begin{tehtava} 
Ratkaise epäyhtälöt.
		\alakohdat{
		§ $-3x < 6$
		§ $5x-2 > 7x+3$
		§ $ -2(x+3) \leq x-(5-x)$
		}
    \begin{vastaus}
		\alakohdat{
		§ $ x > -2$	
		§ $x < -\frac{5}{2}$
		§ $x \geq -\frac{1}{4}$
		}
    \end{vastaus}
\end{tehtava}

\begin{tehtava}
    Ratkaise seuraavat yhtälöt tai epäyhtälöt.
    \alakohdat{
        § $-2r+6=0$
        § $-2r+6\leq 0$
        § $5y-2<y+6$
        § $8(x+2)\geq -5(5-x)+3$
        § $\frac{x+3}{2}+\frac{-2x+1}{3}>\frac{x-9}{4}$
    }
    \begin{vastaus}
        \alakohdat{
            § $r=3$
            § $r\geq 3$
            § $y<2$
            § $x=-12\frac{2}{3}$
            § $x<9\frac{4}{5}$
        }
    \end{vastaus}
\end{tehtava}

\begin{tehtava}
Ratkaise kaksoisepäyhtälö $2x-1 < x \leq 3+5x$.
    \begin{vastaus}
        $-\frac{3}{4} \leq x < 1$
    \end{vastaus}
\end{tehtava}

\begin{tehtava}
Polkupyörävuokraamo A laskuttaa pyörän vuokrasta $5$\,€ ja lisäksi $2,5$\,€ jokaisesta täydestä tunnista. Vuokraamo B laskuttaa $11,50$\,€ ja lisäksi $1,5$\,€ jokaisesta täydestä tunnista. Kuinka monen tunnin vuokrassa vuokraamo B on edullisempi?
    \begin{vastaus}
	$7$\,h:n tai sitä pidemmissä vuokrissa
    \end{vastaus}
\end{tehtava}

\begin{tehtava}
Kirahvi kantaa suussaan karahvia ja kävelee vaa'alle, jonka toisella puolella on $7$ karahvia ja sattuu niin, että vaaka on tasapainossa. Mikä on karahvin paino kirahveina?
	\begin{vastaus}
			Yksi karahvi painaa yhden kuudesosakirahvin.
	\end{vastaus}
\end{tehtava}
	
\begin{tehtava} 
Ratkaise epäyhtälö $ax \geq a-2x$ parametrin $a$ kaikilla arvoilla.
    \begin{vastaus}
        $x \geq \frac{a}{a+2}$, kun $a > -2$ \\
        $x \leq \frac{a}{a+2}$, kun $a < -2$ \\
    $x \in \mathbb{R}$, kun $a = -2$ \\
	\end{vastaus}
\end{tehtava}

\subsubsection*{Toinen aste}

\begin{tehtava} 
Ratkaise yhtälöt.
		\alakohdat{
		§ $x^2-13=0$ 
		§ $5x^+2x=0$  
		§ $2x^2+5x-3=0$
		}
    \begin{vastaus}
		\alakohdat{
		§ $x= \pm \sqrt{13}$
		§ $x=0$ tai $x=-\frac{2}{5}$
		§ $x=-3$ tai $x= \frac{1}{2}$			
		}
    \end{vastaus}
\end{tehtava}

\begin{tehtava}
    Ratkaise yhtälöt.
    \alakohdat{
        § $x^2+3x+2=0$
        § $2x^2+5x-12=0$
        § $3x^2-7x-20=0$
        § $x^2+3x-5=0$
        § $x^2+5x-24=0$
    }
    \begin{vastaus}
        \alakohdat{
            § $x=-2$ tai $x=-1$
            § $x=3/2$ tai $x=-4$
            § $x=4$ tai $x=-5/3$
            § $x=\frac{3\pm\sqrt{29}}{2}$
            § $x=3$ tai $x=-8$
        }
    \end{vastaus}
\end{tehtava}

\begin{tehtava}
    Ratkaise yhtälöt.
    \alakohdat{
        § $x^2+3x-5=4x+8$
        § $8x^2-5x+1=-36$
        § $-3x^2-4x+2=-5x^2+3$
        § $-3x^2+4x+13=-5x^2+10x+9$
    }
    \begin{vastaus}
        \alakohdat{
            § $\frac{1\pm\sqrt{53}}{2}$
            § Ei ratkaisua reaalilukujen joukossa.
            § $1\pm\frac{\sqrt{6}}{2}$
            § $x=1$ tai $x=2$
        }
    \end{vastaus}
\end{tehtava}

\begin{tehtava}
Ratkaise yhtälö $x - 3 = \frac{1}{x}$.
    \begin{vastaus}
    $x =\frac{3 \pm \sqrt{13}}{2}$
    \end{vastaus}
\end{tehtava}

\begin{tehtava} 
Ratkaise epäyhtälöt.
		\alakohdat{
		§ $x^2-x-6<0$ 
		§ $x^2 \geq 5x$  
		}
    \begin{vastaus}
		\alakohdat{
		§ $ -2 < x < 3 $
		§ $x \leq 0$ tai $x \geq 5$
	}
    \end{vastaus}
\end{tehtava}

\begin{tehtava} 
Maijan ja Veeran ikien summa on $30$. Ikien tulo on yli $186$. Minkä ikäisiä tytöt voivat olla?
    \begin{vastaus}
	Vähintään $9$, korkeintaan $21$. Ratkeaa epäyhtälöstä $x(30-x)>186$.
	\end{vastaus}
\end{tehtava}

\begin{tehtava} 
Millä vakion $k$ arvolla yhtälöllä $kx^2+kx=x-3$ on tasan yksi ratkaisu?
    \begin{vastaus}
		$k = 7 \pm 4 \sqrt{3}$. Diskriminantti on $D = (k-1)^2-4\cdot k \cdot 3$.
    \end{vastaus}
\end{tehtava}

\begin{tehtava} 
Polynomilla $P(x)=x^2-3x+c$ on tekijä $x+5$. Mikä on $c$?
    \begin{vastaus}
		$c=-40$
    \end{vastaus}
\end{tehtava}

\begin{tehtava} 
Suorakulmion $A$ sivut ovat $9$ ja $x^2+1$, suorakulmion $B$ sivut $5x+5$
ja $5-x$. Millä luvun $x$ arvoilla suorakulmion $A$ ala on suurempi?
    \begin{vastaus}
	$-1 < x < -\frac{4}{7}$ tai $2 < x < 5$. Huomioi, että suorakulmion $B$
    sivut ovat positiiviset vain, kun $-1<x<5$.
    \end{vastaus}
\end{tehtava}

\begin{tehtava} % toisen asteen yhtälö
Neliön muotoisen taulun sivu on $36$\,cm. Taululle tehdään tasalevyinen kehys, jonka nurkat on pyöristettu neljännesympyrän muotoisiksi. Kuinka leveä kehys on, kun sen pinta-ala on puolet taulun pinta-alasta? Ympyrän pinta-ala on $\pi r^2$.
    \begin{vastaus}
     $7,7$\,cm
    \end{vastaus}
\end{tehtava}

\subsubsection*{Korkeampi aste}

\begin{tehtava} 
Ratkaise yhtälö $2x^7=5x^2$.
    \begin{vastaus}
		$x=0$ tai $x=\sqrt[5]{\frac{5}{2}}$
    \end{vastaus}
\end{tehtava}

\begin{tehtava} 
Anna esimerkki (jos mahdollista)
		\alakohdat{
		§ $4$. asteen yhtälöstä, jolla ei ole ratkaisua 
		§ $5$. asteen yhtälöstä, jolla on tasan kaksi ratkaisua
		§ $3$. asteen yhtälöstä, jolla on tasan neljä ratkaisua
		}
    \begin{vastaus}
		\alakohdat{
		§ esimerkiksi $x^4=-1$ 
		§ esimerkiksi $x^4(x+1)=0$ 
		§ mahdotonta		
		}
    \end{vastaus}
\end{tehtava}

\begin{tehtava} 
Ratkaise yhtälö $x^4=x^2+6$.
    \begin{vastaus}
		$x=\pm \sqrt{3}$
    \end{vastaus}
\end{tehtava}

\begin{tehtava} 
Ratkaise yhtälö
$x^7=5x^5-x^6$.
     \begin{vastaus}
		$x=0$ tai $x\frac{-1 \pm \sqrt{21}}{2}$ 
    \end{vastaus}
\end{tehtava}

\begin{tehtava}
    Ratkaise yhtälöt.
    \alakohdat{
        § $x^4 + 7x^3 = 0$
        § $2x^3 - 16x^2 + 32x = 0$
        § $x^6 + 6x^5 = -9x^4$
        § $x^3 - 2x^5 = 0$
    }
    \begin{vastaus}
        \alakohdat{
        	§ $x = 0$ tai $x = -7$
        	§ $x = 0$ tai $x = 4$
        	§ $x = 0$ tai $x = -3$
            § $x = 0$ tai $x = \pm\dfrac{1}{\sqrt{2}}$
        }
    \end{vastaus}
\end{tehtava}

\begin{tehtava}
Mitkä luvut ovat kuutiotaan suurempia?
    \begin{vastaus}
	Luvut, jotka ovat pienempiä kuin $-1$ ja luvut välillä $]0,1[$.
    \end{vastaus}
\end{tehtava}

\begin{tehtava} 
Ratkaise epäyhtälöt
		\alakohdat{
		§ $x^4 < 5x $
		§ $2x^3 \leq 3x-5x^2$
		}
     \begin{vastaus}
		\alakohdat{
		§ $0<x<\sqrt[3]{5}$
		§ $x<-3$ tai $0<x<\frac{1}{2}$
		}
    \end{vastaus}
\end{tehtava}

\begin{tehtava} 
Kuution tilavuus (kuutiometreinä) on sama kuin sen särmien pituuksien summa (metreinä). Kuinka pitkä on kuution särmä?
    \begin{vastaus}
		$\sqrt{12}$\,m $\approx 3,46$\,m
    \end{vastaus}
\end{tehtava}

\begin{tehtava} 
Ratkaise yhtälö $2x^5-(x+6)^5=0$.
    \begin{vastaus}
	$x=\frac{6}{\sqrt[5]{2}-1}$
    \end{vastaus}
\end{tehtava}

\begin{tehtava}
Etsi yhtälön $x^5-5x^4+6x^3-1=0$ kaikkien kolmen ratkaisun likiarvot laskimella tai tietokoneen avulla. Anna vastaukset kahden desimaalin tarkkuudella.
    \begin{vastaus}
	$x \approx 0,69$, $x \approx 1,86$ tai $x \approx 3,03$.
    \end{vastaus}
\end{tehtava}
