Yhtälöistä yksinkertaisin on ensimmäisen asteen yhtälö. Käydään se lyhyesti läpi kertauksen vuoksi.

Ensimmäisen asteen yhtälössä ratkaistavana on vain mahdollisesti vakiolla kerrottu muuttuja. Yhtälön ratkaisemiseksi riittää neljää peruslaskutoimitusta: yhteen-, vähennys-, kerto ja jakolasku. Yhtälöä muokataan tekemällä yhtälön kummallekin puolelle sama laskutoimitus.

%Aluksi yhtälön molemmille puolille 
%lisätään tai vähennetään jokin luku, niin että 
%vasemmalle puolelle saadaan jäämään pelkkä vakiolla kerrottu muuttuja.
%Sen jälkeen jaetaan yhtälön molemmat puolet muuttujan kertoimella, jolloin
%yhtälön ratkaisu jää oikealle puolelle.

% fixme termiä juuri ei vielä esitellä

\begin{esimerkki}
Ratkaise yhtälö $4x + 5 = 2 + 2x$.
	\begin{esimratk}
\begin{align*}
    4x + 5 &= 2 + 2x && \ppalkki -2x \\
    2x + 5 &= 2      && \ppalkki -5 \\
        2x &= -3     && \ppalkki :2 \\
         x &= -\frac{3}{2}
 \end{align*}
	\end{esimratk}
	\begin{esimvast}
$x=-\frac{3}{2}$
	\end{esimvast}
\end{esimerkki}

\begin{esimerkki}
Ratkaise yhtälö $3x - 6 = 0$.
	\begin{esimratk}
  \begin{align*}
    3x - 6 &= 0 && \ppalkki +6 \\
        3x &= 6 && \ppalkki :3 \\
         x &= \frac{6}{3}  = 2 \\
  \end{align*}
	\end{esimratk}
	\begin{esimvast}
$x=2$
	\end{esimvast}
\end{esimerkki}

\begin{esimerkki}
Ratkaise yhtälö $\dfrac{x}{2}-\dfrac{x+3}{2}=3$.
	\begin{esimratk}
  \begin{align*}
    \frac{x}{2}-\frac{x+10}{2}&=5 && \ppalkki \cdot 2 \\
        x-(x+10) &= 10  \\
         x-x-10 &= 10 \\
         -10 &= 10 && \text{ ristiriita, ei ratkaisua.}
  \end{align*}
	\end{esimratk}
	\begin{esimvast}
ei ratkaisua
	\end{esimvast}
\end{esimerkki}

Yleisesti ensimmäisen asteen yhtälö on muotoa $ax + b = 0$. Kaikki $1$. asteen yhtälöt voidaan muokata tähän yleiseen muotoon siirtämällä kaikki termit yhtälön vasemmalle puolelle ja sieventämällä.

Yleisen yhtälön $ax + b = 0$ ratkaisu on

\begin{align*}
	 ax + b &= 0  && \ppalkki -b\\
	 ax &= -b  && \ppalkki : a \ (\neq 0)\\
  x &= -\frac{b}{a}.
\end{align*}

%Erityisesti kannattaa huomata, että kaikkia 1. asteen yhtälöitä ei tarvitse
%saattaa yleiseen muotoon. Esimerkiksi jos vakiotermit ovat valmiiksi
%oikealla puolella, yhtälön ratkaisemiseksi riittää luonnollisesti jakaa
%molemmat puolet muuttujan kertoimella.