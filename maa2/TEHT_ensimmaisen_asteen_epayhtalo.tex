\begin{tehtavasivu}

\subsubsection*{Opi perusteet}
%
%\begin{luoKuva}{teht1}
%lukusuora.pohja(-12, 12, 3, varaa_tila = False)
%lukusuora.kohta(0, "$0$")
%lukusuora.vali(-10, 10, False, False, "$-10$", "$10$")
%\end{luoKuva}
%
%\begin{luoKuva}{teht2}
%lukusuora.pohja(-11, 9, 3, varaa_tila = False)
%lukusuora.kohta(0, "$0$")
%lukusuora.vali(-9, 7, False, True, "$-9$", "$7$")
%\end{luoKuva}
%
%\begin{luoKuva}{teht3}
%lukusuora.pohja(-2, 12, 3, varaa_tila = False)
%lukusuora.kohta(0, "$0$")
%lukusuora.vali(5, None, True, False, "$5$", "$turha$")
%\end{luoKuva}
%
%\begin{luoKuva}{teht4}
%lukusuora.pohja(-1, 9, 3, varaa_tila = False)
%lukusuora.kohta(0, "$0$")
%lukusuora.vali(5, 7.5, True, True, "$5$", "$7,5$")
%\end{luoKuva}
%
%\begin{luoKuva}{teht5}
%lukusuora.pohja(-1, 7, 3, varaa_tila = False)
%lukusuora.kohta(0, "$0$")
%lukusuora.vali(1, 5, False, True, "$1$", "$5$")
%\end{luoKuva}

\begin{tehtava}
    Ratkaise seuraavat epäyhtälöt.
    \alakohdat{
        § $3x+6<4x$
        § $3x-6<2x+57$
        § $5y-2<12$
        § $3\leq y+9$
        § $z-5\geq-888$
		§ $2z+5\leq 42z-995$
    }
    \begin{vastaus}
        \alakohdatm{
            § $x>6$
            § $x<63$
            § $y<2,8$
            § $y\geq -6$
            § $z\geq -883$
			§ $z\geq 25$
        }
    \end{vastaus}
\end{tehtava}

\begin{tehtava}
Maalipurkki sisältää $10$ litraa maalia. Maalin riittoisuus on noin $6\,\text{m}^2/\text{l}$. Talon ulkoseinän korkeus on $4,5$\,m. Ulkoseinälle tulevat laudat on maalattava kahteen kertaan. Riittääkö maali, jos maalattavan seinän pituus on
	\alakohdat{
		§ $5$\,m
		§ $10$\,m
		§ Kuinka pitkälle seinälle yhden purkillisen sisältämä maali riittää?
	}
	\begin{vastaus}
		\alakohdat{
			§ riittää ($22,5~\text{m}^2 < 30~\text{m}^2$)
			§ ei riitä ($45~\text{m}^2 > 30~\text{m}^2$)
			§ noin $6,7$\,m seinälle
		}

	\end{vastaus}
\end{tehtava}

\begin{tehtava}
    Ratkaise epäyhtälöt.
    \alakohdat{
        § $3x+6<2x\leq 9-x$
        § $3x+6<2x\leq 1+3x$
    }
    \begin{vastaus}
        \alakohdatm{
            § $x<-6$
            § ei ratkaisua
        }
    \end{vastaus}
\end{tehtava}

\subsubsection*{Hallitse kokonaisuus}

\begin{tehtava}
	Millä $x$:n arvoilla luvut $2x - 5$, $-x$ ja $x + 4$ ovat erisuuria ja $2x - 5$ on luvuista
	\alakohdat{
		§ suurin
		§ toiseksi suurin
		§ pienin?
	}
	\begin{vastaus}
		\alakohdatm{
			§ $x > 9$
			§ $\frac{5}{3} < x < 9$
			§ $x < \frac{5}{3}$
		}
	\end{vastaus}
\end{tehtava}

\begin{tehtava}
	Tietyn auton käyttövoimavero on $450$\,€/vuosi, ja keskimääräinen kulutus dieselöljyä käytettäessä on $5$ litraa/$100$\,km. Saman valmistajan vastaava bensiinikäyttöinen auto kuluttaa $8$ litraa/$100$\,km. Diesel maksaa $1,55$\,€/litra, ja bensiini maksaa $1,65$\,€/litra. Kun vain annetut tiedot huomioidaan, niin kuinka paljon esimerkin dieselajoneuvolla tulee vähintään ajaa vuodessa, jotta se on edullisempi? Autojen ostohintoja ei huomioida.
    \begin{vastaus}
        $8\,257$\,km
    \end{vastaus}
\end{tehtava}

\subsubsection*{Lisää tehtäviä}

\begin{tehtava}
    Ratkaise seuraavat epäyhtälöt.
    \alakohdat{
        § $33x+2\geq 27x+6$
        § $3x-6\geq 4x-6$
        § $5y+5\geq 15$
        § $3y+2\geq 2y-1$
        § $z\geq 2z+1\,000$
		§ $z-1\geq z+1$
    }
    \begin{vastaus}
        \alakohdatm{
            § $x\geq \frac{2}{3}$
            § $x\leq 0$
            § $y\geq 2$
            § $y\geq -3$
            § $z\leq -1000$
			§ ei ratkaisuja
        }
    \end{vastaus}
\end{tehtava}

\begin{tehtava}
Lukion päättötodistuksessa aineen arvosana määräytyy aineen pakollisten ja syventävien kurssien keskiarvosta pyöristettynä kokonaisluvuksi tavallisten sääntöjen mukaan. Opiskelija haluaa filosofian päättöarvosanakseen $7$ tai paremman. Opiskelija aikoo osallistua kolmelle filosofian kurssille. Kahden kurssin jälkeen hänen arvosanojensa keskiarvo on $6$. Mikä arvosana on opiskelijan vähintään saatava kolmannesta kurssista? Kurssit arvioidaan asteikolla 
$4$--$10$.
\begin{vastaus}
Muodostettava epäyhtälö on muotoa $\frac{2\cdot 6+n}{3}\geq 6,5$, missä $n \in \lbrace 4,5,6,7,8,9,10 \rbrace$, josta ratkaisuna saadaan $n\geq 7,5$. Opiskelija tarvitsee siis vähintään arvosanan $8$.
\end{vastaus}
\end{tehtava}

\end{tehtavasivu}