\begin{tehtavasivu}

\subsubsection*{Opi perusteet}

\begin{tehtava}
    Sievennä.
    \alakohdat{
        § $3x+5x $
        § $4x^2+7x^2$
        § $-6y+2y $
        § $3x-(-2x)$
    }
    \begin{vastaus}
        \alakohdat{
            § $8x$
            § $11x^2$
            § $-4y$
            § $5x$
        }
    \end{vastaus}
\end{tehtava}

\begin{tehtava}
    Sievennä.
    \alakohdat{
    	§ $5x-2+2x+7$
        § $5x-3y-y-2x$
        § $2x^2+x+x^2-5x$
        § $y^3 - 2y^2+4y^3-y $
    }
    \begin{vastaus}
        \alakohdat{
        	§ $7x+5$
            § $3x-4y$
            § $3x^2-4x$
            § $5y^3-2y^2-y$
        }
    \end{vastaus}
\end{tehtava}

\begin{tehtava}
    Sievennä.
    \alakohdat{
        § $(x^2 - 2x + 1) + (-x^2 + x) $
        § $(3y^3 + 2y^2  + y) - (-y^2 + y)$
        § $(z^{10} - z^6 + z^2 + 1) + (z^{10} + 2z^8 - 3z^6)$
    }
    \begin{vastaus}
        \alakohdat{
            § $-x + 1$
            § $3y^3 + 3y^2$
            § $2z^{10} + 2z^8 - 4z^6 + z^2 + 1$
        }
    \end{vastaus}
\end{tehtava}

\begin{tehtava}
    Olkoot $P(x)=x^2+3x+4$ ja $Q(x)=x^3-10x+1$. Sievennä
    \alakohdat{
        § $P(x)+Q(x)$
        § $P(x)-Q(x)$
        § $Q(x)-P(x)$
        § $2P(3)+Q(2)$.
    }
    \begin{vastaus}
        \alakohdat{
            § $x^3+x^2-7x+5$ % x^2+3x+4 + x^3-10x+1
            § $-x^3+x^2+13x+3$ % x^2+3x+4 -(x^3-10x+1) = x^2+3x+4 -x^3+10x-1
            § $x^3-x^2-13x-3$ % 
            § $33$ % 2*(3^2+3*3+4) +  2^3-10*2+1 = 2*(9+9+4)+8-20+1 =44-11 =33
        }
    \end{vastaus}
\end{tehtava}

\begin{tehtava}
	Sievennä polynomifunktiot ja laske funktioiden arvot muuttujan arvoilla $1$, $-1$ ja $3$.
	\alakohdat{
		§ $P(x)=(x^3-4x+5)+(-x^3+x^2+4x-2)$
		§ $Q(x)=(2x^3+x^2-10x)-(3x^3-4x^2+5x)$
	}
	\begin{vastaus}
		\alakohdat{
			§ $P(x)=x^2+3$, $P(1)=4$, $P(-1)=4$ ja $P(3)=12$
			§ $Q(x)=-x^3+5x^2-15x$, $Q(1)=21$, $Q(-1)=-9$ ja $R(3)=-27$
		}
	\end{vastaus}
\end{tehtava}

\begin{tehtava}
     Sievennä
     \alakohdat{
         § $(2x + 3) + x $
         § $(3x - 1) + (-x + 1)$
         § $(5x + 10) + (6x - 6) - (x + 3)$
     }
     \begin{vastaus}
         \alakohdat{
             § $3x + 3$
             § $2x$
             § $10x + 1$
         }
     \end{vastaus}
 \end{tehtava}

\begin{tehtava}
	Sievennä polynomifunktiot ja laske funktioiden arvot muuttujan arvoilla $1$, $-1$ ja $3$.
	\alakohdat{
		§ $R(x)=4(2x-4)+(-x^3+1)$
		§ $S(x)=-(2x^2-x+8)+6(x^5-3x^2+1)$ \\ $-2(3x^5-10x^2)$
	}
	\begin{vastaus}
		\alakohdat{
			§ $R(x)=-x^3+8x-15$, $R(1)=-8$, $R(-1)=-22$ ja $R(3)=-18$
			§ $S(x)=-2$, $S(1)=-1$, $S(-1)=-3$ ja $S(3)=1$
		}
	\end{vastaus}
\end{tehtava}

\begin{tehtava}
    Sievennä.
    \alakohdat{
        § $x\cdot x^2$
        § $5x\cdot 3x$
        § $-2(-5x^3)$
        § $3x^2\cdot(-6x^4)$
    }
    \begin{vastaus}
        \alakohdat{
            § $x^3$
            § $15x^2$
            § $10x^3$
            § $-18x^6$
        }
    \end{vastaus}
\end{tehtava}

\begin{tehtava}
    Sievennä.
    \alakohdat{
        § $2(x+3)$
        § $x(x - 2)$
        § $3x(1-2x)$
        § $x^2(x + 5)$
    }
    \begin{vastaus}
        \alakohdat{
            § $2x+6$
            § $x^2 - 2x$
            § $3x-6x^2$
            § $x^3 + 5x^2$
        }
    \end{vastaus}
\end{tehtava}

\begin{tehtava}
    Sievennä.
    \alakohdat{
        § $3(x+2y-4)$
        § $(x+2)(x + 3)$
        § $(3-x)(2x-1)$
}
    \begin{vastaus}
        \alakohdat{
            § $3x+6y-12$
            § $x^2 +5x+6$
            § $-2x^2+7x-3$
        }
    \end{vastaus}
\end{tehtava}

\begin{tehtava}
    Sievennä lauseke $(x^2+1)(x^3-2x)$. Mikä on polynomin aste?
    \begin{vastaus}
        Lauseke sievenee muotoon $x^5-x^3-2x$. Polynomin aste on $5$.
    \end{vastaus}
\end{tehtava}

\subsubsection*{Hallitse kokonaisuus}

\begin{tehtava}
	Mitkä seuraavista polynomilausekkeista esittävät samaa polynomifunktiota kuin
	$x^3+2x+1$?
	\alakohdat{
		§ $2x+x^3+1$
		§ $x^2+2x+1$
		§ $x+2x^3+1-(x^3+x)$
		§ $15+x^4+2x+x^3-x^4-14$
	}
	\begin{vastaus}
		a) ja d)
	\end{vastaus}
\end{tehtava}

\begin{tehtava}
	Mitkä ovat seuraavien polynomifunktioiden asteet, ts. sievennettyjen muotojen asteet?
	\alakohdat{
		§ $x+5-x$
		§ $x^2+x-2x^2$
		§ $4x^5+x^2-4-x^2$
		§ $x^4-2x^3+x-1+x^3-x^4+x^3$
	}
	\begin{vastaus}
		\alakohdat{
			§ $0$
			§ $2$
			§ $5$
			§ $1$
		}
	\end{vastaus}
\end{tehtava}

\begin{tehtava}
    Pohdi ja määritä sulkuja avaamatta lausekkeen $(x^2+1)(x^3-2x)$
    \alakohdat{m
        § aste
        § vakiotermi.
    }
    \begin{vastaus}
        \alakohdat{
            § Polynomin aste on kunkin tekijän korkeimpien asteiden summa, tässä siis $2+3=5$.
            § Polynomin vakiotermi on kunkin tekijän vakiotermien tulo, tässä siis $1\cdot 0=0$.
        }
    \end{vastaus}
\end{tehtava}

\begin{tehtava}
    Sievennä.
    \alakohdat{
        § $(-2x)(4x - 1)\cdot 3$
        § $(-x^3)(10x - 2)$
        § $5(-2x + 1)(-9x) $
        § $2x(x-3)+1$
    }
    \begin{vastaus}
        \alakohdat{
            § $-24x^2 + 6x$
            § $-10x^4 + 2x^3$
            § $90x^2 - 45x$
            § $2x^2-6x+1$
        }
    \end{vastaus}
\end{tehtava}

\begin{tehtava}
    Sievennä.
    \alakohdat{
        § $(2y+5)(y+7)$
        § $(x-1)(x+4)x$
    }
    \begin{vastaus}
        \alakohdat{
            § $2y^2 + 19y + 35$
            § $x^3 + 3x^2 - 4x$
        }
    \end{vastaus}
\end{tehtava}

\begin{tehtava}
Ensimmäisen asteen polynomiausekkeen yleinen muoto on $ax+b$, missä $x$ on muuttuja, ja $a$ ja $b$ ovat (reaalisia) vakioita. Osoita, että jos $P$ ja $Q$ ovat ensimmäisen asteen polynomifunktioita, niin tällöin $P(Q(x))$ ja $Q(P(x))$ ovat myös ensimmäistä astetta.

	\begin{vastaus}
Voidaan merkitään $P(x)=ax+b$ ja $Q(x)=cx+d$, koska molemmat funktiot ovat ensimmäisen asteen polynomifunktioita. (Käytämme kertoimien merkitsemiseen eri kirjaimia, sillä lausekkeiden vakiot saattavat olla keskenään erisuuret.) Sievennetään $P(Q(x))$: $P(Q(x))=P(cx+d)=a(cx+d)+b=acx+ad+b$. Todetaan, että jos $a$ ja $c$ ovat reaalilukuvakioita, niin myös niiden tulo on reaalilukuvakio. Sama pätee vakioille $a$ ja $d$, joiden tulo ja myös kyseisen tulon ja $b$:n summa on vakio. Merkitään havainnollistamisen vuoksi $ac=e$, ja $ad+b=f$. Nyt lauseke $acx+ad+b$ voidaan kirjoittaa uudelleen muodossa $ex+f$, josta nähdään selvästi kyseessä olevan ensimmäisen asteen polynomi.

Väite osoitetaan järjestykselle $Q(P(x))$ samalla tavalla, minkä jälkeen väite on todistettu.

	\end{vastaus}
\end{tehtava}

\subsubsection*{Lisää tehtäviä}

\begin{tehtava}
	Sievennä. 
	\alakohdat{
		§ $(x-3)(2x^3-3x+4)$
		§ $(x^2+1)(x^3-2x-4)$
		§ $(x-1)(x^4+x^3+x^2+x+1)$
		§ $(\frac{x}{5}-\frac{2}{3})(x^2+x+1)$  %TÄMÄ TÄYTYY TARKISTAA
	}
	\begin{vastaus}
		\alakohdat{
			§ $2x^4-6x^3-3x^2+13x-12$
			§ $x^5-x^3-4x^2-2x-4$
			§ $x^5-1$
			§ $\frac{1}{5}x^3-\frac{7}{15}x^2-\frac{7}{15}x-\frac{2}{3}$
		}
	\end{vastaus}
\end{tehtava}

	\begin{tehtava}
Sievennä $\sqrt{\frac{(a+b)^2 + (a-b)^2}{a^2 + b^2}}$.
	\begin{vastaus}
$\sqrt{2}$
	\end{vastaus}
	\end{tehtava}

\begin{tehtava}
    Olkoon $P(x)=-x^4+2x$ reaalifunktio. Sievennä lausekkeet.
    \alakohdat{
		§ $P(\sqrt{2})$
        § $P(x)^2$
        § $P(-x)$
        § $P(2t)$
    }
    \begin{vastaus}
        \alakohdat{
            § $-4 + 2\sqrt{2}$
            § $x^8 - 4x^5 + 4x^2$
            § $-x^4-2x$
            § $-16t^4+4t$
        }
    \end{vastaus}
\end{tehtava}

\begin{tehtava}
    Olkoot $P(x)=x^2$ ja $Q(x)=x+1$ reaalifunktioita. Sievennä lausekkeet.
    \alakohdat{
        § $P(x+1)$
        § $Q(x-1)$
        § $P(Q(x))$
        § $Q(P(x))$
    }
    \begin{vastaus}
        \alakohdat{
            § $P(x+1) = (x+1)^2 = x^2+2x+1$
            § $Q(x-1) = (x-1)+1 = x$
            § $P(Q(x)) = P(x+1) = (x+1)^2 = x^2+2x+1$
            § $Q(P(x)) = Q(x^2) = x^2+1$
        }
    \end{vastaus}
\end{tehtava}

\end{tehtavasivu}