\subsection*{Osaatko selittää?}
\begin{multicols}{2}
\subsubsection*{Luvut ja laskutoimitukset}

\begin{tehtava}
Mitä eroa on luvulla ja numerolla?
\begin{vastaus}
Luvulla on suuruus, ja niitä esitetään numeroiden eli numeromerkkien avulla. Kymmenjärjestelmässä käytetyt numeromerkit ovat 0, 1, 2, 3, 4, 5, 6, 7, 8, ja 9. (Huomaa myös vakiintuneet nimitykset kuten postinumero, puhelinnumero, \ldots)
\end{vastaus}
\end{tehtava}

\begin{tehtava}
Miten vähennyslasku ja jakolasku määritellään?
\begin{vastaus}
Kahden luvun $a$ ja $b$ vähennyslasku (erotus) $a-b$ määritellään summana $a+(-b)$, eli $a$:han on lisätty $b$:n vastaluku. Jakolasku (osamäärä) $\frac{a}{b}$ määritellään tulona $a\cdot \frac1b$, eli $a$ on kerrottu $b$:n käänteisluvulla.
\end{vastaus}
\end{tehtava}

\begin{tehtava}
Minkälainen ristiriita seuraisi, jos nollalla jakaminen olisi määritelty reaaliluvuille?
\begin{vastaus}
Kaikilla luvuilla $x$ pätee $0 \cdot x = 0$, eli mikä tahansa luku kerrottuna nollalla on aina nolla. Kuitenkin, jos nollalla voisi jakaa (eli sille olisi määritelty käänteisluku $0^{-1}=\frac10$), olisi voimassa $0 \cdot\frac10 = 1$, mikä ei kuitenkaan samanaikaisesti voi pitää paikkansa sen kanssa, että, kun mikä tahansa luku kerrotaan nollalla, saadaan tuloksi nolla.
\end{vastaus}
\end{tehtava}

\begin{tehtava}
Mitä eroa on kertomisella ja laventamisella?
\begin{vastaus}
Laventaminen muuttaa lausekkeen (tavallisesti rationaaliluvun) eri muotoon, mutta pitää sen suuruuden ennallaan. Kertominen on laskutoimitus, joka voi muuttaa luvun suuruutta. Esimerkiksi $\frac34$ lavennettuna viidellä on $\frac{15}{20}$ mutta sama luku kerrottuna viidellä on $\frac{15}{4}$.
\end{vastaus}
\end{tehtava}

\begin{tehtava}
Mitä tarkoitetaan alkuluvulla?
\begin{vastaus}
Alkuluku on luonnollinen luku, jota ei voi esittää kahden pienemmän luonnollisen tulona. Pienin alkuluku on $2$. %parempi määritelmä?
\end{vastaus}
\end{tehtava}

%FIXME termikoherenssi
\begin{tehtava}
Mitä tarkoitetaan luvun alkulukukehitelmällä (tai alkutekijäkehitelmällä)?
\begin{vastaus}
Alkutekijäkehitelmä on luvun esitys alkulukujen tulona, esimerkiksi $45=3\cdot 3 \cdot 5$ tai potenssien avulla kirjoitettuna $45=3^2\cdot 5$.
\end{vastaus}
\end{tehtava}

\begin{tehtava}
Miksei reaalilukuja käytettäessä negatiivisesta luvusta voi ottaa neliöjuurta?
\begin{vastaus}
Koska $x^2 \geq 0$ kaikilla reaaliluvuilla $x$, eli ei ole olemassa reaalilukua, joka kerrottuna itsellään tuottaisi negatiivisen luvun.
\end{vastaus}
\end{tehtava}

\begin{tehtava}
Mitä tarkoitetaan, kun sanotaan luvun olevan epänegatiivinen?
\begin{vastaus}
Kyseessä \textit{ei} ole vaikea tapa ilmaista sana positiivinen. Reaaliluvut voivat olla joko negatiivisia, positiivisia tai nolla (joka ei ole kumpaakaan), joten epänegatiivinen-ilmaisu sulkee pois ainoastaan negatiivisen -- luku voi olla nolla tai positiivinen. Ilmaisuun törmää usein määrittelyehtojen yhteydessä. Esimerkiksi parillisia juuria voi ottaa reaalilukuja käytettäessä vain epänegatiivisista luvuista.
\end{vastaus}
\end{tehtava}

\begin{tehtava}
Miksi mielivaltaisen, nollasta poikkeavan luvun nollas potenssi on arvoltaan yksi?
\begin{vastaus}
Koska käyttämällä potenssisääntöjä saadaan: $x^0=x^{1-1}=\frac{x^1}{x^1}=\frac{x}{x}=1$, mikä pätee kaikille luvuille paitsi $x=0$. Tämä on ainut tapa määritellä nollas potenssi niin, että se ei olisi ristiriidassa muiden potenssin ominaisuuksien (potenssisääntöjen) kanssa. (Toinen variantti: $x=x^1=x^{0+1}=x^0 \cdot x^1 = x^0 \cdot x$, mistä tulos seuraa.)
\end{vastaus}
\end{tehtava}

\begin{tehtava}
Mitä tarkoitetaan johdannaissuureella?
\begin{vastaus}
Johdannaissuure on suure, joka voidaan ilmaista (SI-järjestelmän) perussuureiden avulla niiden tuloina tai osamäärinä.
\end{vastaus}
\end{tehtava}

\begin{tehtava}
Mikä nyanssiero on saman luvun esitysmuodoilla $9,1$ ja $9,10$?
\begin{vastaus}
Jälkimmäisessä esityksessä on kolme merkitsevää numeroa -- ensimmäisessä vain kaksi. Jos kyseessä on mittaustulos, jälkimmäistä lukua voidaan pitää tarkempana.
\end{vastaus}
\end{tehtava}

\begin{tehtava}
Tutkit luvun desimaalikehitelmää ja huomaat siinä äärettömästi toistuvan jakson 2104. Minkälainen luku on kyseessä?
\begin{vastaus}
Kyseessä on rationaaliluku, sillä kaikkien rationaalilukujen (ja ainoastaan rationaalilukujen) desimaaliesitys sisältää äärettömästi toistuvaan jaksoon.
\end{vastaus}
\end{tehtava}

\begin{tehtava}
Mikä yhteys murtopotensseilla ja juurilla on?
\begin{vastaus}
Positiivisilla reaaliluvuilla $x$ pätee $\sqrt[n]{x^m} = x^{\frac{m}{n}}$. Tämä mahdollistaa potenssisääntöjen käyttämisen juurilausekkeille.
\end{vastaus}
\end{tehtava}
%tehtäviä liitännäisyydestä, vaihdannaisuudesta, refleksiivisyydestä, transitiivisuudesta ja symmetrisyydestä

\subsubsection*{Yhtälöt}

\begin{tehtava}
Mitä tarkoitetaan yhtälön ratkaisulla?
\begin{vastaus}
Muuttujan arvoa (tai muuttujien arvoja), jolla yhtälö on tosi
\end{vastaus}
\end{tehtava}

\begin{tehtava}
Jos lukua kasvatetaan $p$:llä prosentilla ja sen jälkeen pienennetään $p$:llä prosentilla, miksi ei päädytä takaisin alkuperäiseen lukuun? (Oletetaan, että $p\neq 0$.)
	\begin{vastaus}
	Kyse on absoluuttisista, ei suhteellisista muutoksista. Jos lisätään jokin luku ja sitten vähennetään sama luku, laskutoimitukset kumoavat aina toisensa. Sen sijaan jos mielivaltaista lukua kasvatetaan suhtellisella osuudella, kasvun suuruus riippuu alkuperäisestä luvusta. Kun kasvatetusta luvusta otetaan sama suhteellinen osa pois, kyseinen osa onkin nyt suurempi, joten päädytään alkuperäistä pienempään lukuun.
	\end{vastaus}
\end{tehtava}

\subsubsection*{Funktiot}

\begin{tehtava}
Mitä eroa on funktion määrittelyjoukolla ja määrittelyehdolla?
\begin{vastaus}
Määrittelyjoukko on, kuten nimikin sanoo, joukko. Se siis esittää kaikki mahdolliset luvut (tai muunlaiset syötteet), jotka funktion muuttuja voi saada arvokseen, niin että funktion arvon laskeminen on mielekästä. Määrittelyehto kertoo määrittelyjoukon epäsuorasti antamalla jonkinlaisen ehdon muuttujalle, yleensä epäyhtälön muodossa. Esimerkiksi, jos määrittelyjoukko olisi $\mathbb{R}_+$ (positiiviset reaaliluvut), niin sitä vastaava määrittelyehto olisi $x>0$, missä $x$ on kyseisen funktion muuttuja.
\end{vastaus}
\end{tehtava}

\begin{tehtava}
Mitä eroa on funktion maali- ja arvojoukolla?
	\begin{vastaus}
Maalijoukko voi olla laajempi kuin arvojoukko, mutta sisältää aina arvojoukon kokonaisuudessaan. Arvojoukko on niiden alkioiden (yleensä lukujen) joukko, jotka funktio voi arvokseen saada. Maalijoukko koostuu alkioista, jotka funktio \textit{saattaa} saada arvokseen. Maalijoukkoa ei voi päätellä funktion mahdollisesta lausekkeesta, mutta arvojoukon voi. (Maalijoukon tärkeys ei tule kunnolla esille lukion matematiikassa -- ei huolta, vaikkei sen merkitys kunnolla aukeaisikaan. Arvojoukko sen sijaan tulee ymmärtää, ja sitä käsitellään lisää tulevilla kursseilla.)
	\end{vastaus}
\end{tehtava}

\end{multicols}

\newpage
\subsection*{Monivalinta}

Valitse kussakin monivalintatehtävässä yksi paras vastaus.


\subsubsection*{Luvut ja laskutoimitukset}
\begin{multicols}{2}

\begin{tehtava}
Mikä laskuista antaa suurimman tuloksen?
\alakohdatm{
§ $21\cdot 20-18+19$
§ $18\cdot 19-21+20$
§ $20\cdot 18-19+21$
§ $21\cdot 18-20+19$
§ $20\cdot19-21+18$
}
	\begin{vastaus}
	 a) $21\cdot 20-18+19$
	\end{vastaus}
\end{tehtava}

\begin{tehtava}
Luvun $180$ alkulukukehitelmä on
\alakohdatm{
§ $2\cdot3\cdot5$
§ $2^2\cdot3^2\cdot5^2$
§ $4\cdot3^2\cdot5$
§ $4\cdot9\cdot5$
§ $2^2\cdot3^2\cdot5$.
}
	\begin{vastaus}
	 e) $2^2\cdot3^2\cdot5$
	\end{vastaus}
\end{tehtava}

\begin{tehtava}
Kun lausekkeeseen $x(x-2)$ sovelletaan osittelulakia, saadaan sille jokin alla olevista esityksistä. Mikä?
	\alakohdatm{
		§ $x^2-2$
		§ $2x-2$
		§ $2x-2x$
		§ $x^2-2x$
		§ Osittelulakia ei voi soveltaa tässä tapauksessa.
	}
    \begin{vastaus}
	d) $x^2-2x$
    \end{vastaus}
\end{tehtava}

\begin{tehtava}
Karinilla on neljä koiraa. Jos jokaiselle pitää ostaa kaulapanta ($p$) ja kuusi pakkausta ruokaa ($r$), niin mikä seuraavista lausekkeista vastaa ostosten kokonaismäärää?
	\alakohdatm{
		§ $4pr$
		§ $p+4r$
		§ $4p+4r$
		§ $4p+6r$
		§ $4(p+6r)$
		§ $4+p+r$
	}
    \begin{vastaus}
	e) $4(p+6r)$
    \end{vastaus}
\end{tehtava}

\begin{tehtava}
Kun lausekkeeseen $bc-2ac+3b-6a$ sovelletaan osittelulakia, saadaan sille siistimpi esitys, joka on jokin allaolevista. Mikä?
	\alakohdatm{
		§ $(b-3a)(c+3)$
		§ $(b-2a)(c+3)$
		§ $2(b-a)(c+3)$
		§ $(b-a)(2c+3)$
		§ Osittelulakia ei voi soveltaa tässä tapauksessa.
	}
    \begin{vastaus}
	b) $(b-2a)(c+3)$
    \end{vastaus}
\end{tehtava}

\begin{tehtava}
Kuinka monta kolmasosaa mahtuu yhdeksään?
\alakohdatm{
§ $3$
§ $9$
§ $18$
§ $27$
§ $30$
}
    \begin{vastaus}
	 b) $9$
    \end{vastaus}
\end{tehtava}

\begin{tehtava}
Mikä seuraavista luvuista on pienin?
\alakohdatm{
§ $0,7$
§ $0,77$
§ $0,707$
§ $7,7$
§ $7,07$
§ $77,0$
}
    \begin{vastaus}
	 a) $0,7$
    \end{vastaus}
\end{tehtava}

\begin{tehtava} %fixme todo ja toisinpäin
Sekaluku $7\frac{3}{8}$ ilmaistuna murtolukuna on
\alakohdatm{
§ $\frac{31}{4}$
§ $\frac{21}{8}$
§ $\frac{5}{4}$
§ $\frac{59}{8}$
§ $\frac{53}{8}$
}
    \begin{vastaus}
	 d) $\frac{59}{8}$
    \end{vastaus}
\end{tehtava}
%muokannut jaakko viertiö 22.3.2014

\begin{tehtava}
Kahdelle reaaliluvulle $s$ ja $t$ pätee ehto $4s=t$. Tämän perusteella voidaan sanoa, että
\alakohdat{
§ $s$ on suurempi kuin $t$
§ $s$ on pienempi kuin $t$.
§ Ei ehdottomasti kumpikaan edellisistä -- riippuu tilanteesta.
}
    \begin{vastaus}
	 b) $s$ on pienempi kuin $t$
    \end{vastaus}
\end{tehtava}

\begin{tehtava}
Millä lausekkeista on suurin arvo?
\alakohdatm{
§ $2^2\cdot2^2$
§ ${2^2}^{2\cdot 2}$
§ $2:2^{2^2}$
§ $2^{2^2}:2$
§ $(2^2)^{2^2}$
§ $2^{2^{2^2}}$
}
	\begin{vastaus}
	 f) $2^{2^{2^2}}$
	\end{vastaus}
\end{tehtava}

\begin{tehtava}
Potenssi $\left( \dfrac{2}{3} \right)^{-2}$ sievennettynä on jokin seuraavista. Mikä?
\alakohdatm{
§ $36$
§ $\frac{1}{36}$
§ $\frac{4}{9}$
§ $-\frac{4}{9}$
§ $\frac{9}{4}$
§ $-\frac{9}{4}$
}
\begin{vastaus}
e) $\frac{9}{4}$
\end{vastaus}
\end{tehtava}
%muokannut jaakko viertiö 22.3.2014

\begin{tehtava}
Kuinka monta nollaa on biljoonassa?
\alakohdatm{
§ $6$
§ $9$
§ $12$
§ $15$
§ ei mikään annetuista vaihtoehdoista
}
\begin{vastaus}
c) $12$
\end{vastaus}
\end{tehtava}

\begin{tehtava}
Mitä on $0,34$\,nm esitettynä metreinä kymmenpotenssimuodossa (''oikeaoppisesti'', \textit{scientific notation})?
\alakohdatm{
§ $0,34 \cdot 10^{-9}$\,m
§ $3,4 \cdot 10^{-10}$\,m
§ $34 \cdot 10^{-9}$\,m
§ $3,4 \cdot 10^{-9}$\,m
§ $34 \cdot 10^{-10}$\,m
}
\begin{vastaus}
b) $3,4 \cdot 10^{-10}$\,m (standardimuodossa kymmenen potenssin kertoimena on luku itseisarvoltaan nollan ja kymmenen väliltä)
\end{vastaus}
\end{tehtava}

\begin{tehtava}
Kuinka monta merkitsevää numeroa on luvussa $1,0240$?
\alakohdatm{
§ ei yhtään
§ yksi
§ kolme
§ neljä
§ viisi
}
\begin{vastaus}
e) viisi
\end{vastaus}
\end{tehtava}

\begin{tehtava}
Kuinka monta merkitsevää numeroa on luvussa $-0,0100$?
\alakohdatm{
§ ei yhtään
§ yksi
§ kolme
§ neljä
§ viisi
}
\begin{vastaus}
c) kolme
\end{vastaus}
\end{tehtava}

\begin{tehtava}
Luku $-78,2449$ pyöristettynä sadasosien tarkkuuteen on
\alakohdatm{
§ $-78,245$
§ $-78,25$
§ $-78,24$
§ $-78,2$
}
\begin{vastaus}
c) $-78,24$ (tuhannesosia pienempiä desimaaleja ei huomioida lainkaan; pyöristys nollaan päin)
\end{vastaus}
\end{tehtava}

\begin{tehtava}
Pyöristä luku neljäkymmentäseitsemäntuhatta yhdeksänsataakahdeksankymmentäkolme lähimpään tuhanteen.
\alakohdatm{
§ $47\,000$
§ $50\,000$
§ $100\,00$
§ $48\,000$
§ $49\,000$
§ $48\,900$
}
\begin{vastaus}
d) $48\,000$
\end{vastaus}
\end{tehtava}

\begin{tehtava}
Kaksi litraa kuutiosenttimetreinä on	
\alakohdatm{
§ $2$\,cm$^3$
§ $20$\,cm$^3$
§ $200$\,cm$^3$
§ $2\,000$\,cm$^3$.
§ Ei mikään annetuista vaihtoehdoista.
}
	\begin{vastaus}
	d) $2\,000$\,cm$^3$
	\end{vastaus}
\end{tehtava}

\begin{tehtava}
Jauhosäkki voisi painaa
\alakohdatm{
§ $2$\,g
§ $2\,000$\,mg
§ $2$ litraa
§ $2$\,kg
§ $2\cdot10^6$\,g.
}
	\begin{vastaus}
	d) $2$\,kg
	\end{vastaus}
\end{tehtava}

\begin{tehtava}
Nopeus $60$\,km/h ilmaistuna yksiköissä m/s on suunnilleen
\alakohdatm{
§ $10$
§ $17$
§ $30$
§ $36$
§ $216$.
}
	\begin{vastaus}
	b) $17$
	\end{vastaus}
\end{tehtava}

\begin{tehtava}
Neliöjuuri luvusta $8$ mahdollisimman pitkälle sievennettynä on
\alakohdatm{
§ $\sqrt{8}$
§ $2\sqrt{2}$
§ $4$
§ $16$
§ $64$.
}
  \begin{vastaus}
	 b) $2\sqrt{2}$
    \end{vastaus}
\end{tehtava}
%muokannut jaakkoviertiö 22.3.2014
\end{multicols}

\subsubsection*{Yhtälöt}
\begin{multicols}{2}

\begin{tehtava}
Yhtälö $x = x-1$ on
\alakohdatm{
§ aina tosi
§ joskus tosi
§ ei koskaan tosi
}
\begin{vastaus}
c) ei koskaan tosi
\end{vastaus}
\end{tehtava}

\begin{tehtava}
Yhtälö $x = 4$ on
\alakohdatm{
§ aina tosi
§ joskus tosi
§ ei koskaan tosi
}
\begin{vastaus}
b) joskus tosi
\end{vastaus}
\end{tehtava}

\begin{tehtava}
Yhtälö $x = x$ on
\alakohdatm{
§ aina tosi
§ joskus tosi
§ ei koskaan tosi
}
\begin{vastaus}
a) aina tosi
\end{vastaus}
\end{tehtava}

\begin{tehtava}
Milloin yhtälön molemmat puolet voi kertoa samalla luvulla niin, että tosi yhtälö säilyy totena?
\alakohdat{
§ Aina muttei millä tahansa luvulla
§ Aina ja millä tahansa luvulla
§ Vain joskus tietyissä tilanteissa
§ Ei koskaan
}
	\begin{vastaus}
b) Aina ja millä tahansa luvulla
	\end{vastaus}
\end{tehtava}

\begin{tehtava}
Milloin yhtälön molemmat puolet voi kertoa samalla luvulla niin, että yhtälö säilyy yhtäpitävänä?
\alakohdat{
§ Aina muttei nollalla
§ Aina ja millä tahansa luvulla
§ Vain joskus tietyissä tilanteissa
§ Ei koskaan
}
\begin{vastaus}
a) Aina muttei nollalla (nollalla kertominen kadottaa informaatiota, eikä alkuperäistä yhtälöä enää saa ratkaistua)
\end{vastaus}
\end{tehtava}

\begin{tehtava}
Jos muuttujat $x$ ja $y$ ovat suoraan verrannolliset, ja muuttuja $x$ kasvaa, mitä tapahtuu muuttujalle $y$?
\alakohdat{
§ Muuttuja $y$ kasvaa myös mutta suhteessa vähemmän.
§ Muuttuja $y$ kasvaa myös mutta suhteessa enemmän.
§ Muuttuja $y$ kasvaa myös -- ja samassa suhteessa.
§ Muuttuja $y$ pienee -- ja samassa suhteessa.
§ Muuttujan $y$ arvo pysyy vakiona.
}
\begin{vastaus}
e) Muuttuja $y$ kasvaa myös -- ja samassa suhteessa.
\end{vastaus}
\end{tehtava}

\begin{tehtava}
Jos muuttujat $x$ ja $y$ ovat suoraan verrannolliset, ja muuttujat $y$ ja $z$ ovat kääntäen verrannolliset, millainen on muuttujien $x$ ja $z$ välinen suhde?
\alakohdat{
§ $x$ ja $z$ ovat suoraan verrannolliset.
§ $x$ ja $z$ ovat kääntäen verrannolliset.
§ $x$ on verrannollinen $z$:n neliöön.
§ $z$ on verrannollinen $x$:n neliöön.
§ Mikään annetuista vaihtoehdoista ei ole oikein.
}
	\begin{vastaus}
b) $x$ ja $z$ ovat kääntäen verrannolliset.
	\end{vastaus}
\end{tehtava}

\begin{tehtava}
Yhtälön $a^2+a^{-2}=0$ (eräs) juuri on
\alakohdatm{
§ $-1$
§ $0$
§ $1$
§ $\sqrt{2}$
§ ei mikään mainituista vaihtoehdoista.
}
	\begin{vastaus}
e) ei mikään mainituista vaihtoehdoista
	\end{vastaus}
\end{tehtava}

\begin{tehtava}
Kuinka monta eri reaaliratkaisua yhtälöllä $x^{10}=-2$ on?
\alakohdatm{
§ ei yhtään
§ yksi
§ kaksi
§ kymmenen
§ ei mikään mainituista vaihtoehdoista
}
	\begin{vastaus}
a) ei yhtään
	\end{vastaus}
\end{tehtava}

\begin{tehtava}
Yhtälönratkaisun vaiheessa
\begin{align*}
2x+x&=4 \\
3x&=4
\end{align*}
käytettiin avuksi:
\alakohdat{
§ reaalilukujen vaihdantalakia
§ reaalilukujen liitäntälakia
§ reaalilukujen osittelulakia
§ kaikkia edellä mainittuja
§ ei mitään edellä mainituista.
}
	\begin{vastaus}
	c) reaalilukujen osittelulakia
	\end{vastaus}
\end{tehtava}

%\begin{tehtava} %verrannollisuus
%
%\alakohdatm{
%§
%§
%§
%§
%}
%	\begin{vastaus}
%	
%	\end{vastaus}
%\end{tehtava}
%
%\begin{tehtava}%verrannollisusmonivalintoja
%
%\alakohdatm{
%§
%§
%§
%§
%}
%	\begin{vastaus}
%	
%	\end{vastaus}
%\end{tehtava}
%
%\begin{tehtava}%verrannollisusmonivalintoja
%
%\alakohdatm{
%§
%§
%§
%§
%}
%	\begin{vastaus}
%	
%	\end{vastaus}
%\end{tehtava}

\begin{tehtava}
$0,46$\,\permil\;esitettynä desimaalilukuna on
	\alakohdatm{
	§ $0,046$
	§ $0,0046$
	§ $0,00046$
	§ $0,000046$
	§ Ei mikään annetuista vaihtoehdoista.
	}
	\begin{vastaus}
	c) $0,00046$
	\end{vastaus}
\end{tehtava}

\begin{tehtava}
Jos lukua $a$ pienennetään $3$\,\%, laskutoimitusta vastaava lauseke on
\alakohdatm{
§ $0,03a$
§ $0,97a$
§ $a-0,03$
§ $a-0,97$
§ $\frac{a}{0,03}$
§ $\frac{a}{1,03}$.
}
	\begin{vastaus}
b) $0,97a$
	\end{vastaus}
\end{tehtava}

\begin{tehtava}
Erään tuotteen hinta on noussut, ja vertaamme uutta ja vanhaa hintaa toisiinsa. Mikä seuraavista \textit{ei} tarkoita samaa kysymyksenasettelun ''Kuinka monta prosenttia hinta on noussut'' kanssa?
\alakohdat{
§ ''Kuinka monta prosenttia uusi hinta on suurempi kuin vanha hinta?''
§ ''Kuinka monta prosenttia uusi hinta on vanhaa hintaa suurempi?''
§ ''Kuinka monta prosenttia vanha hinta on uutta hintaa pienempi?''
§ ''Mikä on uuden ja vanhan hinnan suhteellinen erotus, kun verrataan vanhaan hintaan?''
§ ''Kuinka paljon vanhan ja uuden hinnan suhde poikkeaa yhdestä?''
}
	\begin{vastaus}
c) ''Kuinka monta prosenttia vanha hinta on uutta hintaa pienempi?''
	\end{vastaus}
\end{tehtava}

\begin{tehtava}
Ratauudistusten jälkeen junien matka-ajat ovat lyhentyneet. Merkataan vanhaa matkan kestoa $t_1$ ja uudistusten jälkeistä kestoa $t_2$. Kysymykseen ''Kuinka monta prosenttia matkan kesto on lyhentynyt?'' (sopivalla vastauksen tulkinnalla) vastaa lauseke
\alakohdatm{
§ $t_1-t_2$
§ $\dfrac{t_1}{t_2}$
§ $\dfrac{t_2}{t_1}$
§ $\dfrac{t_1-t_2}{t_1}$
§ $\dfrac{t_1-t_2}{t_2}$
§ $\dfrac{t_1}{t_2}-1$
§ $\dfrac{t_2}{t_1}+1$.
}
	\begin{vastaus}
d) $\frac{t_1-t_2}{t_1}$
	\end{vastaus}
\end{tehtava}

\begin{tehtava}
Jos pääoma $K$ kasvaa korkoa $n$ kertaa korkokannalla $p$, niin talletuksen suuruus on lopulta
\alakohdatm{
§ $nK$
§ $npK$
§ $n\frac{p}{100}k$
§ $n(1+\frac{p}{100})K$
§ $(1+\frac{p}{100})^nK$
§ $(1+\frac{p^n}{100})K$
}
	\begin{vastaus}
e) $(1+\frac{p}{100})^nK$
	\end{vastaus}
\end{tehtava}

\end{multicols}

\subsubsection*{Funktiot}
\begin{multicols}{2}

\begin{tehtava}
Funktio $f$ määritellään yhtälöllä $f(x)=\frac{1}{x-1}$. Funktion määrittelyjoukko on tällöin
\alakohdatm{
§ $\mathbb{R}$
§ $\mathbb{R}_+$
§ $\mathbb{R}_-$
§ $\mathbb{R}\setminus \lbrace 0 \rbrace$
§ $\mathbb{R}\setminus \lbrace 1 \rbrace$
§ $\mathbb{R}\setminus \lbrace -1 \rbrace$.
}
    \begin{vastaus}
	 e) $\mathbb{R}\setminus \lbrace 1 \rbrace$
    \end{vastaus}
\end{tehtava}

\begin{tehtava}
Funktio $g$ määritellään yhtälöllä $g(y)=\sqrt{y}$. Funktion määrittelyehto on tällöin
\alakohdatm{
§ $y<0$
§ $y\leq 0$
§ $y\in \mathbb{R}$
§ $y\geq 0$
§ $y>0$.
}
    \begin{vastaus}
	 d) $y\geq 0$
    \end{vastaus}
\end{tehtava}

\begin{tehtava}
Pallon tilavuus lasketaan pallon säteen funktiona kaavalla $V(r)=\frac{4}{3}\pi r^3$. Funktiolle $V$ sopivin arvojoukko on tällöin
\alakohdatm{
§ $\mathbb{N}$
§ $\mathbb{Q}$
§ $\mathbb{R}$
§ $\mathbb{R}_-$
§ $\mathbb{R}_+$.
}
	\begin{vastaus}
	e) $\mathbb{R}_+$ (tilavuus ei voi olla negatiivinen)
	\end{vastaus}
\end{tehtava}

\begin{tehtava}
Arvosanojen $6$, $7$ ja $10$ aritmeettinen keskiarvo on
	\alakohdatm{
	§ $7$
	§ $7,25$
	§ $7\,\frac{2}{3}$
	§ $8,25$
	§ Ei mikään annetuista vaihtoehdoista.
	}
	\begin{vastaus}
	c) $7\,\frac{2}{3}$
	\end{vastaus}
\end{tehtava}

\begin{tehtava}
Tarkastellaan reaalifunktiota $f$, jonka arvot määritellään kaavalla $f(x)=x^2-2$. Mikä väitteistä on epätosi?
	\alakohdat{
	§ Funktio saa negatiivisia arvoja.
	§ Funktio saa mielivaltaisen suuria arvoja.
	§ Funktion määrittelyjoukoksi sopii kaikkein reaalilukujen joukko.
	§ Funktion määrittelujoukoksi sopii luonnollisten lukujen joukko.
	§ Funktio ei saa irrationaaliarvoja.
	}
	\begin{vastaus}
	e) Funktio ei saa irrationaaliarvoja.
	\end{vastaus}
\end{tehtava}

\begin{tehtava}
Tarkastellaan reaalifunktiota $g$, jonka arvot määritellään kaavalla $g(x)=\pi$. Mikä väitteistä on epätosi?
	\alakohdat{
	§ Funktion arvojoukko koostuu yhdestä luvusta.
	§ Funktion pienin arvo on $\pi$.
	§ Funktion suurin arvo on $\pi$.
	§ Funktio ei saa irrationaaliarvoja.
	§ Funktiolla ei ole nollakohtia.
	}
	\begin{vastaus}
	d) Funktio ei saa irrationaaliarvoja.
	\end{vastaus}
\end{tehtava}

\end{multicols}
\newpage

\subsection*{Tosi vai epätosi?}

Pitävätkö väitteet paikkansa?

\subsubsection*{Luvut ja laskutoimitukset}
\begin{multicols}{2}
\begin{tehtava}
\alakohdat{
§ Kahden eri alkuluvun osamäärä ei voi olla kokonaisluku.
§ Triljoona kirjoitetaan kymmenpotenssimuodossa $10^{15}$.
§ Lukujen osamäärä on laskutoimituksena liitännäinen.
§ Erotus on laskutoimituksena vaihdannainen.
§ Kaikki kokonaisluvut ovat rationaalilukuja.
§ Lämpötila on skalaarisuure.
§ Jos tiheys kerrotaan tilavuudella, saadaan massa.
§ Suurin osa luonnollisista luvuista on suurempia kuin $10^{100}$.
§ On olemassa yksiselitteinen pienin positiivinen rationaaliluku.
§ Nolla on epänegatiivinen luku.
}
	\begin{vastaus}
	\alakohdatm{
	§ Tosi
	§ Tosi
	§ Epätosi
	§ Epätosi
	§ Tosi
	§ Tosi
	§ Tosi
	§ Tosi
	§ Epätosi
	§ Tosi
	}
	\end{vastaus}
\end{tehtava}
	
\begin{tehtava}
\alakohdat{	
§ Mistään aidosta murtoluvusta (ei-kokonaisluku) ei voi kokonaislukupotenssiin korottamalla saada kokonaislukua. 
§ Jokaisen rationaaliluvun välistä löytyy uusia rationaalilukuja.
§ Irrationaaliluvun voi joissain tilanteissa esittää kahden kokonaisluvun suhteena.
§ Kahden positiivisen luvun suuri absoluuttinen ero johtaa aina myös suureen suhteelliseen eroon. %tarkista sisältö JA kielen koherenssi; korjaa myös prosenttilukulukuun
§ Kahden positiivisen luvun suuri suhteellinen ero johtaa aina myös suureen absoluuttiseen eroon.
§ Jos kahden muuttujan osamäärä on vakio, myös niiden tulo on vakio.
}
	\begin{vastaus}
	\alakohdatm{
	§ Epätosi %(esim. $\left(\frac{1}{3}\right)^{-1}=3$)
	§ Tosi
	§ Tosi %(erikoistapaus: $p$ tai alkuperäinen luku (tai molemmat) on arvoltaan nolla)
	§ Epätosi
	§ Epätosi
	§ Epätosi
 	}
	\end{vastaus}
\end{tehtava}
\end{multicols}

\newpage

\subsubsection*{Yhtälöt}
\begin{multicols}{2}

\begin{tehtava}
\alakohdat{
§ Jos keskinopeus nousee $5$\,\%, matka-aika lyhenee $5$\,\%.
§ Jos $a$ on $p$ prosenttia suurempi kuin $b$, niin $b$ on $p$ prosenttia pienempi kuin $a$.
§ Yhtälö $2(3x^2-x-1)=3(2x^2-x-1)$ on ensimmäisen asteen yhtälö.
§ Verrannollisuuskertoimen lukuarvoa ei tarvitse tuntea, jos halutaan selvittää vain suhteellisia (esimerkiksi prosentuaalisia) muutoksia.
§ Yhtälöllä voi olla äärettömän monta juurta.
§ Yhtälöllä $y=y-1$ ei ole ratkaisuja.
§ Jos muuttuja $x$ on kääntäen verrannollinen muuttujaan $y$, niin $y$ on myös kääntäen verrannollinen $x$:ään.
§ Jos Pythagoraan lauseena tunnetusta yhtälöstä $a^2+b^2=c^2$ ratkaistaan $b$, kyseessä on potenssiyhtälö.
§ Erikoistapauksessa on mahdollista, että jos lukua kasvatetaan ensin $p$ prosenttia ja sitten pienennetään $p$ prosenttia, päädytään takaisin alkuperäiseen lukuun.
§ Jos punaisia palloja on $50$ prosenttia enemmän kuin keltaisia palloja, niin keltaisia palloja on $50$ prosenttia vähemmän kuin punaisia palloja.

}
	\begin{vastaus}
	\alakohdatm{
	§ Epätosi
	§ Epätosi
	§ Tosi
	§ Tosi
	§ Tosi
	§ Tosi
	§ Tosi
	§ Tosi
	§ Tosi
	§ Epätosi
	}
	\end{vastaus}
\end{tehtava}
\end{multicols}

\subsubsection*{Funktiot}
\begin{multicols}{2}
\begin{tehtava}
\alakohdat{
§ Jos funktio on relaationa symmetrinen, sen määrittely- ja arvojoukko ovat samat.
§ Funktiot $f$ ja $g$ ovat täsmälleen samat, jos ne määritellään kaavoilla $f(z)=1$ ja $g(z)=\frac{z}{z}$.
§ Jos eksponenttifunktion kantalukuna on positiivinen reaaliluku, ja funktion määrittelyjoukko on $\mathbb{R}$, tällöin funktion arvojoukko on myös $\mathbb{R}$.
}
	\begin{vastaus}
	\alakohdatm{
	§ Tosi
	§ Epätosi
	§ Epätosi
	}
	\end{vastaus}
\end{tehtava}

\end{multicols}
\newpage