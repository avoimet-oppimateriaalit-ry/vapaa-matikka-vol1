\begin{tehtavasivu}

\begin{tehtava}%Laati Henri Ruoho 10-11-2013
Sijoita lukusuoralle luvut
\[
\mbox{$-3$, $1$, $4$, $\sqrt{2}$, $\sqrt{3}$, $\pi$ ja $\sqrt{\pi}$.}
\]
	\begin{vastaus}
Järjestys on \mbox{$-3$,$1$,$\sqrt{2}$,$\sqrt{3}$,$\sqrt{\pi}$,$\pi,4$}
	\end{vastaus}
\end{tehtava}

\begin{tehtava} %Laati Henri Ruoho 10-11-2013
Piirrä samankokoiset kuvat lukusuoran väleistä, joiden päätepisteet ovat 
\alakohdat{
§ $0$ ja $1$
§ $0$ ja $0,1$
§ $0$ ja $0,01$.
}
Miten b-kohdan väli sijoittuu a-kohdan välille? Entä c-kohdan väli b-kohdan välille? Mikä olisi samaa logiikkaa käyttäen seuraava väli ja miten se sijoittuisi c-kohdan välille? Havainnollista piirtämällä.
	\begin{vastaus}
%\alakohdat{
%§
\begin{kuva}
	lukusuora.pohja(-0.2,1.2,6)
	lukusuora.piste(0)
	lukusuora.piste(1)
	lukusuora.vali(0,1)
\end{kuva}
%§
\begin{kuva}
	lukusuora.pohja(-0.2,1.2,6)
	lukusuora.piste(0)
	lukusuora.piste(0.1)
	lukusuora.vali(0,0.1)
\end{kuva}
%§
\begin{kuva}
	lukusuora.pohja(-0.2,1.2,6)
	lukusuora.piste(0)
	lukusuora.piste(0.01)
	lukusuora.vali(0,0.01)
\end{kuva}
%}
	\end{vastaus}
\end{tehtava}

\begin{tehtava}
Mihin joukoista $\nn$, $\zz$, $\qq$, $\rr$ seuraavat luvut kuuluvat?
\alakohdat{
§ $-5$
§ $\frac82$
§ $\pi$
§ $0,888\ldots$
§ $2,995$
§ $\sqrt{2}$
}
	\begin{vastaus}
\alakohdat{
§ $\zz$, $\qq$ ja $\rr$
§ $\nn$, $\zz$, $\qq$ ja $\rr$
§ vain $\rr$
§$\qq$ ja $\rr$
§ $\qq$ ja $\rr$
§ vain $\rr$
}
	\end{vastaus}
\end{tehtava}

\begin{tehtava}
Ovatko seuraavat luvut rationaalilukuja vai irrationaalilukuja? Kunkin desimaalit noudattavat yksinkertaista sääntöä.
\alakohdat{
§ $0,123456789101112131415\ldots$
§ $2,415115115115115115115\ldots$
§ $1,010010001000010000010\ldots$
}
\begin{vastaus}
\alakohdat{
§ irrationaaliluku
§ rationaaliluku (jakso on 151)
§ irrationaaliluku
}
\end{vastaus}
\end{tehtava}

\begin{tehtava}%Laati Henri Ruoho 10-11-2013
\alakohdat{
§ Luku $144$ on luonnollinen luku ja siten reaaliluku. Onko sen vastaluku luonnollinen luku?
§ Luku $144$ on kokonaisluku ja siten reaaliluku. Onko sen käänteisluku kokonaisluku?
§ Luku $144$ on rationaaliluku ja siten reaaliluku. Onko sen käänteisluku rationaaliluku?
}
	\begin{vastaus}
\alakohdat{
§ Ei ole
§ Ei ole
§ On
}
	\end{vastaus}
\end{tehtava}

\begin{tehtava}
Mikä on pienin lukua $-3$ suurempi luku
\alakohdat{§ kokonaislukujen § luonnollisten lukujen § reaalilukujen joukossa?}
	\begin{vastaus}
\alakohdat{§ $-2$ § $0$ § Sellaista ei ole. Jos nimittäin $a > -3$, niin keskiarvo $\frac{-3+a}{2}$ on vielä lähempänä lukua $-3$.}
	\end{vastaus}
\end{tehtava}

\begin{tehtava}
Kuinka monta 
\alakohdat{
§ luonnollista lukua
§ kokonaislukua
§ reaalilukua
}
on reaalilukujen $-\pi$ ja $\pi$ välillä?
	\begin{vastaus}
\alakohdat{
§ $4$
§ $7$
§ äärettömän monta
}
	\end{vastaus}
\end{tehtava}

\begin{tehtava}
$\star$ Osoita, että
\alakohdat{
§ jokaisen kahden rationaaliluvun välissä on rationaaliluku
§ jokaisen kahden rationaaliluvun välillä on irrationaaliluku
§ jokaisen kahden irrationaaliluvun välissä on rationaaliluku
§ jokaisen kahden irrationaaliluvun välissä on irrationaaliluku.
§ Perustele edellisten kohtien avulla, että minkä tahansa kahden luvun
välissä on äärettömän monta rationaali- ja irrationaalilukua.
}
	\begin{vastaus}
Vihjeitä:
\alakohdat{
§ keskiarvo
§ $\sqrt{2}$ on irrationaaliluku. Käytä painotettua keskiarvoa. (Joudut luultavasti etsimään lisätietoja.)
§ pyöristäminen
§ hyödynnä c-kohtaa
§ keskiarvot %painotettua keskiarvoa ei ole opetettu!
}
Ohjeita:
\alakohdat{
§ $\frac{a+b}{2}$
§ $\frac{\sqrt{2}a+b}{\sqrt{2}+1}$
§ Pyöristä pienempi luku ylöspäin sen desimaalin tarkkuudella, jossa lukujen kymmenjärjestelmäesitykset eroavat ensimmäisen kerran.
§ Jos rationaaliluku $r$ on irrationaalilukujen $a$ ja $b$ välissä, niin $\frac{a+r}{2}$ kelpaa.
§ Jos kahden rationaaliluvun välissä olisi vain äärellinen määrä rationaalilukuja, löytyisi kahden vierekkäisen välistä a-kohdan perusteella vielä yksi. Koska kahden rationaaliluvun välissä on ääretön määrä rationaalilukuja, b-kohdasta seuraa, että niiden välissä on äärettömän monta irrationaalilukua. Irrationaalilukujen välissä olevien rationaali- ja irrationaalilukujen äärettömyys seuraa vastaavasti.
}
	\end{vastaus}
\end{tehtava}

\end{tehtavasivu}