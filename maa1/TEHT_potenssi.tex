\begin{tehtavasivu}

\subsubsection*{Opi perusteet}

\begin{tehtava}
Mikä seuraavissa potenssimerkinnöissä on kantaluku, eksponentti ja potenssin arvo?
    \alakohdatm{
	§ $5^3$
	§ $2^5$
	§ $9$
	§ $1$
	§ $(-4)^3$
	§ $(-1)^{10}$
	§ $6^{-2}$
    }
	\begin{vastaus}
 \alakohdat{
 	§ kantaluku $5$, eksponentti $3$, potenssin arvo $125$
	§ kantaluku $2$, eksponentti $5$, potenssin arvo $32$
	§ kantaluku $9$, eksponentti $1$, potenssin arvo $9$
	§ kantaluku $1$, eksponentti $1$, potenssin arvo $1$
	§ kantaluku $-4$, eksponentti $3$, potenssin arvo $-64$
	§ kantaluku $-1$, eksponentti $10$, potenssin arvo $1$
	§ kantaluku $6$, eksponentti $-1$, potenssin arvo $\frac{1}{36}$
 }
	\end{vastaus}
\end{tehtava}

 \begin{tehtava}
        Esitä potenssin avulla.
        \alakohdatm{
        § $a\cdot a\cdot a$
        § $a\cdot a\cdot a\cdot b\cdot b\cdot b\cdot b$
        § $a\cdot b\cdot a\cdot b\cdot a\cdot b\cdot a$
        }
        
        \begin{vastaus}
        \alakohdatm{
            § $a^3$
            § $a^3b^4$
            § $a^4b^3$
        }
        \end{vastaus}
    \end{tehtava}

\begin{tehtava}
    Sievennä lausekkeet.
        \alakohdatm{
        	§ $a^2\cdot a^3$ 
       		§ $(a^2)^3$
        	§ $\frac{a^7}{a^5}$
        	§ $a^0$
      		§ $a^3\cdot a^2\cdot a^5$
       		§ $(a^2a^3)^4$
 		}
        \begin{vastaus}
        \alakohdatm{
            § $a^5$
            § $a^6$
            § $a^2$
            § $1$, ei määritelty kun $a=0$
            § $a^{10}$
            § $a^{20}$
        }
        \end{vastaus}
\end{tehtava}
    
\begin{tehtava}
        Laske
        \alakohdatm{
        § $2^3\cdot2^3$
        § $4^3$
        § $(2^2)^3$
        § $2^{2+2+2}$.
        }

        \begin{vastaus}
        \alakohdatm{
            § $64$
            § $64$
            § $64$
            § $64$
            }
        \end{vastaus}
\end{tehtava}    
    
\begin{tehtava}
    Laske
        \alakohdatm{
        § $2^4$ 
       	§ $(-2)^4$
        § $-2^4$
 		}
        \begin{vastaus}
        \alakohdatm{
            § $16$
            § $16$
            § $-16$
        }
        \end{vastaus}
\end{tehtava}    
    
\begin{tehtava}
  		Laske
        \alakohdatm{
            § $-7^2$
            § $(-7)^2$
        	§ $7^{-2}$
        	§ $(-7)^{-2}$.
 		}
        \begin{vastaus}
        \alakohdatm{
            § $-49$
            § $49$
            § $\frac{1}{7^2}=\frac{1}{49}$
            § $\frac{1}{(-7)^2}=\frac{1}{49}$
        }
        \end{vastaus}
\end{tehtava}

%        \begin{tehtava}
%    Muunna lauseke muotoon, jossa ei ole sulkuja.
%        \alakohdat{
%        	§ $(ab)^2$
%       		§ $\left( \frac{a}{b} \right)^3$
% 		}
%        \begin{vastaus}
%        \alakohdat{
%            § $a^2b^2$
%            § $\frac{a^3}{b^3}$
%        }
%        \end{vastaus}
%    \end{tehtava}

\begin{tehtava}
  		Kirjoita murtolukuna
        \alakohdatm{
        	§ $3^{-1}$
      		§ $5^{-2}$
      		§ $\left(\frac{1}{7}\right)^{-1} $
        	§ $\left(\frac{3}{4}\right)^{-1}$.

 		}
       \begin{vastaus}
        \alakohdatm{
            § $\frac{1}{3}$
            § $\frac{1}{25}$
            § $\frac{7}{1}$
            § $\frac{4}{3}$
        }
        \end{vastaus}
\end{tehtava}

\begin{tehtava}
  		Sievennä
        \alakohdatm{
        	§ $b^3(ab^0)^2$
       		§ $(ab^3)^0$
        	§ $(aa^4)^3a^2$.
 		}
        \begin{vastaus}
        \alakohdatm{
            § $a^2b^3$
            § $1$ 
            § $a^{17}$
        }
        \end{vastaus}
\end{tehtava}

\begin{tehtava}
        Laske.
                \alakohdatm{
        § $\frac{a^3}{a^2}$
        § $\frac{b^4}{b^2}$
        § $\frac{c^3}{c^1}$
        § $\frac{d^3}{d^0}$
        § $\frac{e^3}{e^4}$
        § $\frac{f^3}{f^5}$
       	§ $\frac{k^3}{k^{-5}}$
        }
        \begin{vastaus}
                \alakohdatm{
            § $a$
            § $b^2$
            § $c^2$
            § $d^3$
            § $e^{-1}$
            § $f^{-2}$
            § $k^8$
            }
        \end{vastaus}
\end{tehtava}  
    
\begin{tehtava}
Esitä lausekkeet sievennettynä ilman sulkeita.
\alakohdatm{
§ $x^2(x-1)$
§ $k^3(k^2-3k)$
§ $t^{100}(\frac{1}{t^{98}}-t^2-1)$
}
	\begin{vastaus}
	\alakohdatm{
	§ $x^3-x^2$
	§ $k^5-3k^4$
	§ $t^2-t^{102}-t^{100}$
	}
	\end{vastaus}
\end{tehtava}
    
\begin{tehtava}
Sievennä.
	\alakohdatm{
	§ $\frac{2x^3}{x}$
	§ $\frac{3x^3y^2}{xy}$
	§ $\frac{x^2yz}{xy^2}$
	§ $\frac{6xy^3z^2}{2xz}$
	}
	\begin{vastaus}
	\alakohdatm{
	§ $2x^2$
	§ $3x^2y$
	§ $\frac{xz}{y}$
	§ $3y^3z$
	}
	\end{vastaus}
\end{tehtava}
   
\subsubsection*{Hallitse kokonaisuus}

\begin{tehtava}
  		Sievennä
        \alakohdatm{
        	§ $(a^2)^{-2}$
       		§ $(ab^{-1})^3$
 		}
        \begin{vastaus}
        \alakohdatm{
            § $\frac{1}{a^4}$
            § $\frac{a^3}{b^3}$
        }
        \end{vastaus}
\end{tehtava}

\begin{tehtava}
  		Sievennä
        \alakohdatm{
        	§ $\frac{10^{n+1}}{10\cdot 10^n} \cdot 10^{-1}$
       		§ $\frac{125^3}{5^4}$
        	§ $\frac{2\cdot 4^n}{4^2}$
        	§ $\frac{8^{2i}}{16^{-3}}$.
 		}
        \begin{vastaus}
        \alakohdatm{
            § $\frac{1}{10}$
            § $5^5 = 3\,125$
            § $2^{2n-3}$
            § $2^{6i+12}$
        }
        \end{vastaus}
\end{tehtava}

\begin{tehtava}
  		Sievennä.
        \alakohdatm{
       		§ $\left(\frac{a^2b^{-2}}{a^2b}\right)^{-3}$
        	§ $\frac{5a^2}{-15a}$
			§ $\left(-(\frac{10^3}{100b})^2 b^{-1} \right )^2$
 		}
        \begin{vastaus}
        \alakohdatm{
            § $b^9$ 
            § $-\frac{1}{3}a = -\frac{a}{3}$
            § $\frac{10\,000}{b^6}$
        }
        \end{vastaus}
\end{tehtava}

\begin{tehtava}
    Sievennä.
        \alakohdatm{
        	§ $a^2(-a^4) $ 
        	§ $(ab^2)^0$ 
        	§ $(3a)^3$ 
        	§ $(a^5b^3)^3$
		}    
        \begin{vastaus}
        \alakohdatm{
            § $-a^6$ 
            § $1$ 
            § $27a^3$ 
            § $a^{15}b^9$
        }
        \end{vastaus}
\end{tehtava}

\begin{tehtava}
        Sievennä.
                \alakohdatm{
        § $(-\frac{ab^2}{a^2b})^3$ 
        § $(-a^4b^4)^2$ 
        § $\left((\frac{a}{b})^4\right)^2$
        }
        \begin{vastaus}
                \alakohdatm{
            § $-\frac{b^3}{a^3}$ 
            § $a^8b^8$ 
            § $\frac{a^8}{b^8}$
            }
        \end{vastaus}
\end{tehtava}
    
\begin{tehtava}
Esitä lausekkeet sievennettynä ilman sulkeita.
\alakohdatm{
§ $(y-1)(y+1)$
§ $(2s^2-1)(1-s)$
§ $(z^2+2)(z^2+2)$
}
	\begin{vastaus}
	\alakohdatm{
	§ $y^2-1$
	§ $-2s^3+2s^2+s-1$
	§ $z^4+4z^2+4$
	}
	\end{vastaus}
\end{tehtava}

\begin{tehtava}
Sievennä.
	\alakohdatm{
	§ $\frac{2x^5+3x^3}{x^2}$
	§ $\frac{6x^2+8y}{2x^2}$
	§ $\frac{3x-2x^2y^3}{xy}$
	§ $\frac{2x^2+3xy^2z-4xz}{2xy^2z}$
	}

\begin{vastaus}
	\alakohdatm{
	§ $2x^3+3x$
	§ $3+4 \frac{y}{x^2}$
	§ $\frac{3}{y} - 2xy^2$
	§ $\frac{x}{y^2z} + \frac{3}{2} + \frac{2}{y^2}$
	}
\end{vastaus}
\end{tehtava}

\begin{tehtava}
Sievennä ryhmittelemällä.
	\alakohdat{
	§ $2x^2+3x+5x^2$
	§ $x^2+3x^3+x^2+x^3+2x^2$
	§ $ax^2+bx+cx$
	§ $ax^3+bx+cy^3+dx+ey^3+fx^3$
	}
	\begin{vastaus}
	\alakohdat{
	§ $7x^2+3x$
	§ $4(x^2+x^3)$ tai $4x^2+4x^3$
	§ $ax^2+(b+c)x$ tai $ax^2+bx+cx$
	§ $(a+f)x^3+(b+d)x+(c+e)y^3$
	}
	\end{vastaus}
\end{tehtava}

\begin{tehtava}
	Millä kokonaisluvun $n$ arvoilla
	\alakohdatm{
		§ $2^n$
		§ $(-3)^n$
		§ $(-1)^{n-1}$
		§ $(-1)^{n-1}(-2)^n$
	}
	on positiivinen?
	\begin{vastaus}
		\alakohdat{
			§ kaikilla kokonaisluvuilla
			§ parillisilla kokonaisluvuilla
			§ parittomilla kokonaisluvuilla
			§ ei millään kokonaisluvulla
		}
	\end{vastaus}
\end{tehtava}

\begin{tehtava}
	Laske
	\alakohdat{
		§ $\left(\frac{4}{8}\right)^{1\,543} 2^{1\,546}$
		§ $\left(\frac{28}{15}\right)^{214} \left(\frac{45}{98}\right)^{109} \left(\frac{5}{8}\right)^{105}$.
	}
	\begin{vastaus}
		\alakohdatm{
			§ $8$
			§ $\left(\frac{2\cdot 3}{7}\right)^4 = \frac{1\,296}{2\,401}$
		}
	\end{vastaus}
\end{tehtava}

\begin{tehtava} %tarkista biografiat
Eukleides Aleksandrialainen oli kuuluisa, noin vuonna $300$ eaa. elänyt kreikkalainen matemaatikko. Hän esitti \textit{Alkeet}-kirjassaan kaavan täydellisten lukujen systemaattiselle laskemiselle (ks. luvun Jaollisuus tehtävät). Nykykielellä ilmaistuna se menisi seuraavasti: 
$$\frac{p(p+1)}{2} \text{ on täydellinen luku, kun}$$
$$p \text{ on muotoa } 2^n-1 \text{ oleva alkuluku.}$$

Muotoa $2^n-1$ olevia alkulukuja, missä todistettavasti välttämättäkin myös $n$ on alkuluku, kutsutaan \termi{Mersennen alkuluku}{Mersennen alkuluvuiksi}. Vasta 1700-luvulla sveitsiläinen Euler todisti, että kaikki parilliset täydelliset luvut voidaan esittää kyseisessä muodossa.
\alakohdat{
§ Laske kaavan avulla kolmas yhtä suurempi täydellinen luku.
§ Osoita, että Eukleideen kaava on yhtäpitävä kaavan $2^{n-1}(2^n-1)$ kanssa, missä $2^n-1$ on Mersennen alkuluku.
}
	\begin{vastaus}
	\alakohdat{
	§ Kolmas alkuluku on $5$, sijoittamalla alkulukukaavaan $n=5$, saadaan $p=2^5-1=31$, josta edelleen täydelliseksi luvuksi $p(p+1)/2=31(31+1)/2=31\cdot32/2=496$.
	§ Sijoitetaan $p=2^n-1$ kaavaan $p(p+1)/2$, niin saadaan sieventämällä $p(p+1)/2=(2^n-1)(2^n-1+1)/2=(2^n-1)2^n/2=2^n(2^n-1)/2=\frac{2^n}{2}(2^n-1)=2^{n-1}(2^n-1)$.
	}
	\end{vastaus}
\end{tehtava}

\subsubsection*{Lisää tehtäviä}

 \begin{tehtava}
        Sievennä
                \alakohdatm{
        § $(1\cdot a)^3$ 
        § $(a\cdot 2)^2$ 
        § $(-2abc)^3$ 
        § $(3a)^4$.
        }

        \begin{vastaus}
                \alakohdatm{
            § $a^3$ 
            § $4a^2$ 
            § $-8a^3b^3c^3$ 
            § $91a^4$
            }
        \end{vastaus}
\end{tehtava}
\begin{tehtava}
        Sievennä.
                \alakohdatm{
        § $-a^3\cdot(-a^2)$ 
        § $a\cdot(-a)\cdot(-b)$ 
        § $a^2\cdot(-a^2)$
        }
        \begin{vastaus}
                \alakohdatm{
            § $a^5$ 
            § $a^2b$ 
            § $-a^4$
            }
        \end{vastaus}
    \end{tehtava}

    \begin{tehtava}
        Sievennä.
                \alakohdatm{
        § $(a^3b^2)^2$ 
        § $a(a^2b^3)^4$ 
        § $(b^2a^4)^5$ 
        § $b(2ab^2)^3$
        }
        \begin{vastaus}
                \alakohdatm{
            § $a^6b^4$
            § $a^9b^{12}$ 
            § $a^{20}b^{10}$ 
            § $8a^3b^7$
            }
        \end{vastaus}
    \end{tehtava}
      
    \begin{tehtava}
        Sievennä lausekkeet.
        \alakohdatm{
        § $\frac{a^2b^2}{ab}$ 
        § $\frac{a^2b}{a^2}$ 
        § $\frac{a^3}{a^3}$ 
        § $\frac{1}{a^0}$ 
        § $\frac{ab^3}{-b^4}$
        }
        \begin{vastaus}
        \alakohdatm{
            § $ab$ 
            § $b$ 
            § $1$ 
            § $1$ 
            § $-\frac{a}{b}$
            }
        \end{vastaus}
    \end{tehtava}
    
\begin{tehtava}
         Sievennä ja kirjoita potenssiksi, jonka eksponentti on positiivinen
         \alakohdatm{
        § $a^{-3}$ 
        § $\frac{a}{a^3}$ 
        § $a^{-2}\cdot a^5$ 
        § $\frac{b}{a^4}b^{-4}$ 
        § $\frac{a^3}{a^{-5}}$.
        }
        \begin{vastaus}
        \alakohdatm{
            § $\frac{1}{a^3}$ 
            § $\frac{1}{a^2}$ 
            § $a^3$ 
            § $\frac{1}{a^4b^3}$ 
            § $a^8$
            }
        \end{vastaus}
\end{tehtava}
    
\begin{tehtava}
Sievennä.
            \alakohdatm{
        § $a^0$
        § $a^0a^0$
        § $aa^1$
        § $aa^0$
        § $a^0a^1$
        }
        \begin{vastaus}
                \alakohdat{
            § $1$  ($a\neq0$, koska $0^0$ ei ole määritelty)
            § $1$ ($a\neq0$, koska $0^0$ ei ole määritelty)
            § $a^2$
            § $a$ ($a\neq0$, koska $0^0$ ei ole määritelty)
            § $a$ ($a\neq0$, koska $0^0$ ei ole määritelty)
            }
        \end{vastaus}
\end{tehtava}
    
\begin{tehtava}
        Esitä ilman sulkuja ja sievennä.
       \alakohdatm{
        § $(\frac{1}{2})^2$ 
        § $(\frac{1}{3})^3$ 
        § $(\frac{a}{b})^4$ 
        § $(\frac{a^2}{b^3})^2$ 
        § $\left(\frac{a^2}{ab^2}\right)^2$.
        }
        
        \begin{vastaus}
        \alakohdatm{
            § $\frac{1}{4}$ 
            § $\frac{1}{27}$ 
            § $\frac{a^4}{b^4}$ 
            § $\frac{a^4}{b^6}$ 
            § $\frac{a^2}{b^4}$
            }
        \end{vastaus}
\end{tehtava}

\begin{tehtava}
Laske tai sievennä.
        \alakohdatm{
        	§ $a^2a^5$
        	§ $\frac{a^5}{a^3}$
        	§ $(a^3)^2$ 
        	§ $12^0$
		}
        \begin{vastaus}
        \alakohdatm{
            § $a^7$
            § $a^2$
            § $a^6$
            § $1$
        }
        \end{vastaus}
\end{tehtava}    
    
\begin{tehtava}
        \alakohdatm{
        	§ $\frac{2^7}{2^9}$ 
        	§ $\frac{a^3}{a}$ 
        	§ $\left(\frac{1}{3}\right)^2$ 
        	§ $\left(\frac{a^{-2}}{ab^4}\right)^4$
		}
        \begin{vastaus}
        \alakohdatm{
            § $\frac{1}{4}$ 
            § $a^2$ 
            § $\frac{1}{9} $ 
            § $\frac{1}{a^{12}b^{16}}$ tai $a^{-12}b^{-16}$
        }
        \end{vastaus}
\end{tehtava}     

\begin{tehtava}
        Sievennä
                \alakohdatm{
        § $a^3\cdot b^2\cdot a^5$  
        § $(-ab^3)^2$  
        § $(a^5a^4)^3$  
        § $10^{2^3}$.
		}
        \begin{vastaus}
                \alakohdatm{
            § $a^8b^2$ 
            § $a^2b^6$
            § $a^{15}b^{12}$
            § $10^8 = 100\,000\,000$
            }
        \end{vastaus}
\end{tehtava}

\begin{tehtava}
     Sievennä.
             \alakohdatm{
        § $(-a)\cdot(-a)$ 
        § $(-a)\cdot(-a)\cdot(-b)^3$ 
        § $(-a^2)\cdot(-a)^2$
		}
        \begin{vastaus}
                \alakohdatm{
            § $a^2$ 
            § $-a^2b^3$ 
            § $-a^4$
            }
        \end{vastaus}
\end{tehtava}
    
\begin{tehtava}
    \alakohdatm{
        § $a^2\cdot a^3$
        § $a^3a^2$
        § $a^2 a$
        § $a a^2 a$
        § $a^2a^1a^3$
        }
        \begin{vastaus}
        \alakohdatm{
            § $a^5$
            § $a^5$
            § $a^3$ 
            § $a^4$
            § $a^6$
            }
        \end{vastaus}
\end{tehtava}
    
\begin{tehtava}
        Sievennä.
                \alakohdatm{
        § $a^1aa^2$ 
        § $aaaa$ 
        § $a^3ba^2$ 
        § $aba^0ba^1$
        }
        \begin{vastaus}
                \alakohdatm{
            § $a^4$ 
            § $a^4$ 
            § $a^5b$ 
            § $a^2b^2$
            }
        \end{vastaus}
\end{tehtava}
    
\begin{tehtava}
    Kerrotaan $x$:n toinen potenssi $y$:n neliöllä, vähennetään siitä $x$:n ja $y$:n toisen potenssin erotus kerrottuna kahdeksalla. Koko lauseke jaetaan piillä. Mitä saamme näin tekemällä aikaiseksi?
    	\begin{vastaus}
    	Onnettoman elämän. Jotain kaksi.\footnote{Ketosen ja Myllyrinteen mukaan: https://www.youtube.com/watch?v=NwNzYJuXzZY \\ Joidenkin mielestä sievin muoto onkin $\frac{x^2y^2-(x-y^2)\cdot 8}{\pi}=\frac{1}{\pi}(x^2y^2-8x+8y^2)=\frac{y^2}{\pi}(x^2-8x+8)$}
    	\end{vastaus}
\end{tehtava}

\begin{tehtava}
%Laatinut ja ratkaissut Matias Jalkanen 10.11.2013 – laajentanut JoonasD6
Tarinan mukaan muuan intialainen ruhtinas pyysi erästä matemaatikkoa kehittämään uuden strategisen lautapelin ja luvannut hänelle pelin keksimisestä suuren palkkion. Tällöin matemaatikko keksi shakin. Ruhtinas ihastui peliin ja kysyi keksijältä, mitä tämä halusi palkkioksi. Keksijä pyysi palkkioksi niin monta vehnän jyvää kuin saadaan koko shakkilaudalta, jos niitä asetetaan sen ensimmäiselle ruudulle yksi, toiselle ruudulle kaksi, kolmannelle neljä, neljännelle kahdeksan ja edelleen jokaiselle ruudulle kaksi kertaa niin monta kuin edelliselle ruudulle.
	\alakohdat{
		§ Kuinka monta vehnän jyvää tulee viidennelle ruudulle?
		§ Kuinka monta vehnän jyvää viimeiselle ruudulle tulee?
		§ Kuinka monta vehnän jyvää tulee $n$:nnelle ruudulle?
		§ $\star$ Kuinka montaa vehnän jyvää keksijä pyysi yhteensä?
	}
	\begin{vastaus}
	      \alakohdat{
		    § $2^4 = 16$
		    § $2^{63}$
		    § $2^{n-1}$
		    § $2^{64} - 1$ \\ $= 18\,446\,744\,073\,709\,551\,615$ (Huomataan, että kullakin ruudulla on yksi jyvä vähemmän kuin kaikilla edellisillä ruuduilla yhteensä.)
	      }
	\end{vastaus}
\end{tehtava}

\begin{tehtava} $\star$
Sievennä $(2ab+4b^2-8b^3c):(a+2b-4b^2c)$.
	\begin{vastaus}
	$2b$
	\end{vastaus}
\end{tehtava}
 
\begin{tehtava}
$\star$ Niin sanottu \termi{tetraatio}{tetraatio} on lyhennysmerkintä ''potenssitornille'', jossa esiintyy vain yhtä lukua. Se määritellään kaavalla
\[^na = \underbrace{{a^{a^{a^{\mathstrut^{.^{.^{.^{a}}}}}}}}}_{n\,\textrm{kpl}}.\]
Laske
\alakohdat{
§ $^42$
§ $^35$.
%§ Onko $^0a$ yksikäsitteisesti määritelty? ($a$ ei ole nolla.)
%§ $^{-2}(-2)$
	}
	\begin{vastaus}
	\alakohdatm{
§ $^42 = 2^{2^{2^2}}$ \\ $=2^{2^4}=2^{16}$ \\ $=65\ 536$
§ $^35 = 4^{4^4}$ \\ $= 4^{256}$ \\ $\approx 1,34 \cdot 10^{154}$
%§ 
%§ 
	}
	\end{vastaus}
\end{tehtava}

\begin{tehtava}
  		$\star$ Sievennä lauseke
$$\left[ \frac{(a^2b^{-2}c)^{-3}:\left(c^2\cdot (ab^{-2})^0 \cdot a^{-4}\right)}
{\left((c^{-1}\cdot a^3)^{-1}:a^2\right)^3(ab^2c^{-3})^3} \right]^2.$$
	\begin{vastaus}
	$a^{20}c^2$
	\end{vastaus}
\end{tehtava} 

\begin{tehtava}
$\star$ Kompleksilukujen niin sanottu \termi{imaginääriyksikkö}{imaginääriyksikkö} $i$ määritellään seuraavalla (täysin kaikista reaaliluvuista poikkeavalla) ominaisuudella:

$$i^2=-1$$

Sievennä määritelmän avulla seuraavat lausekkeet.
\alakohdat{
§ $i^3$
§ $i^4$
§ $i^{4k+1}$, missä $k$ on kokonaisluku
§ $i^{1\,001}$
§ $\frac{1}{i}$ (Ohje: voit laventaa.)
§ $(1+i)(1-i)$
§ $(x+iy)(x-iy)$, missä $x$ ja $y$ ovat mielivaltaisia reaalilukuja
}
	\begin{vastaus}
	\alakohdatm{
§ $-i$
§ $1$
§ $i$
§ $i$
§ $-i$
§ $2$
§ $x^2+y^2$
}
	\end{vastaus}
\end{tehtava}

\end{tehtavasivu}