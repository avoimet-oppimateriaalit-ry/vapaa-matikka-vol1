Seuraavaksi tutkitaan potenssin käsitteen laajentamista tilanteeseen, jossa eksponenttina on rationaaliluku – erityisesti murtoluku. Esimerkiksi voidaan pohtia, mitä tarkoittaa merkintä $2^\frac{1}{3}$. Potenssin laskusääntöjen perusteella
\[
(2^{m})^n = 2^{mn},\textrm{ kun }m,n\in \zz.
\]
Siten on luonnollista ajatella, että \termi{murtopotenssi}{murtopotenssille} $2^\frac{1}{3}$ pätee
\[
(2^\frac{1}{3})^3 = 2^\frac{3}{3} = 2^1=2.
\]
Koska luvun $2$ kuutiojuuri toteuttaa yhtälön $(\sqrt[3]{2})^3=2$, täytyy siis asettaa $2^\frac{1}{3}=\sqrt[3]{2}$. Yleisemmin asetetaan $a^\frac{1}{n} =\sqrt[n]{a}$, kun $n$ on positiivinen kokonaisluku ja $a\ge 0$.

Kaavan on edelleen luontevaa ajatella yleistyvän niin, että esimerkiksi
\[
(2^{\frac{1}{3}})^2 = 2^{\frac{2}{3}}.
\]
Yleisemmin otetaan murtopotenssin $a^\frac{m}{n}$ määritelmäksi
\[
a^\frac{m}{n} = (a^{\frac{1}{n}})^m = (\sqrt[n]{a})^m,
\]
kun $m$ ja $n$ ovat kokonaislukuja ja $n>0$. 

\laatikko[Murtopotenssimerkinnät]{
\[
a^\frac{1}{n} = \sqrt[n]{a},\textrm{ kun }a\geq 0. 
\]
Erityisesti $a^\frac{1}{2}=\sqrt{a}$.
\[
a^\frac{m}{n} =  (\sqrt[n]{a})^m,\textrm{ kun }a > 0.
\]
}

Kun murtolukueksponentit määritellään näin, kaikki aikaisemmat potenssien laskusäännöt ovat sellaisenaan voimassa myös niille. Esimerkiksi kaavat
\[ a^q\cdot a^q = a^{p+q}, \quad (a^p)^q = a^{pq}, \quad (ab)^q=a^qb^q \]
pätevät kaikille rationaaliluvuille $p$ ja $q$.

\laatikko[Huomautus määrittelyjoukosta]{Murtopotenssimerkintää käytettäessä vaaditaan, että $a>0$ myös silloin, kun $n$ (eli juuren kertaluku) on pariton. Syy tähän on seuraava: Esimerkiksi $\sqrt[3]{-1}=-1$, koska $(-1)^3=-1$, mutta lauseketta $(-1)^\frac{1}{3}$ ei ole tällöin määritelty. Murtopotenssimerkinnän määrittelystä voi tällöin seurata yllättäviä ongelmia:
\[
 -1 = \sqrt[3]{-1} = (-1)^\frac{1}{3} = (-1)^\frac{2}{6}
= ((-1)^2)^\frac{1}{6} = 1^\frac{1}{6} = \sqrt[6]{1} = 1. 
\]
Luvun $\frac{1}{3}$ lavennus muotoon $\frac{2}{6}$ on ongelman ydin, mutta murtolukujen lavennussäännöistä luopuminen olisi myös hankalaa. Siksi sovitaan, ettei murtopotenssimerkintää käytetä, jos kantaluku on negatiivinen.

Nolla kantalukuna taas aiheuttaa ongelmia, jos eksponentti on negatiivinen. Esimerkiksi
\[
 0^{-\tfrac{1}{2}}=\frac{1}{0^\frac{1}{2}}=\frac{1}{0},
\]
jota ei ole määritelty.}

\begin{esimerkki}
Muuta lausekkeet murtopotenssimuotoon.
\alakohdatm{
§ $\sqrt[5]{3}$
§ ${\sqrt[4]{a}}^7, a > 0$
}
	\begin{esimratk}
\alakohdatm{
§ $\sqrt[5]{3} = 3^\frac{1}{5}$
§ $(\sqrt[4]{a})^7 = (a^\frac{1}{4})^7=a^\frac{7}{4}$
}
	\end{esimratk}
\end{esimerkki}

\begin{esimerkki}
Sievennä lausekkeet.
\alakohdatm{
§ $8^\frac23$
§ $\left( \frac12 \right)^{-\frac12}$
}
	\begin{esimratk}
	\alakohdatm{
§ $8^\frac23 = (\sqrt[3]{8})^2 = 2^2 = 4$
§ $\left( \frac14 \right)^{-\frac12}=4^{\frac{1}{2}}=2$
}
 	\end{esimratk}
\end{esimerkki}

Sieventämiskäytäntö (joka pitää samalla huolta määrittelyjoukosta) on, että jos alkuperäinen lauseke on annettu juurimuodossa, niin myös lopetetaan juurimuotoon, ja toisinpäin: jos alkuperäinen lauseke on murtopotenssi, niin lopputulosta ei kirjoiteta juurena.

%esim. + tehtäviä vastaavia esimerkkejä