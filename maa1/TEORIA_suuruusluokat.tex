%Löytyy tällä hetkellä taas yksikköluvusta
%Suuruusluokkien ymmärtäminen (kuvaajien ja tilastojen tulkitsemisen ohella) on tärkeä osa \textit{luku}taitoa -- näiden ymmärtäminen estää sinua tulemasta huijatuksi ja harhaanjohdetuksi.\footnote{Kirjallisuutta aiheesta tarjoaa muun muassa John Allen Paulosin teos Numerotaidottomuus. (Kirja on nimetty hassusti, ottaen huomioon, että siinä erityisesti puhutaan \textit{luvuista}, ei numeroista -- oletamme tämän huonoksi käännökseksi, koska englanninkielen sana \textit{number} tarkoittaa lukua.}
%%monilla menee milligrammat ja mikrogrammat samaan kastiin "vähän"
%
%Kymmenpotenssimuoto eli eksponenttimuoto on merkintätapa, jossa luku ilmoitetaan kertoimen sekä jonkin kymmenen kokonaislukupotenssin tulona. Kymmenpotenssimuodosta on hyötyä, kun halutaan kirjoittaa suuria tai pieniä lukuja lyhyesti.Myös mittaustarkkuuden ilmoittamiseksi tarvitaan joskus kymmenpotenssimuotoa. On yleinen tapa (standardi) valita esitysmuoto niin, että potenssin kerroin on vähintään yksi mutta alle kymmenen. Kymmenpotenssimuotoa $a\cdot 10^n$, missä $a$ on jotain ydehn ja kymmenen väliltä, kutsutaan joskus myös tieteelliseksi merkintätavaksi, engl. \termi{scientific notation}{scientific notation}. Sama luku voidaan kuitenkin esittää kymmenpotensismuodossa äärettömän monella tavalla, jos luovutaan kertoimen suuruusehdosta.
%
%Yleisellä ilmaisulla ``\termi{pilkun siirtäminen}{pilkun siirtäminen}'' tarkoitetaan käytännössä luvun kymmenpotenssiesityksen muokkaamista. Pilkku siirtyy oikealle yhtä monta pykälää kuin kymmenen potenssi pienenee ja päinvastoin. Pilkuttoman luvun tapauksessa pilkku ``kuvitellaan luvun perään''. Koska $10^0 = 1$, voidaan ilman kymmenpotenssia esitetyn luvun perään tarvittaessa lisätä tämä kerroin mielessä. $10$ tarkoittaa luonnollisesti samaa asiaa kuin $10^1$. 
%
%%\begin{esimerkki}
%%Laske luvun $10$ potensseja: $10^1, 10^2, 10^3, 10^4, \ldots$ Kuinka monta nollaa on luvussa $10^n$?
%%	\begin{vastaus}
%%$10^1 = 10, 10^2 = 100, 10^3 = 1\,000, 10^4 = 10\,000$. Luvussa $10^n$ on $n$ kappaletta nollia. 
%%	\end{vastaus}
%%\end{esimerkki}
%
%\laatikko[Joitakin esitysmuotoja samoille luvuille]{
%\begin{center}
%\begin{tabular}{c|c|c|c|c|c|c}
%\hline
%$1\,234\,000\cdot10^{-3}$ & $14\cdot10^{-3}$ & $130\,000\,000\,000\cdot10^{-3}$ & $1\cdot10^{-5}$ \\
%$123\,400\cdot10^{-2}$ & $1,4\cdot10^{-2}$ & $13\,000\,000\,000\cdot10^{-2}$ & $0,1\cdot10^{-4}$ \\
%$12\,340\cdot10^{-1}$ & $0,14\cdot10^{-1}$ &  $1\,300\,000\,000\cdot10^{-1}$ &  $0,001\cdot10^{-2}$ \\
%$1\,234$ & $0,014$ & $130,000\,000$ & $0,00001$ \\
%$123,4\cdot10$ & $0,0014\cdot10$ & $130\cdot10^{6}$ & $0,000001\cdot10$ \\
%$12,34\cdot10^{2}$ & $0,00014\cdot10^{2}$ & $1,3\cdot10^{2}$ & $0,0000001\cdot10^{2}$ \\
%$1,234\cdot10^{3}$ & $0,000014\cdot10^{3}$ & $0,13\cdot10^{9}$ & $0,00000001\cdot10^{3}$ \\
%    	 \end{tabular}
%  \end{center}
%}
%\begin{esimerkki}
%\alakohdat{
%§ Maan massa on noin $5\,974\,000\,000\,000\,000\,000\,000\,000$\,kg. Luvussa on ensimmäisen numeron (5) jälkeen vielä $24$ numeroa, eli kymmenpotenssimuodoksi saadaan $5,974\cdot10^{24}$\,kg.
%§ Vetymolekyylissä H$_2$ ytimien välinen etäisyys on noin $0,000000000074$\,m eli $7,4\cdot10^{11}$\,m.  
%§ Jos juoksuradan pituudeksi ilmoitetaan $1\,000$\,m, ei lukija voi tietää, millä tarkkuudella mittaus on tehty. Mittaustarkkuus välittyy kymmenpotenssin avulla: jos etäisyys ilmoitetaan muodossa $1,00\cdot10^{3}$\,m, on mitattu kymmenen metrin tarkkuudella, ja jos muodossa $1,0\cdot10^{3}$\,m, on mitattu sadan metrin tarkkuudella jne.
%}
%\end{esimerkki}
%
%Suurten lukujen nimet kannattaa opetella, ja myös huomata, että eri kielten välillä on joitakin poikkeuksia.
%
%\begin{esimerkki}
%Amerikanenglanniksi miljardi on billion, ja biljoona on trillion.
%\end{esimerkki}
%% Suomessa on käytössä niin sanottu lyhyt asteikko...
%
%%long ja short scale -selitys! +harjoitustehtäviä englannista suomeen ja toisinpäin