\begin{tehtavasivu}

\subsubsection*{Opi perusteet}

\begin{tehtava}
Onko annettu suure skalaarisuure? Jos kyllä, mainitse esimerkkinä jokin yksikkö (koko nimi ja lyhenne), jota kyseisen ominaisuuden mittaamiseen käytetään.
\alakohdat{
§ energia
§ lämpötila
§ kiihtyvyys
§ massa
§ tiheys
}
\begin{vastaus}
\alakohdat{
§ On: joule, J tai kalori, cal
§ On: celsiusaste, \textdegree C tai kelvin, K
§ Ei ole.
§ On: gramma, g
§ On: kilogrammaa per kuutiometri, kg/m$^3$
}
\end{vastaus}
\end{tehtava}

% FIXME \todo{TEHTÄVÄ: sanallinen tehtävä, jossa pitää laskea esim. km ja m yhteen}
\begin{tehtava}
Esitä ilman etuliitettä tai kymmenpotenssimuotoa.
\alakohdat{
§ $0,5$\,dl
§ $233$\,mm
§ $33$ senttimetriä
§ $16$\,kg
§ $2$\,MJ
§ $4$ kibitavua
§ $0,125$\,Mit
}
	\begin{vastaus}
\alakohdat{
§ $0,05$\,l
§ $0,233$\,m
§ $0,33$ metriä
§ $16\,000$\,g
§ $2\,000\,000$\,J
§ $4\,096$ tavua
§ $131\,072$\,t
}
	\end{vastaus}
\end{tehtava}

	\begin{tehtava}
	\alakohdat{
§ Elektroniikkayhtiö on ilmoittanut, että laitteen täyteen ladattu akku kestää käyttöä $450$ minuuttia. Laitetta on käytetty lataamisen jälkeen $3$\,h $30$\,min. Kuinka monta tuntia akun voi olettaa vielä kestävän?
§ Elokuva kestää $142$ minuuttia ja näytös alkaa klo $17.30$. Jos elokuvan alussa on lisäksi (tasan) $15$ minuuttia mainoksia, moneltako elokuva loppuu?
} %kuinka monta minuuttia on vuodessa? vastaukseen footnote: lähde: Rent
\begin{vastaus}
\alakohdat{
§ $4$ tuntia
§ kello $21.07$
}
	\end{vastaus}
\end{tehtava}

%teh: esitä kymmenpotenssimuodossa

%taulukko tilavuuden ja pinta-alanyksikkömuunnoksista

\begin{tehtava}
Muuta seuraavat pituudet SI-muotoon ($1$ tuuma$= 2,54$\,cm, $1$ jaardi$=0,914$\,m, $1$ jalka$= 0,305$\,m, $1$ maili$ = 1,609$\,km).
\alakohdat{
§ $5$ tuumaa senttimetreiksi
§ $0,3$ tuumaa millimetreiksi
§ $79$ jaardia metreiksi
§ $80$ mailia kilometreiksi
§ $5$ jalkaa ja $7$ tuumaa senttimetreiksi
§ $330$ jalkaa kilometreiksi
}
	\begin{vastaus}
\alakohdat{
§ $12,7$\,cm
§ $7,62$\,mm
§ $72,206$\,m
§ $128,72$\,km
§ $170,28$\,cm
§ $100,65$\,m
}
	\end{vastaus}
\end{tehtava}
 
\begin{tehtava}
\alakohdat{
§ Laske pöydän pinta-ala neliösenttimetreinä mittaustarkkuus huomioiden, kun suorakaiteen muotoisen pöytälevyn sivujen pituuksiksi on mitattu $72,5$\,cm ja $81,5$\,cm.
§ Kuinka monta litraa maitoa mahtuu $1\,\textrm{m}\times 1\,\textrm{m}\times 1\,\textrm{m}$ -laatikkoon?
}
	\begin{vastaus}
	\alakohdat{
§ $5\,910$\,cm$^2$
§	$1\,000$\,litraa
}
	\end{vastaus}
\end{tehtava}

\begin{tehtava}
Laske ja sievennä.
\alakohdat{
§ $3,0\,\textrm{cm} \cdot 2,5\,\textrm{\,cm} \cdot 6,0\,\textrm{cm}$
§ $0,4\,\textrm{km} : 8\,\textrm{m/s}$
§ $2\,\textrm{g} : 50\,\textrm{mg/ml}$
§ $3,0\,\textrm{ml} \cdot 25\,\textrm{mg/ml}$
§ $150\,\textrm{m} : 3\,\textrm{m/s}$
§ $3,0\,\textrm{g} : 2,0\,\textrm{dm}^2$
}
	\begin{vastaus}
\alakohdat{
§ $45$\,cm$^3$
§ $50$\,s
§ $40$\,ml
§ $75$\,mg
§ $50$\,s
§ $1,5$\,g/dm$^2$
}
		\end{vastaus}
\end{tehtava}

\begin{tehtava}
Laske $4,0\,\mathrm{\frac{MJ}{kg}}\cdot 2,0\,\mathrm{\frac{g}{cm^3}}\cdot 3,0\,\mathrm{mm^3}$.
	\begin{vastaus}
	$24,0\,\mathrm{J}$
	\end{vastaus}
\end{tehtava}

\begin{tehtava}
Ruokakaupassa perunoiden kilohinta on $0,81\,\frac{\textrm{\textrm{€}}}{\textrm{kg}}$ ja banaanien kilohinta $1,04\,\frac{\textrm{€}}{\textrm{kg}}$. Pikku-Kallella on kaupparahaa $5$ euroa, ja hän ostaa kolme kiloa perunoita. Keskimääräisen banaanin massa on $120$ grammaa. Rahamäärät esitetään sentin tarkkuudella ja massat kymmenen gramman tarkkuudella. %kerro jossain teoriaosuudessa, että monivaiheisesta tehtävästä voi kysyä myös vain viimeisen, jolloin pitää itse keksiä, mitä välitietoja ensin pitää selvittää
\alakohdat{
§ Kuinka paljon rahaa perunoihin kuluu?
§ Kuinka paljon rahaa jää perunoiden oston jälkeen jäljelle?
§ Kuinka monta kilogrammaa banaaneja hän voisi vielä ostaa perunoiden lisäksi?
§ Kuinka monta (kokonaista) banaania hän saisi korkeintaan ostettua?
§ Kuinka paljon banaanit maksaisivat?
§ Kuinka paljon rahaa jäisi jäljelle banaanienkin oston jälkeen?
}
	\begin{vastaus}
	\alakohdat{
	§ Perunoihin kuluu rahaa $0,81\,\frac{\textrm{€}}{\textrm{kg}}\cdot 3\,\textrm{kg}=2,43\,\textrm{€}$
	§ $5\,\textrm{€}-2,43\,\textrm{€}=2,57\,\textrm{€}$
	§ $2,57\,\textrm{€}: (1,04\,\frac{\textrm{€}}{\textrm{kg}})=\frac{2,57}{1,04}\,\frac{\textrm{€}}{\frac{\textrm{€}}{\textrm{kg}}}=\frac{2,57}{1,04}\,\textrm{kg}\approx 2,47\,\textrm{kg}$
	§ Käytetään edellisen kohdan \textit{tarkkaa} tulosta: $\frac{2,57}{1,04}\,\textrm{kg}: (120\,\textrm{g})=\frac{2,57}{1,04}\cdot1\,000\,\textrm{g}: (120\,\textrm{g})=\frac{2\,570}{1,04}\,\textrm{g}:(120\,\textrm{g})=\frac{2\,570}{1,04}\,\textrm{g} \cdot \frac{1}{120\,\textrm{g}}=\frac{2\,570}{1,04\cdot 120}\approx 20,6$, eli kokonaisia banaaneja saa ostettua $20$.
	§ $20\cdot 120\,\textrm{g}\cdot 1,04\,\frac{\textrm{€}}{\textrm{kg}}=20\cdot 0,120\,\textrm{kg}\cdot 1,04\,\frac{\textrm{€}}{\textrm{kg}}=20\cdot 0,120\cdot 1,04\,\textrm{€}=2,496\,\textrm{€}\approx 2,50\,\textrm{€}$
	§ $2,57\,\textrm{€}-2,496\,\textrm{€} \approx 0,07\,\textrm{€}$
	}
	\end{vastaus}
\end{tehtava}

\begin{tehtava}
Liuoksen pitoisuus (massakonsentraatio) voidaan laskea kaavalla $c=m/V$, missä $c$ on pitoisuus, $m$ on liuenneen aineen massa ja $V$ on liuoksen tilavuus.
\alakohdat{
§ Jos kaavassa massan yksikkönä on milligrammat ja tilavuuden yksikkönä millilitrat, niin saadaan pitoisuuden yksiköksi?
§ Laske suolaliuoksen pitoisuus, kun $4,5$ grammaa suolaa on liuotettu $500$ millilitraan liuosta. 
} %tarkennus pitoisuuksista (massapitoisuus, ei molaarinen)
	\begin{vastaus}
\alakohdat{
§ $[c]=\textrm{mg/ml}$ (tai sievennettynä g/l)
§ $9$\,mg/ml (tai $9$\,mg/ml)
}
	\end{vastaus}
\end{tehtava}

\begin{tehtava}
Auto ajaa maantiellä nopeudella $90$\,km/h.
\alakohdat{
§ Esitä nopeus metreinä sekuntia kohden.
§ Ihmisen reaktioaika (aika tilanteen huomaamisesta jarrutuksen alkuun) on tavallisesti noin yksi sekunti -- väsyneenä ja humalassa paljon pidempi. Jos kuljettajan reaktioaika on etanolin vaikutuksen alaisena $2,2$ sekuntia ja tielle hyppää hirvi, niin kuinka monta metriä auto on jo ehtinyt edetä ennen kuin jarrutus alkaa?
}
	\begin{vastaus}
	\alakohdat{
	§ $25$\,m/s (tasan)
	§ $55$ metriä (tasan)
	}
	\end{vastaus}
\end{tehtava}

\begin{tehtava}
Maailman nopein kamera vuonna 2014 otti $4,4$ biljoonaa kuvaa sekunnissa.
\alakohdat{
§ Esitä kymmenpotenssimuodossa sievennettynä tarkkana arvona, kuinka monta sekuntia yhden kuvan ottaminen (keskimäärin) kestää.
§ Kuinka monta kokonaista nanosekuntia kestää ottaa kyseisellä kameralla miljoonaa kuvaa?
}
	\begin{vastaus}
	\alakohdat{
	§ $\frac{5}{22}\cdot 10^{-12}$\,s
	§ $227$\,ns
	}
	\end{vastaus}
\end{tehtava}

\begin{tehtava}
USB (engl. \textit{Universal Serial Bus}) on yleinen teknologia erilaisten lisälaitteiden yhdistämiseksi tietokoneisiin. Vuonna 2008 esitelty USB:n versio $3.0$ on yli kymmenen kertaa nopeampi kuin USB $2.0$, ja USB $3.0$:n suurin tiedonsiirtonopeus on jopa $5$\,Gb/s.
\alakohdat{
§ Kuinka nopea USB $3.0$ on (nopeimmillaan) megatavuina sekunnissa?
§ Kuinka kauan kyseisellä nopeudella kestää siirtää $13$\,Git:n kokoinen Blu-ray-elokuva ulkoiselle kiintolevylle? Oletetaan, että mikään muu siirtovaihe ei toimi pullonkaulana.
}
	\begin{vastaus}
	\alakohdat{
	§ $625$\,Mt/s
	§ USB $3.0$:n tiedonsiirtonopeus gigatavuina sekunnissa on $\frac{5}{8}$\,Gb/s$=0,625\,$Gt/s. Elokuvan koko gigatavuina on $13$\,Git$=13\cdot 2^{30}$\,t$\approx 13,959\,$Gt. Jakamalla tämä koko siirtonopeudella saadaan vastaukseksi sekunteina $13,959\,\text{Gt}:0,625\,\text{Gt/s}\approx 22,3$ sekuntia.
	}
	\end{vastaus}
\end{tehtava}

\begin{tehtava}
(YO K00/3) Suorakulmaisen särmiön muotoinen hautakivi on $80$\,cm korkea, $2,10$\,m pitkä ja $32$\,cm leveä. Voidaanko kivi nostaa nosturilla, joka pystyy nostamaan enintään kahden tonnin painoisen kuorman? Hautakiven tiheys on $2,7 \cdot 10^3$\,kg/m$^3$. [Suorakulmaisen särmiön tilavuus on sen kolmen pituusdimension eli korkeuden, leveyden ja syvyyden tulo.]
	\begin{vastaus}
Kyllä voidaan: hautakiven tilavuus on noin $0,54$\,m$^3$, joten se painaa noin $1,5$ tonnia.
	\end{vastaus}
\end{tehtava}

\subsubsection*{Hallitse kokonaisuus}

\begin{tehtava}
%Tehtävän laatinut Johanna Rämö 9.11.2013.
%Ratkaisun tehnyt Johanna Rämö 9.11.2013.
Pudotat pallon kädestäsi lattialle. Pallo pomppaa ensin metrin korkeudelle ja sen jälkeen jokaisen pompun korkeus on aina puolet edellisestä korkeudesta. Kuinka korkea on pallon $5$. pomppu? Entä $13$. pomppu?     
        \begin{vastaus}
Viidennen pompun korkeus on $(1/2)^4\,\textrm{m}=1/16\,\textrm{m}\approx 0,06$ metriä. Pallon $13$. pompun korkeus on $(1/2)^{12}\,\textrm{m} \approx 0,0002$ metriä eli $0,2$ millimetriä.
        \end{vastaus}
\end{tehtava}

\begin{tehtava}
%Tehtävän laatinut Johanna Rämö 9.11.2013.
%Ratkaisun tehnyt Johanna Rämö 9.11.2013.
Ympyrän mallisen pöytäliinan halkaisija on $1,2$\,m. Mikä on pöytäliinan pinta-ala? Ympyrän pinta-ala $A$ voidaan laskea kaavalla $A=\pi\text{r}^2$, missä $r$ on pöytäliinan säde eli puolet halkaisijasta.
        \begin{vastaus}
        Pinta-ala on $\pi \cdot 0,6^2 \approx 1,1\,$cm$^2$.
        \end{vastaus}
\end{tehtava}

\begin{tehtava} %tarkista vastaustarkkuus
Jemina on menossa bussilla lukioon. Kahden bussipysäkin, joiden välillä oikea rakennuksen sisäänkäynti sijaitsee, välinen etäisyys on $600$ metriä. Sisäänkäynti sijaitsee tulosuunnasta mitattuna $200$ metriä ennen jälkimmäistä pysäkkiä. Bussin keskinopeus pysäkkien välillä on $28$\,km/h ja Jemina itse viipottaa pysäkiltä kohteeseensa nopeudella $8$\,km/h. Kummalla pysäkillä kannattaa jäädä pois, jotta hän pääsee mahdollisimman nopeasti (=aikaisin) perille?
	\begin{vastaus}
	Jälkimmäisellä. (Aikaisemmalta pysäkiltä kävely kestää $180$ sekuntia. Ajaminen jälkimmäiselle pysäkille ja kävely sieltä kestää vain noin $170$ sekuntia.) 
	\end{vastaus}
\end{tehtava}

%\begin{tehtava}
%Lääkehiiltä eli aktiivihiiltä käytetään muun muassa ripulin ja myrkytysten akuuttihoitoon, sillä se sitoo suuren pinta-alansa vuoksi tehokkaasti nesteitä ja useita lääkeaineita ja myrkkyjä. Eräässä $50,0$ gramman pakkauksessa ohjeena oli: ''käytetään $3$ ruokalusikallista hiiliraetta kymmentä painokiloa kohti''. Yksi ruokalusikallinen on tilavuudeltaan noin $15\,$ml, ja aktiivihiilen tiheys on $2,26$\,g\,cm$^{-1}$.
%\alakohdat{
%§ Kuinka monta ruokalusikallista lääkehiiltä $40$-kiloinen lapsi tarvitsee?
%§ Kuinka monta ruokalusikallista hiiltä pakkaukseen jää a-kohdan käytön jälkeen?
%}
%	\begin{vastaus}
%\alakohdat{
%§ ($3\,\textrm{rkl}):(10\,\textrm{kg})\cdot 40\,\textrm{kg}=12\,\textrm{rkl}$
%§
%}
%	\end{vastaus}
%\end{tehtava}

\begin{tehtava}
Astmalääke Bricanylia annetaan injektiona $30$\,kg painavalle lapselle. Bricanyl-injektionesteen vahvuus on $0,35$\,mg/ml ja annos $11$\,$\upmu$g/kg. Kuinka monta mikrolitraa injektionestettä lapsi saa? Anna vastaus $1$\,$\mu$l:n tarkkuudella. %UPMU?
 	\begin{vastaus}
$943$\,$\mu$l. (Lapselle annettava Bricanyl-annos on $330$\,$\mu$l.) %UPMU?
	\end{vastaus}
\end{tehtava}

\begin{tehtava}
Erään antibioottimikstuuran vahvuus on $0,23$\,mg/ml. Lapselle annostus on $1.$ päivänä $42$\,mg (aloitusannos), ja $2$.--$7$. päivänä $12$\,mg (ylläpitoannokset). Mikstuuran pakkauskoko on $90,0$\,ml. Kuinka monta pakkausta tarvitaan, ja montako millilitraa jää käyttämättä? Ilmoita tulos $0,1$\,ml:n tarkkuudella.
 \begin{vastaus}
Tarvitaan $6$ pakkausta. Mikstuuraa jää käyttämättä $44,3$\,ml.
 \end{vastaus}
\end{tehtava}

\subsubsection*{Lisää tehtäviä}

\begin{tehtava}
Muuta sekunneiksi
\alakohdat{
§ $1$\,h $42$\,min
§ $3$\,h $32$\,min
§ $1,25$\,h
§ $4,5$\,h.
}
	\begin{vastaus}
\alakohdat{
§ $6\,120$\,s
§ $12\,720$\,s
§ $4\,500$\,s
§ $16\,200$\,s
}
	\end{vastaus}
\end{tehtava}

\begin{tehtava}
Muuta minuuteiksi
\alakohdat{
§ $1$\,h $17$\,min
§ $2$\,h $45$\,min
§ $1,5$\,h
§ $1,75$\,h
}
	\begin{vastaus}
\alakohdat{
§ $77$\,min
§ $165$\,min
§ $90$\,min
§ $105$\,min
}
	\end{vastaus}
\end{tehtava}

\begin{tehtava}
Muuta seuraavat ajat tunneiksi kolmen desimaalin tarkkuudella.
	\alakohdat{
		§ $73$ minuuttia
		§ $649$ sekuntia
		§ $15$ minuuttia ja $50$ sekuntia
		§ $42$ minuuttia ja $54$ sekuntia
	}
	\begin{vastaus}
		\alakohdat{
			§ $1,217\ldots$ tuntia
			§ $0,180\ldots$ tuntia
			§ $0,264\ldots$ tuntia
			§ $0,715$ tuntia
		}
	\end{vastaus}
\end{tehtava}

\begin{tehtava}
Muuta tunneiksi ja minuuteiksi
\alakohdat{
§ $125$\,min
§ $667$\,min
§ $120$\,min
§ $194$\,min.
}
	\begin{vastaus}
\alakohdat{
§ $2$\,h $5$\,min
§ $11$\,h $7$\,min
§ $2$\,h
§ $3$\,h $14$\,min
}
	\end{vastaus}
\end{tehtava}

\begin{tehtava}
Laske ja ilmoita tulos sopivissa yksiköissä niin, että luvun kokonaisosa on mahdollisimman pieni ja siinä on vähintään yksi numero.
\alakohdat{
§ $0,3$\,km$ +200$\,m
§ $0,04$\,m$ + 10$\,mm
§ $0,4$\,km$ + 7\,000$\,m$ + 1\,000\,000$\,mm
§ $0,2$\,cm$ + 4$\,cm.
}
	\begin{vastaus}
\alakohdat{
§ $500$\,m
§ $5$\,cm
§ $8,4$\,km
§ $4,2$\,cm
}
	\end{vastaus}
\end{tehtava}

\begin{tehtava} %vastaavaa esimerkki, plz...
Nimeltä mainitsematon henkilö käveli kilometrin kahdessatoista minuutissa. Esitä tämä nopeus
\alakohdat{
§ kilometreinä tunnissa
§ metreinä sekunnissa.
§ Kuinka monta sekuntia kyseisellä kävelynopeudella kestää kulkea kymmenen metrin matka?
}
	\begin{vastaus}
	\alakohdat{
§ $5,0$\,km/h
§ $1,4$\,m/s
§ $11$\,s
}
	\end{vastaus}
\end{tehtava}

\begin{tehtava}
Jasper-Korianteri ja Kotivalo vertailivat keppejään. Jasper-Korianteri mittasi oman keppinsä $5,9$ tuumaa pitkäksi ja Kotivalo omansa $14,8$ senttimetriä pitkäksi. Kummalla on pidempi keppi?
	\begin{vastaus}
Jasper-Korianterilla, sillä $5,9$ tuumaa = $14,986$\,cm
	\end{vastaus}
\end{tehtava}

\begin{tehtava}
Liisa ohjelmoi tietokoneensa sammumaan $12\,400$ sekunnin kuluttua, kun kello on $1.45$. Moneltako Liisan tietokone sammuu -- minuutin tarkkuudella?
	\begin{vastaus}
Kello $5.12$
	\end{vastaus}
\end{tehtava}

\begin{tehtava}
Laske oma pituutesi ja painosi tuumissa ja paunoissa. %vai laitetaanko "kiinalaistenkeskipituus on..." ja sitten vertaa itseesi :)
%\begin{vastaus}
%Esimerkiksi $168$\,cm...
%\end{vastaus}
\end{tehtava}

\begin{tehtava}
(Muunnelma fyysikko Enrico Fermin esittämästä ongelmasta.) Chicagossa asuu osapuilleen $5\,000\,000$ asukasta. Kussakin kotitaloudessa asuu keskimäärin kaksi asukasta. Karkeasti joka kahdennessakymmenennessä kotitaloudessa on säännöllisesti viritettävä piano. Säännöllisesti viritettävät pianot viritetään keskimäärin kerran vuodessa. Pianonvirittäjällä kestää noin kaksi tuntia virittää piano, kun matkustusajat huomioidaan. Kukin pianonvirittäjä työskentelee kahdeksan tuntia päivässä, viisi päivää viikossa ja $50$ viikkoa vuodessa. Mikä arvio Chicagon pianonvirittäjien lukumäärälle saadaan näillä luvuarvoilla?
	\begin{vastaus}
$125$ pianonvirittäjää
	\end{vastaus}
\end{tehtava}

%\begin{tehtava}
%Kun lämpötila $T$ (kelvineissä), kaasusäiliön tilavuus $V$ (litroissa) ja sen sisällä olevan kaasun ainemäärä $n$ (mooleissa) tiedetään, säiliössä olevan kaasun paine voidaan laskea kaavalla $P=\frac{nRT}{V}$, missä $•$
%\end{tehtava}

\begin{tehtava}
Newtonin painovoimalain mukaan kaksi kappaletta, joiden välinen etäisyys (metreinä) on $R$ ja joiden massat (kilogrammoina) ovat $m_1$ ja $m_2$, vetävät toisiaan puoleensa voimalla $F=\gamma \frac{m_1m_2}{R^2}$, missä $\gamma$ on niin sanottu universaali painovoimavakio ja voima $F$ ilmoitetaan newtoneina. Päättele painovoimavakion yksikkö, newtoneita sisältämättömässä muodossa, kun lisäksi tiedetään, että $F=ma$, missä massa $m$ on kilogrammoina ja kiihtyvyys $a$ on yksikössä $\mathrm{\frac{m}{s^2}}$.
	\begin{vastaus}
	$[\gamma]=\mathrm{\frac{m^3}{kg\,s^2}}$
	\end{vastaus} 
\end{tehtava}

\end{tehtavasivu}