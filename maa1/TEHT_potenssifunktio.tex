\begin{tehtavasivu}

\subsubsection*{Opi perusteet}

%Tarkistanut V-P Kilpi 2013-11-10
\begin{tehtava}
Mikä on seuraavien potenssifunktioiden aste?
\alakohdat{
§ $f(x) = x$
§ $f(x) = 5x^5$
§ $f(x) = \frac{1}{2x}$
§ $f(x) = x^{-2}$
}
\begin{vastaus}
\alakohdat{
§ $1$
§ $5$
§ $-1$
§ $-2$
}
\end{vastaus}
\end{tehtava}

%Laatinut V-P Kilpi 2013-11-10
\begin{tehtava}
Olkoon $g(x)=-2x^{-2}$. Laske
\alakohdat{
§ $g(4)$
§ $g(\frac{1}{4})$
§ $g(-\frac{1}{4})$
§ $g(-4)$.
}
\begin{vastaus}
\alakohdat{
§ $-\frac{1}{8}$
§ $-32 $
§ $-32 $
§ $-\frac{1}{8}$
}
\end{vastaus}
\end{tehtava}

%Tarkistanut V-P Kilpi 2013-11-10
\begin{tehtava}
Olkoon $f(x)=x^{-3}$. Laske
\alakohdat{
§ $f(1)$
§ $f(2)$
§ $f(-\frac{1}{3})$.
}
\begin{vastaus}
\alakohdat{
§ $1$
§ $\frac{1}{8}$
§ $-27$
}
\end{vastaus}
\end{tehtava}

%Tarkistanut V-P Kilpi 2013-11-10
\begin{tehtava}
Saako potenssifunktio $f$ negatiivisia arvoja, jos
\alakohdat{
§ $f(x) = x^2$
§ $f(x) = -2x^7$
§ $f(x) = 3x^{-3}$
§ $f(x) = -6x^{-4}$
§ $f(x) = x^{-6}$?
}
\begin{vastaus}
\alakohdat{
§ Ei saa.
§ Saa.
§ Saa.
§ Saa.
§ Ei saa.
}
\end{vastaus}
\end{tehtava}

%Laatinut V-P Kilpi 2013-11-10
\begin{tehtava}
Saavatko funktiot, jotka määritellään lausekkeilla $f(x)=x^3-1$ ja $g(x)=x^3+1$, missään kohdassa samaa arvoa?
\begin{vastaus}
Eivät saa.
\end{vastaus}
\end{tehtava}

\begin{tehtava}
Yhdistä funktioiden kuvaajat oikeaan funktion lausekkeeseen.
\alakohdat{
	§ $g(x)=x$
	§ $c(x)=x^{10}$
	§ $a(x)=\sqrt[3]{x}$
	§ $b(x)=x^{\frac{1}{10}}$
	§ $f(x)=x^{\frac{3}{2}}$
	§ $d(x)=x^{-1}$
	§ $h(x)=x^{-8}$
}

\begin{center}
	\begin{kuvaajapohja}{0.9}{-3.5}{3.5}{-3.5}{3.5}
	
	\newcommand{\kuvaajaneg}[3]{
\draw[smooth,color=#3,thick,domain=\kuvaajaminx:-0.01,scale=\kuvaajascale,samples=300] plot function{(#1) < \kuvaajamaxy ? ((#1) > \kuvaajaminy ? (#1) : NaN) : NaN} node[right] {#2};}
\newcommand{\kuvaajapos}[3]{
\draw[smooth,color=#3,thick,domain=0.01:\kuvaajamaxx,scale=\kuvaajascale,samples=300] plot function{(#1) < \kuvaajamaxy ? ((#1) > \kuvaajaminy ? (#1) : NaN) : NaN} node[right] {#2};}
	 \kuvaaja{x**10}{2}{red}
	 \kuvaaja{x**1.5}{1}{green}
	 \kuvaaja{x}{3}{black}
	 \kuvaaja{x**0.1}{5}{purple}
	 \kuvaajapos{x**0.33333333333}{7}{blue}
	 \kuvaajaneg{-(-x)**0.33333333333}{}{blue}
	 \kuvaajaneg{x**(-1)}{}{violet}
	 \kuvaajaneg{x**(-8)}{6}{orange}
	 \kuvaajapos{x**(-1)}{4}{violet}
	 \kuvaajapos{x**(-8)}{}{orange}
	\end{kuvaajapohja}
\end{center}

	\begin{vastaus}
	\alakohdat{
		§ 3
		§ 2
		§ 7
		§ 5
		§ 1
		§ 4
		§ 6
	}
	\end{vastaus}
\end{tehtava}

\subsubsection*{Hallitse kokonaisuus}

\begin{tehtava}
Olkoon $f(x)=x^2$. Millä $x$:n arvoilla
\alakohdat{
§ $f(x)=4$
§ $f(x)=25$
§ $f(x)=121$?
}
	\begin{vastaus}
\alakohdat{
§ $2$
§ $5$
§ $11$
}
	\end{vastaus}
\end{tehtava}

\begin{tehtava}
Minkä kahden peräkkäisen kokonaisluvun välissä yhtälön $x^2 = 12$ positiivinen ratkaisu on?
	\begin{vastaus}
Ratkaisu on kokonaislukujen $3$ ja $4$ välissä.
	\end{vastaus}
\end{tehtava}

%Laatinut V-P Kilpi 2013-11-10
\begin{tehtava}
Vapaassa pudotuksessa olevan kappaleen etäisyydelle metreinä pudotuskorkeudesta antaa hyvän arvion funktio $ s(t)=4,9t^{2}$, missä $t$ on putoamisaika sekunteina. Kuinka monen sekunnin päästä pudotettu kappale on pudonnut $100$ metriä?
	\begin{vastaus}
Noin $4,5$ sekunnin päästä
	\end{vastaus}
\end{tehtava}

\begin{tehtava}
Suorakulmaisen kolmion molempien kateettien pituus on $x$.
\alakohdat{
§ Muodosta kolmion pinta-alan lauseke.
§ Mikä on kolmion pinta-ala, jos $x=10$?
§ Kolmion pinta-ala on $72$. Mikä on kateettien pituus?
}
	\begin{vastaus}
\alakohdat{
§ $\frac{1}{2}x^2$
§ $50$
§ $12$
}
	\end{vastaus}
\end{tehtava}

\subsubsection*{Lisää tehtäviä}

%Laatinut V-P Kilpi 2013-11-10
\begin{tehtava}
Tasasivuisen kolmion pinta-ala sivun pituuden funktiona on $A(s) = \frac{\sqrt{3}}{4}s^{2}$. Jos tasasivuisen kolmion sivun pituus on $10$, mikä on sen pinta-ala?
\begin{vastaus}
Kolmion pinta-ala on $25\sqrt{3}\approx43,3$.
\end{vastaus}
\end{tehtava}

%Laatinut V-P Kilpi 2013-11-10
\begin{tehtava}
Asta kehitti funktion $t(p)=\frac{1}{10}p^{\frac{6}{5}}$, joka ennustaa palapelin kokoamiseen kuluvaa aikaa minuutteina palojen määrän $p$ funktiona. Kuinka kauan mallin mukaan kuluu aikaa
\alakohdat{
§ $100$ palan palapelin kokoamiseen
§ $1\,000$ palan palapelin kokoamiseen
§ kahden palan palapelin kokoamiseen?
}
\begin{vastaus}
\alakohdat{
§ Noin $25$ minuuttia
§ Noin $6$ tuntia ja $40$ minuuttia
§ Noin $14$ sekuntia
}
\end{vastaus}
\end{tehtava}

%Laatinut V-P Kilpi 2013-11-10
\begin{tehtava}
Stefanin--Boltzmannin lain mukaan kappaleen säteilyteho tiettyä pinta-alaa kohden on suoraan verrannollinen sen absoluuttisen lämpötilan neljänteen potenssiin. Lakia kuvaa funktio $ P(T)=A\varepsilon\sigma T^4 $, missä $A$, $\varepsilon$ ja $\sigma $ ovat vakioita. Jos absoluuttinen lämpötila kaksinkertaistuu, kuinka moninkertaiseksi säteilyteho kasvaa?
\begin{vastaus}
$16$-kertaiseksi
\end{vastaus}
\end{tehtava}

%Laatinut V-P Kilpi 2013-11-10
\begin{tehtava}
Säteilyn intensiteetti säteilylähteen etäisyyden $r$ funktiona on $ I(r)=\frac{I_0}{r^{2}}$, missä $I_0$ on vakio, joka kuvaa säteilyn intensiteettiä etäisyydellä $0$. Arvioi funktion avulla kuinka monta prosenttia Auringon säteilyn intensiteetti Maan pinnalla vähenee, jos Maan ja Auringon välinen etäisyys kasvaa $20$ prosenttia?
\begin{vastaus}
Säteilyn intensiteetti Maan pinnalla vähenee noin $31$ prosenttia.
\end{vastaus}
\end{tehtava}

\end{tehtavasivu}