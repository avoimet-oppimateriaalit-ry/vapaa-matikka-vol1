\begin{tehtavasivu}

\subsubsection*{Opi perusteet}

Laske
\begin{tehtava}
\alakohdatm{
§ $\sqrt{64}$
§ $\sqrt{-64}$
§ $\sqrt[3]{64}$
§ $\sqrt[3]{-64}$
§ $\sqrt{100}$
}
	\begin{vastaus}
\alakohdatm{§ $8$ § Ei määritelty § $4$ § $-4$ § $10$}
	\end{vastaus}
\end{tehtava}

\begin{tehtava}
\alakohdatm{
§ $\sqrt[4]{81}$
§ $\sqrt[4]{-81}$
§ $\sqrt[5]{32}$
§ $\sqrt[5]{-32}$
§ $\sqrt[3]{1\,000}$
}
	\begin{vastaus}
	\alakohdatm{
	§ $3$
	§ Ei määritelty
	§ $2$
	§ $-2$
	§ $10$
	}
	\end{vastaus}
\end{tehtava}
%Tehtävät laatinut Aleksi Sipola 10.11.2013 %Ratkaisut tehnyt Aleksi Sipola 10.11.2013
	\begin{tehtava}
\alakohdatm{
§ $\sqrt{81}$
§ $\sqrt[3]{27}$
§ $\sqrt[4]{-625}$
§ $\sqrt[8]{256}$ 
}
	\begin{vastaus}
\alakohdatm{
§ $9$
§ $3$
§ Ei määritelty
§ $2$
}
	\end{vastaus}
\end{tehtava}

%Tehtävät laatinut Aleksi Sipola 09.11.2013
%Ratkaisut tehnyt Aleksi Sipola 09.11.2013
\begin{tehtava}
Laske. 
\alakohdatm{
§ $\sqrt{2}\sqrt{8}$
§ $\sqrt{3}\sqrt{5}$
§ $\sqrt{3}\sqrt{27}$
§ $\sqrt[3]{3}\sqrt[3]{25}$
}
	\begin{vastaus}
\alakohdatm{
§ $4$
§ $\sqrt{15}$ (Ei sievene enempää.)
§ $9$
§ $5$
}
	\end{vastaus}
\end{tehtava}

%Tehtävät laatinut Aleksi Sipola 09.11.2013
%Ratkaisut tehnyt Aleksi Sipola 09.11.2013
\begin{tehtava} 
Laske. 
\alakohdat{
§ $\frac{\sqrt{8}}{\sqrt{2}}$
§ $ \frac{\sqrt[3]{3} \cdot \sqrt[3]{9}}{\sqrt[3]{8}}$
§ $ \frac{\sqrt{3} \cdot \sqrt{6}}{\sqrt{2}}$
§ $ \frac {\sqrt[3]{2}}{\sqrt[3]{16}} \cdot \sqrt[3]{4}$ 
}
	\begin{vastaus}
\alakohdatm{
§ $2$
§ $\frac{3}{2}$
§ $3$
§ $\frac{1}{\sqrt[3]{2}}$
}
	\end{vastaus}
\end{tehtava}

\begin{tehtava}
Laske $\sqrt{8\,100}$ päässä ajattelemalla juurrettava luku sopivana tulona.
	\begin{vastaus}
$\sqrt{8\,100}=\sqrt{81\cdot100}=\sqrt{81}\sqrt{100}=9\cdot 10=90$.
	\end{vastaus}
\end{tehtava}

%Tehtävät laatinut Aleksi Sipola 09.11.2013
%Ratkaisut tehnyt Aleksi Sipola 09.11.20133
\begin{tehtava}
Pienen kuution sivu on puolet ison kuution sivusta. Kuinka pitkä pienen kuution sivu on, jos ison kuution tilavuus on $64$ yksikköä? Kuution särmän pituus on $a$, jos ja vain jos kuution tilavuus on $a^3$.
	\begin{vastaus}
Pienen kuution sivu on $\sqrt[3]{64}/2=2$.
	\end{vastaus}
\end{tehtava}

\subsubsection*{Hallitse kokonaisuus}

\begin{tehtava}
Laske laskimella neliöjuurten arvot sadasosien tarkkuudella.
\alakohdatm{
§ $\sqrt{320}$
§ $\sqrt{15}$
§ $\sqrt{71}$
}
	\begin{vastaus}
	\alakohdatm{
	§ $17,89$
	§ $3,87$
	§ $8,43$
	}
	\end{vastaus}
\end{tehtava}

Laske.
%Tehtävät laatinut Aleksi Sipola 09.11.2013
%Ratkaisut tehnyt Aleksi Sipola 09.11.2013
\begin{tehtava}
\alakohdatm{
§ $ \frac{\sqrt{15}}{\sqrt{7}} \cdot  \frac{\sqrt{27}}{\sqrt{35}}$
§ $ \frac{\sqrt{64}-\sqrt{9}}{\sqrt{5}}$
§ $ \frac{2 \cdot \sqrt[3]{32}}{\sqrt[3]{4}}$
}
	\begin{vastaus}
\alakohdatm{
§ $9/7$
§ $\sqrt{5}$
§ $4$
}
	\end{vastaus}
\end{tehtava}

%Tehtävät laatinut Aleksi Sipola ja Jaakko Viertiö 10.11.2013
%Ratkaisut tehnyt Aleksi Sipola ja Jaakko Viertiö 10.11.2013
\begin{tehtava} 
Laske. 
\alakohdatm{
§ $ \frac{3\sqrt{5}+5\sqrt{3}-\sqrt{15}}{\sqrt{15}}$
§ $ \frac{2-\sqrt{2}}{\sqrt{2}-1}$
}
	\begin{vastaus}
\alakohdatm{
§ $\sqrt{3}+\sqrt{5}-1$
§ $\sqrt{2}$
}
	\end{vastaus}
\end{tehtava}

\begin{tehtava}
%Tehtävän laatinut Johanna Rämö 9.11.2013. %Ratkaisun tehnyt Johanna Rämö 9.11.2013.
Selvitä ilman laskinta, kumpi luvuista $3\sqrt{2}$ ja $2\sqrt{3}$ on suurempi. 
        \begin{vastaus}
Korotetaan luvut toiseen potenssiin, jolloin päästään eroon juurista: $(3\sqrt{2})^2=3^2\cdot\sqrt{2}^2=9 \cdot 2=18$ ja $(2\sqrt{3})^2=2^2\cdot\sqrt{3}^2=4 \cdot 3=12$. Koska $3\sqrt{2}$ ja $2\sqrt{3}$ ovat molemmat lukua $1$ suurempia, voidaan niiden keskinäinen suuruusjärjestys lukea neliöiden suuruusjärjestyksestä. Edellisten laskujen perusteella $(3\sqrt{2})^2 > (2\sqrt{3})^2$, joten $3\sqrt{2} > 2\sqrt{3}$.
        \end{vastaus}
\end{tehtava}

%Tehtävät laatinut Aleksi Sipola 09.11.2013
%Ratkaisut tehnyt Aleksi Sipola 09.11.2013
\begin{tehtava} Kumpi juurista on suurempi? Arvaa ja laske.
\alakohdatm{
§ $\sqrt[3]{8}$ vai $\sqrt{16}$
§ $\sqrt{8}$ vai $2\sqrt{2}$
§ $\sqrt[3]{64}$ vai $\sqrt[5]{32}$
§ $\sqrt[2]{121}$ vai $\sqrt[5]{243}$ 
}
\begin{vastaus}
\alakohdatm{
§ $\sqrt[3]{8}=2<\sqrt{16}=4$
§ $\sqrt{8}=\sqrt{4}\sqrt{2}=2\sqrt{2} = 2\sqrt{2}$
§ $\sqrt[3]{64}=4>\sqrt[5]{32}=2$
§ $\sqrt[2]{121}=11 >\sqrt[5]{243}=3$
}
\end{vastaus}
\end{tehtava}

\begin{tehtava}
%Tehtävän laatinut Johanna Rämö 9.11.2013.
%Ratkaisun tehnyt Johanna Rämö 9.11.2013.
Mitä tapahtuu neliöjuuren arvolle, kun juurrettavaa kerrotaan luvulla $100$? Miten tulos yleistyy?
        \begin{vastaus}
Neliöjuuri kasvaa $10$-kertaiseksi. Reaalilukujen $a$ ja $100a$ neliöjuuret ovat nimitäin $\sqrt{a}$ ja $\sqrt{100a}=\sqrt{100}\sqrt{a}=10\sqrt{a}$. Yleisesti jos juurrettavaa kerrotaan reaaliluvulla $k$, neliöjuuri kasvaa $\sqrt{k}$-kertaiseksi.
        \end{vastaus}
\end{tehtava}

\begin{tehtava}
Oletetaan, että suorakaiteen leveyden suhde korkeuteen on $2$ ja suorakaiteen pinta-ala on $10$ (mielivaltaista pinta-alayksikköä). Mikä on suorakaiteen leveys ja korkeus?
\begin{vastaus}
Suorakaide muodostuu kahdesta vierekkäisestä neliöstä, joiden pinta-ala on $5$. Tämän neliön sivun pituus on $\sqrt{5}$. Siis suorakaiteen korkeus on $\sqrt{5}$ ja leveys $2\sqrt{5}$.
\end{vastaus}
\end{tehtava}

\begin{tehtava}
Etsi luku $a>0$, jolle $a^4=83\,521$.
	\begin{vastaus}
$a=\sqrt[4]{83\,521}$
	\end{vastaus}
\end{tehtava}

\begin{tehtava}
Laske ilman laskinta.
\alakohdatm{
§ $\sqrt[6]{1\,000\,000}$
§ $\sqrt[10]{10\,000\,000\,000}$
§ $\sqrt[5]{10\,000\,000\,000}$
}
	\begin{vastaus}
	\alakohdat{
	§ $\sqrt[6]{1\,000\,000}=\sqrt[6]{10^6}=10$
	§ $\sqrt[10]{10\,000\,000\,000}=\sqrt[10]{10^{10}}=10$
	§ $\sqrt[5]{10\,000\,000\,000}=\sqrt[5]{10^{10}}=\sqrt[5]{10^{2\cdot 5}}=\sqrt[5]{(10^2)^5}=10^2=100$
	}
	\end{vastaus}
\end{tehtava}

\begin{tehtava}
Onko annettu juuri määritelty kaikilla luvuilla $a$? Millaisia arvoja juuri voi saada luvusta $a$ riippuen?
\alakohdatm{
§ $\sqrt[4]{a^2}$
§ $\sqrt[4]{-a^2}$
§ $\sqrt[4]{(-a)^2}$
§ $-\sqrt[4]{a^2}$
}
	\begin{vastaus}
\alakohdat{
	§ Juuri on määritelty kaikilla luvuilla $a$, koska kaikkien lukujen neliöt ovat vähintään nolla. Vastaus on aina epänegatiivinen.
	§ Juuri on määritelty vain luvulla $a = 0$. Muilla $a$:n arvoilla $-a^2$ on negatiivinen, jolloin parillinen juuri ei ole määritelty. Ainoa vastaus, joka voidaan saada, on siis $\sqrt[4]{0} = 0$.
	§ Juuri on määritelty kaikilla luvuilla $a$, koska $(-a)^2$ on aina vähintään nolla. Vastaus on aina epänegatiivinen.
	§ Juuri on määritelty kaikilla luvuilla $a$, koska kaikkien lukujen neliöt ovat vähintään nolla. Vastaus on aina ei-positiivinen, koska $\sqrt[4]{a^2}$ on aina epänegatiivinen.
}
	\end{vastaus}
\end{tehtava}

\begin{tehtava}
Onko annettu juuri määritelty kaikilla luvuilla $a$? Millaisia arvoja lauseke voi saada luvusta $a$ riippuen?
\alakohdatm{
§ $\sqrt[5]{a^2}$
§ $\sqrt[5]{a^3}$
§ $\sqrt[5]{-a^2}$
§ $-\sqrt[5]{a^2}$
}
	\begin{vastaus}
\alakohdat{
	§ Juuri on määritelty kaikilla luvuilla $a$. Vastaus on aina epänegatiivinen (eli nolla tai positiivinen).
	§ Juuri on määritelty kaikilla luvuilla $a$. Vastaus voi olla mikä tahansa luku (negatiivinen, nolla tai positiivinen).
	§ Juuri on määritelty kaikilla luvuilla $a$. Vastaus on aina epäpositiivinen (eli negatiivinen tai nolla).
	§ Juuri on määritelty kaikilla luvuilla $a$. Vastaus on aina epäpositiivinen (eli negatiivinen tai nolla).
}
	\end{vastaus}
\end{tehtava}

%Tehtävät laatinut Aleksi Sipola 09.11.2013
%Ratkaisut tehnyt Aleksi Sipola 09.11.2013
\begin{tehtava} Etsi positiivinen rationaaliluku $a$, jolla
\alakohdatm{
§ $\sqrt{a} \sqrt{2} = 4$
§ $ \sqrt{a}\cdot{\sqrt{3}} =9 $.
}
	\begin{vastaus}
\alakohdatm{
§ $a=8$
§ $a=27$
}
	\end{vastaus}
\end{tehtava}

\begin{tehtava}
Anna esimerkki sellaisista luvuista $a$ ja $b$, että $\sqrt{a^b}\neq {\sqrt{a}}^b$.
	\begin{vastaus}
Esimerkiksi $a=-1$ ja $b=2$. Tällöin annetun ehdon vasemman lausekkeen arvoksi tulee $1$, mutta oikea ei ole reaaliluvuilla määritelty, jolloin se ei voi olla yhtä suuri vasemman puolen kanssa.
	\end{vastaus}
\end{tehtava}

%Tehtävät laatinut Aleksi Sipola 09.11.2013
%Ratkaisut tehnyt Aleksi Sipola 09.11.2013
\begin{tehtava} Etsi jotkin erisuuret ykköstä suuremmat luonnolliset luvut $a$ ja $b$, jolla
\alakohdat{
§ $\frac{\sqrt{a}}{\sqrt{b}}$ on jokin luonnollinen luku
§ $\sqrt[b]{a}=2$.
}
	\begin{vastaus}
\alakohdatm{
§ Esimerkiksi $a=27$ ja $b=3$
§ Esimerkiksi $a=16$ $b=4$ 
}
	\end{vastaus}
\end{tehtava}

\begin{tehtava}
Yksinkertaisen heilurin jaksonaika $T$ voidaan laskea kaavalla $T=2\pi\sqrt{\frac{l}{g}}$, missä $l$ on heilurin (langan tai varren) pituus metreinä, ja $g$ on putoamiskiihtyvyys $9,81\,$m\,s$^{-2}$. Näytä sieventämällä yksiköt ja neliöjuuri, että jaksonajan yksiköksi saadaan sekunnit.
	\begin{vastaus}
	Lukuvakiot $2$ ja $\pi$ eivät vaikuta yksikköön. $[T]=\sqrt{\frac{\text{m}}{\frac{\text{m}}{\text{s}^2}}}=\sqrt{\text{m}\cdot \frac{\text{s}^2}{m}}=\sqrt{\text{s}^2}=\text{s}$
	\end{vastaus} 
\end{tehtava}

\subsubsection*{Lisää tehtäviä}

\begin{tehtava}
Sievennä $(3+\sqrt{3}x)^4:(\sqrt{3}+x)^3$.
	\begin{vastaus}
$9(3 + \sqrt{3}x)$
	\end{vastaus}
\end{tehtava}

%Tehtävät laatinut Aleksi Sipola 09.11.2013
%Ratkaisut tehnyt Aleksi Sipola 09.11.2013
\begin{tehtava} Laske laskimella likiarvot viiden merkitsevän luvun tarkkuudella. Mikä yhteys luvuilla on?
\alakohdat{
§ $ \sqrt{1^2+1^2}$
§ $ \frac {\sqrt{3^2+3^2}}{3}$
§ $ \frac {\sqrt{9^2+9^2}}{9}$
}
	\begin{vastaus}
	\alakohdatm{
§ $1,41421$
§ $1,41421$
§ $1,41421$
}
Kyseessä on tietysti $\sqrt{2}$ ja annetut laskut mukailevat neliön halkaisijan suhdetta sivuun. (Tästä lisää geometrian kurssilla.)
	\end{vastaus}
\end{tehtava}

%Tehtävät laatinut Aleksi Sipola 09.11.2013
%Ratkaisut tehnyt Aleksi Sipola 09.11.2013
\begin{tehtava} Etsi jotkin erisuuret luonnolliset luvut $a$ ja $b$, jolla $\frac{\sqrt{\sqrt[a]{b}}}{\sqrt{a}}=1$.
	\begin{vastaus}
Esimerkiksi $a=3$ ja $b=27$ 
	\end{vastaus}
\end{tehtava}

\begin{tehtava}
	Osoita vastaesimerkin avulla, että seuraavat opiskelijat ovat erehtyneet:
	\alakohdat{
		§ Mikael: ''Jos $n$:s juuri korotetaan johonkin muuhun potenssiin kuin $n$, myös tuloksessa esiintyy välttämättä juuri. Esimerkiksi $\sqrt[3]{5}^2 = \sqrt[3]{25}$ ja $\sqrt[5]{3}^6 = 3 \sqrt[5]{3}$.''
		§ Raisa: ''Neliöjuuria sisältävän summalausekkeen voi aina korottaa toiseen, jolloin neliöjuurista pääsee eroon. Esimerkiksi ja $(\sqrt{2} + \sqrt{8})^2 = 18$.''
	}
	\begin{vastaus}
		\alakohdat{
			§ Esimerkiksi $\sqrt[4]{3}^8 = 16$
			§ Esimerkiksi $(\sqrt{2} + \sqrt{3})^2 = 5 + 2 \sqrt{6}$
		}
	\end{vastaus}
\end{tehtava}

\end{tehtavasivu}