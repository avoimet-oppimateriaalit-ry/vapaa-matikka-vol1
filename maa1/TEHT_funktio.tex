\begin{tehtavasivu}

%PALJON LISÄÄ KUVAAJIA!!! (sekä tehtäviin että VASTAUKSIIN!!!!!!!!!!! MUR VASTAUKSIIN!)

\subsubsection*{Opi perusteet}
\begin{tehtava}
Määritä kuvaajasta katsomalla funktion $f$ nollakohdat. Saako funktio $f$ negatiivisia ja positiivisia arvoja?
\begin{kuva}
    kuvaaja.pohja(-2, 2, -2, 2)
    kuvaaja.piirra("x**3-x", nimi = "$f$")
\end{kuva}
\begin{vastaus}
Funktion $f$ nollakohdat ovat $x$-koordinaatiston kohdissa $-1$, $0$ ja $1$. Funktio saa sekä positiivisia että negatiivisia arvoja.
\end{vastaus}
\end{tehtava}

\begin{tehtava}
Määritä kuvaajasta katsomalla funktion $f$ nollakohdat. Saako funktio $f$ sekä negatiivisia ja positiivisia arvoja?
\begin{kuva}
    kuvaaja.pohja(-2, 2, -4, 2)
    kuvaaja.piirra("2*x**4-5*x**2+1", nimi = "$f$")
\end{kuva}
	\begin{vastaus}
Funktion $f$ nollakohdat ovat suunnilleen $x$:n pisteissä: $-1,5$, $-0,5$, $0,5$ ja $1,5$. Funktio saa sekä positiivisia että negatiivisia arvoja.
	\end{vastaus}
\end{tehtava}

\begin{tehtava}
Määritä kuvaajasta funktion $f$ arvot muuttujan arvoilla $0$ ja $1$. Millä muuttujan arvoilla $f$ saa arvon $3$? Entä arvon $0$?
\begin{kuva}
    kuvaaja.pohja(-3, 3, -2, 4)
    kuvaaja.piirra("x**2-1", nimi = "$f$")
\end{kuva}
\begin{vastaus}
 $f(0)=-1$, $f(1)=0$, $f(x)=3$ kun $x=-2$ tai $x=2$, ja $f(x)=0$, kun $x=-1$ tai $x=1$
\end{vastaus}
\end{tehtava}

\begin{tehtava}
Hahmottele funktion kuvaaja koordinaatistoon. Voit myös käyttää tietokonetta.
\alakohdatm{
§ $f(x) = 2$
§ $f(x) = 3x+2$
§ $f(x) = x^2$
§ $f(x) = \frac{2}{x}$
}
%	\begin{vastaus}
%	\alakohdat{
%	
%	\begin{kuva}
%    kuvaaja.pohja(-3, 3, -2, 4)
%    kuvaaja.piirra("x**2-1", nimi = "$f$")
%	\end{kuva}
%	\end{vastaus}
%	}
\end{tehtava} %VASTAUKSET!

\begin{tehtava}
  Laske taulukkoon funktion arvo pisteissä $x=-2$, $x=-1$, $x=0$, $x=1$ ja $x=2$. Hahmottele näiden tietojen avulla funktion kuvaaja.
  \alakohdatm{
    § $f(x)= x^2+x+1$
    § $f(x)= 3x^2$
    § $f(x)= x^3 +2x+2$
  }
  \begin{vastaus}
    \alakohdat{
      § $f(-2)=3$, $f(-1)=1$, $f(0)=1$, $f(1)=3$ ja $f(2)=7$
      § $f(-2)=12$, $f(-1)=3$, $f(0)=0$, $f(1)$ ja $f(2)=12$
      § $f(-2)=6$, $f(-1)=1$, $f(0)=2$, $f(1)=5$ ja $f(2)=14$
    }
  \end{vastaus}
\end{tehtava}

\begin{tehtava} %Laatinut Emilia Välttilä 9.11.2013
Mikä on funktion $f(x)=\frac{x+1}{2x+5}$ laajin reaalinen määrittelyjoukko?
  \begin{vastaus}
   $\mathbb{R} \backslash \lbrace \frac{5}{2} \rbrace$ eli reaaliluvut pois lukien rationaaliluku $\frac{5}{2}$
  \end{vastaus}
\end{tehtava}

\begin{tehtava} % Emilia Välttilä 9.11.2013
	Olkoon $f(x) = 2x-1$ ja $g(x) = 2-x$. Millä muuttujan $x$ arvolla $f(x) = g(x)$? Piirrä funktioiden kuvaajat ja tarkista vastauksesi kuvasta.
    \begin{vastaus}
    $x=1$
    \end{vastaus}
\end{tehtava}

\begin{tehtava} % Emilia Välttilä 9.11.2013
	Missä pisteessä funktio $P(t)=-2t+4$ saa arvon
	\alakohdatm{
	  § $4$
	  § $-2$
	  § $8$?
	  § Mikä on funktion $P$ nollakohta?
	 }
	 \begin{vastaus}
	  \alakohdatm{
	   § pisteessä $(0,4)$
	   § pisteessä $(3,-2)$
	   § pisteessä $(-2,8)$
	   § Funktiolla on nollakohta $t$:n arvolla $2$, eli pisteessä $(2,0)$.
	  }
	 \end{vastaus}
\end{tehtava}

\begin{tehtava} % Emilia Välttilä 10.11.2013
	Tuuli ostaa joka viikko pussin vadelmaveneitä. Yksi pussi maksaa $2,5$ euroa. Tuulin rahatilannetta euroina havainnollistaa funktio $f(x) = -2,5x+20$, jossa muuttuja $x$ kuvaa aikaa viikkoina.
	  \alakohdat{
	  § Kuinka paljon rahaa Tuulilla on aluksi?
	  § Kuinka paljon rahaa Tuulilla on viiden viikon jälkeen?
	  § Milloin Tuulilla on jäljellä enää $5$ euroa?
	  § Milloin rahat loppuvat?
	  § Piirrä funktion $f$ kuvaaja ja tarkista vastauksesi siitä.
	  } 
  \begin{vastaus}
	\alakohdat{
	  § $20$ euroa ($f(0)=20$)
	  § $7,5$ euroa
	  § Kuuden viikon jälkeen. (Kun $f(x)=5$, $x=6$)
	  § Kahdeksan viikon jälkeen. (Kun $f(x)=0$, $x=8$)
	% §
}
  \end{vastaus} 
\end{tehtava}

\subsubsection*{Hallitse kokonaisuus}

%Laatinut V-P Kilpi 2013-11-09
\begin{tehtava}
  Olkoon $f(x)=\frac{x}{3}-4$ ja $g(x)=3x-7$. Millä $x$:n arvolla pätee $ f(x)=g(x)$?
  \begin{vastaus}
$x=\frac{9}{8}$
  \end{vastaus}
\end{tehtava}

\begin{tehtava}
%Laatinut Emilia Välttilä 9.11.2013
  Mikä on funktion määrittelyehto?
  \alakohdatm{
    § $f(x)= \sqrt{x+6}$
    § $f(x)= \frac{2x}{\sqrt{x+1}}$
    § $f(x)= \sqrt{x-\frac{1}{2}}$
  }
  \begin{vastaus}
  \alakohdatm{
  § $x\geq-6$
  § $x>-1$
  §$x\geq\frac12$
  }
  \end{vastaus}
\end{tehtava}

\begin{tehtava}
	Määritä funktion $g$ arvojoukko, kun
	\alakohdatm{
		§ $g(x)=x-\sqrt{2}$
		§ $g(x)=\frac{3}{x-2}$
		§ $g(x)=\sqrt[4]{x}$
		§ $g(x)=\sqrt[3]{x}$.
	}
	\begin{vastaus}
	\alakohdatm{
	§ $\mathbb{R}$
	§ $\mathbb{R}$ $\setminus \lbrace 0 \rbrace$
	§ $\mathbb{R}_+$ (tai $\mathbb{R}_{\geq 0}$ tai $[0, \infty [$)
	§ $\mathbb{R}$
	}
	\end{vastaus}
\end{tehtava}
%laatinut Jaakko Viertiö 14.12.2013

\begin{tehtava}
	Lääketieteen valintakoneessa on monivalintatehtävä, jossa kustakin oikeasta vastauksesta saa $0,25$ pistettä ja kustakin väärästä vastauksesta pistemäärästä vähennetään $0,5$ pistettä. Monivalintakohtia on $28$. Esitä kokonaispistemäärä $p$ väärien vastauksien määrän $v$ funktiona olettaen kokelaan vastaavan jokaiseen tehtävään. Kuinka moneen kohtaan kokelas on korkeintaan vastannut väärin, jos hän saa positiivisen kokonaispistemäärän?
	\begin{vastaus}
		$p(v) = 7-\frac{3v}{4}$. Kokelas voi antaa korkeintaan $9$ väärää vastausta menemättä nollille.
	\end{vastaus}
\end{tehtava}

%\begin{tehtava}
%Mitä epämääräisyyksiä on seuraavissa ilmaisuissa?
%\alakohdat{
%§ ''Funktion ratkaisu on $3$.''
%§ ''Yhtälön nollakohta on $4$.''
%§ ''Funktio on nouseva.''
%§ ''Funktio leikkaa vaaka-akselin kohdassa $80$.''
%}
%	\begin{vastaus}
%	\alakohdat{
%	§
%	§
%	§
%	§
%	}
%	\end{vastaus}
%\end{tehtava}

\begin{tehtava}
%Laatinut Emilia Välttilä 10.11.2013
Suorakaiteen muotoisen aitauksen toinen sivu on $2$\,m pidempi kuin toinen. Muodosta aitauksen pinta-alan funktio $A(x)$. Mikä on aitauksen pinta-ala, kun lyhyemmän sivun pituus on
  \alakohdatm{
    § $0$\,m
    § $5$\,m
    § $20$\,m?
    § Mikä on funktion $A(x)$ määrittelyehto?
  }
  \begin{vastaus}
    $A(x) = x(x+2)=x^2+2x$
      \alakohdatm{
	§ $0$\,m
	§ $35$\,m
	§ $440$\,m
	§ $x\geq0$
      }
  \end{vastaus}
\end{tehtava}

\begin{tehtava}
%Laatinut Emilia Välttilä 10.11.2013
  Olkoon $f(x) = \frac{7s + 5x}{6} + 2s - 1$ ja $g(x) = -2x + \frac{1}{2}$. Määritä vakio $s$, kun
  \alakohdat{
    § $f(0)=g(0)$
    § $f\left(\frac{1}{2}\right) = g\left(\frac{1}{2}\right)$.
   }
   Määritä vielä $f(6)$ ja $g(f(6))$, kun $s=1$. 
  \begin{vastaus}
      \alakohdat{
      § $s = -\frac{3}{19}$ 
      § $s = \frac{1}{38}$
      }
   Jos $s=1$, niin $f(6) = 3$ ja $g(3) = -\frac{11}{2}$
  \end{vastaus}
\end{tehtava}

\begin{tehtava} % Emilia Välttilä 9.11.2013
	Olkoon $f(x)=-2x$. Mitä on
	\alakohdatm{
	  § $f(x+3)$
	  § $f(x)+3$
	  § $f(3x)$
  	  § $3f(x)$?
	 }
	\begin{vastaus}
		\alakohdatm{
	  	§ $-2x-6$
	  § $-2x+3$
	  § $-6x$
  	  § $-6x$
	 }
	\end{vastaus}
\end{tehtava}

\begin{tehtava} % Emilia Välttilä 9.11.2013
	Olkoon $z(x) = \frac{x^2+\frac{1}{2}x}{x}-x+5$. Määritä
	\alakohdat{
	  § $z\left(\frac{1}{2}\right)$
	  § $z(-1)$
	  § $z(2\,013)$.
	 }
	 Osoita, ettei funktion $z$ arvo riipu muuttujan arvosta.
  \begin{vastaus}
	\alakohdatm{
	  § $\frac{11}{2}$
	  § $\frac{11}{2}$
	  § $\frac{11}{2}$
	}
	Vihje: Voit supistaa. $z(x) = \frac{x^2+\frac{1}{2}x}{x}-x+5 = \frac{x(x+\frac{1}{2})}{x}-x+5 = \frac{1}{2}+5 = \frac{11}{5}$
  \end{vastaus}
\end{tehtava}

\begin{tehtava} % Emilia Välttilä 9.11.2013
	Millä vakion $a$ arvolla funktio $g(x) = \frac{2x+1+a}{5x}$ saa arvon $5$ kohdassa $x = 3$?
	\begin{vastaus}
$a=68$
	\end{vastaus}
\end{tehtava}
	 
\subsubsection*{Lisää tehtäviä}

\begin{tehtava}
	Olkoon $f(x)=\frac{x^2+3x+1}{x^2-x}$. Laske, mikäli mahdollista:
	\alakohdatm{
		§ $f(2)$
		§ $f(1)$
		§ $f(0)$
		§ $f(-1)$
	}
	\begin{vastaus}
		\alakohdatm{
			§ $\frac{11}{3}$
			§ ei määritelty
			§ ei määritelty
			§ $\frac{-1}{2}$
		}
	\end{vastaus}
\end{tehtava}

\begin{tehtava}
	Määritellään funktio $f$ lausekkeella \[f(x)=\frac{1}{2}x^2+2x-1.\] Funktion $f$ kuvaaja löytyy alta. Oletetaan lisäksi, että $a>1$. Päättele lukujen suuruusjärjestys.
	\alakohdat{
		§ $f(a)$ ja $f(a+2)$
		§ $f(-a)$ ja $f(a)$
	}
	\begin{center}
	\begin{kuvaajapohja}{0.7}{-1}{5}{-3}{3}
		\kuvaaja{-0.5*x**2+2*x-1}{\qquad$f(x)=-\frac{1}{2}x^2+2x-1$}{black}
	\end{kuvaajapohja}
\end{center}
    \begin{vastaus}
	\alakohdat{
		§ $f(a)>f(a+2)$
		§ $f(-a)<f(a)$
	}
    \end{vastaus}
\end{tehtava}

\begin{tehtava} Keksi jokin sellainen funktio, joka toteuttaa annetun ehdon. Jonkinlaisen kuvan hahmottelusta on apua.
	\alakohdat{
		§ $f(3)=2$ ja $f(24)=16$
		§ Jos $f(a)=b$, niin $f(a+1)=b+1$
		§ Jos $f(a)=b$, niin $f(a+1)=b-1$
		§ Jos $f(0)=0$ ja $z>0$, niin $f(\pm z)=z$. (Vinkki: $|\pm a|=a$)
	}
    \begin{vastaus}
	\alakohdatm{
		§ $f(x)=\frac{2}{3}x$
		§ $f(x)=x$
		§ $f(x)=-x$
		§ $f(x)=|x|$
	}
    \end{vastaus}
\end{tehtava}

\begin{tehtava}
$\star$ Millä muuttujan $x$ arvoilla yhtälö $f(f(x)) = x$ pätee, kun
	\alakohdatm{
		§ $f(x) = 1$
		§ $f(x) = x$
		§ $f(x) = x+1$
		§ $f(x) = 2x+1$?
	}
	\begin{vastaus}
		\alakohdatm{
			§ $x = 1$
			§ kaikilla $x\in\rr$
			§ ei ratkaisuja
			§ $x = -1$
		}
	\end{vastaus}
\end{tehtava}

\begin{tehtava}$\star$ % Emilia Välttilä 9.11.2013 (Sampo Tiensuun idea)
	Olkoon $f(x) =\sqrt{x} $ ja 
	\begin{equation*}
	g(x)=
		\begin{cases}
		  0, & x \in \mathbb{Q},\\
		  1, & x \notin \mathbb{Q}.
		\end{cases}
	\end{equation*}
	Laske
	\alakohdatm{
			§ $g(f(2))$
			§ $g(f(0))$.
	}
    \begin{vastaus}
	\alakohdatm{
		§ $1$
		§ $0$
	}
    \end{vastaus}
\end{tehtava}

%Laatinut V-P Kilpi 2013-11-09
\begin{tehtava}$\star$
  Olkoon $f(x)=\frac{x}{5}$ ja $g(x)=2x-3$. Laske
  %Muodosta yhdistetyt funktiot ja laske minkä arvon ne saavat kohdassa $ x=4 $.
\alakohdatm{
			§ $f(f(4))$
			§ $f(g(4))$
			§ $g(f(4))$
			§ $g(f(x))$
			§ $f(g(x))$.

%			§ $f(f(x))$
%			§ $f(g(x))$
%			§ $g(f(x))$
		}
  \begin{vastaus}
\alakohdatm{
    § $\frac{4}{25}$
	§ $1$
	§ $-\frac{7}{5}$	
	§ $2x/5-3$
	§ $2x/5-3/5$			
}
  \end{vastaus}
\end{tehtava}

\end{tehtavasivu}