Matematiikka on kaikkien luonnontieteiden perusta. Se tarjoaa työkaluja asioiden täsmälliseen jäsentämiseen, päättelyyn ja mallintamiseen, joita ilman yhteiskunnan teknologinen ja tieteellinen kehittyminen ei olisi mahdollista. Alasta riippuen käsittelemme matematiikassa erilaisia \textbf{objekteja}: Geometriassa tarkastelemme kaksiulotteisia \textbf{tasokuvioita} ja kolmiulotteisia \textbf{avaruuskappaleita}. Algebra tutkii \textbf{lukualueita} ja niissä määriteltäviä \textbf{laskutoimituksia}. Todennäköisyyslaskenta tarkastelee satunnaisten \textbf{tapahtumien} esiintymistä ja suhdetta toisiinsa. Matemaattinen analyysi tutkii \textbf{funktioita} ja niiden ominaisuuksia, esimerkiksi \textbf{jatkuvuutta}, \textbf{derivoituvuutta} ja \textbf{integroituvuutta}. Matemaattista analyysiä käsittelevät pitkässä matematiikassa kurssit 7, 8, 10 ja 13 sekä osin kurssit 9 ja 12. Voidaankin sanoa, että analyysi on keskeisin aihealue lukion pitkässä matematiikassa.\footnote[1]{Tämä Suomessa käytetty lähestymistapa on käytössä monissa länsimaissa. Sen sijaan esimerkiksi Balkanilla geometrialle ja matemaattiselle todistamiselle annetaan edelleen paljon suurempi painoarvo.} Matematiikka opettaa loogista päättelytaitoa ja luovaa ongelmanratkaisukykyä, mistä on hyötyä niin opinnoissa kuin elämässä yleensäkin. Sitä hyödynnetään jollakin tavalla jokaisella tieteenalalla ja myös taiteissa. 

Pitkän matematiikan ensimmäinen kurssi MAA1 Funktiot ja yhtälöt käsittelee lähinnä lukuja ja niiden operaatioita eli laskutoimituksia. Kirjassa esittelemme luvun käsitteen ja yleisimmin käytetyt lukualueet laskutoimituksineen, ja jatkamme niistä \textbf{yhtälöihin} ja funktioihin. Kurssilla luodaan tietopohja nimensä mukaisiin aiheisiin ja opitaan välttämättömimpiä työkaluja, joita tarvitaan kaikilla myöhemmillä kursseilla. Ensimmäisen kurssin tiedot riittävät myös jo monipuolisesti monien arkielämän tilanteiden -- esimerkiksi korkolaskujen -- tarkasteluun, ja suureiden sekä yksiköiden syvällinen käsittely antaa hyvän laskennallisen pohjan fysiikan ja kemian kursseille.

Tässä kirjassa käymme läpi opetussuunnitelman mukaiset keskeiset sisällöt, joita ovat
\luettelo{
§ potenssifunktio
§ potenssiyhtälön ratkaiseminen
§ juuret ja murtopotenssi
§ eksponenttifunktio.
}

Opetussuunnitelman mukaiset kurssin keskeiset tavoitteet ovat, että opiskelija
\luettelo{
§ vahvistaa yhtälön ratkaisemisen ja prosenttilaskennan taitojaan
§ syventää verrannollisuuden, neliöjuuren ja potenssin käsitteiden ymmärtämistään
§ tottuu käyttämään neliöjuuren ja potenssin laskusääntöjä
§ syventää funktiokäsitteen ymmärtämistään tutkimalla potenssi- ja eksponenttifunktioita
§ oppii ratkaisemaan potenssiyhtälöitä.
}
