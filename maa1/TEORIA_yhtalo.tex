\subsection{Peruskäsitteitä}

\termi{yhtälö}{Yhtälö} on väite kahden lausekkeen yhtäsuuruudesta.

\begin{esimerkki}
	\alakohdat{
		§ $x+3=\frac{5}{7}+2$ on yhtälö, joka väittää lausekkeiden $x+3$ ja $\frac{5}{7}+2$ olevan lukuarvoltaan yhtä suuret.
		§ $2=\frac{4}{2}$ on yhtälö, joka väittää, että luku $2$ voidaan esittää myös muodossa $\frac{4}{2}$.
	}
\end{esimerkki}

\laatikko[Yhtälön päteminen]{
	Jos yhtälön kummankin puolen lausekkeen arvo on sama, sanotaan, että \termi{yhtälön päteminen}{yhtälö pätee} tai että yhtälö on tosi.
}

\begin{esimerkki}
	\alakohdat{
		§ Yhtälö $5=3$ ei päde.
		§ Yhtälö $t=t$ on tosi kaikilla muuttujan $t$ lukuarvoilla, sillä jokainen luku on varmasti yhtä suuri itsensä kanssa.
		§ Yhtälö $x=x+1$ on epätosi kaikilla $x$:n arvoilla, koska mikään luku ei voi olla yhtä suuri kuin sitä yhtä suurempi luku.
		§ Yhtälö $x+2=0$ pätee vain siinä tapauksessa, että $x=-2$.
	}
\end{esimerkki}

%yhtä suuri ERIKSEEN, ei yhdyssana

%Virkerakenne ja yhtälöiden luokittelulaatikon otsikointi\ldots
\laatikko[Yhtälöiden luokittelu]{
	\begin{description}
		\item[Aina tosi yhtälö] Esimerkiksi yhtälöt $8=8$ ja $x=x$. Tilanteesta kytetään myös termiä \termi{identtisesti tosi}{identtisesti tosi}.
		\item[Joskus tosi yhtälö] Esimerkiksi yhtälö $x+4=7$ on tosi, kun $x=3$, ja epätosi muulloin.
		\item[Ei koskaan tosi yhtälö] Esimerkiksi yhtälö $0=1$
	\end{description}
}

Matematiikan sovelluksissa näistä tärkeimpiä ovat joskus (eli ehdollisesti) todet yhtälöt.

\laatikko[Yhtälöihin liittyviä käsitteitä]{
	\begin{description}
		\item[Tuntematon] Luku, jonka arvoa ei tiedetä. Tuntemattomia merkitään vaihtelevilla symboleilla. Jos tuntemattomia on vain yksi, sitä merkitään 	yleensä kirjaimella $x$ (alunperin kreikkalainen $\chi$, \textit{khi}).
		\item[Yhtälön ratkaisu(t)] Muuttujan arvo(t), jolla yhtälö pätee. Kutsutaan myös yhtälön juuriksi. (Nimityksellä ei ole mitään tekemistä juurenottolaskutoimituksen kanssa.)
		\item[Yhtälön ratkaiseminen] Kaikkien yhtälön ratkaisujen selvittäminen.
	\end{description}
}
Huomioi, että yhtälössä voi olla useitakin eri tuntemattomia. Kaikki kirjaimet eivät myöskään ole välttämättä tuntemattomia, vaan niillä voidaan tarkoittaa myös vakioita eli muuttumattomia lukuarvoja. Esimerkiksi fysiikassa käytetään usein eri kirjaimia merkitsemään eri luonnonvakioita.

\begin{esimerkki}
Tarkastellaan legendaarista fysiikan yhtälöä $E=mc^2$. Tässä yhtälössä on vakio $c$ eli valonnopeus, joka on luonnonvakio eikä siis (nykytiedon mukaan) muutu. Lisäksi yhtälössä on muuttujat $E$ eli energia ja $m$ eli massa. Riippuu käsiteltävästä tilanteesta, ovatko energia ja massa vakioita vai muuttujia, ja ratkaistavana tuntemattomana voi eri tilanteessa toimia mikä tahansa: $E$, $m$ tai $c$.
\end{esimerkki}

\newpage % VIIMEISTELY
\subsection{Yhtälön muokkaaminen}
%mainitse yhtälön puolet ja lausekkeideinsieventäminen, välivaiheet... ohjeita välivaiheiden kirjoittamiseen,,,
\laatikko[Yhtälön muokkaaminen]{

Koska yhtälön puolet ovat yhtä suuret, eli tarkoittavat täsmälleen samaa asiaa, niin jonkin yhtälön molemmille puolille tehdyn laskutoimituksen jälkeen ne ovat edelleen yhtä suuret.
}

%selitetään, mitä on puolittain kertominen jne., puolittain\ldots\ldots

\begin{esimerkki}
%Lisännyt Jaakko Viertiö 2013-11-10
	\luettelo{
		§ Yhtälö $3x-12=\frac{6}{2x-2}$ on yhtäpitävä yhtälön $0=0$ kanssa, sillä se saadaan tähän muotoon 
		tekemällä yhtäpitävyyden molemmille puolille aina samat laskutoimitukset. Yhtälön yksinkertaisemmasta esityksestä $0=0$ nähdään, että muuttujan $x$ arvo ei vaikuta yhtälön ratkaisuihin. Toisin sanoen puolet ovat aina samat, olipa muuttuja $x$ mikä tahansa luku. Yhtälö on siis aina tosi.
		§ Yhtälö $3x-11=\frac{6}{2x-2}$ on yhtäpitävä yhtälön $0=1$ kanssa. Jälkimmäisestä muodosta nähdään, ettei yhtälö voi olla tosi millään muuttujan $x$ arvolla, sillä yhtälön muodossa $0=1$ ei esiinny muuttujaa $x$. Täten $x$ ei vaikuta yhtälön ratkaisuihin. Yhtälö on siis aina epätosi.
		§ Yhtälö $3x-11=6(x-2)$ on yhtäpitävä yhtälön $x=\frac{1}{3}$ kanssa. Yhtälö on siis joskus tosi: täsmälleen silloin kun $x=\frac{1}{3}$. Tämä voidaan tarkistaa sijoittamalla yhtälöön $x$:n paikalle arvo $\frac{1}{3}$, jolloin yhtälön vasen ja oikea puoli ovat yhtä suuret.
	}
\end{esimerkki}

%esimerkki yhtälön ratkaisun tarkistamisesta!
%Esimerkki yhtälöstä, jolla on useita ratkaisuja! annettujen juuriehdotusten tarkistaminen
%esimerkki, että yhtälön ratkaisun voi myös arvata!


%yhtälöpariesimerkki, viittaus transitiivisuuteen

Yhtälöitä käyttämällä voidaan mallintaa monia käytännön tilanteita.

\begin{esimerkki}
Yksinkertainen esimerkki yhtälön käyttämisestä on matkaan kuluvan ajan ratkaiseminen matkan pituuden ja nopeuden avulla, mitä teimme jo yksikköluvussa perustuen siihen, että saamme tulokseksi sopivan yksikön. Kuljettu matka on nopeus kerrottuna kuljetulla ajalla. Jos matka on $15$ kilometriä ja nopeus $5$\,km/h, voimme merkitä $5\cdot t=15$, jossa $t$ on matkaan kuluvaa aikaa kuvaava \termi{muuttuja}{muuttuja}.
\end{esimerkki}

Toisaalta yhtälöjä voidaan käyttää myös abstraktimpien pulmien ratkaisuun.

\begin{esimerkki}
\alakohdat{
§ Halutaan tietää, onko olemassa lukua, jonka kolmas potenssi jaettuna kolmella on yhtä suuri kuin sen toinen potenssi jaettuna kahdella. Tällöin merkitään $\frac{x^3}{3}=\frac{x^2}{2}$. $x$:n valitseminen tuntemattoman luvun symboliksi oli mielivaltaista.
§ Minkä kahden peräkkäisen kokonaisluvun tulo on $70$? Ongelmaa lähdetään ratkaisemaan kirjoittamalla tilanteen ehto yhtälönä. Merkitään kahdesta kokonaisluvusta pienempää $n$:llä. Tällöin seuraava kokonaisluku on $n+1$. Yhtälöksi saadaan $n(n+1)=70$.
}
\end{esimerkki}
	%%\begin{esimerkki}
%%	Merkitään lausekkeet $5x+\sqrt{x}$ ja $7x+7$ yhtäsuuriksi, jolloin saadaan yhtälö $5x+\sqrt{x} = 7x+7$.
%%\end{esimerkki}

%Lisännyt Jaakko Viertiö 2013-11-10
\begin{esimerkki}
Tarkastellaan yhtälöä $x^3-2x^2-x+2=0$. Yhtälössä on yksi muuttuja $x$ ja neljä termiä: $x^3$, $-2x^2$, $-x$ ja $+2$. Yhtälön \textit{eräs} ratkaisu on $x=1$, sillä $1^3-2\cdot{1^2}-1+2=0$. Myös $x=2$ on kyseisen yhtälön ratkaisu, sillä $2^3-2\cdot{2^2}-2+2=0$. Tuntemattomalle $x$ voidaan siis joissain tapauksissa löytää useita arvoja, jotka toteuttavat yhtälön. 
\end{esimerkki}

\subsection{Yhtälön ratkaiseminen}

Tyypillinen tapa ratkaista yhtälöitä on kirjoittaa ne ilmaistuna toisella tavalla. Käytännössä tämä tarkoittaa niiden muokkaamista siten, ettei alkuperäisen yhtälön paikkansapitävyys muutu. Tällä tavalla saadaan eri yhtälö, jolla on kuitenkin samat ratkaisut. Yleensä tavoitteena on saada yhtälö muotoon, jossa yhtäsuuruusrelaation toisella puolella on vain haluttu tuntematon ja toisella puolella kaikki muu. % Yhtälön ratkaisemiseksi näitä muunnoksia toistetaan, kunnes yhtälön ratkaisu on helposti luettavissa.

\begin{esimerkki} %TARKENNUS JA KOVENTIO-OPPIA! + laatikko: mihin pyritään yhdistä tuntemattomat, omalle puolelleyhtälöä, yksin omalla puolellaan
Ratkaistaan aiemman esimerkin matkan kesto. Yhtälö on siis $5t=15$, jossa $t$ on aika, nopeus on $5$\,km/h ja matkan pituus $15$\,km. Jakamalla laskusääntöjen mukaan yhtälön molemmat puolet viidellä, saadaan yhtälö muotoon $t=3$, joka on samalla kyseisen yhtälön ratkaisu.
\end{esimerkki}

%yksiköiden kirjoittaminen yhtälöihin

Yhtälön ratkaisemista voidaan ajatella havainnollisemmin kuvittelemalla orsivaaka, joka on tasapainossa. Vasemmalla ja oikealla puolella on eripainoisia esineitä, mutta ne painavat yhteensä yhtä paljon. Jos molemmille puolille lisätään nyt saman verran painoa, vaaka on yhä tasapainossa. Samalla tavalla yhtälön molemmille puolille on sallittua lisätä sama luku.

\begin{esimerkki}
Kuvassa oleva vaaka on tasapainossa. Toisessa vaakakupissa on kahden kilon siika ja toisessa puolen kilon ahven sekä tuntematon määrä lakritsia. Kuinka paljon vaakakupissa on lakritsia?
	\begin{center}
		\includegraphics[scale=0.6]{pictures/Kuva10-1-vaaka.pdf} % CC-BY Lilja Tamminen
		%\includegraphics{unused/kala-vaaka.png} % CC-BY Hannu Köngäs
		% FIXME: Hannun piirros on viimeistellympi (smoothit värjäykset jne.), Liljan piirros taas konsistentimpi
		% kirjan muun kuvituksen kanssa. Kumpi jätetään? Nyt päällä Liljan kuva.
	\end{center}
	\begin{esimratk}
Merkitään lakritsin määrää tuntemattomalla $x$. Tilannetta kuvaa yhtälö $2 = 0,5 + x$, joka muodostetaan vaa'an toiminnan ymmäryksen avulla: tasapainossa molemmissa vaakakupeissa tulee olla massaltaan yhtä paljon ainetta, eli massat (määrät) voidaan merkitä yhtä suuriksi.

Ratkaistaan yhtälö.
		\begin{align*}
			2 &= 0,5 + x &&\text{|| $-0,5$} \\
			2-0,5 &= 0,5 + x -0,5 &&\text{|| sievennetään} \\
			1,5 &= x && \text{|| yhtälön voi halutessaan kääntää}\\
			x &= 1,5 &&
		\end{align*}
	\end{esimratk}
	\begin{esimvast}
		Lakritsia on $1,5$\,kg.
	\end{esimvast}
\end{esimerkki}

Huomioi aina, asettaako yhtälön alkuperäinen muotoilu joitakin rajoituksia tuntemattoman mahdollisille arvoille.

\begin{esimerkki}
Ratkaistaan tuntematon $a$ yhtälöstä $\frac{1}{a}=\frac{1}{a-1}$. Huomioidaan, että yhtälössä jaetaan $a$:lla -- $a$ ei voi olla nolla. Koska yhtälössä jaetaan myös $a-1$:llä, $a$ ei voi saada myöskään arvoa yksi.
		\begin{align*}
			\frac{1}{a}&=\frac{2}{a-1} && \text{|| kerrotaan $a$:lla}\\
			\frac{1}{a}\cdot a&=\frac{2}{a-1}\cdot a && \text{|| sievennetään}\\	
			1&=\frac{2a}{a-1} && \text{|| kerrotaan $a-1$:llä}\\
			1\cdot (a-1)&=\frac{2a}{a-1}\cdot (a-1) && \text{|| sievennetään}\\
			a-1&=2a && \text{|| $-2a$}\\
			a-1-2a&=2a-2a && \text{|| sievennetään}\\
			-a-1&=0 && \text{|| $+1$}\\
			-a-1+1&=0+1 && \text{|| $+1$}\\
			-a&=1 && \text{||  sievennetään}\\
			-a&=1 && \text{||  $\cdot (-1)$}\\
			-a\cdot (-1)&=1\cdot(-1) && \text{|| sievennetään}\\
			a&=-1 && \text{|| sievennetään}\\
\end{align*}

Yhtälöllä on siis ratkaisu $a=-1$. Tämä ei ole ristiriidassa alkuperäisten rajoitusten kanssa, joten se hyväksytään sellaisenaan.
\end{esimerkki}

Yhtälönratkaisun sievennysvaiheita voi hypätä yli sitä mukaa, kun varmuus kehittyy.

Mitään varsinaista ei virhettä yhtälönratkaisussa ei ole tehty niin kauan kuin sama operaatio tehdään aina yhtälön molemmille puolille, mutta vaatii rutiinia, jotta oppii, mitä laskutoimituksia kannattaa tehdä ja missä järjestyksessä. Seuraavassa joitakin yleisiä toimenpiteitä yhtälöiden ratkaisuun, joita voi soveltaa päästäkseen eteenpäin. Kyseessä ei ole mikään aina toimiva ratkaisuohje, vaan yleisiä vinkkejä.

\laatikko[Yleisiä yhtälönratkaisuperiaatteita]{
	\begin{description}
		\item[1)] Kerro tuntematonta sisältävät lausekkeet pois nimittäjistä.
		\item[2)] Yhdistä useat murtolausekkeet yhdeksi laventamalla ne samannimisiksi.
		\item[3)] Siirrä tuntemattomia sisältävät yhtälön osat samalle puolelle ja ota tuntematon yhteiseksi tekijäksi.
		\item[4)] Kumoa juuret ja murtopotenssit korottamalla yhtälö puolittain sopivaan potenssiin.
		\item[5)] Sulkuja ei välttämättä aina kannata kertoa auki.
		\item[6)] Yhtälö on ratkaistu vasta, kun jäljellä on enää vain yksi kappale tuntematonta suuretta, ja se sijaitsee yksin omalla puolellaan yhtälöä.
	\end{description}
	
}

Usein on myös perusteltua yksinkertaisesti kokeilla tai arvata jokin yhtälön ratkaisu.