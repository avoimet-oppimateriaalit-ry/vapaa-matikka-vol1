\begin{tehtavasivu}
%tehtäviä päättelystä ja syntaksin ymmärtämisestä?

\subsubsection*{Opi perusteet}

%\begin{tehtava}
%Mitkä seuraavista esimerkeistä ovat lukuja?
%\alakohdat{
%
%}
%	\begin{vastaus}
%	\alakohdat{
%	§
%	§
%	}
%	\end{vastaus}
%\end{tehtava}

\begin{tehtava}
Kirjoita luvut oikeaoppisesti.
\alakohdat{
§ kolmekymmentäseitsemäntuhatta kuusisataakuusitoista ja kolme tuhannesosaa
§ miinus kuusikymmentäseitsemänmiljoonaa kolmekymmentäkuusituhatta kahdeksan
}
	\begin{vastaus}
	\alakohdat{
	§ $37\,616,003$
	§ $-67\,360\,008$
	}
	\end{vastaus}
\end{tehtava}

\begin{tehtava}
Olkoon ihmisten joukossa relaatio ''$a$ on $b$:n sisar'', jota merkitään $a \bowtie b$. Perustele, onko relaatio $\bowtie$
\alakohdat{
	§ refleksiivinen
	§ symmetrinen
	§ transitiivinen.
}
	\begin{vastaus}
	\alakohdat{
		§ Kukaan ei ole itsensä sisar, joten relaatio ei ole refleksiivinen.
		§ Sisaruus on aina molemminpuolista (jos $a$ on $b$:n sisar, niin myös $b$ on $a$:n sisar), joten relaatio on symmetrinen.
		§ Jokainen mielivaltaisen henkilön sisar on myös muiden kyseisen henkilön sisarten sisar (sisaruus ''välittyy eteenpäin''), joten relaatio on transitiivinen.
	}
	\end{vastaus}
\end{tehtava}

\begin{tehtava}
Onko relaatio refleksiivinen, symmetrinen tai transitiivinen?
\alakohdat{
§ Kahden luvun välinen relaatio $\leq$
§ Kahden luvun välinen relaatio $>$
§ ''$a$ on $b$:n lapsi'', missä $a$ ja $b$ edustavat ihmisiä 
§ "$a$ on $b$:n työkaveri'', missä $a$ ja $b$ edustavat ihmisiä, jotka töissä paikassa, missä on töissä muitakin
}
	\begin{vastaus}
	\alakohdat{
	§ refleksiivinen ja transitiivinen
	§ transitiivinen
	§ ei mikään vaihtoehdoista
	§ symmetrinen (ei transitiivinen, koska oma työkaveri voi olla töissä lisäksi jossain muualla)
	}
	\end{vastaus}
\end{tehtava}

\begin{tehtava}
Olkoot $l_1$ ja $l_2$ tason suoria. Määritellään niiden välinen relaatio $l_1 \parallel l_2$, joka luetaan ''suora $l_1$ on yhdensuuntainen suoran $l_2$ kanssa''. Onko relaatio refleksiivinen, symmetrinen tai transitiivinen?
%§ Miksi suorien nimeämisessä käytettiin alain
	\begin{vastaus}
	Relaatio on symmetrinen, symmetrinen ja transitiivinen.
	\end{vastaus}
\end{tehtava}
\end{tehtavasivu}