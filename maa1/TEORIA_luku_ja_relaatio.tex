\qrlinkki{http://opetus.tv/maa/maa1/lukujoukot-ja-lukujen-ominaisuuksia/}
{Opetus.tv: \emph{Lukujoukot ja lukujen ominaisuuksia} ($7.17$, $5.08$, $6.56$ ja $7.47$)}

\subsection{Luvun käsite}

\termi{luku}{Luku} on käsite, jolla voidaan ilmaista esimerkiksi suuruutta, lukumäärää tai järjestystä. Erityisesti on huomattava, että luku ja \termi{numero}{numero} eivät ole synonyymejä; numero on yksittäinen lukujärjestelmän merkki. Luvulla siis on aina suuruus mutta numerolla ei.

\begin{esimerkki}
\alakohdat{
	§ Luku $715\,531$ koostuu numeroista 7, 1, 5, 5, 3 ja 1.
	§ Luku $9$ koostuu ainoastaan vastaavasta numeromerkistä 9.
	}
\end{esimerkki}

Joskus numerosarjoja ei tulkita minkään asian suuruudeksi tai lukumääräksi, vaan ne ovat vain joitakin asioita koodaavia merkkijonoja. Esimerkiksi postinumero 00900 merkitsee postinkantajalle tiettyä aluetta, jolle postia jaetaan. Tämä ei kuitenkaan tarkoita, että esimerkiksi postinumero 46900 olisi jollakin tapaa suurempi kuin postinumero 00900.

%\begin{esimerkki}
%\alakohdat{
%§ Virkkeessä "Joakimilla on lemmikkinä $2$ kania" luku $2$... 
%§ järjestys
%§ indeksointi
%§
%}
%\end{esimerkki}

Englannin kielen sana \textit{number} voi viitata sekä numeroon että lukuun. Sana \textit{digit} tarkoittaa pelkästään yhtä numeromerkkiä. Ruotsiksi luku on \textit{tal}, lukumäärä \textit{antal} ja numeroa tai lukumäärää tarkoittamatonta numeroyhdistelmää kuvaa suomen kielen tapaan sana \textit{nummer}. Huomattavaa on myös, että suomen kielen verbi laskea voi tarkoittaa sekä lukumäärän laskemista että abstraktimpien laskutoimitusten suorittamista -- sen sijaan englannissa näille on omat verbinsä: lukumäärälle \textit{count} ja vastaavasti \textit{calculate} laskutoimituksille.

\termi{tuhaterotin}{Tuhaterottimena} käytetään suomenkielisessä tekstissä välilyöntiä, ei pilkkua tai pistettä. (Kieliopillisena poikkeuksena vuosiluvut kuten 2014 ja erilaiset koodit kirjoitetaan ilman tuhaerotinta.) Tuhaterotin erottaa kaikki kolmen numeron ryhmät luvussa mahdollista desimaaliosaa lukuun ottamatta. Jos lukuja kirjoitetaan sanoilla, tulee sanaväli samaan paikkaan kuin tuhaterotinkin. \termi{desimaalierotin}{Desimaalierotin} puolestaan on suomenkielisessä tekstissä pilkku -- ei piste kuten yhdysvaltalaisissa laskimissa. Tekniikan alan teksteissä suositellaan kuitenkin nykyään myös amerikanenglannissa käytettäväksi tuhaterottimena ohutta väliä pilkun tai pisteen sijaan. %lähdeviite %FIXME: TARKISTA TUO DESIMAALIEN JAOTTELEMATTOMUUS!

\begin{esimerkki}
\alakohdat{
§ Luku $2\,030,0005$ lausutaan ja on kirjoitettuna ilman numeroita ''kaksituhatta kolmekymmentä ja viisi tuhannesosaa''.
§ Luku kaksimiljoonaa kolmesataaseitsemänkymmentäkaksituhatta neljäsataayhdeksänkymmentäkaksi kirjoitettuna numeroilla on $2\,372\,492$.
§ Jos näet yhdysvaltalaisessa yhteydessä ilmaisun $1,000.50$ dollaria, tämä tarkoittaa tuhatta dollaria ja $50$ senttiä. Suomenkielisessä tekstissä sama kirjoitettaisiin $1\,000,50$ dollaria.
}
\end{esimerkki}

Matematiikan merkintätavat eivät usein ole mitään niin sanottua luonnollista kieltä kuten suomi tai tanska, vaan asiat pyritään merkitsemään tiivisti ja mahdollisimman yleisin merkinnöin niin, että samoja merkintöjä ymmärtäisi joku muukin kuin samaa äidinkieltä puhuva. Matematiikan kehitys on sitä myöten tuonut mukanaan suuren määrän erilaisia symboleita ja historian varrella muodostuneita käytäntöjä.

Matematiikassa lukuja merkitään tai pikemminkin symboloidaan usein kirjaimilla. Tästä on se etu, että kirjain voi tarkoittaa mitä lukua tahansa, joten kirjainten avulla pystytään ilmaisemaan kaikille luvuille päteviä yleisiä sääntöjä tarvitsematta välittää siitä, mikä luku johonkin paikkaan kuuluu. Matematiikassa pyritään yleistämään ilmaisut mahdollisimman kattaviksi, jotta poikkeuksia on mahdollisimman vähän. %formalisointi--- muodollistaminen? :S

\begin{esimerkki}
\alakohdat{
§ Voidaan tiivistää $x=x$ ilmaisemaan, että jokainen luku on yhtä suuri itsensä kanssa, eikä meidän tarvitse erikseen luetella mahdollisia (kaikkiaan ääretöntä määrää) lukuja: $0=0$, $1=1$, $2=2$, \ldots
§ Kaikille (ainakin lukiotasollla käsiteltäville) luvuille $x$ ja $y$ pätee $x+y=y+x$, eli yhteenlaskun järjestyksellä ei ole väliä. Tämä on paljon helpompaa kuin luetella (äärettömän monta tapausta), esimerkiksi $2+3=3+2$, $2+4=4+2$, $3+4=4+3$, $2+5=5+2$ ja niin edelleen.
}
\end{esimerkki}

Yleensä samalla kirjaimella tarkoitetaan samassa asiayhteydessä aina samaa lukua. Esimerkiksi edellisessä esimerkissä $x$ tarkoittaa samaa lukua yhtälön kummallakin puolella. $y$ tarkoittaa myös jotain lukua, joka on sama yhtälön kummallakin puolella. $y$ voi tarkoittaa eri lukua kuin $x$, mutta ei välttämättä. On myös mahdollista, että $x$ ja $y$ voivat tarkoittaa samaa lukua -- siispä yllä mainittu sääntö kertoo myös, että esimerkiksi $2+2=2+2$. 

%tää alarivi v on vähän hassu (symbolilla lukuarvo?)
Symboleja, joiden lukuarvo voi vaihdella, on tapana kutsua \termi{muuttuja}{muuttujiksi}, ja sellaisia, joiden lukuarvon ajatellaan pysyvän muuttumattomana, on tapana kutsua \termi{vakio}{vakioiksi}.

\begin{esimerkki}
Luku $4$ on vakio, koska sen suuruus on aina sama asiayhteydestä riippumatta.
%selitys muuttujasta ja parametristä; kirjain voi myös viitata vakioon!
\end{esimerkki}

%kellonajoista? versionumeroista?

\subsection{Relaatiot ja laskutoimitukset}

Matematiikassa olemme kiinnostuneita tutkimuksen kohteen (yleensä lukujen) ominaisuuksista ja niiden suhteista toisiin asioihin. Riippuvuussuhteita kutsutaan matematiikassa \termi{relaatio}{relaatioiksi}. Tulimme jo käyttäneeksi kahta erilaista relaatiota, joilla kummallakin on oma merkintänsä: yhteenlasku ja yhtäsuuruus. Yhteenlasku $+$ yhdistää kaksi lukua yhdeksi uudeksi luvuksi, ja yhtäsuuruus $=$ vertailee kahta lukua keskenään.

Harjoituksena muodollisemmasta (formaalimmasta) matematiikasta esitämme tärkeitä yhtäsuuruusrelaation ominaisuuksia. Yhtäsuuruus on kahden luvun välinen riippuvuus, jolla on seuraavat kolme tärkeää ominaisuutta:

\numerointilaatikko{Yhtäsuuruusrelaation ominaisuudet}{
§ \termi{refleksiivisyys}{Refleksiivisyys}: Jokainen luku on yhtä suuri itsensä kanssa, eli $x=x$ kaikilla luvuilla $x$.
§ \termi{symmetrisyys}{Symmetrisyys}: Yhtäsuuruus toimii molempiin suuntiin, eli jos $x=y$, niin myös $y=x$ kaikilla $x$ ja $y$.
§ \termi{transitiivisuus}{Transitiivisuus}: Jos $x=y$ ja $y=z$, niin $x=z$ kaikilla $x, y$ ja $z$.
}

\newpage % VIIMEISTELY
\begin{esimerkki}
\alakohdat{
§ Ei ole väliä, kirjoitetaanko $x=3$ vai $3=x$, koska nämä tarkoittavat yhtäsuurudeen symmetrisyyden perusteella täysin samaa asiaa.
§ Jos $t=3$ ja toisaalta $y=t$, niin silloin välttämättäkin transitiivisuuden perusteella $y=t$.
}
\end{esimerkki}

On hyvinkin mahdollista, että nämä ominaisuudet voivat tuntua lukijasta \termi{triviaali}{triviaaleilta} eli niin yksinkertaisilta ja itsestään selviltä, ettei niitä tarvitse perustella tai edes mainita. Matematiikka on kuitenkin kumulatiivinen eli edellisen päälle rakentuva tieteenala, ja ominaisuudet kuten refleksiivisyys on tärkeää nimetä, jotta niihin voidaan myöhemmin viitata ja vedota. Matematiikassa pyritään myös tarkkuuteen, ja kaikki asiat -- myös triviaalit -- määritellään hyvin. Hyvin määritelty tarkoittaa yksinkertaista ja yksikäsitteistä.

Peruskoulussakin käytettyjä relaatioita ovat myös $<$ (pienempi kuin), $>$ (suurempi kuin), $\leq$ (pienempi tai yhtä suuri kuin) ja $\geq$ (suurempi tai yhtä suuri kuin), joilla vertaillaan kahden luvun välistä suuruutta.

\begin{esimerkki}
	\alakohdat{
§ Väite $4<5$ pitää paikkansa, mutta $5<4$ ei. Kyseinen relaatio ei siis ole symmetrinen toisin kuin yhtäsuuruus, koska vasemman- ja oikeanpuoleista lukua ei voida vaihtaa keskenään.
§ Edustakoot $a$ ja $b$ ihmisiä, ja relaatiota ``$a$ rakastaa $b$:tä'' merkitään $a \heartsuit b$. Tällöin relaatio $\heartsuit$ ei ole symmetrinen (tunteet eivät aina ole molemminpuolisia), refleksiivinen (kaikki eivät rakasta itseään) eikä transitiivinen (rakastaminen ei välity kolmansille osapuolille).
}
\end{esimerkki}

Huomaa, että refleksiivisyyden, symmetrisyyden ja transitiivisuuden määritelmissä käytetty maininta ''-- -- kaikilla $x$'' on hyvin olennainen. Relaation ominaisuuksien täytyy toimia aina ja kaikissa tilanteissa. Relaatio ja muuttujat täytyy määritellä selvästi, jottei epävarmuuksille jää sijaa!

%\begin{esimerkki}
%Urheilujoukkueiden $X$ ja $Y$ välinen relaatio ''$X$ on joukkueen $Y$ vihollinen'' ei ole refleksiivinen. Se on symmetrinen, mutta transitiivisuus riippuu siitä, 
%\end{esimerkki}
Yhteenlasku on esimerkki relaatiosta, joka yhdistää kaksi lukua johonkin kolmanteen lukuun -- tällaisia relaatioita kutsutaan \termi{laskutoimitus}{laskutoimituksiksi}. (Yhteenlaskun tapauksessa tätä kolmatta lukua kutsutaan yhteenlaskun summaksi.) Kyseessä on tärkeä ero: kyseinen relaatio ei siis vain liitä kahta lukua toisiinsa kuten $=$ tai $>$. Vielä yleisemmin puhutaan \termi{operaatio}{operaatioista}, kun relaation kohteena eivät välttämättä ole luvut vaan jotkin abstraktimmat oliot.
