\begin{tehtavasivu}

\begin{tehtava}
Laske ja totea murtolukujen desimaaliesitysten paikkansapitävyys.
\alakohdatm{
§ $ \frac{1}{10} = 0,1$
§ $ \frac{1}{100} = 0,01$
§ $ \frac{1}{2} = 0,5$
§ $ \frac{1}{4} = 0,25$ 
§ $ \frac{3}{4} = 0,75$
}
%Oli toivottu tehtäviä jakokulmasta, joten voisiko tämän tehtävän tehtävänantoon lisätä 
%``desimaaliesitysten paikkansapitävyys laskemalla jakokulmassa.'' (P Thitz 2014-02-08.)
\end{tehtava}

\begin{tehtava}
Esitä murtolukuna
\alakohdatm{
§ $0,5$
§ $0,333\ldots$
§ $1,234$
§ $-0,2$.
}
	\begin{vastaus}
\alakohdatm{
§ $\frac{1}{2}$
§ $\frac{1}{3}$
§ $\frac{1\,234}{1\,000}=\frac{617}{500}$
§ $-\frac{1}{5}$
}
	\end{vastaus}
\end{tehtava}

\begin{tehtava}
Esitä desimaalilukuna
\alakohdatm{
§ $-\frac{3}{2}$
§ $\frac{3}{4}$
§ $\frac{12}{20}$
§ $-1\frac{1}{6}$.
}
	\begin{vastaus}
\alakohdatm{
§ $-1,5$
§ $0,75$
§ $0,6$
§ $-1,1666\ldots$
}
	\end{vastaus}
\end{tehtava}

\begin{tehtava}
Muuta murtolukumuotoon (murtolukua ei tarvitse sieventää)
	\alakohdatm{
		§ $43,532$
		§ $5,031$
		§ $0,23$
		§ $0,3002$
		§ $0,101$.
	}
	\begin{vastaus}
	\alakohdatm{
		§ $ \frac{43\,532}{1\,000}$
		§ $ \frac{5\,031}{1\,000}$
		§ $ \frac{23}{100}$
		§ $ \frac{3\,002}{1\,000}$
		§ $ \frac{101}{1\,000}$
	}
	\end{vastaus}
\end{tehtava}

\begin{tehtava}
Muuta murtolukumuotoon ja sievennä
	\alakohdatm{
		§ $0,01$
		§ $0,0245$
		§ $0,004$
		§ $0,001004$.
	}
	\begin{vastaus}
	\alakohdatm{
		§ $ \frac{1}{100}$
		§ $ \frac{49}{200}$
		§ $ \frac{1}{250}$
		§ $ \frac{251}{250\,000}$
	}
	\end{vastaus}
\end{tehtava}

\begin{tehtava}
Muuta murtolukumuotoon
	\alakohdatm{
		§ $0,77777\ldots$
		§ $0,151515 \ldots$
		§ $2,05\overline{631}$
		§ $0,99999\ldots$.
	}
	\begin{vastaus}
	\alakohdatm{
		§ $\frac{7}{9}$ 
		§ $\frac{15}{99}=\frac{5}{33}$
		§ $\frac{205\,426}{99\,900} = \frac{102\,713}{49\,950}$
		§ $\frac{9}{9} = 1$
	}
	\end{vastaus}
\end{tehtava}

\begin{tehtava}
Muuta desimaaliluvuksi
	\alakohdatm{
		§ $\frac{151}{250}$
		§ $\frac{251}{625}$
		§ $\frac{386}{1\,250}$
		§ $\frac{493}{500}$.
	}
	\begin{vastaus}
	\alakohdatm{
		§ $0,604$
		§ $0,4016$
		§ $0,3088$
		§ $0,986$
	}
	\end{vastaus}
\end{tehtava}

\begin{tehtava}
Muuta murtoluvuksi
	\alakohdat{
		§ $0,\overline{649}$
		§ $0,\overline{2154}$.
	}
	\begin{vastaus}
	\alakohdat{
		§ $\frac{649}{999}$
		§ $\frac{718}{3\,333}$.
	}
	\end{vastaus}
\end{tehtava}

\begin{tehtava}
Muuta desimaaliluvuksi
	\alakohdatm{
		§ $\frac{42}{11}$
		§ $\frac{37}{13}$
		§ $\frac{38}{99}$
		§ $\frac{14}{15}$.
	}
	\begin{vastaus}
	\alakohdatm{
		§ $3,\overline{81}$
		§ $2,\overline{846153}$
		§ $0,\overline{38}$
		§ $0,9\overline{3}$
	}
	\end{vastaus}
\end{tehtava}

\begin{tehtava}
Perustele, miksi $0,9999\ldots=1$.
	\begin{vastaus}
Koska tunnetusti $\frac{1}{3}=0,3333\ldots$, niin kertomalla molemmat lausekkeet kolmella saadaan toisaalta $0,9999\ldots$ ja toisaalta $3\cdot \frac{1}{3}=\frac{3}{3}=1$. Siis $0,9999\ldots=1$.
	\end{vastaus}
\end{tehtava}

%\begin{tehtava}    %TEHDÄÄN NÄISTÄ ESIMERKKEJÄ teoriaosuuteen
%	Tarkastellaan jaksollista desimaalilukua \(a=0,1212\ldots\) Jakson pituus on 2 ja \(100a=12,1212\ldots\) Nyt \(100a-a=12\), joten \[a=\frac{12}{99}.\] Vastaavasti jos \(b=0,314314\ldots\), niin jakson pituus on 3 ja \(1000b=314,314314\ldots\) Siis \(999b=314\) ja niinpä \[b=\frac{314}{999}.\] Määritä samalla tekniikalla jaksollisten desimaalilukujen
%	\alakohdat{
%		§ $0,2020\ldots$
%		§ $0,118118\ldots$
%		§ $0,333\ldots$
%		§ $3,1414\ldots$
%	}
%murtolukuesitykset.	
%	\begin{vastaus}
%		\alakohdat{
%			§ $20/99$
%			§ $118/999$
%			§ $1/3$
%			§ $3+14/99$
%		}
%	\end{vastaus}
%\end{tehtava}

\end{tehtavasivu}