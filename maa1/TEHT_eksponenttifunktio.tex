\begin{tehtavasivu}

\subsubsection*{Opi perusteet}

%Tarkistanut V-P Kilpi 2013-11-10
\begin{tehtava}
Olkoon $f(x) = 4^x$. Laske
\alakohdatm{
§ $f(0)$
§ $f(3)$
§ $f(\frac{1}{2})$.
}
\begin{vastaus}
\alakohdatm{
§ $1$
§ $64$
§ $2$
}
\end{vastaus}
\end{tehtava}

%Laatinut V-P Kilpi 2013-11-10
\begin{tehtava}
Olkoon $g(x) = (\frac{1}{4})^x$. Laske
\alakohdatm{
§ $f(0)$
§ $f(4)$
§ $f(\frac{1}{2})$.
}
\begin{vastaus}
\alakohdatm{
§ $1$
§ $\frac{1}{256}$
§ $\frac{1}{2}$
}
\end{vastaus}
\end{tehtava}

%Tarkistanut V-P Kilpi 2013-11-10
\begin{tehtava}
Olkoon $f(x) = 10^x$. Millä $x$:n arvoilla
\alakohdatm{
§ $f(x) = 1\,000$
§ $f(x) = \frac{1}{100}$
§ $f(x) = -1$?
}
\begin{vastaus}
\alakohdatm{
§ $3$
§ $-2$
§ Ei ratkaisua.
}
\end{vastaus}
\end{tehtava}

%Laatinut V-P Kilpi 2013-11-10
\begin{tehtava}
Eksponenttifunktio $ e(x)=2^{x}$ kuvaa ihmisen esivanhempien määrää, kun $ x $ on välissä olevien sukupolvien määrä. Kuinka monta esivanhempaa ihmisellä on
\alakohdat{
§ kolmen sukupolven päässä
§ kymmenen sukupolven päässä
§ kahdenkymmenen sukupolven päässä?
} 
\begin{vastaus}
\alakohdatm{
§ $8$
§ $1\,024$
§ $1\,048\,576$
} 
\end{vastaus}
\end{tehtava}

%Laatinut V-P Kilpi 2013-11-10
\begin{tehtava}
Leipuri haluaa tehdä tuulihattuja varten lehtitaikinaa, jossa on yli tuhat kerrosta. Alussa kerroksia on yksi ja yhdellä taikinan taitoksella ja kaulinnalla kerrosten määrä aina kolminkertaistuu. Kuinka monen taitoksen jälkeen lehtitaikinassa on yli tuhat kerrosta?
\begin{vastaus}
Seitsemän taitoksen jälkeen, jolloin kerroksia on $2\,187$
\end{vastaus}
\end{tehtava}

%Laatinut V-P Kilpi 2013-11-10
\begin{tehtava}
Biologi vei autiosaarelle sata jänistä tutkiakseen, miten hyvin funktio $ p(t)=100 \cdot 1,05^{t}$, missä $t$ on aika viikkoina, kuvaa populaation kasvua. Jos malli toimii hyvin, miten paljon jäniksiä tulisi saarella olla vuoden kuluttua kokeilun aloittamisesta?
\begin{vastaus}
Noin $1\,300$
\end{vastaus}
\end{tehtava}

%Tarkistanut V-P Kilpi 2013-11-10
\begin{tehtava}
Olkoon $f(t) = 20 \cdot 2^t$ bakteerien lukumäärä soluviljelmässä ajanhetkellä $t$. Millä ajanhetkellä bakteerien lukumäärä on tasan $160$?
\begin{vastaus}
Ajanhetkellä $t=3$ %yhdyssana?
\end{vastaus}
\end{tehtava}

%Tarkistanut V-P Kilpi 2013-11-10
\begin{tehtava}
Miten muokkaisit edellisen tehtävän funktiota, jos bakteerien lukumääräksi halutaan alussa $5$?
\begin{vastaus}
muutetaan kerroin: $f(t) = 5 \cdot 2^t$
\end{vastaus}
\end{tehtava}

\subsubsection*{Hallitse kokonaisuus}

%Laatinut V-P Kilpi 2013-11-10
\begin{tehtava}
Eräälle eksponenttifunktiolle, joka on muotoa $ f(x)=k\cdot a^{x}  $, missä $ a $ ja $ k $ ovat vakioita, pätee $ f(0)=12 $ ja $ f(2)=192 $ Määritä vakiot $a$ ja $k$.
\begin{vastaus}
$f(x)=12\cdot 4^{x}$
\end{vastaus}
\end{tehtava}

%Tarkistanut V-P Kilpi 2013-11-10
\begin{tehtava}
Minkä kahden peräkkäisen kokonaisluvun välissä yhtälön $10^x = 500$ ratkaisu on?
\begin{vastaus}
Ratkaisu on lukujen $2$ ja $3$ välissä.
\end{vastaus}
\end{tehtava}

%Laatinut V-P Kilpi 2013-11-10
\begin{tehtava}
Eräänä vuonna omenapuuhun tuli $90$ omenaa, seuraavana vuonna $108$ ja sitä seuraavana vuonna $43$ omenaa. Voiko eksponenttifunktiolla kuvata tämän omenapuun omenoiden määrää ajan funktiona? Miksi/Miksei?
\begin{vastaus}
Ei voi, sillä eksponenttifunktiot ovat joko kasvavia tai väheneviä.
\end{vastaus}
\end{tehtava}

%Laatinut V-P Kilpi 2013-11-10
\begin{tehtava}
Funktio $k(t) = 10 \cdot 5^t$, missä $t$ on aika tunteina kuvaa hauskan kissavideon katselukertojen määrää ajan funktiona. Kuinka monen tunnin kuluttua katselukertoja on kertynyt yli miljoona?
\begin{vastaus}
Kahdeksan tunnin kuluttua
\end{vastaus}
\end{tehtava}

%Laatinut V-P Kilpi 2013-11-10
\begin{tehtava}
Villen levykokoelma kasvaa eksponentiaalisesti. Alussa levyjä oli $200$, vuoden päästä $300$ ja kahden vuoden päästä $450$. Muodosta funktio, joka kuvaa levykokoelman kasvamista ajan funktiona ja laske arvio sille, miten monta levyä kokoelmassa on $10$ vuoden kuluttua kokoelman keräämisen aloittamisesta.
\begin{vastaus}
Funktio on $f(t) = 200 \cdot 1,5^t$, ja kymmenen vuoden kuluttua kokoelmassa on yhteensä noin $11\,000$ levyä.
\end{vastaus}
\end{tehtava}

%Tarkistanut V-P Kilpi 2013-11-10
\begin{tehtava} %ei varsinaisesti pienene-... muuttuvat vain toisikseen – voivat jopa lisääntyä! T: JoonasD6
Jos atomiydinten, joiden määrä pienenee radioaktiivisen hajoamisen seurauksena, määrä ajanhetkellä $t=0$ on $f(t)=k$, atomiydinten määrää hetkellä $t$ kuvaava malli voidaan kirjoittaa \[f(t) = k \cdot \left( \frac{1}{2} \right)^t, t \ge 0,\] jossa $f(t)$ on atomiydinten lukumäärä ajanhetkellä $t$. Millä ajanhetkellä atomiydinten määrä on alle $1/200$ alkuperäisestä?
\begin{vastaus}
Ajanhetkellä $t = 8$
\end{vastaus}
\end{tehtava}

%Tehtävän laatinut Paula Thitz 2014-02-09.
\begin{tehtava}
Kaino-Vieno saa perinnöksi maatilan. Hän päättää jättää työnsä kirjanpitäjänä, muuttaa maalle ja alkaa viljelemään luomuperunaa. Ensimmäisenä keväänä Kaino-Vieno kylvää maahan $10$ siemenperunaa. Kesän aikana kukin siemenperuna työntää esiin perunanverson ja muodostaa keskimäärin $6$ uutta perunan mukulaa, samalla kun alun perin istutettu siemenperuna mätänee. Perunan noston jälkeen Kaino-Vieno päättää säästää $40$\,\% perunoista seuraavan kevään siemenperunoiksi. Kuinka pitkään Kaino-Vienon täytyy jatkaa luomuviljelyä, jotta perunasato ylittää $1\,000$ perunan rajan? 
	\begin{vastaus}
	Vähintään $4$ kesää. (Voidaan muodostaa perunasadon määrää kuvaava lauseke $S(n)=0,4\cdot6^n\cdot10$, missä $n$ kuvaa viljelyvuosien määrää. Halutaan selvittää, millä $n$:n arvolla $S(n)\geq1\,000$. Koska $6\geq1$, perunasatoa kuvaava funktio on kasvava -- riittää siis ratkaista $n$ yhtälöstä $S(n)=1\,000$. Saadaan $6^n=250$ ja kokeilemalla nähdään, että $6^3=216$ ja $6^4=1\,296$. Kun $n\geq4$, $S(n)\geq1\,000$.)
	\end{vastaus}
\end{tehtava}

%Laatinut V-P Kilpi 2013-11-10
\begin{tehtava}
Funktion $ h(x)=(-2)^{x}$ maalijoukko ei ole reaalilukujen joukko. Anna esimerkki muuttujan arvosta, jolla funktion arvo ei ole reaalinen.
	\begin{vastaus}
Esimerkiksi kohdassa $x = \frac{1}{2}$
	\end{vastaus}
\end{tehtava}

\subsubsection*{Lisää tehtäviä}

%Laatinut V-P Kilpi 2013-11-10
\begin{tehtava}
Villiintynyt 3D-printteri alkaa tulostamaan uusia 3D-printtereitä, jotka taas toimivat vastaavasti. Funktio $ d(x)=4^{x} $, missä $ x $ on aika päivinä, kuvaa printterien määrää ajan funktiona. Kuinka monen päivän kuluttua 3D-printtereitä on maapallolla enemmän kuin ihmisiä, kun ihmisiä on noin $7$ miljardia?
\begin{vastaus}
$17$ päivän kuluttua
\end{vastaus}
\end{tehtava}

%Laatinut V-P Kilpi 2013-11-10
\begin{tehtava}
Todennäköisyys nopalla peräkkäin heitetyille kutosille noudattaa funktiota $g(x)=(\frac{1}{6})^{x} $, missä $x$ on peräkkäisten kutosten lukumäärä. Mikä on todennäköisyys heittää nopalla kymmenen kutosta peräkkäin?
\begin{vastaus}
$ \frac{1}{6^{10}}\approx0,000000017$
\end{vastaus}
\end{tehtava}

%Laatinut V-P Kilpi 2013-11-10
\begin{tehtava}
Sukujuhliin on ostettu viiden kilon kääretorttu. Vieraita on 30, joista ensimmäinen leikkaa itselleen kahdeskymmenesosan tortusta ja seuraava aina kahdeskymmenesosan sillä hetkellä jäljellä olevasta tortusta. Muodosta funktio, joka kuvaa jäljellä olevan tortun määrää sitä leikanneiden vieraiden määrän funktiona. Kuinka paljon torttua on jäljellä, kun kaikki vieraat ovat leikanneet yhden palan?
\begin{vastaus}
Kääretorttua on jäljellä noin yksi kilogramma.
\end{vastaus}
\end{tehtava}

%Laatinut V-P Kilpi 2013-11-10
\begin{tehtava} %rephrase (funktiot f(x)... yäk)
Määritä yksi kohta, jossa funktiot $g(x)=5^{x} $ ja $ h(x)=4^{x} $ saavat saman arvon.
\begin{vastaus}
Kohdassa $x=0$
\end{vastaus}
\end{tehtava}

%Laatinut V-P Kilpi 2013-11-10
\begin{tehtava}
Funktio $ k(x)=29^x$ kuvaa kaikkien mahdollisten kirjainyhdistelmien määrää, kun yhdistelmän pituus on $x$ ja käytössä ovat suomen kielen aakkoset. Jos Suomen kymmenen merkin mittaiset henkilötunnukset korvataan kirjainyhdistelmillä, kuinka lyhyet yhdistelmät riittäisivät, jotta kaikki $5,4$ miljoonaa kansalaista saavat muista poikkeavan yhdistelmän?
\begin{vastaus}
Viisi merkkiä riittää. Tällöin erilaisia yhdistelmiä on $20\,511\,149$.
\end{vastaus}
\end{tehtava}

\end{tehtavasivu}