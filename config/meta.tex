\renewcommand{\kurssinNimi}{Vapaa matikka}
\renewcommand{\kurssinTunnus}{MAA1-2}

\renewcommand{\metasivu}{
\laatikko[Avoimet oppimateriaalit ry]{
Yhdistys tuottaa ja julkaisee oppimateriaaleja ja kirjoja, jotka ovat kaikille ilmaisia ja vapaita käyttää, levittää ja muokata. Vapaa matikka -sarja on suunnattu lukion pitkän matematiikan kursseille ja täyttää valtakunnallisen opetussuunnitelman vaatimukset.}
\vspace*{\fill}
\begin{flushleft}
	\sffamily
	Jos olet kiinnostunut Vapaa matikka -sarjan kirjoittamisesta,
	voit tehdä pull requestin muutoksillesi GitHub-palvelussa tai
	osallistua yhdistyksen toimintaan. \\
	\vspace*{25pt}
	Lisätietoja Vapaa matikka -kirjasarjasta työryhmältä,
	\href{mailto:vapaamatikka@avoimetoppimateriaalit.fi}{vapaamatikka@avoimetoppimateriaalit.fi}. \\
	\vspace*{25pt}
	1. painos 2015 \\
	\vspace*{25pt}
	ISBN 978-952-7010-06-8 (painettu kirja, MAA1-2) \\
	ISBN 978-952-7010-07-5 (sähköinen julkaisu, MAA1-2) \\
	ISBN 978-952-7010-08-2 (sähköinen julkaisu, MAA1) \\
	ISBN 978-952-7010-09-9 (sähköinen julkaisu, MAA2) \\
	\vspace*{25pt}
	YKL 51 \\
	UDK 51 \\
	\vspace*{25pt} Painotalo C-4, Vaajakoski, 2015
\end{flushleft}
}
