\renewcommand{\luku}[2]{\section{#2} \lukufilter{#1}{\input{maa1/TEORIA_#1} \input{maa1/TEHT_#1}}} % luku
\renewcommand{\nluku}[2]{\section*{#2} \addcontentsline{toc}{section}{#2} \lukufilter{#1}{\input{maa1/#1}}} % numeroimaton luku

\newpage

% siirrä tiedostoon kayttoohje.tex
\subsection*{Kirjan käyttämisestä}

Kirjan kirjoituksessa on nähty erityistä vaivaa siinä, että \textit{mitään ei oleteta osattavan etukäteen}. Esitietovaatimuksia ei ole melkeinpä lainkaan:

\luettelo{
§ intuitio yhteen-, vähennys-, kerto- ja jakolaskusta sekä kyky käyttää laskinta näiden laskemiseen; kellonajat
§ geometriaa: kolmio, suorakulmio; pituus, pinta-ala, tilavuus
§ lukusanoja: kymmenen, sata, tuhat, miljoona -- käytetään kirjan alussa, ja myöhemmin selitetään perusteellisemmin
§ desimaali -- sanaa käytetään kirjan alussa, mutta myöhemmin määritellään perusteellisemmin
}

Jos kirja ei anna tarpeeksi esimerkkejä jostakin asiasta, \textbf{ota yhteyttä}, ja asia korjataan välittömästi. \textbf{Oletusarvoisesti ymmärtämättömyyden syy ei ole sinun vaan oppimateriaalin.} Ota toki myös yhteyttä, jos sinulla on mielenkiintoisia tehtäväideoita!

Kirjan tehtävät on suurimmassa osaa luvuista jaettu kolmeen kategoriaan: Perusteet, kokonaisuuden hallitseminen ja lisätehtävät. Perustehtävät tarkistavat luvun määritelmien ja keskeisten tulosten ymmärtämisen sekä perustavanlaatuiset taidot -- nämä täytyy osata kurssin läpäisemiseen. Hallitse kokonaisuus -osiossa tehtävät ovat soveltavampia, ja ne yhdistelevät monipuolisesti luvun eri menetelmiä ja myös aiemmissa luvuissa opittuja asioita. Jos kaipaat lisäharjoitusta, sekä perustehtäviä että sekalaisia sovelluksia löytyy lisää kolmannesta osiosta. Yhteiskunnallisia tilanteita soveltavien tehtävien tiedot ovat faktuaalisia, ellei toisin mainita. Lisää tehtäviä -osion tähdellä $\star$ merkityt tehtävät ovat selvästi opetussuunnitelman ulkopuolisia tehtäviä, erittäin soveltavia, oppiainerajat ylittäviä tehtäviä, pitkiä tai tekijöiden mielestä muuten koettu mielenkiintoisiksi tai haastaviksi. Näiden tehtävien osaaminen ei missään nimessä ole vaatimus kympin kurssiarvosanaan.

Kirjan lopusta löytyy ylimääräinen teorialiite lukujärjestelmistä, kertaustehtäviä, harjoituskokeita, vanhoja yo- ja eri alojen valintakokeita, harjoitustehtävien vastaukset sekä asiasanahakemisto.

\newpage
\nosa{Lähtötasotesti}
\lukufilter{#1}{Oheinen lähtötasotesti auttaa selvittämään, mihin asioihin kurssilla kannattaa erityisesti perehtyä. Testin tehtävien tekemiseen pitäisi kulua aikaa yhteensä alle varttitunti. Kirjoita ratkaisut välivaiheineen ja palauta testi opettajalle ensimmäisellä tunnilla arviointia varten. Itsearviointia varten näiden tehtävien ratkaisut löytyvät kirjan lopusta. Ei hätää, jos et osaa joitain näistä tehtävistä -- jokaiseen tehtävätyyppiin paneudutaan kyllä kurssin aikana, eikä varsinaisia ennakkovaatimuksia ole.

%ratkaisut!
\begin{multicols}{2}
\begin{tehtava}
	Sievennä. 
	\alakohdat{
        § $-2^4-(-2)^3$
		§ $4\cdot \frac{2}{5} + \frac{2}{3}\cdot \frac{3}{5}$
		§ $\frac{2}{5} : \frac{3}{2}$
		§ $-2(4x^2-x)-x$
		§ $\frac{ab^2}{a^3b}$
	}
	\begin{vastaus}
		\alakohdat{
			§ $-16-(-8)=-16+8=-8$
			§ $\frac{8}{5} + \frac{2}{5}=\frac{10}{5} = 2$
			§ $\frac{2}{5} \cdot \frac{2}{3}=\frac{4}{15}$
			§ $-8x^2+x$
			§ $\frac{b}{a^2}$
		}
	\end{vastaus}
\end{tehtava}

\begin{tehtava}
	Mitä on $20$ kuutiodesimetriä
	\alakohdat{
        § litroina
		§ kuutiometreinä?
	}
	\begin{vastaus}
		\alakohdat{
			§ $20$, sillä $1\,\text{l}=1\,\text{dm}^3$
			§ $20\,\text{dm}^3=20\cdot(\text{dm})^3=20\cdot\text{d}^3\cdot\text{m}^3=20\cdot (\frac{1}{10})^3\,\text{m}^3=20\cdot\frac{1}{1\,000}\, \text{m}^3=\frac{20}{1\,000}\,\text{m}^3=\frac{2}{100}\,\text{m}^3=0,02\,\text{m}^3$
		}
	\end{vastaus}
\end{tehtava}

\begin{tehtava}
	Ratkaise yhtälöt.
	\alakohdat{
		§ $2x+5=-\frac{1}{2}$
		§ $x^2=9$
		§ $3x^3=-1$
	}
	\begin{vastaus}
		\alakohdat{
			§ $x=-\frac{11}{4}$
			§ $x=-3$ tai $x=3$
			§ $x=-\frac{1}{\sqrt[3]{3}}$
		}
	\end{vastaus}
\end{tehtava}

\begin{tehtava}
	Funktioiden $f$ ja $g$ arvot on määritelty seuraavasti: $f(x)= x^2+3x$, $g(x)=2x-8$, missä $x$ on reaaliluku.
	\alakohdat{
		§ Laske $f(-3)$.
	%	§ Piirrä funktion $g$ kuvaaja.
		§ Määritä funktion $g$ nollakohta.
	}	
	\begin{vastaus}
		\alakohdat{
			§ $(-3)^2+3\cdot(-3)=0$
 	%		§ \begin{kuvaajapohja}{0.7}{-4}{4}{-4}{4
%		\kuvaaja{2x-8}{\qquad $g(x)=2x-8$}{black}
 %			      \end{kuvaajapohja}}
			§ $x=4$
		}
	\end{vastaus}
\end{tehtava}

\begin{tehtava}
	Matka $s$ ja aika $t$ (eli kuljetun matkan kesto) ovat suoraan verrannollisia toisiinsa. Aika $t$ ja nopeus $v$ ovat kääntäen verrannollisia toisiinsa.
	\alakohdat{
		§ Jos matka kaksinkertaistuu, niin mitä käy ajalle?
		§ Jos nopeus kaksinkertaistuu, niin mitä käy ajalle?
	}
	\begin{vastaus}
		\alakohdat{
		§ Aika kaksinkertaistuu.
		§ Aika puolittuu. %RATKAISUT
		}
	\end{vastaus}
\end{tehtava}

\begin{tehtava}
Kaupassa on $15$ prosentin alennusmyynti.
	\alakohdat{
		§ Kuinka paljon $25,95$ euron paita maksaa alennuksessa? Ilmoita vastaus pyöristettynä viiden sentin tarkkuuteen.
		§ Tuotteen alennettu hinta on $21,25$ euroa. Mikä oli tuotteen alkuperäinen hinta?
	}
	\begin{vastaus}
		\alakohdat{
			§ Alennettu hinta on $25,95\,€-25,95\cdot0,15=25,95\,€\cdot(1-0,15)=25,95\,€\cdot0,75=22,06\,€\approx 22,05$ euroa
			§ Merkataan tuotteen alkuperäistä hintaa (esimerkiksi) $x$:llä, jolloin saadaan yhtälö $x\cdot 0,75=21,25\,€$. Tästä saadaan ratkaistua jakolaskulla alkuperäiseksi hinnaksi $x=\frac{21,25\,€}{0,75}=25,00\,€$
		}
	\end{vastaus}
\end{tehtava}

\end{multicols}}

\newpage
\nosa{MAA1 -- Funktiot ja yhtälöt}
\lukufilter{#1}{Matematiikka on kaikkien luonnontieteiden perusta. Se tarjoaa työkaluja asioiden täsmälliseen jäsentämiseen, päättelyyn ja mallintamiseen, joita ilman yhteiskunnan teknologinen ja tieteellinen kehittyminen ei olisi mahdollista. Alasta riippuen käsittelemme matematiikassa erilaisia \textbf{objekteja}: Geometriassa tarkastelemme kaksiulotteisia \textbf{tasokuvioita} ja kolmiulotteisia \textbf{avaruuskappaleita}. Algebra tutkii \textbf{lukualueita} ja niissä määriteltäviä \textbf{laskutoimituksia}. Todennäköisyyslaskenta tarkastelee satunnaisten \textbf{tapahtumien} esiintymistä ja suhdetta toisiinsa. Matemaattinen analyysi tutkii \textbf{funktioita} ja niiden ominaisuuksia, esimerkiksi \textbf{jatkuvuutta}, \textbf{derivoituvuutta} ja \textbf{integroituvuutta}. Matemaattista analyysiä käsittelevät pitkässä matematiikassa kurssit 7, 8, 10 ja 13 sekä osin kurssit 9 ja 12. Voidaankin sanoa, että analyysi on keskeisin aihealue lukion pitkässä matematiikassa.\footnote[1]{Tämä Suomessa käytetty lähestymistapa on käytössä monissa länsimaissa. Sen sijaan esimerkiksi Balkanilla geometrialle ja matemaattiselle todistamiselle annetaan edelleen paljon suurempi painoarvo.} Matematiikka opettaa loogista päättelytaitoa ja luovaa ongelmanratkaisukykyä, mistä on hyötyä niin opinnoissa kuin elämässä yleensäkin. Sitä hyödynnetään jollakin tavalla jokaisella tieteenalalla ja myös taiteissa. 

Pitkän matematiikan ensimmäinen kurssi MAA1 Funktiot ja yhtälöt käsittelee lähinnä lukuja ja niiden operaatioita eli laskutoimituksia. Kirjassa esittelemme luvun käsitteen ja yleisimmin käytetyt lukualueet laskutoimituksineen, ja jatkamme niistä \textbf{yhtälöihin} ja funktioihin. Kurssilla luodaan tietopohja nimensä mukaisiin aiheisiin ja opitaan välttämättömimpiä työkaluja, joita tarvitaan kaikilla myöhemmillä kursseilla. Ensimmäisen kurssin tiedot riittävät myös jo monipuolisesti monien arkielämän tilanteiden -- esimerkiksi korkolaskujen -- tarkasteluun, ja suureiden sekä yksiköiden syvällinen käsittely antaa hyvän laskennallisen pohjan fysiikan ja kemian kursseille.

Tässä kirjassa käymme läpi opetussuunnitelman mukaiset keskeiset sisällöt, joita ovat
\luettelo{
§ potenssifunktio
§ potenssiyhtälön ratkaiseminen
§ juuret ja murtopotenssi
§ eksponenttifunktio.
}

Opetussuunnitelman mukaiset kurssin keskeiset tavoitteet ovat, että opiskelija
\luettelo{
§ vahvistaa yhtälön ratkaisemisen ja prosenttilaskennan taitojaan
§ syventää verrannollisuuden, neliöjuuren ja potenssin käsitteiden ymmärtämistään
§ tottuu käyttämään neliöjuuren ja potenssin laskusääntöjä
§ syventää funktiokäsitteen ymmärtämistään tutkimalla potenssi- ja eksponenttifunktioita
§ oppii ratkaisemaan potenssiyhtälöitä.
}
}

\osa{Luvut ja laskutoimitukset}
    \luku{luku_ja_relaatio}{Luku ja relaatio}
    \luku{lukualueet}{Johdatus lukualueisiin ja aritmetiikkaan}
    \luku{jaollisuus}{Jaollisuus ja tekijöihinjako}
    \luku{sieventaminen}{Laskujärjestys ja lausekkeiden sieventäminen}
    \luku{rationaaliluvut}{Murtoluvuilla laskeminen}
%	\luku{rationaalisovelluksia}{Laskutoimitukset sovelluksissa}
	\luku{potenssi}{Potenssi}
    \luku{desimaaliluvut}{Luvun desimaaliesitys}
    \luku{pyoristaminen}{Pyöristäminen ja vastaustarkkuus}
%    \luku{suuruusluokat}{Suuruusluokat ja kymmenpotenssimuoto}
    \luku{yksikot}{Suureet ja yksiköt}
    \luku{juuret}{Juuret}
    \luku{murtopotenssi}{Murtopotenssi}
    \luku{reaaliluvut}{Reaaliluvut lukusuoralla}

\osa{Yhtälöt}
    \luku{yhtalo}{Yhtälö}
    \luku{ensimmainen}{Ensimmäisen asteen yhtälö}
    \luku{prosenttilaskenta}{Prosenttilaskenta}
    \luku{potenssiyhtalot}{Potenssiyhtälöt}
    \luku{verrannollisuus}{Suoraan ja kääntäen verrannollisuus}

\osa{Funktiot}
    \luku{funktio}{Funktio ja sen esitystavat}
    \luku{potenssifunktio}{Potenssifunktio}
    \luku{eksponenttifunktio}{Eksponenttifunktio}
	
\osa{Kertausosio}
    \nluku{LIITE_testaatietosi}{Testaa tietosi!}
    \nluku{LIITE_kertausteht}{Kertaustehtäviä}
    \nluku{LIITE_harjoituskokeet}{Harjoituskokeita}
    \nluku{LIITE_ylioppilaskokeet}{Ylioppilaskoetehtäviä}
    \nluku{LIITE_paasykokeet}{Pääsy- ja valintakoetehtäviä}

